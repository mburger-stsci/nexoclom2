%% Generated by Sphinx.
\def\sphinxdocclass{report}
\documentclass[letterpaper,10pt,english]{sphinxmanual}
\ifdefined\pdfpxdimen
   \let\sphinxpxdimen\pdfpxdimen\else\newdimen\sphinxpxdimen
\fi \sphinxpxdimen=.75bp\relax
\ifdefined\pdfimageresolution
    \pdfimageresolution= \numexpr \dimexpr1in\relax/\sphinxpxdimen\relax
\fi
%% let collapsible pdf bookmarks panel have high depth per default
\PassOptionsToPackage{bookmarksdepth=5}{hyperref}

\PassOptionsToPackage{booktabs}{sphinx}
\PassOptionsToPackage{colorrows}{sphinx}

\PassOptionsToPackage{warn}{textcomp}
\usepackage[utf8]{inputenc}
\ifdefined\DeclareUnicodeCharacter
% support both utf8 and utf8x syntaxes
  \ifdefined\DeclareUnicodeCharacterAsOptional
    \def\sphinxDUC#1{\DeclareUnicodeCharacter{"#1}}
  \else
    \let\sphinxDUC\DeclareUnicodeCharacter
  \fi
  \sphinxDUC{00A0}{\nobreakspace}
  \sphinxDUC{2500}{\sphinxunichar{2500}}
  \sphinxDUC{2502}{\sphinxunichar{2502}}
  \sphinxDUC{2514}{\sphinxunichar{2514}}
  \sphinxDUC{251C}{\sphinxunichar{251C}}
  \sphinxDUC{2572}{\textbackslash}
\fi
\usepackage{cmap}
\usepackage[T1]{fontenc}
\usepackage{amsmath,amssymb,amstext}
\usepackage{babel}



\usepackage{tgtermes}
\usepackage{tgheros}
\renewcommand{\ttdefault}{txtt}



\usepackage[Bjarne]{fncychap}
\usepackage{sphinx}

\fvset{fontsize=auto}
\usepackage{geometry}


% Include hyperref last.
\usepackage{hyperref}
% Fix anchor placement for figures with captions.
\usepackage{hypcap}% it must be loaded after hyperref.
% Set up styles of URL: it should be placed after hyperref.
\urlstyle{same}

\addto\captionsenglish{\renewcommand{\contentsname}{Contents}}

\usepackage{sphinxmessages}
\setcounter{tocdepth}{1}



\title{nexoclom2}
\date{Jan 30, 2026}
\release{0.1.0}
\author{Matthew Burger}
\newcommand{\sphinxlogo}{\vbox{}}
\renewcommand{\releasename}{Release}
\makeindex
\begin{document}

\ifdefined\shorthandoff
  \ifnum\catcode`\=\string=\active\shorthandoff{=}\fi
  \ifnum\catcode`\"=\active\shorthandoff{"}\fi
\fi

\pagestyle{empty}
\sphinxmaketitle
\pagestyle{plain}
\sphinxtableofcontents
\pagestyle{normal}
\phantomsection\label{\detokenize{index::doc}}



\chapter{Neutral EXosphere and CLOud Model, Version 2}
\label{\detokenize{index:neutral-exosphere-and-cloud-model-version-2}}
\sphinxAtStartPar
\sphinxstylestrong{NEXOCLOM2} is the Neutral EXosphere and CLOud Model, Version 2. The high
level goals and requirements for NEXOCLOM2 can be found in the
Requirements Document{]}(nexoclom2/NEXOCLOM\_Requirements\_Document)

\begin{sphinxadmonition}{note}{Note:}
\sphinxAtStartPar
This project is under active development.
\end{sphinxadmonition}

\sphinxstepscope


\section{NEXOCLOM Goals and Requirements}
\label{\detokenize{nexoclom2/NEXOCLOM_Requirements_Document:nexoclom-goals-and-requirements}}\label{\detokenize{nexoclom2/NEXOCLOM_Requirements_Document::doc}}

\subsection{Statement of Purpose}
\label{\detokenize{nexoclom2/NEXOCLOM_Requirements_Document:statement-of-purpose}}
\sphinxAtStartPar
The Neutral EXOsphere and CLoud Model is a numerical model for simulating
collisionless exospheres and neutral clouds around astronomical bodies. These
bodies may be planets (in this solar system or around other stars), satellites
of planets, or small bodies. The exosphere may be surface\sphinxhyphen{}bounded or start at a
pre\sphinxhyphen{}defined (possibly spatially or time variable) exobase above the surface.
Users can use predefined or custom initial spatial and energy distributions from
the surface or exobase. Inputs to the model are defined in a plain\sphinxhyphen{}text \sphinxcode{\sphinxupquote{.input}}
file. The documentation species the structure of the \sphinxcode{\sphinxupquote{.input}} file, which
parameters are required, and default value for optional parameters.

\sphinxAtStartPar
A full test suite with unit, integration, and regression tests is implemented to
ensure that physical processes are implemented correctly, changes in the code do
not change or break functionality, and that saved results can be retrieved
correctly.


\subsection{High Level Requirements}
\label{\detokenize{nexoclom2/NEXOCLOM_Requirements_Document:high-level-requirements}}

\subsubsection{Utilities}
\label{\detokenize{nexoclom2/NEXOCLOM_Requirements_Document:utilities}}
\sphinxAtStartPar
\sphinxstylestrong{Configuration}
\begin{enumerate}
\sphinxsetlistlabels{\arabic}{enumi}{enumii}{}{.}%
\item {} 
\sphinxAtStartPar
Configuration file is a text file containing a set of \sphinxcode{\sphinxupquote{key = value}} pairs.

\item {} 
\sphinxAtStartPar
Filename is set by an environment variable.

\item {} 
\sphinxAtStartPar
Required variable: \sphinxcode{\sphinxupquote{savepath}} = path to directory where data is saved

\item {} 
\sphinxAtStartPar
Optional variable: \sphinxcode{\sphinxupquote{database}} = Name of the TinyDB file containing the
outputs database. Default = \sphinxcode{\sphinxupquote{thesolarsystemmb.db}}

\item {} 
\sphinxAtStartPar
\sphinxcode{\sphinxupquote{user}} is required if it is not set by the operating system (e.g. with
Windows)

\end{enumerate}

\sphinxAtStartPar
\sphinxstylestrong{Database Operations}
\begin{itemize}
\item {} 
\sphinxAtStartPar
sets the path to the database correctly

\item {} 
\sphinxAtStartPar
Format as a TinyDB record
\begin{itemize}
\item {} 
\sphinxAtStartPar
Strip the input classes into dict with only floats, strings, ints,
bools, and dicts.

\item {} 
\sphinxAtStartPar
One\sphinxhyphen{}to\sphinxhyphen{}one mapping between class and cleaned up version

\end{itemize}

\item {} 
\sphinxAtStartPar
Search
\begin{itemize}
\item {} 
\sphinxAtStartPar
Builds a query only retrieves records that match the inputs.

\end{itemize}

\item {} 
\sphinxAtStartPar
insert
\begin{itemize}
\item {} 
\sphinxAtStartPar
Only inserts records that aren’t currently in the table. This won’t be
100\% foolproof since there could be multiple processes accessing the
database, but that’s acceptable.

\item {} 
\sphinxAtStartPar
Table containing outputs information that can point back to doc\_ids of
the individual input classes

\end{itemize}

\end{itemize}


\subsubsection{Input/Output}
\label{\detokenize{nexoclom2/NEXOCLOM_Requirements_Document:input-output}}
\sphinxAtStartPar
\sphinxstyleemphasis{\sphinxstylestrong{System geometry}}
\begin{enumerate}
\sphinxsetlistlabels{\arabic}{enumi}{enumii}{}{.}%
\item {} 
\sphinxAtStartPar
Specify the system geometry either with a time stamp or specifying true
anomaly angle, central meridian longitude, etc.

\item {} 
\sphinxAtStartPar
Central object (e.g., planet or Sun) for the simulation is fixed

\item {} 
\sphinxAtStartPar
StartPoint is either the central object or a satellite of the central object

\item {} 
\sphinxAtStartPar
For the constant step size model, all objects are assumed to be fixed. This
is because each packet has to be launched in the same system configuration
(especially the loss environment). Will need to give a demonstration/more
complete explanation of why this must be true.

\end{enumerate}

\sphinxAtStartPar
\sphinxstyleemphasis{\sphinxstylestrong{Energy distributions}}
\begin{enumerate}
\sphinxsetlistlabels{\arabic}{enumi}{enumii}{}{.}%
\item {} 
\sphinxAtStartPar
Support for user defined sources either as a probability distribution or
user\sphinxhyphen{}defined function

\end{enumerate}

\sphinxAtStartPar
\sphinxstyleemphasis{\sphinxstylestrong{Spatial distributions}}
\begin{enumerate}
\sphinxsetlistlabels{\arabic}{enumi}{enumii}{}{.}%
\item {} 
\sphinxAtStartPar
Support for surface maps in either solar\sphinxhyphen{}fixed or body\sphinxhyphen{}fixed coordinates

\end{enumerate}

\sphinxAtStartPar
\sphinxstyleemphasis{\sphinxstylestrong{Saved results}}
\begin{enumerate}
\sphinxsetlistlabels{\arabic}{enumi}{enumii}{}{.}%
\item {} 
\sphinxAtStartPar
Initial conditions specified by plain text input file in the form
\sphinxcode{\sphinxupquote{parameter = value}} or as a JSON file.

\item {} 
\sphinxAtStartPar
Inputs will be saved in a NoSQL database

\item {} 
\sphinxAtStartPar
When the database is queried only outputs consistent with the queried inputs
are returned.

\end{enumerate}
\begin{itemize}
\item {} 
\sphinxAtStartPar
Some input parameters may involve a comparison (e.g., 5º \textless{}= TAA \textless{} 6º)

\end{itemize}
\begin{enumerate}
\sphinxsetlistlabels{\arabic}{enumi}{enumii}{}{.}%
\setcounter{enumi}{3}
\item {} 
\sphinxAtStartPar
Outputs need to be saved in a format that can be used by packages that allow
for larger than memory datasets.

\item {} 
\sphinxAtStartPar
There should be a way to include customized source distributions specified by
the user.

\end{enumerate}
\begin{itemize}
\item {} 
\sphinxAtStartPar
When doing fitted models, don’t need any of the inputs like packets or Input

\end{itemize}


\subsubsection{Physics}
\label{\detokenize{nexoclom2/NEXOCLOM_Requirements_Document:physics}}
\sphinxAtStartPar
\sphinxstylestrong{Forces}
\begin{itemize}
\item {} 
\sphinxAtStartPar
Gravity

\item {} 
\sphinxAtStartPar
Radiation pressure

\end{itemize}

\sphinxAtStartPar
\sphinxstylestrong{Loss processes}

\sphinxAtStartPar
\sphinxstyleemphasis{\sphinxstylestrong{Coordinate systems}}
\begin{itemize}
\item {} 
\sphinxAtStartPar
Solar\sphinxhyphen{}fixed
\begin{itemize}
\item {} 
\sphinxAtStartPar
\(\hat{\mathbf{x}}\) in planet’s (starting point’s) equatorial plane toward
Sun. If the object’s tilt is included this will not necessarily point
directly at the Sun.

\item {} 
\sphinxAtStartPar
\(\hat{\mathbf{y}}\) in planet’s (starting point’s) equatorial plane duskward

\item {} 
\sphinxAtStartPar
\(\hat{\mathbf{z}}\) points along spin axis;
\(\hat{\mathbf{x}} \times \hat{mathbf{y}}\)

\item {} 
\sphinxAtStartPar
longitude: angle from \(+\hat{x}\) axis to projection of position vector
on XY plane. Increases in counter\sphinxhyphen{}clockwise direction
\begin{itemize}
\item {} 
\sphinxAtStartPar
0º = sub\sphinxhyphen{}solar point; 90º = dusk point, 180º = midnight, 270º = dawn

\end{itemize}

\item {} 
\sphinxAtStartPar
Local time increases counter\sphinxhyphen{}clockwise from 0h (midnight).
\(local\_time = ((latitude + 180^\circ) \times \frac{24}{360^\circ}) \mod 24\)

\item {} 
\sphinxAtStartPar
Latitude ranges from \sphinxhyphen{}90º (south pole) to 90º (north pole)

\end{itemize}

\item {} 
\sphinxAtStartPar
body\sphinxhyphen{}fixed
\begin{itemize}
\item {} 
\sphinxAtStartPar
TBD

\end{itemize}

\end{itemize}


\subsubsection{Solar System Module}
\label{\detokenize{nexoclom2/NEXOCLOM_Requirements_Document:solar-system-module}}\begin{itemize}
\item {} 
\sphinxAtStartPar
SSObject class
\begin{itemize}
\item {} 
\sphinxAtStartPar
Objects that contain all the physical data for solar system objects needed
by the nexoclom2

\item {} 
\sphinxAtStartPar
Required data
\begin{itemize}
\item {} 
\sphinxAtStartPar
Object name

\item {} 
\sphinxAtStartPar
object type (star, planet, moon). Not really important.

\item {} 
\sphinxAtStartPar
What the object orbits (central object to be used for simulations that
extend outside the Hill sphere

\item {} 
\sphinxAtStartPar
Names of satellites

\item {} 
\sphinxAtStartPar
radius

\item {} 
\sphinxAtStartPar
mass

\item {} 
\sphinxAtStartPar
GM (for gravitational acceleration)

\item {} 
\sphinxAtStartPar
semi\sphinxhyphen{}major axis

\item {} 
\sphinxAtStartPar
eccentricity

\item {} 
\sphinxAtStartPar
axis tilt

\item {} 
\sphinxAtStartPar
rotational period

\item {} 
\sphinxAtStartPar
orbital period

\item {} 
\sphinxAtStartPar
NAIF ID

\end{itemize}

\item {} 
\sphinxAtStartPar
Quantities with units stored as astropy quanitites

\item {} 
\sphinxAtStartPar
Data is pre\sphinxhyphen{}determined and stored in a text file included with the
nexoclom2 package.

\item {} 
\sphinxAtStartPar
Sources for data need to be included (listed in the datafile).

\item {} 
\sphinxAtStartPar
If object not found, return object with unknown quantities as None

\end{itemize}

\item {} 
\sphinxAtStartPar
Function or class to object geometry relative to the Sun for a solar
system object.
\begin{itemize}
\item {} 
\sphinxAtStartPar
Can be determined either from a time stamp using SPICE kernels or from a
specified planet true anomaly angle (TAA) or satellite orbital position.

\item {} 
\sphinxAtStartPar
Necessary quantities
\begin{itemize}
\item {} 
\sphinxAtStartPar
Distance from Sun

\item {} 
\sphinxAtStartPar
Radial velocity relative to Sun

\item {} 
\sphinxAtStartPar
Object\sphinxhyphen{}Sun vector taking into account planet’s tilt.

\end{itemize}

\item {} 
\sphinxAtStartPar
Additional functions needed to access SPICE kernels.

\end{itemize}

\item {} 
\sphinxAtStartPar
Possible future work
\begin{itemize}
\item {} 
\sphinxAtStartPar
Extended to include information about exoplanet systems.

\item {} 
\sphinxAtStartPar
Extract data from spice kernels when possible, especially for objects
that aren’t in the included table.

\end{itemize}

\end{itemize}


\subsection{Test Requirements}
\label{\detokenize{nexoclom2/NEXOCLOM_Requirements_Document:test-requirements}}
\sphinxAtStartPar
\sphinxstylestrong{NexoclomConfig}

\sphinxstepscope


\section{Installation}
\label{\detokenize{nexoclom2/install:installation}}\label{\detokenize{nexoclom2/install::doc}}
\sphinxAtStartPar
Installation of NEXOCLOM2 Monte Carlo model of neutral clouds and exospheres.

\sphinxAtStartPar
If properly installed you should see this:

\begin{sphinxVerbatim}[commandchars=\\\{\}]
\PYG{g+gp}{\PYGZgt{}\PYGZgt{}\PYGZgt{} }\PYG{k+kn}{import}\PYG{+w}{ }\PYG{n+nn}{nexoclom2}
\PYG{g+gp}{\PYGZgt{}\PYGZgt{}\PYGZgt{} }\PYG{n+nb}{print}\PYG{p}{(}\PYG{n}{nexoclom2}\PYG{o}{.}\PYG{n}{\PYGZus{}\PYGZus{}version\PYGZus{}\PYGZus{}}\PYG{p}{)}
\PYG{g+go}{0.1.1}
\end{sphinxVerbatim}

\sphinxstepscope


\section{NEXOCLOM Test Cases}
\label{\detokenize{nexoclom2/TestCases:nexoclom-test-cases}}\label{\detokenize{nexoclom2/TestCases::doc}}
\sphinxAtStartPar
Documentation for all the unit, system, and regression tests performed in the nexoclom test suite.


\subsection{Utilities}
\label{\detokenize{nexoclom2/TestCases:utilities}}

\subsubsection{NexoclomConfig}
\label{\detokenize{nexoclom2/TestCases:nexoclomconfig}}\begin{itemize}
\item {} 
\sphinxAtStartPar
Successful runs
\begin{enumerate}
\sphinxsetlistlabels{\arabic}{enumi}{enumii}{}{.}%
\item {} 
\sphinxAtStartPar
Only savepath is given ➔ nexoclom2a

\item {} 
\sphinxAtStartPar
savepath, database, user are given ➔ nexoclom2b

\item {} 
\sphinxAtStartPar
savepath, extra unused value are given ➔ nexoclom2c

\end{enumerate}

\item {} 
\sphinxAtStartPar
Failure runs
\begin{enumerate}
\sphinxsetlistlabels{\arabic}{enumi}{enumii}{}{.}%
\item {} 
\sphinxAtStartPar
savepath not given ➔ nexoclom2d

\item {} 
\sphinxAtStartPar
user not given and is not a environment variable ➔ nexoclom2e
\begin{itemize}
\item {} 
\sphinxAtStartPar
Also contains a line that isn’t read in to complete the code coverage

\end{itemize}

\item {} 
\sphinxAtStartPar
NEXOCLOMCONFIG environment variable not set

\item {} 
\sphinxAtStartPar
NEXOCLOMCONFIG file does not exist ➔ nexoclom2f

\end{enumerate}

\end{itemize}


\subsubsection{DatabaseOperations}
\label{\detokenize{nexoclom2/TestCases:databaseoperations}}\begin{itemize}
\item {} 
\sphinxAtStartPar
\sphinxstylestrong{init}
\begin{itemize}
\item {} 
\sphinxAtStartPar
sets the path to the database correctly

\end{itemize}

\item {} 
\sphinxAtStartPar
make\_acceptable
\begin{itemize}
\item {} 
\sphinxAtStartPar
Can strip the input classes into dict with only floats, strings, ints,
bools, and dicts.

\item {} 
\sphinxAtStartPar
One\sphinxhyphen{}to\sphinxhyphen{}one mapping between class and cleaned up version

\end{itemize}

\item {} 
\sphinxAtStartPar
insert\_parts
\begin{itemize}
\item {} 
\sphinxAtStartPar
Only inserts records that aren’t currently in the table. This won’t be
100\% foolproof since there could be multiple processes accessing the
database, but that’s acceptable.

\end{itemize}

\item {} 
\sphinxAtStartPar
query\_parts

\item {} 
\sphinxAtStartPar
insert outputs

\item {} 
\sphinxAtStartPar
query\_outputs

\end{itemize}


\subsection{Solar System}
\label{\detokenize{nexoclom2/TestCases:solar-system}}

\subsubsection{SSObject}
\label{\detokenize{nexoclom2/TestCases:ssobject}}\begin{enumerate}
\sphinxsetlistlabels{\arabic}{enumi}{enumii}{}{.}%
\item {} 
\sphinxAtStartPar
Case\sphinxhyphen{}insensitive object name

\item {} 
\sphinxAtStartPar
Object with NAIF ID but not in table

\item {} 
\sphinxAtStartPar
Object that doesn’t exist at all

\item {} 
\sphinxAtStartPar
Object equality

\item {} 
\sphinxAtStartPar
Object length (number of satellites + 1)

\item {} 
\sphinxAtStartPar
Planets and satellites

\end{enumerate}


\subsection{Initial State}
\label{\detokenize{nexoclom2/TestCases:initial-state}}

\subsubsection{Input}
\label{\detokenize{nexoclom2/TestCases:input}}\begin{itemize}
\item {} 
\sphinxAtStartPar
Don’t need to test every configuration. That’s done for the individual
pieces. Just need to verfiy that it parses the file correctly.

\end{itemize}


\subsubsection{Geometry}
\label{\detokenize{nexoclom2/TestCases:geometry}}\begin{enumerate}
\sphinxsetlistlabels{\arabic}{enumi}{enumii}{}{.}%
\item {} 
\sphinxAtStartPar
Setting everything
\begin{enumerate}
\sphinxsetlistlabels{\arabic}{enumii}{enumiii}{}{.}%
\item {} 
\sphinxAtStartPar
With modeltime

\item {} 
\sphinxAtStartPar
Without modeltime

\end{enumerate}

\item {} 
\sphinxAtStartPar
Setting only required

\item {} 
\sphinxAtStartPar
Starting from Sun, Planet, Satellite

\item {} 
\sphinxAtStartPar
Exceptions are properly raised.

\end{enumerate}


\subsubsection{Forces}
\label{\detokenize{nexoclom2/TestCases:forces}}\begin{enumerate}
\sphinxsetlistlabels{\arabic}{enumi}{enumii}{}{.}%
\item {} 
\sphinxAtStartPar
Basic comparisons on equality and non\sphinxhyphen{}equality of different settings

\end{enumerate}

\sphinxstepscope


\section{Input File Format}
\label{\detokenize{nexoclom2/inputfiles:input-file-format}}\label{\detokenize{nexoclom2/inputfiles:inputfiles}}\label{\detokenize{nexoclom2/inputfiles::doc}}
\sphinxAtStartPar
Input files are plain text files in the form:

\begin{sphinxVerbatim}[commandchars=\\\{\}]
\PYG{n}{category}\PYG{o}{.}\PYG{n}{parameter} \PYG{o}{=} \PYG{n}{setting}
\end{sphinxVerbatim}

\sphinxAtStartPar
Lines in the input file that can not be parsed in this manner are ignored.
Comments can be entered with a “\#”. Everything in a line
after a comment character is ignored. There are currently eight categories
that can be set: geometry, surface\_interaction,
forces, spatialdist, speeddist, angulardist, loss\_information, and options. The required
parameters for each category are not fixed; i.e., which parameters are needed
depends somewhat on the settings chosen. Below, all possible parameters for
each category are defined. Input files are case insensitive.


\subsection{Geometry}
\label{\detokenize{nexoclom2/inputfiles:geometry}}\label{\detokenize{nexoclom2/inputfiles:id1}}
\sphinxAtStartPar
The geometry can be defined either with a timestamp (i.e., determine the
geometry values at a defined epoch), or without a timestamp (i.e., by
specifying important values). If running the model from MESSENGERdata.model(),
the geometry is determined from from the data and geometry settings in the
input file are ignored.
\begin{description}
\sphinxlineitem{geometry.center {[}Required{]}}
\sphinxAtStartPar
Central object for the model.

\sphinxlineitem{geometry.startpoint {[}Optional{]}}
\sphinxAtStartPar
Object from which packets are ejected. This must be an object in the
planetary system (the planet or one of its moons). If the
\sphinxcode{\sphinxupquote{geometry.planet == Sun}}, the starting point must either be the Sun or
a planet. Default = geometry.center

\sphinxlineitem{geometry.include {[}Optional{]}}
\sphinxAtStartPar
Objects to include in calculations given as comma\sphinxhyphen{}separated list of
bodies in the planetary system. For example, if
\sphinxcode{\sphinxupquote{geometry.objects = Jupiter, Io}}, the gravity effects of the other moons
would not be included, nor would collisions with their surfaces.
Default = geometry.planet, geometry.startpoint

\end{description}


\subsubsection{Geometry With Time Stamp}
\label{\detokenize{nexoclom2/inputfiles:geometry-with-time-stamp}}\label{\detokenize{nexoclom2/inputfiles:geometrytime}}\begin{description}
\sphinxlineitem{geometry.modeltime {[}Required{]}}
\sphinxAtStartPar
Time at which the model is simulated (model end time) in a format that can be
read by
\sphinxhref{https://docs.astropy.org/en/stable/api/astropy.time.Time.html\#astropy.time.Time}{astropy.time.Time}
(YYYY\sphinxhyphen{}MM\sphinxhyphen{}DD HH:MM:SS.S or YYYY\sphinxhyphen{}MM\sphinxhyphen{}DDTHH:MM:SS.S are best).
The true anomaly angle, subsolar point, and orbital position of all objects
are determined using the
\sphinxhref{https://spiceypy.readthedocs.io/en/stable/}{SpiceyPy} python wrapper for the
\sphinxhref{https://naif.jpl.nasa.gov/naif/toolkit.html}{SPICE} toolkit
(\sphinxhref{https://doi.org/10.21105/joss.02050}{Annex et al., (2020). SpiceyPy: a Pythonic Wrapper for the SPICE Toolkit. Journal of Open Source Software, 5(46), 2050}.

\end{description}


\subsubsection{Geometry Without Time Stamp}
\label{\detokenize{nexoclom2/inputfiles:geometry-without-time-stamp}}\label{\detokenize{nexoclom2/inputfiles:geometrynotime}}
\begin{sphinxadmonition}{warning}{Warning:}
\sphinxAtStartPar
The coordinate transformations for GeometryNoTime are not set up correctly
at the moment.
\end{sphinxadmonition}
\begin{description}
\sphinxlineitem{geometry.TAA {[}Optional{]}}
\sphinxAtStartPar
Planet’s True Anomaly Angle in degrees. This is used to determine the
planet’s distance and radial velocity relative to the Sun. Default = 0º

\sphinxlineitem{geometry.phi {[}Required if satellites are included in the simulation{]}}
\sphinxAtStartPar
Orbital phase of each included satellite relative to the Sun in degrees given
as a comma\sphinxhyphen{}separated list.
Measured from 0º to 360º where 0º is superior conjunction and
90º is over the planet’s dawn terminator. The number of values must be
equal to the number of non\sphinxhyphen{}planet objects included.

\sphinxlineitem{geometry.subsolarpoint {[}Optional{]}}
\sphinxAtStartPar
The sub\sphinxhyphen{}solar longitude and latitude over the planet’s surface in degrees
given as comma\sphinxhyphen{}separated values. For Jupiter, use System\sphinxhyphen{}III central
meridian longitude. Sub\sphinxhyphen{}solar latitude isn’t used for anything currently,
but could in the future be used to include effects of the planetary system’s
tilt relative to the Sun.

\sphinxlineitem{geometry.dtaa {[}Optional{]}}
\sphinxAtStartPar
Tolerance for true anomaly differences in model searches in degrees.
Default = 2º

\end{description}


\subsection{SurfaceInteraction}
\label{\detokenize{nexoclom2/inputfiles:surfaceinteraction}}\label{\detokenize{nexoclom2/inputfiles:surfaceinteractions}}
\sphinxAtStartPar
The SurfaceInteraction class defines interactions between packets and body
surfaces. The available parameters depend on the interactions desired.
If no values are provided, 100\% sticking is assumed.


\subsubsection{Constant Sticking Coefficient}
\label{\detokenize{nexoclom2/inputfiles:constant-sticking-coefficient}}\label{\detokenize{nexoclom2/inputfiles:constantsticking}}
\sphinxAtStartPar
surfaceinteracction.type {[}Default = constant{]}
\begin{description}
\sphinxlineitem{surfaceinteraction.stickcoef {[}Default = 1{]}}
\sphinxAtStartPar
Sticking coefficient to be used uniformly across the body’s surface.
For complete surface sticking, set \sphinxcode{\sphinxupquote{surfaceinteraction.stickcoef = 1.}}.
For no sticking (100\% of packets are reemitted from the surface, set
\sphinxcode{\sphinxupquote{surfaceinteraction.stickcoef = 0}}.

\sphinxlineitem{surfaceinteraction.accomfactor {[}Required if stickcoef \textless{} 1{]}}
\sphinxAtStartPar
Surface accommodation factor. 1 = Fully accommodated to local surface
temperature. 0 = Elastic scattering.

\end{description}


\subsubsection{Temperature Dependent Sticking Coefficient}
\label{\detokenize{nexoclom2/inputfiles:temperature-dependent-sticking-coefficient}}\label{\detokenize{nexoclom2/inputfiles:tempdependsticking}}
\begin{sphinxadmonition}{note}{Note:}
\sphinxAtStartPar
This has not been implemented yet.
\end{sphinxadmonition}

\sphinxAtStartPar
The sticking coefficient follows the functional form (Yakshinskiy \& Madey 2005):
\begin{equation*}
\begin{split}S(T) = A_0 e^{A_1 T} + A_2\end{split}
\end{equation*}
\sphinxAtStartPar
where the coefficients are species dependent. For Na,
\begin{quote}

\sphinxAtStartPar
Surface accommodation factor. 1 = Fully accommodated to local surface
temperature. 0 = Elastic reemission.
\end{quote}


\subsubsection{Sticking Coefficient from a Surface Map}
\label{\detokenize{nexoclom2/inputfiles:sticking-coefficient-from-a-surface-map}}\label{\detokenize{nexoclom2/inputfiles:surfacemapsticking}}
\begin{sphinxadmonition}{note}{Note:}
\sphinxAtStartPar
This has not been implemented yet.
\end{sphinxadmonition}


\subsection{Forces}
\label{\detokenize{nexoclom2/inputfiles:forces}}\label{\detokenize{nexoclom2/inputfiles:id2}}
\sphinxAtStartPar
The Forces class determines which forces are included in the simulation.
Currently, the model only includes gravity and radiation pressure. If
no forces are set in the input file both are included by default.
\begin{description}
\sphinxlineitem{forces.gravity {[}Optional{]}}
\sphinxAtStartPar
Default = True

\sphinxlineitem{forces.radpres {[}Optional{]}}
\sphinxAtStartPar
Default = True

\end{description}


\subsection{SpatialDist}
\label{\detokenize{nexoclom2/inputfiles:spatialdist}}\label{\detokenize{nexoclom2/inputfiles:id3}}
\sphinxAtStartPar
The SpatialDist class specifies the initial spatial distribution of packets
in the system. Currently, three spatial distribution types are defined, all of
which place packets over the surface (or exobase) of \sphinxtitleref{geometry.StartingPoint}.
More distributions may defined upon request.

\sphinxAtStartPar
Three coordinate systems (frames) are available for each body: the body\sphinxhyphen{}fixed
IAU frame, a SOLAR frame with the \sphinxstyleemphasis{z}\sphinxhyphen{}axis aligned with the rotation axis and
the \sphinxstyleemphasis{x}\sphinxhyphen{}axis directed toward the Sun, and a SOLARFIXED frame with the \sphinxstyleemphasis{x}\sphinxhyphen{}axis
pointed directly at the Sun. These are described further at
{\hyperref[\detokenize{nexoclom2/coordinate_systems:coordinate-systems}]{\sphinxcrossref{\DUrole{std}{\DUrole{std-ref}{Planetary Coordinate Systems and System Geometry}}}}}.


\subsubsection{Uniform Surface}
\label{\detokenize{nexoclom2/inputfiles:uniform-surface}}\label{\detokenize{nexoclom2/inputfiles:uniformspatdist}}
\sphinxAtStartPar
Distribute packets randomly across a region of the surface or exobase with
a uniform probability distribution.
\begin{description}
\sphinxlineitem{spatialdist.type {[}Required{]}}
\sphinxAtStartPar
Set \sphinxtitleref{spatialdist.type = uniform}.

\sphinxlineitem{spatialdist.longitude {[}Optional{]}}
\sphinxAtStartPar
Longitude range on the surface to place packets in degrees given as
\sphinxstyleemphasis{long0, long1} where \(0^\circ \leq long0,long1 \leq 360^\circ\).
If \sphinxstyleemphasis{long0} \textgreater{} \sphinxstyleemphasis{long1}, the region wraps around. Default = 0, 360

\sphinxlineitem{spatialdist.latitude {[}Optional{]}}
\sphinxAtStartPar
Latitude range on the surface to place packets in degrees given as
\sphinxstyleemphasis{lat0, lat1} where \(90^\circ \leq lat0 \leq lat1 \leq 90^\circ\).

\sphinxlineitem{spatialdist.exobase {[}Optional{]}}
\sphinxAtStartPar
Location of the exobase in units of the starting point’s radius.
Default = 1.

\sphinxlineitem{spatialdist.frame {[}Optional{]}}
\sphinxAtStartPar
SPICE frame to use. Options are “IAU”, “SOLAR”, “SOLARFIXED”. The correct
body is added to the name for use in the SPICE toolkit.

\end{description}

\sphinxAtStartPar
To eject all packets from a single point, set \sphinxstyleemphasis{long0 = long1} and
\sphinxstyleemphasis{lat0 = lat1}; i.e., to eject all packets from the sub\sphinxhyphen{}solar point of a planet,
set:

\begin{sphinxVerbatim}[commandchars=\\\{\}]
\PYG{n}{spatialdist}\PYG{o}{.}\PYG{n}{longitude} \PYG{o}{=} \PYG{l+m+mi}{0}\PYG{p}{,}\PYG{l+m+mi}{0}
\PYG{n}{spatialdist}\PYG{o}{.}\PYG{n}{latitude} \PYG{o}{=} \PYG{l+m+mi}{0}\PYG{p}{,}\PYG{l+m+mi}{0}
\PYG{n}{spatialdist}\PYG{o}{.}\PYG{n}{frame} \PYG{o}{=} \PYG{n}{SolarFixed}
\end{sphinxVerbatim}


\subsubsection{Golden Spiral Spatial Distribution}
\label{\detokenize{nexoclom2/inputfiles:golden-spiral-spatial-distribution}}\label{\detokenize{nexoclom2/inputfiles:goldenspiralspatdist}}
\sphinxAtStartPar
This approximates a regular grid of packets on a sphere using the
\sphinxhref{https://docs.astropy.org/en/stable/coordinates/angles.html\#on-a-grid}{global spiral method} implemented in \sphinxtitleref{astropy.coordinates}.
\begin{description}
\sphinxlineitem{spatialdist.exobase {[}Optional{]}}
\sphinxAtStartPar
Location of the exobase in units of the starting point’s radius.
Default = 1.

\end{description}


\subsubsection{Spatial Distribution from a Surface Map}
\label{\detokenize{nexoclom2/inputfiles:spatial-distribution-from-a-surface-map}}\label{\detokenize{nexoclom2/inputfiles:surfacemapspatdist}}
\begin{sphinxadmonition}{note}{Note:}
\sphinxAtStartPar
This has not been set up yet.
\end{sphinxadmonition}

\sphinxAtStartPar
Distribute packets according to a probability distribution given by a
pre\sphinxhyphen{}defined surface map.
\begin{description}
\sphinxlineitem{spatialdist.type {[}Required{]}}
\sphinxAtStartPar
Set \sphinxtitleref{spatialdist.type = surface map}.

\sphinxlineitem{spatialdist.mapfile {[}Optional{]}}
\sphinxAtStartPar
Set this to a pickle or IDL savefile containing the map information, or
set to ‘default’ to use the default surface composition map.

\sphinxAtStartPar
The sourcemap is saved as a dictionary with the fields:
\begin{itemize}
\item {} 
\sphinxAtStartPar
longitude: longitude axis in degrees

\item {} 
\sphinxAtStartPar
latitude: latitude axis in degrees

\item {} 
\sphinxAtStartPar
abundance: surface abundance map

\item {} 
\sphinxAtStartPar
coordinate\_system: planet\sphinxhyphen{}fixed, solar\sphinxhyphen{}fixed, or moon\sphinxhyphen{}fixed

\end{itemize}

\sphinxAtStartPar
If not given, the default, planet\sphinxhyphen{}fixed surface composition map is used.

\sphinxlineitem{spatialdist.exobase {[}Optional{]}}
\sphinxAtStartPar
Location of the exobase in units of the starting point’s radius.
Default = 1.

\end{description}


\subsubsection{Surface\sphinxhyphen{}Spot Spatial Distribution}
\label{\detokenize{nexoclom2/inputfiles:surface-spot-spatial-distribution}}\label{\detokenize{nexoclom2/inputfiles:surfacespotspatdist}}
\begin{sphinxadmonition}{note}{Note:}
\sphinxAtStartPar
This has not been set up yet.
\end{sphinxadmonition}

\sphinxAtStartPar
Distribute packets with a spatial distribution that drops off exponentially
from a central point.
\begin{description}
\sphinxlineitem{spatialdist.type {[}Required{]}}
\sphinxAtStartPar
Set \sphinxtitleref{spatialdist.type = surface spot}.

\sphinxlineitem{spatialdist.longitude {[}Required{]}}
\sphinxAtStartPar
Longitude of the source center in degrees.

\sphinxlineitem{spatialdist.latitude {[}Required{]}}
\sphinxAtStartPar
Latitude of the soruce center in degrees.

\sphinxlineitem{spatialdist.sigma {[}Required{]}}
\sphinxAtStartPar
Angular e\sphinxhyphen{}folding width of the source in degrees.

\sphinxlineitem{spatialdist.exobase {[}Optional{]}}
\sphinxAtStartPar
Location of the exobase in units of the starting point’s radius.
Default = 1.

\end{description}


\subsection{SpeedDist}
\label{\detokenize{nexoclom2/inputfiles:speeddist}}\label{\detokenize{nexoclom2/inputfiles:id4}}
\sphinxAtStartPar
The SpeedDist class defines the one\sphinxhyphen{}dimensional initial speed distribution
of the packets. Currently implemented speed distributions are
Maxwellian, sputtering, and flat. More can be added upon request.


\subsubsection{Gaussian (Normal) distribution}
\label{\detokenize{nexoclom2/inputfiles:gaussian-normal-distribution}}\label{\detokenize{nexoclom2/inputfiles:gaussianspeeddist}}
\begin{sphinxadmonition}{note}{Note:}
\sphinxAtStartPar
This has not been set up yet.
\end{sphinxadmonition}

\sphinxAtStartPar
Packets speeds are chosen from a normal distribution. See
{\color{red}\bfseries{}\textasciigrave{}}numpy.random.normal
for more information on the implementation.
\begin{description}
\sphinxlineitem{speeddist.type {[}Required{]}}
\sphinxAtStartPar
Set \sphinxtitleref{speeddist.type = gaussian}

\sphinxlineitem{speeddist.vprob {[}Required{]}}
\sphinxAtStartPar
Mean speed of the distribution in km/s.

\sphinxlineitem{speeddist.sigma {[}Required{]}}
\sphinxAtStartPar
Standard deviation of the distribution in km/s.

\end{description}


\subsubsection{Maxwellian Distribution}
\label{\detokenize{nexoclom2/inputfiles:maxwellian-distribution}}\label{\detokenize{nexoclom2/inputfiles:maxwellianspeeddist}}\begin{eqnarray*}
f(v) & \propto & v^3 \exp(-v^2/v_{th}^2) \\
v_{th}^2 & = & 2Tk_B/m
\end{eqnarray*}\begin{description}
\sphinxlineitem{speeddist.type {[}Required{]}}
\sphinxAtStartPar
Set \sphinxtitleref{speeddist.type = maxwellian}

\sphinxlineitem{speeddist.temperature {[}Required{]}}
\sphinxAtStartPar
Temperature of the distribution in K. Set \sphinxtitleref{speeddist.temperature = 0} to
use a pre\sphinxhyphen{}defined surface temperature map (Not implemented yet).

\end{description}


\subsubsection{Sputtering Distribution}
\label{\detokenize{nexoclom2/inputfiles:sputtering-distribution}}\label{\detokenize{nexoclom2/inputfiles:sputterspeeddist}}
\sphinxAtStartPar
Packet speeds are chosen from a sputtering distribution in the form:
\begin{eqnarray*}
f(v) & \propto & \frac{v^{2\beta + 1}}{(v^2 + v_b^2)^\alpha} \\
v_b & = & \left(\frac{2U}{m} \right)^{1/2}
\end{eqnarray*}\begin{description}
\sphinxlineitem{speeddist.type {[}Required{]}}
\sphinxAtStartPar
Set \sphinxtitleref{speeddist.type = sputtering}

\sphinxlineitem{speeddist.alpha {[}Required{]}}
\sphinxAtStartPar
\(\alpha\) parameter.

\sphinxlineitem{speeddist.beta {[}Required{]}}
\sphinxAtStartPar
\(\beta\) parameter.

\sphinxlineitem{speeddist.U {[}Required{]}}
\sphinxAtStartPar
Surface binding energy in eV.

\end{description}


\subsubsection{Flat Distribution}
\label{\detokenize{nexoclom2/inputfiles:flat-distribution}}\label{\detokenize{nexoclom2/inputfiles:flatspeeddist}}
\sphinxAtStartPar
Chooses the initial speed uniformly with \(v_{min} \leq v < v_{max}\).
\begin{description}
\sphinxlineitem{speeddist.type {[}Required{]}}
\sphinxAtStartPar
Set \sphinxtitleref{speeddist.type = flat}

\sphinxlineitem{speeddist.vmin {[}Optional{]}}
\sphinxAtStartPar
Minimum speed in km/s. Default = 0 km/s

\sphinxlineitem{speeddist.vmax {[}Optional{]}}
\sphinxAtStartPar
Maximum speed in km/s. Default = 10 km/s

\end{description}


\subsection{AngularDist}
\label{\detokenize{nexoclom2/inputfiles:angulardist}}\label{\detokenize{nexoclom2/inputfiles:id7}}
\sphinxAtStartPar
The AngularDist class defines the initial angular distribution of packets.
The options are radial and isotropic. More distributions can be added upon
request. If not given, an isotropic distribution into the outward facing
hemisphere is assumed.


\subsubsection{Radial Distribution}
\label{\detokenize{nexoclom2/inputfiles:radial-distribution}}\label{\detokenize{nexoclom2/inputfiles:radialangulardist}}
\sphinxAtStartPar
Packets are ejected radially from the surface.
\begin{description}
\sphinxlineitem{angulardist.type {[}Required{]}}
\sphinxAtStartPar
Set \sphinxtitleref{angulardist.type = radial}.

\end{description}


\subsubsection{Isotropic Distribution}
\label{\detokenize{nexoclom2/inputfiles:isotropic-distribution}}\label{\detokenize{nexoclom2/inputfiles:isotropicangulardist}}
\sphinxAtStartPar
Packets are ejected isotropically into the outward facing hemisphere (if the
packets are starting from the surface) or the full hemisphere.
\sphinxtitleref{angulardist.type} is not given, an isotropic distribution is assumed and
all other options are ignored (i.e., altitude and azimuth can not be specified).
\begin{description}
\sphinxlineitem{angulardist.type {[}Optional{]}}
\sphinxAtStartPar
Set \sphinxtitleref{angulardist.type = isotropic}.

\sphinxlineitem{angulardist.altitude {[}Optional{]}}
\sphinxAtStartPar
Used to limit the altitude range of the distribution. Given as a
comma\sphinxhyphen{}separated list of \sphinxstyleemphasis{altmin, altmax} in degrees measured from the
surface tangent to the surface normal.

\sphinxlineitem{angulardist.azimuth {[}Optional{]}}
\sphinxAtStartPar
Used to limit the azimuth range of the distribution. Given as a
comma\sphinxhyphen{}separated list of \sphinxstyleemphasis{az0, az1} in degrees. This should be measured with
azimuth = 0 rad pointing to north, but I’m not sure if it actually works.
Use of this option is not recommended.

\end{description}


\subsection{Loss Information}
\label{\detokenize{nexoclom2/inputfiles:loss-information}}\label{\detokenize{nexoclom2/inputfiles:lossinfo}}
\sphinxAtStartPar
The LossInformation class configures the processes responsible for loss not due
to collisions with the surface. Three processes can be included: photoionization
or dissociation, electron\sphinxhyphen{}impact ionization or dissociation, and charge exchange.
The physical processes and determination of relevant plasma parameters are
described at {\hyperref[\detokenize{nexoclom2/lossrates:lossrates}]{\sphinxcrossref{\DUrole{std}{\DUrole{std-ref}{Neutral Loss Rates}}}}}.
\begin{description}
\sphinxlineitem{lifetime {[}Optional{]}}
\sphinxAtStartPar
Assigns a constant lifetime everywhere in the system. Default=False
(constant not applied). If this is set, it provides an loss floor; i.e.,
other loss mechanisms can still be included with the loss rates additive.

\sphinxlineitem{photoionization {[}Optional{]}}
\sphinxAtStartPar
Set to True to include photoionization. Default = True

\sphinxlineitem{photo\_lifetime {[}Optional{]}}
\sphinxAtStartPar
If 0, then uses the calculated photoionzitaion rate. If \textgreater{}0, sets the
photoionization lifetime to that value in seconds. Default = 0

\sphinxlineitem{photo\_factor {[}Optional{]}}
\sphinxAtStartPar
Constant factor by which to modify the photoionization lifetime. Useful for
testing the effect of changed photoionization rates. Default = 1

\sphinxlineitem{electron\_impact {[}Optional{]}}
\sphinxAtStartPar
Set to True to include loss due to electron impacts. Default = True for
planets with known plasma parameters, False otherwise.

\sphinxlineitem{eimp\_factor {[}Optional{]}}
\sphinxAtStartPar
Constant factor by which to modify the electron impact loss rates.
Default = 1

\sphinxlineitem{charge\_exchange {[}Optional{]}}
\sphinxAtStartPar
Set to True to include charge exchange. Default = True for planets with
known plasma parameters, False otherwise.

\sphinxlineitem{chx\_factor {[}Optional{]}}
\sphinxAtStartPar
Constant factor by which to modify the charge exchange rates. Default = 1

\end{description}


\subsection{Options}
\label{\detokenize{nexoclom2/inputfiles:options}}\label{\detokenize{nexoclom2/inputfiles:id8}}
\sphinxAtStartPar
The Options class sets runtime options that don’t fit into other categories.
\begin{quote}

\sphinxAtStartPar
The total simulated runtime for the model. Generally chosen to be several
times the lifetime of the species.
\end{quote}
\begin{description}
\sphinxlineitem{options.runtime {[}Required{]}}
\sphinxAtStartPar
Total simulation run time in seconds.

\sphinxlineitem{options.species {[}Required{]}}
\sphinxAtStartPar
The species to be simulated.

\sphinxlineitem{options.outer\_edge {[}Optional{]}}
\sphinxAtStartPar
Distance from \sphinxstyleemphasis{geometry.startpoint} to simulate in object radii. Default =
infinite; i.e., no outer edge is given to the simulation.

\sphinxlineitem{options.step\_size {[}Optional{]}}
\sphinxAtStartPar
Step size in seconds. If 0, uses the adaptive step size integrator. Otherwise
it uses the constant step size integrator. Default = 0.

\sphinxlineitem{options.resolution {[}Optional{]}}
\sphinxAtStartPar
Required precision for the adaptive step size integrator.
Default = \(10^{-5}\).

\sphinxlineitem{options.start\_together {[}Optional{]}}
\sphinxAtStartPar
If using the adaptive step size integrator, \sphinxtitleref{options.start\_together = True}
starts all packets at the same time. Otherwise, start times are
randomly distributed between the start and end of the simulation.
Default = False

\sphinxlineitem{options.random\_seed {[}Optional{]}}
\sphinxAtStartPar
Seed for the random number generator. Default = None

\end{description}

\sphinxstepscope


\section{Solar System Geometries}
\label{\detokenize{nexoclom2/planet_distance:solar-system-geometries}}\label{\detokenize{nexoclom2/planet_distance:planetdistance}}\label{\detokenize{nexoclom2/planet_distance::doc}}
\sphinxAtStartPar
The positions of each body relative to another is handled with the \sphinxcode{\sphinxupquote{SSObject}}
object combined with a \sphinxcode{\sphinxupquote{Geometry}} object.

\sphinxAtStartPar
Sun frame: x\sphinxhyphen{}axis points toward perihelion, y\sphinxhyphen{}axis points toward TAA=90º,
z\sphinxhyphen{}axis points along planet rotational axis. Origin = Solar center

\sphinxAtStartPar
Planet frame: x\sphinxhyphen{}axis points toward Sun, y\sphinxhyphen{}axis points toward dusk, z\sphinxhyphen{}axis
points along rotational axis. Origin = Planet center.

\sphinxAtStartPar
Moon frame: x\sphinxhyphen{}axis points toward Sun, y\sphinxhyphen{}axis points toward dusk, z\sphinxhyphen{}axis
points along rotational axis. Origin = Moon center.


\subsection{Computing Distance and Radial Velocity of a Planet Relative to the Sun}
\label{\detokenize{nexoclom2/planet_distance:computing-distance-and-radial-velocity-of-a-planet-relative-to-the-sun}}
\sphinxAtStartPar
The distance from the central body as a function of true anomaly is given by:
\begin{equation*}
\begin{split}r = a \frac{1 - e^2}{1 + e \cos \nu}\end{split}
\end{equation*}
\sphinxAtStartPar
where \sphinxstyleemphasis{r} is the distance from the central body, \sphinxstyleemphasis{a} is the semi\sphinxhyphen{}major axis of
the orbit, \sphinxstyleemphasis{e} is the eccentriciy, and \sphinxstyleemphasis{ν} is the true anomaly.

\sphinxAtStartPar
To calculate the \(dr/dt\), the radial velocity relative to the central
body, it is necessary to determine the mean anomaly from the true anomaly as
the true anomaly does not increase at a constant rate. The mean anomaly, \sphinxstyleemphasis{M} does
increase at a constant rate. The mean anomaly as a function of time is simply
\begin{equation*}
\begin{split}M(t) = 2\pi \frac{t}{P}\end{split}
\end{equation*}
\sphinxAtStartPar
where \sphinxstyleemphasis{P} is the planet’s orbital period. The true anomaly can be approximated
from the mean anomaly with:G
\begin{equation*}
\begin{split}\nu = M + (2e - \frac{1}{4}e^3) \sin M + \frac{5}{4}e^2 \sin 2M +
\frac{13/12} e^3 \sin 3M + \mathcal{O}(e^4)\end{split}
\end{equation*}
\sphinxAtStartPar
This allows one to compute \(dr/dt\) using the equation for \sphinxstyleemphasis{r} above.

\sphinxAtStartPar
A comparison of distance and radial velocity relative to the Sun computed
using the above equations and retrieved from
\sphinxhref{https://astroquery.readthedocs.io/en/latest/jplhorizons/jplhorizons.html}{JPL Horizons}
for Mercury, Earth, Jupiter and Saturn are shown here:

\noindent\sphinxincludegraphics{{nexoclom2/distance_and_velocity_Mercury}.png}

\noindent\sphinxincludegraphics{{nexoclom2/distance_and_velocity_Earth}.png}

\noindent\sphinxincludegraphics{{nexoclom2/distance_and_velocity_Jupiter}.png}

\noindent\sphinxincludegraphics{{nexoclom2/distance_and_velocity_Saturn}.png}

\sphinxstepscope


\section{Planetary Coordinate Systems and System Geometry}
\label{\detokenize{nexoclom2/coordinate_systems:planetary-coordinate-systems-and-system-geometry}}\label{\detokenize{nexoclom2/coordinate_systems:coordinate-systems}}\label{\detokenize{nexoclom2/coordinate_systems::doc}}

\subsection{Mercury}
\label{\detokenize{nexoclom2/coordinate_systems:mercury}}
\sphinxAtStartPar
Mercury’s rotational period and orbital period are in a 2:3 resonance: two
Mercury days are equal to three Mercury years. The consequence of this is that
the sub\sphinxhyphen{}solar longitude on Mercury is a double\sphinxhyphen{}valued function of true anomaly
angle. If the sub\sphinxhyphen{}solar longitude is \(\lambda\) at true anomaly \(\nu\)
one year, then it will be \(\lambda + 180^\circ\) (modulo 360º) the
following year and \(\lambda\) the year after that. This is illustrated
below:

\noindent\sphinxincludegraphics{{nexoclom2/planet_geometry_Mercury}.png}

\sphinxAtStartPar
Also shown is Mercury’s sub\sphinxhyphen{}solar latitude, which is always near\sphinxhyphen{}zero due to the
lack of any significant axial tilt.


\subsection{Coordinate Systems on a Planet}
\label{\detokenize{nexoclom2/coordinate_systems:coordinate-systems-on-a-planet}}
\sphinxAtStartPar
The coordinate system used for the object’s latitude and longitude depends
on whether the packets are ejected from a planet or a moon. For planets, a
solar\sphinxhyphen{}fixed coordinate system is used where the longitude increases in the
positive direction from the sub\sphinxhyphen{}solar point (noon) point to dusk point:

\begin{sphinxVerbatim}[commandchars=\\\{\}]
sub\PYGZhy{}solar (noon) point = 0 rad = 0°
dusk point (leading) = π/2 rad = 90°
anti\PYGZhy{}solar (midnight) point = π rad = 180°
dawn point (trailing) = 3π/2 rad = 270°
\end{sphinxVerbatim}

\sphinxAtStartPar
Latitude ranges from \sphinxhyphen{}π/2 rad to π/2 rad for the south pole to the north pole.
All angular values are given in radians in the input file.


\subsection{Coordinate Systems on a Moon}
\label{\detokenize{nexoclom2/coordinate_systems:coordinate-systems-on-a-moon}}
\sphinxAtStartPar
For satellites, the coordinate system is planet\sphinxhyphen{}fixed from the sub\sphinxhyphen{}planet
point increasing positive through the leading point:

\begin{sphinxVerbatim}[commandchars=\\\{\}]
sub\PYGZhy{}planet point = 0 rad = 0°
leading point = π/2 rad = 90°
anti\PYGZhy{}planet point = π rad = 180°
trailing point = 3π/2 rad = 270°
\end{sphinxVerbatim}

\sphinxAtStartPar
Latitude ranges from \sphinxhyphen{}π/2 rad to π/2 rad for the south pole to the north pole.
All angular values are given in radians in the input file.

\sphinxstepscope


\section{Interactions between incident atoms and surfaces}
\label{\detokenize{nexoclom2/surface_interaction:interactions-between-incident-atoms-and-surfaces}}\label{\detokenize{nexoclom2/surface_interaction::doc}}
\sphinxAtStartPar
The locations and speeds of atoms that hit the surface are tracked. When a
particle is beneath the surface at a time step, it is necessary to find where
on the surface the particle hit.
\begin{equation*}
\begin{split}r^2 & = & \sum (x_i - X_i)^2 \\
R^2 & = & \sum (x'_i - X_i)^2\end{split}
\end{equation*}
\sphinxAtStartPar
where
\begin{equation*}
\begin{split}x'_i = x_i - v_i t\end{split}
\end{equation*}
\sphinxAtStartPar
where \(x_i, v_i\) are the components of the position and velocity found
in the model, \(x'_i\) are the components of the position where the
particle hit the surface, :math::\sphinxtitleref{R’ is the radius of the object located at
:math::\textasciigrave{}X\_i}, and \(t\) needs to be solved for. This problem
can be reformulated as a quadratic equation in \(t\):
\begin{equation*}
\begin{split}\sum(v_i^2) t^2 + \sum(x_i v_i)t - 2\sum x_i^2 - R^2 = 0\end{split}
\end{equation*}
\sphinxAtStartPar
which can be solved through the standard methods.

\sphinxAtStartPar
The impact speed is determined from conservation of energy:
\begin{equation*}
\begin{split}KE + PE & = & KE' + PE' \\
\frac{\sum v_i^2}{2} + \frac{GM}{r} & = & \frac{\sum v_i^{'2}}{2} + \frac{GM}{R}\end{split}
\end{equation*}
\sphinxAtStartPar
where \(KE = \frac{\sum v_i^2}{2}\) is the kinetic energy per atom,
\(PE = \frac{GM}{r}\) is the potential energy per atom, and \(R\) is the
object radius. Solving for the updated velocity \(v'\) gives:
\begin{equation*}
\begin{split}\sum v_i^{'2} = \sum v_i^2 + 2 G M \left(\frac{1}{r} - \frac{1}{R} \right)\end{split}
\end{equation*}
\sphinxstepscope


\section{Neutral Loss Rates}
\label{\detokenize{nexoclom2/lossrates:neutral-loss-rates}}\label{\detokenize{nexoclom2/lossrates:lossrates}}\label{\detokenize{nexoclom2/lossrates::doc}}
\sphinxAtStartPar
Neutral species ejected from solar system bodies are lost through collisions
with surfaces or rings, gravitational escape from the system, or interactions
with the surrounding environment. The last of these is discussed here. Three
loss process are included here: photoionization or dissociation, electron impact
ionization or dissociation, and charge\sphinxhyphen{}exchange. These depend on the local
solar photon flux, electron distribution, and ion distributions respectively.

\sphinxAtStartPar
Photoionziation rates for for atomic species have been calculated by Huebner \&
Mukherjee (2015). The accuracy of these rates may be questionable in some cases,
but for now these are the default photoionization rates. See Killen et al.,
Mercury book, for a discussion of photoionization rates for Na, Ca, and Mg.

\sphinxstepscope


\section{Runge\sphinxhyphen{}Kutta Integrator \& The Equations of Motion}
\label{\detokenize{nexoclom2/integrator:runge-kutta-integrator-the-equations-of-motion}}\label{\detokenize{nexoclom2/integrator::doc}}

\subsection{The Equations of Motion}
\label{\detokenize{nexoclom2/integrator:the-equations-of-motion}}
\sphinxAtStartPar
Two forces act on atoms ejected into the exosphere: gravity from each object in
the system and radiation pressure. The components of the gravitational force are
given by:
\begin{equation*}
\begin{split}F_i & = & \sum_p GM (x_i - x_{pi})/r_p^3 \\
r_p & = & \sqrt\left(\sum_j (x_j - x_{pj})^2\right)\end{split}
\end{equation*}
\sphinxAtStartPar
where in the first equation \(x_i\) are the components of the particle
location and \(x_{pi}\) are the components of object \(p\) in the
system and the sum is over all objcts included. The second equation gives the
distances between each particle and object.
\begin{equation*}
\begin{split}y_x(t) & = & x(t_0) + \frac{dx}{dt} t \\
y_{vx}(t) & = & v_x(t_0) + \frac{d^2x}{dt} t \\
f(t) & = & e^{-t \nu}\end{split}
\end{equation*}
\sphinxstepscope


\section{Atomic Data}
\label{\detokenize{nexoclom2/atomicdata:atomic-data}}\label{\detokenize{nexoclom2/atomicdata::doc}}

\subsection{Atom}
\label{\detokenize{nexoclom2/atomicdata:atom}}
\sphinxAtStartPar
The \sphinxcode{\sphinxupquote{Atom}} class is intended to store all the atomic data for different species
used in NEXOCLOM2. This includes the specific neutral species being simulated,
but also magnetosopheric ions that ionize and induce emissions in the neutral
species. Currently, relevant \sphinxstyleemphasis{g}\sphinxhyphen{}values, electron impact excitation rate
coefficients, and electron impact ionization rate coefficients are available
through the \sphinxcode{\sphinxupquote{Atom}} class.

\begin{sphinxadmonition}{note}{Note:}
\sphinxAtStartPar
Charge exchange coefficients have not yet been included.
\end{sphinxadmonition}

\begin{sphinxadmonition}{note}{Note:}
\sphinxAtStartPar
The Atom class cannot currently handle molecular species.
\end{sphinxadmonition}


\subsection{Ionization}
\label{\detokenize{nexoclom2/atomicdata:ionization}}

\subsubsection{Photoionization}
\label{\detokenize{nexoclom2/atomicdata:photoionization}}
\sphinxAtStartPar
Loss due to ionization is customized in the input file with the \sphinxcode{\sphinxupquote{lossinfo}}
object. Three loss mechanisms can be specified: a constant lifetime,
photoinization, and electron impact ionization (currently only implemented
at Jupiter). Charge exchange will be added in the future. All three mechanisms
can be included. The rates of each included mechanism are summed.

\sphinxAtStartPar
The photoionization rate, \(\nu_{PI}(r)\), is a function of distance
\(r\)from the Sun according to
\begin{equation*}
\begin{split}\nu_{PI}(r) = \nu_{PI}(r_0) \left(\frac{r_0}{r}\right)^2\end{split}
\end{equation*}
\sphinxAtStartPar
where \(\nu_{PI}(r_0)\) is the photoioniation rate at the reference point
distance from the Sun \(r_0\) (1 AU in \sphinxcite{nexoclom2/atomicdata:huebner2015}). The photoionization
rate is zero when an atom is in the geometric shadow of a planet or moon.

\sphinxAtStartPar
Photoionization and photodissocation rates are taken from \sphinxcite{nexoclom2/atomicdata:huebner2015},
although see the caveats in {[}Killen2018{]} regarding the Na photoionzaion rate.
Atomic species with photoionization rates available are H, He, C, N, O, N, Mg,
S, Cl, K, and Ca. Molecular species with photoionization and or photodissociation
rates included are H$_{\text{2}}$, CH$_{\text{4}}$, NH$_{\text{3}}$, OH, H$_{\text{2}}$O,
N$_{\text{2}}$, O$_{\text{2}}$ , CO$_{\text{2}}$ , and SO$_{\text{2}}$ , although as
noted above, the Atom class cannot currently handle molecular species.


\subsubsection{Electron Impact Ionization}
\label{\detokenize{nexoclom2/atomicdata:electron-impact-ionization}}\label{\detokenize{nexoclom2/atomicdata:eimp-ionization}}
\sphinxAtStartPar
The electron impact ionization rate, \(\nu_{EII}(n_e, T_e)\) is a function
of electron density (\(n_e\)) and temperature (\(T_e\)):
\begin{equation*}
\begin{split}\nu_{EII}(n_e, T_e) = \kappa(T_e) n_e\end{split}
\end{equation*}
\sphinxAtStartPar
where \(\kappa(T_e)\) is the electron impact ionization rate coefficient.

\sphinxAtStartPar
In general, electron impact ionization cross sections are reported, not the
rate coefficients. The rate coefficient as function of electron temperature
is found by
convolving the ionization cross sections over the electron speed distribution
function, assumed here to be thermal (Maxwellian flux):
\begin{equation*}
\begin{split}f(v; T_e) = 4 \pi \left(\frac{m_e}{2\pi k_b T_e}\right)^{3/2} v^3
    e^{-\frac{m_e v^2}{2 k_b T_e}}\end{split}
\end{equation*}
\sphinxAtStartPar
where \(f(v)\) is the probability of an electron having speed
:math\textasciigrave{}v\textasciigrave{}, \(m_e\) is the electron mass, :math\textasciigrave{}k\_b\textasciigrave{} is
Boltzmann’s constant, and \(T_e\) is the electron temperature.

\sphinxAtStartPar
The rate coefficient \(\kappa(T_e)\) is given by:
\begin{equation*}
\begin{split}\kappa(T_e) = \int_0^\infty f(v; T_e) \sigma(v) dv\end{split}
\end{equation*}
\sphinxAtStartPar
The integral is computed numerically for a range of electron temperatures.

\sphinxAtStartPar
Electron impact ionization cross sections and rate coefficients for Na and O are
taken
from {[}Johnston1995{]} and {[}Johnson2005{]}, respectively, and shown in Figure X.

\noindent\sphinxincludegraphics{{ElectronImpactIonization}.png}


\subsubsection{Charge Exchange}
\label{\detokenize{nexoclom2/atomicdata:charge-exchange}}
\sphinxAtStartPar
Charge exchange between ions and neutral atoms has not yet been implemented.


\subsection{Routines}
\label{\detokenize{nexoclom2/atomicdata:routines}}
\sphinxAtStartPar
This is an incomplete list of classes and functions available in the atomic data
module. It is limited to the routines a user might need or want to use, excluding
more behind the scenes classes. The API reference includes all classes and
functions.


\begin{savenotes}\sphinxattablestart
\sphinxthistablewithglobalstyle
\sphinxthistablewithnovlinesstyle
\centering
\begin{tabulary}{\linewidth}[t]{\X{1}{2}\X{1}{2}}
\sphinxtoprule
\sphinxtableatstartofbodyhook
\sphinxAtStartPar
{\hyperref[\detokenize{autoapi/nexoclom2/atomicdata/atom/index:nexoclom2.atomicdata.atom.Atom}]{\sphinxcrossref{\sphinxcode{\sphinxupquote{nexoclom2.atomicdata.atom.Atom}}}}}(species)
&
\sphinxAtStartPar
Class containing all useful atomic data for a neutral or ionic species.
\\
\sphinxbottomrule
\end{tabulary}
\sphinxtableafterendhook\par
\sphinxattableend\end{savenotes}


\begin{savenotes}\sphinxattablestart
\sphinxthistablewithglobalstyle
\sphinxthistablewithnovlinesstyle
\centering
\begin{tabulary}{\linewidth}[t]{\X{1}{2}\X{1}{2}}
\sphinxtoprule
\sphinxtableatstartofbodyhook
\sphinxAtStartPar
{\hyperref[\detokenize{autoapi/nexoclom2/atomicdata/gvalues/index:nexoclom2.atomicdata.gvalues.gValue}]{\sphinxcrossref{\sphinxcode{\sphinxupquote{nexoclom2.atomicdata.gvalues.gValue}}}}}(species)
&
\sphinxAtStartPar
Class to compute g\sphinxhyphen{}values and radiation acceleration
\\
\sphinxbottomrule
\end{tabulary}
\sphinxtableafterendhook\par
\sphinxattableend\end{savenotes}

\sphinxstepscope


\section{Particle Tracking}
\label{\detokenize{nexoclom2/particle_tracking:particle-tracking}}\label{\detokenize{nexoclom2/particle_tracking::doc}}
\sphinxstepscope


\section{References}
\label{\detokenize{nexoclom2/references:references}}\label{\detokenize{nexoclom2/references::doc}}
\sphinxstepscope


\section{API Reference}
\label{\detokenize{autoapi/index:api-reference}}\label{\detokenize{autoapi/index::doc}}
\sphinxAtStartPar
This page contains auto\sphinxhyphen{}generated API reference documentation %
\begin{footnote}[1]\sphinxAtStartFootnote
Created with \sphinxhref{https://github.com/readthedocs/sphinx-autoapi}{sphinx\sphinxhyphen{}autoapi}
%
\end{footnote}.

\sphinxstepscope


\subsection{Geometry}
\label{\detokenize{autoapi/Geometry/index:module-Geometry}}\label{\detokenize{autoapi/Geometry/index:geometry}}\label{\detokenize{autoapi/Geometry/index::doc}}\index{module@\spxentry{module}!Geometry@\spxentry{Geometry}}\index{Geometry@\spxentry{Geometry}!module@\spxentry{module}}

\subsubsection{Classes}
\label{\detokenize{autoapi/Geometry/index:classes}}

\begin{savenotes}\sphinxattablestart
\sphinxthistablewithglobalstyle
\sphinxthistablewithnovlinesstyle
\centering
\begin{tabulary}{\linewidth}[t]{\X{1}{2}\X{1}{2}}
\sphinxtoprule
\sphinxtableatstartofbodyhook
\sphinxAtStartPar
{\hyperref[\detokenize{autoapi/Geometry/index:Geometry.Geometry}]{\sphinxcrossref{\sphinxcode{\sphinxupquote{Geometry}}}}}
&
\sphinxAtStartPar
Solar System geometry information
\\
\sphinxbottomrule
\end{tabulary}
\sphinxtableafterendhook\par
\sphinxattableend\end{savenotes}


\subsubsection{Module Contents}
\label{\detokenize{autoapi/Geometry/index:module-contents}}\index{Geometry (class in Geometry)@\spxentry{Geometry}\spxextra{class in Geometry}}

\begin{fulllineitems}
\phantomsection\label{\detokenize{autoapi/Geometry/index:Geometry.Geometry}}
\pysigstartsignatures
\pysiglinewithargsret
{\sphinxbfcode{\sphinxupquote{class\DUrole{w}{ }}}\sphinxcode{\sphinxupquote{Geometry.}}\sphinxbfcode{\sphinxupquote{Geometry}}}
{\sphinxparam{\DUrole{n}{gparam}\DUrole{p}{:}\DUrole{w}{ }\DUrole{n}{dict\DUrole{p}{,}\DUrole{w}{ }tinydb.table.Document}}}
{}
\pysigstopsignatures
\sphinxAtStartPar
Bases: {\hyperref[\detokenize{autoapi/nexoclom2/initial_state/InputClass/index:nexoclom2.initial_state.InputClass.InputClass}]{\sphinxcrossref{\sphinxcode{\sphinxupquote{nexoclom2.initial\_state.InputClass.InputClass}}}}}

\sphinxAtStartPar
Solar System geometry information

\sphinxAtStartPar
Base class for Geometry inputs. This is not intended to be called by the
user. Sets parameters used jointly by GeometryTime and GeometryNoTime.

\sphinxAtStartPar
Parameters set here
\begin{itemize}
\item {} 
\sphinxAtStartPar
center

\item {} 
\sphinxAtStartPar
startpoint

\item {} 
\sphinxAtStartPar
included

\end{itemize}

\sphinxAtStartPar
See {\hyperref[\detokenize{nexoclom2/inputfiles:geometry}]{\sphinxcrossref{\DUrole{std}{\DUrole{std-ref}{Geometry}}}}} for more information.
\begin{quote}\begin{description}
\sphinxlineitem{Parameters}\begin{description}
\sphinxlineitem{\sphinxstylestrong{gparam}}{[}dict{]}
\sphinxAtStartPar
keys, values for defining system geometry.

\end{description}

\sphinxlineitem{Attributes}\begin{description}
\sphinxlineitem{\sphinxstylestrong{\_\_name\_\_}}{[}‘Geometry’{]}
\sphinxlineitem{\sphinxstylestrong{type}}{[}‘geometry\_base’{]}
\sphinxlineitem{\sphinxstylestrong{center}}{[}str{]}
\sphinxAtStartPar
Central body for the model

\sphinxlineitem{\sphinxstylestrong{startpoint}}{[}str{]}
\sphinxAtStartPar
Object from which packets are ejected.

\sphinxlineitem{\sphinxstylestrong{include}}{[}tuple of str{]}
\sphinxAtStartPar
Objects included in calculations.

\end{description}

\end{description}\end{quote}
\index{\_\_name\_\_ (Geometry.Geometry attribute)@\spxentry{\_\_name\_\_}\spxextra{Geometry.Geometry attribute}}

\begin{fulllineitems}
\phantomsection\label{\detokenize{autoapi/Geometry/index:Geometry.Geometry.__name__}}
\pysigstartsignatures
\pysigline
{\sphinxbfcode{\sphinxupquote{\_\_name\_\_}}\sphinxbfcode{\sphinxupquote{\DUrole{w}{ }\DUrole{p}{=}\DUrole{w}{ }\textquotesingle{}Geometry\textquotesingle{}}}}
\pysigstopsignatures
\end{fulllineitems}

\index{\_\_repr\_\_() (Geometry.Geometry method)@\spxentry{\_\_repr\_\_()}\spxextra{Geometry.Geometry method}}

\begin{fulllineitems}
\phantomsection\label{\detokenize{autoapi/Geometry/index:Geometry.Geometry.__repr__}}
\pysigstartsignatures
\pysiglinewithargsret
{\sphinxbfcode{\sphinxupquote{\_\_repr\_\_}}}
{}
{}
\pysigstopsignatures
\end{fulllineitems}

\index{\_\_str\_\_() (Geometry.Geometry method)@\spxentry{\_\_str\_\_()}\spxextra{Geometry.Geometry method}}

\begin{fulllineitems}
\phantomsection\label{\detokenize{autoapi/Geometry/index:Geometry.Geometry.__str__}}
\pysigstartsignatures
\pysiglinewithargsret
{\sphinxbfcode{\sphinxupquote{\_\_str\_\_}}}
{}
{}
\pysigstopsignatures
\sphinxAtStartPar
Override of superclass \_\_str\_\_

\end{fulllineitems}

\index{\_\_eq\_\_() (Geometry.Geometry method)@\spxentry{\_\_eq\_\_()}\spxextra{Geometry.Geometry method}}

\begin{fulllineitems}
\phantomsection\label{\detokenize{autoapi/Geometry/index:Geometry.Geometry.__eq__}}
\pysigstartsignatures
\pysiglinewithargsret
{\sphinxbfcode{\sphinxupquote{\_\_eq\_\_}}}
{\sphinxparam{\DUrole{n}{other}}}
{}
\pysigstopsignatures
\end{fulllineitems}


\end{fulllineitems}


\sphinxstepscope


\subsection{nexoclom2}
\label{\detokenize{autoapi/nexoclom2/index:module-nexoclom2}}\label{\detokenize{autoapi/nexoclom2/index:nexoclom2}}\label{\detokenize{autoapi/nexoclom2/index::doc}}\index{module@\spxentry{module}!nexoclom2@\spxentry{nexoclom2}}\index{nexoclom2@\spxentry{nexoclom2}!module@\spxentry{module}}
\sphinxAtStartPar
NEXOCLOM2: Neutral EXosphere and CLoud Model v2

\sphinxAtStartPar
Documentation will be online at readthedocs.org


\subsubsection{Submodules}
\label{\detokenize{autoapi/nexoclom2/index:submodules}}
\sphinxstepscope


\paragraph{nexoclom2.atomicdata}
\label{\detokenize{autoapi/nexoclom2/atomicdata/index:module-nexoclom2.atomicdata}}\label{\detokenize{autoapi/nexoclom2/atomicdata/index:nexoclom2-atomicdata}}\label{\detokenize{autoapi/nexoclom2/atomicdata/index::doc}}\index{module@\spxentry{module}!nexoclom2.atomicdata@\spxentry{nexoclom2.atomicdata}}\index{nexoclom2.atomicdata@\spxentry{nexoclom2.atomicdata}!module@\spxentry{module}}
\sphinxAtStartPar
nexoclom2 routines for working with atomic data


\subparagraph{Submodules}
\label{\detokenize{autoapi/nexoclom2/atomicdata/index:submodules}}
\sphinxstepscope


\subparagraph{nexoclom2.atomicdata.atom}
\label{\detokenize{autoapi/nexoclom2/atomicdata/atom/index:module-nexoclom2.atomicdata.atom}}\label{\detokenize{autoapi/nexoclom2/atomicdata/atom/index:nexoclom2-atomicdata-atom}}\label{\detokenize{autoapi/nexoclom2/atomicdata/atom/index::doc}}\index{module@\spxentry{module}!nexoclom2.atomicdata.atom@\spxentry{nexoclom2.atomicdata.atom}}\index{nexoclom2.atomicdata.atom@\spxentry{nexoclom2.atomicdata.atom}!module@\spxentry{module}}

\subparagraph{Classes}
\label{\detokenize{autoapi/nexoclom2/atomicdata/atom/index:classes}}

\begin{savenotes}\sphinxattablestart
\sphinxthistablewithglobalstyle
\sphinxthistablewithnovlinesstyle
\centering
\begin{tabulary}{\linewidth}[t]{\X{1}{2}\X{1}{2}}
\sphinxtoprule
\sphinxtableatstartofbodyhook
\sphinxAtStartPar
{\hyperref[\detokenize{autoapi/nexoclom2/atomicdata/atom/index:nexoclom2.atomicdata.atom.Atom}]{\sphinxcrossref{\sphinxcode{\sphinxupquote{Atom}}}}}
&
\sphinxAtStartPar
Class containing all useful atomic data for a neutral or ionic species.
\\
\sphinxbottomrule
\end{tabulary}
\sphinxtableafterendhook\par
\sphinxattableend\end{savenotes}


\subparagraph{Module Contents}
\label{\detokenize{autoapi/nexoclom2/atomicdata/atom/index:module-contents}}\index{Atom (class in nexoclom2.atomicdata.atom)@\spxentry{Atom}\spxextra{class in nexoclom2.atomicdata.atom}}

\begin{fulllineitems}
\phantomsection\label{\detokenize{autoapi/nexoclom2/atomicdata/atom/index:nexoclom2.atomicdata.atom.Atom}}
\pysigstartsignatures
\pysiglinewithargsret
{\sphinxbfcode{\sphinxupquote{class\DUrole{w}{ }}}\sphinxcode{\sphinxupquote{nexoclom2.atomicdata.atom.}}\sphinxbfcode{\sphinxupquote{Atom}}}
{\sphinxparam{\DUrole{n}{species}\DUrole{p}{:}\DUrole{w}{ }\DUrole{n}{str}}}
{}
\pysigstopsignatures
\sphinxAtStartPar
Class containing all useful atomic data for a neutral or ionic species.
\begin{quote}\begin{description}
\sphinxlineitem{Parameters}\begin{description}
\sphinxlineitem{\sphinxstylestrong{species: str}}
\sphinxAtStartPar
Atomic species. Charge is indicated by the number of ‘+’ or ‘\sphinxhyphen{}’
appended to the chemical symbol

\end{description}

\sphinxlineitem{Attributes}\begin{description}
\sphinxlineitem{\sphinxstylestrong{charge: int}}
\sphinxAtStartPar
Electric charge

\sphinxlineitem{\sphinxstylestrong{symbol: str}}
\sphinxAtStartPar
Chemical symbol

\sphinxlineitem{\sphinxstylestrong{name: str}}
\sphinxlineitem{\sphinxstylestrong{number: int}}
\sphinxlineitem{\sphinxstylestrong{mass: astropy Quantity}}
\sphinxlineitem{\sphinxstylestrong{gvalues: nexoclom2 gValue object}}
\sphinxlineitem{\sphinxstylestrong{photo\_refpt: astropy Quantity}}
\sphinxAtStartPar
Reference point for the photoionziation rate (generally 1 AU)

\sphinxlineitem{\sphinxstylestrong{photo\_rate: astropy Quantity}}
\sphinxAtStartPar
Photoionization rate at the reference point

\sphinxlineitem{\sphinxstylestrong{photo\_reactions: list}}
\sphinxAtStartPar
List of tuples containing each photoreaction included and its rate at
the reference point

\sphinxlineitem{\sphinxstylestrong{wavelengths: tuple}}
\sphinxAtStartPar
List of wavelengths for which there are photon scattering rate
coefficients (\sphinxstyleemphasis{g}\sphinxhyphen{}values) or electron impact excitation rate
coefficients.

\sphinxlineitem{\sphinxstylestrong{\_atom: Element}}
\sphinxAtStartPar
Object from
\sphinxhref{https://periodictable.readthedocs.io/en/latest/periodictable}{periodictable}
saved in case there is additional infomation the user might need.

\sphinxlineitem{\sphinxstylestrong{eimp\_ionization: nexoclom2 EimpIonizationCoef object}}
\sphinxlineitem{\sphinxstylestrong{eimp\_emission: nexoclom2 EimpEmissionCoef object}}
\end{description}

\end{description}\end{quote}
\index{\_atom (nexoclom2.atomicdata.atom.Atom attribute)@\spxentry{\_atom}\spxextra{nexoclom2.atomicdata.atom.Atom attribute}}

\begin{fulllineitems}
\phantomsection\label{\detokenize{autoapi/nexoclom2/atomicdata/atom/index:nexoclom2.atomicdata.atom.Atom._atom}}
\pysigstartsignatures
\pysigline
{\sphinxbfcode{\sphinxupquote{\_atom}}}
\pysigstopsignatures
\end{fulllineitems}

\index{charge (nexoclom2.atomicdata.atom.Atom attribute)@\spxentry{charge}\spxextra{nexoclom2.atomicdata.atom.Atom attribute}}

\begin{fulllineitems}
\phantomsection\label{\detokenize{autoapi/nexoclom2/atomicdata/atom/index:nexoclom2.atomicdata.atom.Atom.charge}}
\pysigstartsignatures
\pysigline
{\sphinxbfcode{\sphinxupquote{charge}}}
\pysigstopsignatures
\end{fulllineitems}

\index{symbol (nexoclom2.atomicdata.atom.Atom attribute)@\spxentry{symbol}\spxextra{nexoclom2.atomicdata.atom.Atom attribute}}

\begin{fulllineitems}
\phantomsection\label{\detokenize{autoapi/nexoclom2/atomicdata/atom/index:nexoclom2.atomicdata.atom.Atom.symbol}}
\pysigstartsignatures
\pysigline
{\sphinxbfcode{\sphinxupquote{symbol}}}
\pysigstopsignatures
\end{fulllineitems}

\index{name (nexoclom2.atomicdata.atom.Atom attribute)@\spxentry{name}\spxextra{nexoclom2.atomicdata.atom.Atom attribute}}

\begin{fulllineitems}
\phantomsection\label{\detokenize{autoapi/nexoclom2/atomicdata/atom/index:nexoclom2.atomicdata.atom.Atom.name}}
\pysigstartsignatures
\pysigline
{\sphinxbfcode{\sphinxupquote{name}}}
\pysigstopsignatures
\end{fulllineitems}

\index{number (nexoclom2.atomicdata.atom.Atom attribute)@\spxentry{number}\spxextra{nexoclom2.atomicdata.atom.Atom attribute}}

\begin{fulllineitems}
\phantomsection\label{\detokenize{autoapi/nexoclom2/atomicdata/atom/index:nexoclom2.atomicdata.atom.Atom.number}}
\pysigstartsignatures
\pysigline
{\sphinxbfcode{\sphinxupquote{number}}}
\pysigstopsignatures
\end{fulllineitems}

\index{mass (nexoclom2.atomicdata.atom.Atom attribute)@\spxentry{mass}\spxextra{nexoclom2.atomicdata.atom.Atom attribute}}

\begin{fulllineitems}
\phantomsection\label{\detokenize{autoapi/nexoclom2/atomicdata/atom/index:nexoclom2.atomicdata.atom.Atom.mass}}
\pysigstartsignatures
\pysigline
{\sphinxbfcode{\sphinxupquote{mass}}}
\pysigstopsignatures
\end{fulllineitems}

\index{gvalues (nexoclom2.atomicdata.atom.Atom attribute)@\spxentry{gvalues}\spxextra{nexoclom2.atomicdata.atom.Atom attribute}}

\begin{fulllineitems}
\phantomsection\label{\detokenize{autoapi/nexoclom2/atomicdata/atom/index:nexoclom2.atomicdata.atom.Atom.gvalues}}
\pysigstartsignatures
\pysigline
{\sphinxbfcode{\sphinxupquote{gvalues}}}
\pysigstopsignatures
\end{fulllineitems}

\index{photo\_refpt (nexoclom2.atomicdata.atom.Atom attribute)@\spxentry{photo\_refpt}\spxextra{nexoclom2.atomicdata.atom.Atom attribute}}

\begin{fulllineitems}
\phantomsection\label{\detokenize{autoapi/nexoclom2/atomicdata/atom/index:nexoclom2.atomicdata.atom.Atom.photo_refpt}}
\pysigstartsignatures
\pysigline
{\sphinxbfcode{\sphinxupquote{photo\_refpt}}}
\pysigstopsignatures
\end{fulllineitems}

\index{eimp\_ionization (nexoclom2.atomicdata.atom.Atom attribute)@\spxentry{eimp\_ionization}\spxextra{nexoclom2.atomicdata.atom.Atom attribute}}

\begin{fulllineitems}
\phantomsection\label{\detokenize{autoapi/nexoclom2/atomicdata/atom/index:nexoclom2.atomicdata.atom.Atom.eimp_ionization}}
\pysigstartsignatures
\pysigline
{\sphinxbfcode{\sphinxupquote{eimp\_ionization}}}
\pysigstopsignatures
\end{fulllineitems}

\index{eimp\_emission (nexoclom2.atomicdata.atom.Atom attribute)@\spxentry{eimp\_emission}\spxextra{nexoclom2.atomicdata.atom.Atom attribute}}

\begin{fulllineitems}
\phantomsection\label{\detokenize{autoapi/nexoclom2/atomicdata/atom/index:nexoclom2.atomicdata.atom.Atom.eimp_emission}}
\pysigstartsignatures
\pysigline
{\sphinxbfcode{\sphinxupquote{eimp\_emission}}}
\pysigstopsignatures
\end{fulllineitems}

\index{wavelengths (nexoclom2.atomicdata.atom.Atom attribute)@\spxentry{wavelengths}\spxextra{nexoclom2.atomicdata.atom.Atom attribute}}

\begin{fulllineitems}
\phantomsection\label{\detokenize{autoapi/nexoclom2/atomicdata/atom/index:nexoclom2.atomicdata.atom.Atom.wavelengths}}
\pysigstartsignatures
\pysigline
{\sphinxbfcode{\sphinxupquote{wavelengths}}}
\pysigstopsignatures
\end{fulllineitems}

\index{charge\_exchange (nexoclom2.atomicdata.atom.Atom attribute)@\spxentry{charge\_exchange}\spxextra{nexoclom2.atomicdata.atom.Atom attribute}}

\begin{fulllineitems}
\phantomsection\label{\detokenize{autoapi/nexoclom2/atomicdata/atom/index:nexoclom2.atomicdata.atom.Atom.charge_exchange}}
\pysigstartsignatures
\pysigline
{\sphinxbfcode{\sphinxupquote{charge\_exchange}}}
\pysigstopsignatures
\end{fulllineitems}

\index{\_\_str\_\_() (nexoclom2.atomicdata.atom.Atom method)@\spxentry{\_\_str\_\_()}\spxextra{nexoclom2.atomicdata.atom.Atom method}}

\begin{fulllineitems}
\phantomsection\label{\detokenize{autoapi/nexoclom2/atomicdata/atom/index:nexoclom2.atomicdata.atom.Atom.__str__}}
\pysigstartsignatures
\pysiglinewithargsret
{\sphinxbfcode{\sphinxupquote{\_\_str\_\_}}}
{}
{}
\pysigstopsignatures
\end{fulllineitems}

\index{\_\_repr\_\_() (nexoclom2.atomicdata.atom.Atom method)@\spxentry{\_\_repr\_\_()}\spxextra{nexoclom2.atomicdata.atom.Atom method}}

\begin{fulllineitems}
\phantomsection\label{\detokenize{autoapi/nexoclom2/atomicdata/atom/index:nexoclom2.atomicdata.atom.Atom.__repr__}}
\pysigstartsignatures
\pysiglinewithargsret
{\sphinxbfcode{\sphinxupquote{\_\_repr\_\_}}}
{}
{}
\pysigstopsignatures
\end{fulllineitems}

\index{\_\_eq\_\_() (nexoclom2.atomicdata.atom.Atom method)@\spxentry{\_\_eq\_\_()}\spxextra{nexoclom2.atomicdata.atom.Atom method}}

\begin{fulllineitems}
\phantomsection\label{\detokenize{autoapi/nexoclom2/atomicdata/atom/index:nexoclom2.atomicdata.atom.Atom.__eq__}}
\pysigstartsignatures
\pysiglinewithargsret
{\sphinxbfcode{\sphinxupquote{\_\_eq\_\_}}}
{\sphinxparam{\DUrole{n}{other}}}
{}
\pysigstopsignatures
\end{fulllineitems}


\end{fulllineitems}


\sphinxstepscope


\subparagraph{nexoclom2.atomicdata.charge\_exchange}
\label{\detokenize{autoapi/nexoclom2/atomicdata/charge_exchange/index:module-nexoclom2.atomicdata.charge_exchange}}\label{\detokenize{autoapi/nexoclom2/atomicdata/charge_exchange/index:nexoclom2-atomicdata-charge-exchange}}\label{\detokenize{autoapi/nexoclom2/atomicdata/charge_exchange/index::doc}}\index{module@\spxentry{module}!nexoclom2.atomicdata.charge\_exchange@\spxentry{nexoclom2.atomicdata.charge\_exchange}}\index{nexoclom2.atomicdata.charge\_exchange@\spxentry{nexoclom2.atomicdata.charge\_exchange}!module@\spxentry{module}}

\subparagraph{Classes}
\label{\detokenize{autoapi/nexoclom2/atomicdata/charge_exchange/index:classes}}

\begin{savenotes}\sphinxattablestart
\sphinxthistablewithglobalstyle
\sphinxthistablewithnovlinesstyle
\centering
\begin{tabulary}{\linewidth}[t]{\X{1}{2}\X{1}{2}}
\sphinxtoprule
\sphinxtableatstartofbodyhook
\sphinxAtStartPar
{\hyperref[\detokenize{autoapi/nexoclom2/atomicdata/charge_exchange/index:nexoclom2.atomicdata.charge_exchange.ChargeExchange}]{\sphinxcrossref{\sphinxcode{\sphinxupquote{ChargeExchange}}}}}
&
\sphinxAtStartPar

\\
\sphinxbottomrule
\end{tabulary}
\sphinxtableafterendhook\par
\sphinxattableend\end{savenotes}


\subparagraph{Functions}
\label{\detokenize{autoapi/nexoclom2/atomicdata/charge_exchange/index:functions}}

\begin{savenotes}\sphinxattablestart
\sphinxthistablewithglobalstyle
\sphinxthistablewithnovlinesstyle
\centering
\begin{tabulary}{\linewidth}[t]{\X{1}{2}\X{1}{2}}
\sphinxtoprule
\sphinxtableatstartofbodyhook
\sphinxAtStartPar
{\hyperref[\detokenize{autoapi/nexoclom2/atomicdata/charge_exchange/index:nexoclom2.atomicdata.charge_exchange.load_charge_exchange}]{\sphinxcrossref{\sphinxcode{\sphinxupquote{load\_charge\_exchange}}}}}(neutral)
&
\sphinxAtStartPar

\\
\sphinxbottomrule
\end{tabulary}
\sphinxtableafterendhook\par
\sphinxattableend\end{savenotes}


\subparagraph{Module Contents}
\label{\detokenize{autoapi/nexoclom2/atomicdata/charge_exchange/index:module-contents}}\index{ChargeExchange (class in nexoclom2.atomicdata.charge\_exchange)@\spxentry{ChargeExchange}\spxextra{class in nexoclom2.atomicdata.charge\_exchange}}

\begin{fulllineitems}
\phantomsection\label{\detokenize{autoapi/nexoclom2/atomicdata/charge_exchange/index:nexoclom2.atomicdata.charge_exchange.ChargeExchange}}
\pysigstartsignatures
\pysiglinewithargsret
{\sphinxbfcode{\sphinxupquote{class\DUrole{w}{ }}}\sphinxcode{\sphinxupquote{nexoclom2.atomicdata.charge\_exchange.}}\sphinxbfcode{\sphinxupquote{ChargeExchange}}}
{\sphinxparam{\DUrole{n}{neutral}}\sphinxparamcomma \sphinxparam{\DUrole{n}{ion}}\sphinxparamcomma \sphinxparam{\DUrole{n}{params}}}
{}
\pysigstopsignatures\index{neutral (nexoclom2.atomicdata.charge\_exchange.ChargeExchange attribute)@\spxentry{neutral}\spxextra{nexoclom2.atomicdata.charge\_exchange.ChargeExchange attribute}}

\begin{fulllineitems}
\phantomsection\label{\detokenize{autoapi/nexoclom2/atomicdata/charge_exchange/index:nexoclom2.atomicdata.charge_exchange.ChargeExchange.neutral}}
\pysigstartsignatures
\pysigline
{\sphinxbfcode{\sphinxupquote{neutral}}}
\pysigstopsignatures
\end{fulllineitems}

\index{ion (nexoclom2.atomicdata.charge\_exchange.ChargeExchange attribute)@\spxentry{ion}\spxextra{nexoclom2.atomicdata.charge\_exchange.ChargeExchange attribute}}

\begin{fulllineitems}
\phantomsection\label{\detokenize{autoapi/nexoclom2/atomicdata/charge_exchange/index:nexoclom2.atomicdata.charge_exchange.ChargeExchange.ion}}
\pysigstartsignatures
\pysigline
{\sphinxbfcode{\sphinxupquote{ion}}}
\pysigstopsignatures
\end{fulllineitems}

\index{reaction (nexoclom2.atomicdata.charge\_exchange.ChargeExchange attribute)@\spxentry{reaction}\spxextra{nexoclom2.atomicdata.charge\_exchange.ChargeExchange attribute}}

\begin{fulllineitems}
\phantomsection\label{\detokenize{autoapi/nexoclom2/atomicdata/charge_exchange/index:nexoclom2.atomicdata.charge_exchange.ChargeExchange.reaction}}
\pysigstartsignatures
\pysigline
{\sphinxbfcode{\sphinxupquote{reaction}}}
\pysigstopsignatures
\end{fulllineitems}

\index{v\_rel (nexoclom2.atomicdata.charge\_exchange.ChargeExchange attribute)@\spxentry{v\_rel}\spxextra{nexoclom2.atomicdata.charge\_exchange.ChargeExchange attribute}}

\begin{fulllineitems}
\phantomsection\label{\detokenize{autoapi/nexoclom2/atomicdata/charge_exchange/index:nexoclom2.atomicdata.charge_exchange.ChargeExchange.v_rel}}
\pysigstartsignatures
\pysigline
{\sphinxbfcode{\sphinxupquote{v\_rel}}}
\pysigstopsignatures
\end{fulllineitems}

\index{T\_i (nexoclom2.atomicdata.charge\_exchange.ChargeExchange attribute)@\spxentry{T\_i}\spxextra{nexoclom2.atomicdata.charge\_exchange.ChargeExchange attribute}}

\begin{fulllineitems}
\phantomsection\label{\detokenize{autoapi/nexoclom2/atomicdata/charge_exchange/index:nexoclom2.atomicdata.charge_exchange.ChargeExchange.T_i}}
\pysigstartsignatures
\pysigline
{\sphinxbfcode{\sphinxupquote{T\_i}}}
\pysigstopsignatures
\end{fulllineitems}

\index{kappa (nexoclom2.atomicdata.charge\_exchange.ChargeExchange attribute)@\spxentry{kappa}\spxextra{nexoclom2.atomicdata.charge\_exchange.ChargeExchange attribute}}

\begin{fulllineitems}
\phantomsection\label{\detokenize{autoapi/nexoclom2/atomicdata/charge_exchange/index:nexoclom2.atomicdata.charge_exchange.ChargeExchange.kappa}}
\pysigstartsignatures
\pysigline
{\sphinxbfcode{\sphinxupquote{kappa}}}
\pysigstopsignatures
\end{fulllineitems}

\index{file (nexoclom2.atomicdata.charge\_exchange.ChargeExchange attribute)@\spxentry{file}\spxextra{nexoclom2.atomicdata.charge\_exchange.ChargeExchange attribute}}

\begin{fulllineitems}
\phantomsection\label{\detokenize{autoapi/nexoclom2/atomicdata/charge_exchange/index:nexoclom2.atomicdata.charge_exchange.ChargeExchange.file}}
\pysigstartsignatures
\pysigline
{\sphinxbfcode{\sphinxupquote{file}}}
\pysigstopsignatures
\end{fulllineitems}

\index{ratecoef() (nexoclom2.atomicdata.charge\_exchange.ChargeExchange method)@\spxentry{ratecoef()}\spxextra{nexoclom2.atomicdata.charge\_exchange.ChargeExchange method}}

\begin{fulllineitems}
\phantomsection\label{\detokenize{autoapi/nexoclom2/atomicdata/charge_exchange/index:nexoclom2.atomicdata.charge_exchange.ChargeExchange.ratecoef}}
\pysigstartsignatures
\pysiglinewithargsret
{\sphinxbfcode{\sphinxupquote{ratecoef}}}
{\sphinxparam{\DUrole{n}{v\_rel}}\sphinxparamcomma \sphinxparam{\DUrole{n}{T\_i}}}
{}
\pysigstopsignatures
\end{fulllineitems}


\end{fulllineitems}

\index{load\_charge\_exchange() (in module nexoclom2.atomicdata.charge\_exchange)@\spxentry{load\_charge\_exchange()}\spxextra{in module nexoclom2.atomicdata.charge\_exchange}}

\begin{fulllineitems}
\phantomsection\label{\detokenize{autoapi/nexoclom2/atomicdata/charge_exchange/index:nexoclom2.atomicdata.charge_exchange.load_charge_exchange}}
\pysigstartsignatures
\pysiglinewithargsret
{\sphinxcode{\sphinxupquote{nexoclom2.atomicdata.charge\_exchange.}}\sphinxbfcode{\sphinxupquote{load\_charge\_exchange}}}
{\sphinxparam{\DUrole{n}{neutral}}}
{}
\pysigstopsignatures
\end{fulllineitems}


\sphinxstepscope


\subparagraph{nexoclom2.atomicdata.eimp\_emission\_coef}
\label{\detokenize{autoapi/nexoclom2/atomicdata/eimp_emission_coef/index:module-nexoclom2.atomicdata.eimp_emission_coef}}\label{\detokenize{autoapi/nexoclom2/atomicdata/eimp_emission_coef/index:nexoclom2-atomicdata-eimp-emission-coef}}\label{\detokenize{autoapi/nexoclom2/atomicdata/eimp_emission_coef/index::doc}}\index{module@\spxentry{module}!nexoclom2.atomicdata.eimp\_emission\_coef@\spxentry{nexoclom2.atomicdata.eimp\_emission\_coef}}\index{nexoclom2.atomicdata.eimp\_emission\_coef@\spxentry{nexoclom2.atomicdata.eimp\_emission\_coef}!module@\spxentry{module}}

\subparagraph{Classes}
\label{\detokenize{autoapi/nexoclom2/atomicdata/eimp_emission_coef/index:classes}}

\begin{savenotes}\sphinxattablestart
\sphinxthistablewithglobalstyle
\sphinxthistablewithnovlinesstyle
\centering
\begin{tabulary}{\linewidth}[t]{\X{1}{2}\X{1}{2}}
\sphinxtoprule
\sphinxtableatstartofbodyhook
\sphinxAtStartPar
{\hyperref[\detokenize{autoapi/nexoclom2/atomicdata/eimp_emission_coef/index:nexoclom2.atomicdata.eimp_emission_coef.EImpEmissionCoef}]{\sphinxcrossref{\sphinxcode{\sphinxupquote{EImpEmissionCoef}}}}}
&
\sphinxAtStartPar
Loads and computes electron impact emission rate coefficients.
\\
\sphinxbottomrule
\end{tabulary}
\sphinxtableafterendhook\par
\sphinxattableend\end{savenotes}


\subparagraph{Module Contents}
\label{\detokenize{autoapi/nexoclom2/atomicdata/eimp_emission_coef/index:module-contents}}\index{EImpEmissionCoef (class in nexoclom2.atomicdata.eimp\_emission\_coef)@\spxentry{EImpEmissionCoef}\spxextra{class in nexoclom2.atomicdata.eimp\_emission\_coef}}

\begin{fulllineitems}
\phantomsection\label{\detokenize{autoapi/nexoclom2/atomicdata/eimp_emission_coef/index:nexoclom2.atomicdata.eimp_emission_coef.EImpEmissionCoef}}
\pysigstartsignatures
\pysiglinewithargsret
{\sphinxbfcode{\sphinxupquote{class\DUrole{w}{ }}}\sphinxcode{\sphinxupquote{nexoclom2.atomicdata.eimp\_emission\_coef.}}\sphinxbfcode{\sphinxupquote{EImpEmissionCoef}}}
{\sphinxparam{\DUrole{n}{species}}}
{}
\pysigstopsignatures
\sphinxAtStartPar
Loads and computes electron impact emission rate coefficients.
\begin{quote}\begin{description}
\sphinxlineitem{Parameters}\begin{description}
\sphinxlineitem{\sphinxstylestrong{species: nexoclom2 Atom}}
\sphinxlineitem{\sphinxstylestrong{wavelength: astropy Quantity}}
\sphinxAtStartPar
Approximate wavelength to be simulated.

\end{description}

\end{description}\end{quote}
\subsubsection*{Methods}


\begin{savenotes}\sphinxattablestart
\sphinxthistablewithglobalstyle
\centering
\begin{tabulary}{\linewidth}[t]{TT}
\sphinxtoprule
\sphinxtableatstartofbodyhook
\sphinxAtStartPar
\sphinxstylestrong{ratecoef(n\_e, T\_e)}
&
\sphinxAtStartPar
Returns rate coefficients as function of input \sphinxcode{\sphinxupquote{n\_e}} and \sphinxcode{\sphinxupquote{T\_e}}.
\\
\sphinxhline
\sphinxAtStartPar
\sphinxstylestrong{wavelengths(species)}
&
\sphinxAtStartPar
\sphinxstyleemphasis{Class Method} Returns the wavelengths for which emission rate coefficients have been included.
\\
\sphinxbottomrule
\end{tabulary}
\sphinxtableafterendhook\par
\sphinxattableend\end{savenotes}
\subsubsection*{Notes}

\sphinxAtStartPar
The emission wavelengths do not need to be exact. The rate coefficients are
returned for the emission line closest to that inputted. Therefore, it is
not necessary to specify vacuum versus air wavelength.
\index{species (nexoclom2.atomicdata.eimp\_emission\_coef.EImpEmissionCoef attribute)@\spxentry{species}\spxextra{nexoclom2.atomicdata.eimp\_emission\_coef.EImpEmissionCoef attribute}}

\begin{fulllineitems}
\phantomsection\label{\detokenize{autoapi/nexoclom2/atomicdata/eimp_emission_coef/index:nexoclom2.atomicdata.eimp_emission_coef.EImpEmissionCoef.species}}
\pysigstartsignatures
\pysigline
{\sphinxbfcode{\sphinxupquote{species}}}
\pysigstopsignatures
\end{fulllineitems}

\index{\_\_repr\_\_() (nexoclom2.atomicdata.eimp\_emission\_coef.EImpEmissionCoef method)@\spxentry{\_\_repr\_\_()}\spxextra{nexoclom2.atomicdata.eimp\_emission\_coef.EImpEmissionCoef method}}

\begin{fulllineitems}
\phantomsection\label{\detokenize{autoapi/nexoclom2/atomicdata/eimp_emission_coef/index:nexoclom2.atomicdata.eimp_emission_coef.EImpEmissionCoef.__repr__}}
\pysigstartsignatures
\pysiglinewithargsret
{\sphinxbfcode{\sphinxupquote{\_\_repr\_\_}}}
{}
{}
\pysigstopsignatures
\end{fulllineitems}

\index{\_\_str\_\_() (nexoclom2.atomicdata.eimp\_emission\_coef.EImpEmissionCoef method)@\spxentry{\_\_str\_\_()}\spxextra{nexoclom2.atomicdata.eimp\_emission\_coef.EImpEmissionCoef method}}

\begin{fulllineitems}
\phantomsection\label{\detokenize{autoapi/nexoclom2/atomicdata/eimp_emission_coef/index:nexoclom2.atomicdata.eimp_emission_coef.EImpEmissionCoef.__str__}}
\pysigstartsignatures
\pysiglinewithargsret
{\sphinxbfcode{\sphinxupquote{\_\_str\_\_}}}
{}
{}
\pysigstopsignatures
\end{fulllineitems}

\index{ratecoef() (nexoclom2.atomicdata.eimp\_emission\_coef.EImpEmissionCoef method)@\spxentry{ratecoef()}\spxextra{nexoclom2.atomicdata.eimp\_emission\_coef.EImpEmissionCoef method}}

\begin{fulllineitems}
\phantomsection\label{\detokenize{autoapi/nexoclom2/atomicdata/eimp_emission_coef/index:nexoclom2.atomicdata.eimp_emission_coef.EImpEmissionCoef.ratecoef}}
\pysigstartsignatures
\pysiglinewithargsret
{\sphinxbfcode{\sphinxupquote{ratecoef}}}
{\sphinxparam{\DUrole{n}{electrons}}\sphinxparamcomma \sphinxparam{\DUrole{n}{wavelength}}}
{}
\pysigstopsignatures
\sphinxAtStartPar
Returns the electron impact excitation coefficient as a function
electron density and temperature
\begin{quote}\begin{description}
\sphinxlineitem{Parameters}\begin{description}
\sphinxlineitem{\sphinxstylestrong{n\_e: astropy Quantity array}}
\sphinxAtStartPar
Electron density. Must be in a unit convertable to cm$^{\text{\sphinxhyphen{}3}}$

\sphinxlineitem{\sphinxstylestrong{T\_e: astropy Quantity array}}
\sphinxAtStartPar
Electron temperature. Must be the same length as \sphinxcode{\sphinxupquote{n\_e}} and
in a unit convertable to eV (K is acceptable).

\end{description}

\sphinxlineitem{Returns}\begin{description}
\sphinxlineitem{Electron impact excitation rate coefficient in units of}
\sphinxlineitem{cm$^{\text{3}}$ s$^{\text{\sphinxhyphen{}1}}$}
\end{description}

\end{description}\end{quote}

\end{fulllineitems}


\end{fulllineitems}


\sphinxstepscope


\subparagraph{nexoclom2.atomicdata.eimp\_ionization\_coef}
\label{\detokenize{autoapi/nexoclom2/atomicdata/eimp_ionization_coef/index:module-nexoclom2.atomicdata.eimp_ionization_coef}}\label{\detokenize{autoapi/nexoclom2/atomicdata/eimp_ionization_coef/index:nexoclom2-atomicdata-eimp-ionization-coef}}\label{\detokenize{autoapi/nexoclom2/atomicdata/eimp_ionization_coef/index::doc}}\index{module@\spxentry{module}!nexoclom2.atomicdata.eimp\_ionization\_coef@\spxentry{nexoclom2.atomicdata.eimp\_ionization\_coef}}\index{nexoclom2.atomicdata.eimp\_ionization\_coef@\spxentry{nexoclom2.atomicdata.eimp\_ionization\_coef}!module@\spxentry{module}}

\subparagraph{Classes}
\label{\detokenize{autoapi/nexoclom2/atomicdata/eimp_ionization_coef/index:classes}}

\begin{savenotes}\sphinxattablestart
\sphinxthistablewithglobalstyle
\sphinxthistablewithnovlinesstyle
\centering
\begin{tabulary}{\linewidth}[t]{\X{1}{2}\X{1}{2}}
\sphinxtoprule
\sphinxtableatstartofbodyhook
\sphinxAtStartPar
{\hyperref[\detokenize{autoapi/nexoclom2/atomicdata/eimp_ionization_coef/index:nexoclom2.atomicdata.eimp_ionization_coef.EimpIonizationCoef}]{\sphinxcrossref{\sphinxcode{\sphinxupquote{EimpIonizationCoef}}}}}
&
\sphinxAtStartPar
Loads and computes electron impact ionization rate coefficients
\\
\sphinxbottomrule
\end{tabulary}
\sphinxtableafterendhook\par
\sphinxattableend\end{savenotes}


\subparagraph{Module Contents}
\label{\detokenize{autoapi/nexoclom2/atomicdata/eimp_ionization_coef/index:module-contents}}\index{EimpIonizationCoef (class in nexoclom2.atomicdata.eimp\_ionization\_coef)@\spxentry{EimpIonizationCoef}\spxextra{class in nexoclom2.atomicdata.eimp\_ionization\_coef}}

\begin{fulllineitems}
\phantomsection\label{\detokenize{autoapi/nexoclom2/atomicdata/eimp_ionization_coef/index:nexoclom2.atomicdata.eimp_ionization_coef.EimpIonizationCoef}}
\pysigstartsignatures
\pysiglinewithargsret
{\sphinxbfcode{\sphinxupquote{class\DUrole{w}{ }}}\sphinxcode{\sphinxupquote{nexoclom2.atomicdata.eimp\_ionization\_coef.}}\sphinxbfcode{\sphinxupquote{EimpIonizationCoef}}}
{\sphinxparam{\DUrole{n}{species}}}
{}
\pysigstopsignatures
\sphinxAtStartPar
Loads and computes electron impact ionization rate coefficients
\begin{quote}\begin{description}
\sphinxlineitem{Parameters}\begin{description}
\sphinxlineitem{\sphinxstylestrong{species}}{[}nexoclom2 Atom{]}
\end{description}

\sphinxlineitem{Attributes}\begin{description}
\sphinxlineitem{\sphinxstylestrong{T\_e: astropy Quantity array}}
\sphinxlineitem{\sphinxstylestrong{kappa: astropy Quantity array}}
\sphinxAtStartPar
Electron impact ionization rate coefficient as function of \sphinxcode{\sphinxupquote{T\_e}}

\end{description}

\end{description}\end{quote}
\subsubsection*{Methods}


\begin{savenotes}\sphinxattablestart
\sphinxthistablewithglobalstyle
\centering
\begin{tabulary}{\linewidth}[t]{TT}
\sphinxtoprule
\sphinxtableatstartofbodyhook
\sphinxAtStartPar
\sphinxstylestrong{ratecoef(T\_e)}
&
\sphinxAtStartPar
Returns rate coefficients as function of input \sphinxcode{\sphinxupquote{T\_e}}.
\\
\sphinxbottomrule
\end{tabulary}
\sphinxtableafterendhook\par
\sphinxattableend\end{savenotes}
\index{species (nexoclom2.atomicdata.eimp\_ionization\_coef.EimpIonizationCoef attribute)@\spxentry{species}\spxextra{nexoclom2.atomicdata.eimp\_ionization\_coef.EimpIonizationCoef attribute}}

\begin{fulllineitems}
\phantomsection\label{\detokenize{autoapi/nexoclom2/atomicdata/eimp_ionization_coef/index:nexoclom2.atomicdata.eimp_ionization_coef.EimpIonizationCoef.species}}
\pysigstartsignatures
\pysigline
{\sphinxbfcode{\sphinxupquote{species}}}
\pysigstopsignatures
\end{fulllineitems}

\index{ratecoef (nexoclom2.atomicdata.eimp\_ionization\_coef.EimpIonizationCoef attribute)@\spxentry{ratecoef}\spxextra{nexoclom2.atomicdata.eimp\_ionization\_coef.EimpIonizationCoef attribute}}

\begin{fulllineitems}
\phantomsection\label{\detokenize{autoapi/nexoclom2/atomicdata/eimp_ionization_coef/index:nexoclom2.atomicdata.eimp_ionization_coef.EimpIonizationCoef.ratecoef}}
\pysigstartsignatures
\pysigline
{\sphinxbfcode{\sphinxupquote{ratecoef}}}
\pysigstopsignatures
\end{fulllineitems}


\end{fulllineitems}


\sphinxstepscope


\subparagraph{nexoclom2.atomicdata.extract\_chX\_coefs}
\label{\detokenize{autoapi/nexoclom2/atomicdata/extract_chX_coefs/index:module-nexoclom2.atomicdata.extract_chX_coefs}}\label{\detokenize{autoapi/nexoclom2/atomicdata/extract_chX_coefs/index:nexoclom2-atomicdata-extract-chx-coefs}}\label{\detokenize{autoapi/nexoclom2/atomicdata/extract_chX_coefs/index::doc}}\index{module@\spxentry{module}!nexoclom2.atomicdata.extract\_chX\_coefs@\spxentry{nexoclom2.atomicdata.extract\_chX\_coefs}}\index{nexoclom2.atomicdata.extract\_chX\_coefs@\spxentry{nexoclom2.atomicdata.extract\_chX\_coefs}!module@\spxentry{module}}
\sphinxAtStartPar
Extract and save ion\sphinxhyphen{}neutral charge exchange rate coefficients

\sphinxAtStartPar
This in noy intended to be run by users and is included here for documentation.


\subparagraph{Attributes}
\label{\detokenize{autoapi/nexoclom2/atomicdata/extract_chX_coefs/index:attributes}}

\begin{savenotes}\sphinxattablestart
\sphinxthistablewithglobalstyle
\sphinxthistablewithnovlinesstyle
\centering
\begin{tabulary}{\linewidth}[t]{\X{1}{2}\X{1}{2}}
\sphinxtoprule
\sphinxtableatstartofbodyhook
\sphinxAtStartPar
{\hyperref[\detokenize{autoapi/nexoclom2/atomicdata/extract_chX_coefs/index:nexoclom2.atomicdata.extract_chX_coefs.species}]{\sphinxcrossref{\sphinxcode{\sphinxupquote{species}}}}}
&
\sphinxAtStartPar

\\
\sphinxhline
\sphinxAtStartPar
{\hyperref[\detokenize{autoapi/nexoclom2/atomicdata/extract_chX_coefs/index:nexoclom2.atomicdata.extract_chX_coefs.chx}]{\sphinxcrossref{\sphinxcode{\sphinxupquote{chx}}}}}
&
\sphinxAtStartPar

\\
\sphinxbottomrule
\end{tabulary}
\sphinxtableafterendhook\par
\sphinxattableend\end{savenotes}


\subparagraph{Module Contents}
\label{\detokenize{autoapi/nexoclom2/atomicdata/extract_chX_coefs/index:module-contents}}\index{species (in module nexoclom2.atomicdata.extract\_chX\_coefs)@\spxentry{species}\spxextra{in module nexoclom2.atomicdata.extract\_chX\_coefs}}

\begin{fulllineitems}
\phantomsection\label{\detokenize{autoapi/nexoclom2/atomicdata/extract_chX_coefs/index:nexoclom2.atomicdata.extract_chX_coefs.species}}
\pysigstartsignatures
\pysigline
{\sphinxcode{\sphinxupquote{nexoclom2.atomicdata.extract\_chX\_coefs.}}\sphinxbfcode{\sphinxupquote{species}}\sphinxbfcode{\sphinxupquote{\DUrole{w}{ }\DUrole{p}{=}\DUrole{w}{ }(\textquotesingle{}Na\textquotesingle{}, \textquotesingle{}S\textquotesingle{}, \textquotesingle{}O\textquotesingle{})}}}
\pysigstopsignatures
\end{fulllineitems}

\index{chx (in module nexoclom2.atomicdata.extract\_chX\_coefs)@\spxentry{chx}\spxextra{in module nexoclom2.atomicdata.extract\_chX\_coefs}}

\begin{fulllineitems}
\phantomsection\label{\detokenize{autoapi/nexoclom2/atomicdata/extract_chX_coefs/index:nexoclom2.atomicdata.extract_chX_coefs.chx}}
\pysigstartsignatures
\pysigline
{\sphinxcode{\sphinxupquote{nexoclom2.atomicdata.extract\_chX\_coefs.}}\sphinxbfcode{\sphinxupquote{chx}}}
\pysigstopsignatures
\end{fulllineitems}


\sphinxstepscope


\subparagraph{nexoclom2.atomicdata.extract\_eimp\_coefs}
\label{\detokenize{autoapi/nexoclom2/atomicdata/extract_eimp_coefs/index:module-nexoclom2.atomicdata.extract_eimp_coefs}}\label{\detokenize{autoapi/nexoclom2/atomicdata/extract_eimp_coefs/index:nexoclom2-atomicdata-extract-eimp-coefs}}\label{\detokenize{autoapi/nexoclom2/atomicdata/extract_eimp_coefs/index::doc}}\index{module@\spxentry{module}!nexoclom2.atomicdata.extract\_eimp\_coefs@\spxentry{nexoclom2.atomicdata.extract\_eimp\_coefs}}\index{nexoclom2.atomicdata.extract\_eimp\_coefs@\spxentry{nexoclom2.atomicdata.extract\_eimp\_coefs}!module@\spxentry{module}}
\sphinxAtStartPar
Extract and save electron impact rates from CHIANTI database

\sphinxAtStartPar
Extracts electron impact ionization rates and excitation coefficients for
electron densities and temperatures relevant to the Io torus.

\sphinxAtStartPar
This is not intended to be run by users and is included here for documentation.
ChiantiPy is not a dependency on nexoclom2 and must be installed by the user
for this to work.


\subparagraph{Attributes}
\label{\detokenize{autoapi/nexoclom2/atomicdata/extract_eimp_coefs/index:attributes}}

\begin{savenotes}\sphinxattablestart
\sphinxthistablewithglobalstyle
\sphinxthistablewithnovlinesstyle
\centering
\begin{tabulary}{\linewidth}[t]{\X{1}{2}\X{1}{2}}
\sphinxtoprule
\sphinxtableatstartofbodyhook
\sphinxAtStartPar
{\hyperref[\detokenize{autoapi/nexoclom2/atomicdata/extract_eimp_coefs/index:nexoclom2.atomicdata.extract_eimp_coefs.n_e}]{\sphinxcrossref{\sphinxcode{\sphinxupquote{n\_e}}}}}
&
\sphinxAtStartPar

\\
\sphinxhline
\sphinxAtStartPar
{\hyperref[\detokenize{autoapi/nexoclom2/atomicdata/extract_eimp_coefs/index:nexoclom2.atomicdata.extract_eimp_coefs.temperature}]{\sphinxcrossref{\sphinxcode{\sphinxupquote{temperature}}}}}
&
\sphinxAtStartPar

\\
\sphinxhline
\sphinxAtStartPar
{\hyperref[\detokenize{autoapi/nexoclom2/atomicdata/extract_eimp_coefs/index:nexoclom2.atomicdata.extract_eimp_coefs.species}]{\sphinxcrossref{\sphinxcode{\sphinxupquote{species}}}}}
&
\sphinxAtStartPar

\\
\sphinxbottomrule
\end{tabulary}
\sphinxtableafterendhook\par
\sphinxattableend\end{savenotes}


\subparagraph{Functions}
\label{\detokenize{autoapi/nexoclom2/atomicdata/extract_eimp_coefs/index:functions}}

\begin{savenotes}\sphinxattablestart
\sphinxthistablewithglobalstyle
\sphinxthistablewithnovlinesstyle
\centering
\begin{tabulary}{\linewidth}[t]{\X{1}{2}\X{1}{2}}
\sphinxtoprule
\sphinxtableatstartofbodyhook
\sphinxAtStartPar
{\hyperref[\detokenize{autoapi/nexoclom2/atomicdata/extract_eimp_coefs/index:nexoclom2.atomicdata.extract_eimp_coefs.compute_EImpCoef}]{\sphinxcrossref{\sphinxcode{\sphinxupquote{compute\_EImpCoef}}}}}(sp, lines)
&
\sphinxAtStartPar

\\
\sphinxbottomrule
\end{tabulary}
\sphinxtableafterendhook\par
\sphinxattableend\end{savenotes}


\subparagraph{Module Contents}
\label{\detokenize{autoapi/nexoclom2/atomicdata/extract_eimp_coefs/index:module-contents}}\index{n\_e (in module nexoclom2.atomicdata.extract\_eimp\_coefs)@\spxentry{n\_e}\spxextra{in module nexoclom2.atomicdata.extract\_eimp\_coefs}}

\begin{fulllineitems}
\phantomsection\label{\detokenize{autoapi/nexoclom2/atomicdata/extract_eimp_coefs/index:nexoclom2.atomicdata.extract_eimp_coefs.n_e}}
\pysigstartsignatures
\pysigline
{\sphinxcode{\sphinxupquote{nexoclom2.atomicdata.extract\_eimp\_coefs.}}\sphinxbfcode{\sphinxupquote{n\_e}}}
\pysigstopsignatures
\end{fulllineitems}

\index{temperature (in module nexoclom2.atomicdata.extract\_eimp\_coefs)@\spxentry{temperature}\spxextra{in module nexoclom2.atomicdata.extract\_eimp\_coefs}}

\begin{fulllineitems}
\phantomsection\label{\detokenize{autoapi/nexoclom2/atomicdata/extract_eimp_coefs/index:nexoclom2.atomicdata.extract_eimp_coefs.temperature}}
\pysigstartsignatures
\pysigline
{\sphinxcode{\sphinxupquote{nexoclom2.atomicdata.extract\_eimp\_coefs.}}\sphinxbfcode{\sphinxupquote{temperature}}}
\pysigstopsignatures
\end{fulllineitems}

\index{species (in module nexoclom2.atomicdata.extract\_eimp\_coefs)@\spxentry{species}\spxextra{in module nexoclom2.atomicdata.extract\_eimp\_coefs}}

\begin{fulllineitems}
\phantomsection\label{\detokenize{autoapi/nexoclom2/atomicdata/extract_eimp_coefs/index:nexoclom2.atomicdata.extract_eimp_coefs.species}}
\pysigstartsignatures
\pysigline
{\sphinxcode{\sphinxupquote{nexoclom2.atomicdata.extract\_eimp\_coefs.}}\sphinxbfcode{\sphinxupquote{species}}}
\pysigstopsignatures
\end{fulllineitems}

\index{compute\_EImpCoef() (in module nexoclom2.atomicdata.extract\_eimp\_coefs)@\spxentry{compute\_EImpCoef()}\spxextra{in module nexoclom2.atomicdata.extract\_eimp\_coefs}}

\begin{fulllineitems}
\phantomsection\label{\detokenize{autoapi/nexoclom2/atomicdata/extract_eimp_coefs/index:nexoclom2.atomicdata.extract_eimp_coefs.compute_EImpCoef}}
\pysigstartsignatures
\pysiglinewithargsret
{\sphinxcode{\sphinxupquote{nexoclom2.atomicdata.extract\_eimp\_coefs.}}\sphinxbfcode{\sphinxupquote{compute\_EImpCoef}}}
{\sphinxparam{\DUrole{n}{sp}}\sphinxparamcomma \sphinxparam{\DUrole{n}{lines}}}
{}
\pysigstopsignatures
\end{fulllineitems}


\sphinxstepscope


\subparagraph{nexoclom2.atomicdata.gvalues}
\label{\detokenize{autoapi/nexoclom2/atomicdata/gvalues/index:module-nexoclom2.atomicdata.gvalues}}\label{\detokenize{autoapi/nexoclom2/atomicdata/gvalues/index:nexoclom2-atomicdata-gvalues}}\label{\detokenize{autoapi/nexoclom2/atomicdata/gvalues/index::doc}}\index{module@\spxentry{module}!nexoclom2.atomicdata.gvalues@\spxentry{nexoclom2.atomicdata.gvalues}}\index{nexoclom2.atomicdata.gvalues@\spxentry{nexoclom2.atomicdata.gvalues}!module@\spxentry{module}}

\subparagraph{Classes}
\label{\detokenize{autoapi/nexoclom2/atomicdata/gvalues/index:classes}}

\begin{savenotes}\sphinxattablestart
\sphinxthistablewithglobalstyle
\sphinxthistablewithnovlinesstyle
\centering
\begin{tabulary}{\linewidth}[t]{\X{1}{2}\X{1}{2}}
\sphinxtoprule
\sphinxtableatstartofbodyhook
\sphinxAtStartPar
{\hyperref[\detokenize{autoapi/nexoclom2/atomicdata/gvalues/index:nexoclom2.atomicdata.gvalues.gValue}]{\sphinxcrossref{\sphinxcode{\sphinxupquote{gValue}}}}}
&
\sphinxAtStartPar
Class to compute g\sphinxhyphen{}values and radiation acceleration
\\
\sphinxbottomrule
\end{tabulary}
\sphinxtableafterendhook\par
\sphinxattableend\end{savenotes}


\subparagraph{Module Contents}
\label{\detokenize{autoapi/nexoclom2/atomicdata/gvalues/index:module-contents}}\index{gValue (class in nexoclom2.atomicdata.gvalues)@\spxentry{gValue}\spxextra{class in nexoclom2.atomicdata.gvalues}}

\begin{fulllineitems}
\phantomsection\label{\detokenize{autoapi/nexoclom2/atomicdata/gvalues/index:nexoclom2.atomicdata.gvalues.gValue}}
\pysigstartsignatures
\pysiglinewithargsret
{\sphinxbfcode{\sphinxupquote{class\DUrole{w}{ }}}\sphinxcode{\sphinxupquote{nexoclom2.atomicdata.gvalues.}}\sphinxbfcode{\sphinxupquote{gValue}}}
{\sphinxparam{\DUrole{n}{species}}}
{}
\pysigstopsignatures
\sphinxAtStartPar
Class to compute g\sphinxhyphen{}values and radiation acceleration

\sphinxAtStartPar
The g\sphinxhyphen{}value is the product of the solar flux at the dopler\sphinxhyphen{}shifted
emission wavelength and the scattering probability per atom. See
\sphinxhref{https://doi.org/10.3847/1538-4365/ac9eab}{Killen, R.M. et al., Ap. J. Supp., 2022} for details on calculating
g\sphinxhyphen{}values for important species in Mercury’s atmosphere.

\sphinxAtStartPar
These g\sphinxhyphen{}values have been calculated for this solarsystem at a reference
distance of 0.352 AU.

\sphinxAtStartPar
Methods are provided to return the \sphinxstyleemphasis{g}\sphinxhyphen{}values for each resonant wavelength
and radiation acceleration as functions of distance from the Sun and
radial velocity relative to the Sun.

\sphinxAtStartPar
Radiation acceleration \(a_r\) is computed by:
\begin{equation*}
\begin{split}a_r = \sum_i g_i(v_r) \times \frac{h}{m \lambda_i}\end{split}
\end{equation*}
\sphinxAtStartPar
where \(g_i\) is \sphinxstyleemphasis{g} at wavelength \(\lambda_i\), \(h\) is
Planck’s constant, \(m\) is the mass of the species in question, and the
sum is over all resonant wavelengths.
\begin{quote}\begin{description}
\sphinxlineitem{Parameters}\begin{description}
\sphinxlineitem{\sphinxstylestrong{species}}{[}str{]}
\sphinxAtStartPar
atomic species

\end{description}

\sphinxlineitem{Attributes}\begin{description}
\sphinxlineitem{\sphinxstylestrong{wavelengths}}{[}astropy Quantity array{]}
\sphinxAtStartPar
Wavelengths are rounded to the nearest Angstrom.

\sphinxlineitem{\sphinxstylestrong{velocity}}{[}astropy Quantity array{]}
\sphinxAtStartPar
Radial velocity relative to the Sun

\end{description}

\end{description}\end{quote}
\subsubsection*{Methods}


\begin{savenotes}\sphinxattablestart
\sphinxthistablewithglobalstyle
\centering
\begin{tabulary}{\linewidth}[t]{TT}
\sphinxtoprule
\sphinxtableatstartofbodyhook
\sphinxAtStartPar
\sphinxstylestrong{gvalue(drdt, r=1*u.au)}
&
\sphinxAtStartPar
Dict with \sphinxstyleemphasis{g} as function of \sphinxcode{\sphinxupquote{drdt}} and \sphinxcode{\sphinxupquote{r}} for each resonant wavelength
\\
\sphinxhline
\sphinxAtStartPar
\sphinxstylestrong{radaccel(drdt, r=1*u.au)}
&
\sphinxAtStartPar
Radiation acceleration as function of \sphinxcode{\sphinxupquote{drdt}} at the reference distance from the Sun
\\
\sphinxbottomrule
\end{tabulary}
\sphinxtableafterendhook\par
\sphinxattableend\end{savenotes}
\index{species (nexoclom2.atomicdata.gvalues.gValue attribute)@\spxentry{species}\spxextra{nexoclom2.atomicdata.gvalues.gValue attribute}}

\begin{fulllineitems}
\phantomsection\label{\detokenize{autoapi/nexoclom2/atomicdata/gvalues/index:nexoclom2.atomicdata.gvalues.gValue.species}}
\pysigstartsignatures
\pysigline
{\sphinxbfcode{\sphinxupquote{species}}}
\pysigstopsignatures
\end{fulllineitems}

\index{gvalue() (nexoclom2.atomicdata.gvalues.gValue method)@\spxentry{gvalue()}\spxextra{nexoclom2.atomicdata.gvalues.gValue method}}

\begin{fulllineitems}
\phantomsection\label{\detokenize{autoapi/nexoclom2/atomicdata/gvalues/index:nexoclom2.atomicdata.gvalues.gValue.gvalue}}
\pysigstartsignatures
\pysiglinewithargsret
{\sphinxbfcode{\sphinxupquote{gvalue}}}
{\sphinxparam{\DUrole{n}{drdt}}\sphinxparamcomma \sphinxparam{\DUrole{n}{r}\DUrole{o}{=}\DUrole{default_value}{1.0 * u.au}}}
{}
\pysigstopsignatures
\sphinxAtStartPar
\sphinxstyleemphasis{g} as function of radial velocity and solar distance
\begin{quote}\begin{description}
\sphinxlineitem{Parameters}\begin{description}
\sphinxlineitem{\sphinxstylestrong{drdt: astropy Quantity array}}
\sphinxAtStartPar
Radial velocity relative to the Sun

\sphinxlineitem{\sphinxstylestrong{r: astropy Quantity array}}
\sphinxAtStartPar
Distance from the Sun. Default = 1 AU

\end{description}

\sphinxlineitem{Returns}\begin{description}
\sphinxlineitem{Dictionary with \sphinxstyleemphasis{g}\sphinxhyphen{}values at each point for each wavelength as function}
\sphinxlineitem{of \sphinxcode{\sphinxupquote{drdt}} and \sphinxcode{\sphinxupquote{r}}}
\end{description}

\end{description}\end{quote}

\end{fulllineitems}

\index{radaccel() (nexoclom2.atomicdata.gvalues.gValue method)@\spxentry{radaccel()}\spxextra{nexoclom2.atomicdata.gvalues.gValue method}}

\begin{fulllineitems}
\phantomsection\label{\detokenize{autoapi/nexoclom2/atomicdata/gvalues/index:nexoclom2.atomicdata.gvalues.gValue.radaccel}}
\pysigstartsignatures
\pysiglinewithargsret
{\sphinxbfcode{\sphinxupquote{radaccel}}}
{\sphinxparam{\DUrole{n}{drdt}}\sphinxparamcomma \sphinxparam{\DUrole{n}{r}\DUrole{o}{=}\DUrole{default_value}{1.0 * u.au}}}
{}
\pysigstopsignatures
\sphinxAtStartPar
radial acceleration as function of radial velocity and solar distance
\begin{quote}\begin{description}
\sphinxlineitem{Parameters}\begin{description}
\sphinxlineitem{\sphinxstylestrong{drdt: astropy Quantity array}}
\sphinxAtStartPar
Radial velocity relative to the Sun

\sphinxlineitem{\sphinxstylestrong{r: astropy Quantity array}}
\sphinxAtStartPar
Distance from the Sun. Default = 1 AU

\end{description}

\sphinxlineitem{Returns}\begin{description}
\sphinxlineitem{radiation acceleration at each point as function of \sphinxcode{\sphinxupquote{drdt}} and \sphinxcode{\sphinxupquote{r}}}
\end{description}

\end{description}\end{quote}

\end{fulllineitems}

\index{\_\_eq\_\_() (nexoclom2.atomicdata.gvalues.gValue method)@\spxentry{\_\_eq\_\_()}\spxextra{nexoclom2.atomicdata.gvalues.gValue method}}

\begin{fulllineitems}
\phantomsection\label{\detokenize{autoapi/nexoclom2/atomicdata/gvalues/index:nexoclom2.atomicdata.gvalues.gValue.__eq__}}
\pysigstartsignatures
\pysiglinewithargsret
{\sphinxbfcode{\sphinxupquote{\_\_eq\_\_}}}
{\sphinxparam{\DUrole{n}{other}}}
{}
\pysigstopsignatures
\end{fulllineitems}

\index{\_\_repr\_\_() (nexoclom2.atomicdata.gvalues.gValue method)@\spxentry{\_\_repr\_\_()}\spxextra{nexoclom2.atomicdata.gvalues.gValue method}}

\begin{fulllineitems}
\phantomsection\label{\detokenize{autoapi/nexoclom2/atomicdata/gvalues/index:nexoclom2.atomicdata.gvalues.gValue.__repr__}}
\pysigstartsignatures
\pysiglinewithargsret
{\sphinxbfcode{\sphinxupquote{\_\_repr\_\_}}}
{}
{}
\pysigstopsignatures
\end{fulllineitems}

\index{\_\_str\_\_() (nexoclom2.atomicdata.gvalues.gValue method)@\spxentry{\_\_str\_\_()}\spxextra{nexoclom2.atomicdata.gvalues.gValue method}}

\begin{fulllineitems}
\phantomsection\label{\detokenize{autoapi/nexoclom2/atomicdata/gvalues/index:nexoclom2.atomicdata.gvalues.gValue.__str__}}
\pysigstartsignatures
\pysiglinewithargsret
{\sphinxbfcode{\sphinxupquote{\_\_str\_\_}}}
{}
{}
\pysigstopsignatures
\end{fulllineitems}


\end{fulllineitems}


\sphinxstepscope


\subparagraph{nexoclom2.atomicdata.lossrate}
\label{\detokenize{autoapi/nexoclom2/atomicdata/lossrate/index:module-nexoclom2.atomicdata.lossrate}}\label{\detokenize{autoapi/nexoclom2/atomicdata/lossrate/index:nexoclom2-atomicdata-lossrate}}\label{\detokenize{autoapi/nexoclom2/atomicdata/lossrate/index::doc}}\index{module@\spxentry{module}!nexoclom2.atomicdata.lossrate@\spxentry{nexoclom2.atomicdata.lossrate}}\index{nexoclom2.atomicdata.lossrate@\spxentry{nexoclom2.atomicdata.lossrate}!module@\spxentry{module}}

\subparagraph{Functions}
\label{\detokenize{autoapi/nexoclom2/atomicdata/lossrate/index:functions}}

\begin{savenotes}\sphinxattablestart
\sphinxthistablewithglobalstyle
\sphinxthistablewithnovlinesstyle
\centering
\begin{tabulary}{\linewidth}[t]{\X{1}{2}\X{1}{2}}
\sphinxtoprule
\sphinxtableatstartofbodyhook
\sphinxAtStartPar
{\hyperref[\detokenize{autoapi/nexoclom2/atomicdata/lossrate/index:nexoclom2.atomicdata.lossrate.lossrate}]{\sphinxcrossref{\sphinxcode{\sphinxupquote{lossrate}}}}}(packets, output)
&
\sphinxAtStartPar
Calculate the loss rates due to photons, electron impacts, and charge\sphinxhyphen{}exchange
\\
\sphinxbottomrule
\end{tabulary}
\sphinxtableafterendhook\par
\sphinxattableend\end{savenotes}


\subparagraph{Module Contents}
\label{\detokenize{autoapi/nexoclom2/atomicdata/lossrate/index:module-contents}}\index{lossrate() (in module nexoclom2.atomicdata.lossrate)@\spxentry{lossrate()}\spxextra{in module nexoclom2.atomicdata.lossrate}}

\begin{fulllineitems}
\phantomsection\label{\detokenize{autoapi/nexoclom2/atomicdata/lossrate/index:nexoclom2.atomicdata.lossrate.lossrate}}
\pysigstartsignatures
\pysiglinewithargsret
{\sphinxcode{\sphinxupquote{nexoclom2.atomicdata.lossrate.}}\sphinxbfcode{\sphinxupquote{lossrate}}}
{\sphinxparam{\DUrole{n}{packets}}\sphinxparamcomma \sphinxparam{\DUrole{n}{output}}}
{}
\pysigstopsignatures
\sphinxAtStartPar
Calculate the loss rates due to photons, electron impacts, and charge\sphinxhyphen{}exchange
with optional constant loss rate.
\begin{quote}\begin{description}
\sphinxlineitem{Parameters}\begin{description}
\sphinxlineitem{\sphinxstylestrong{packets: nexoclom2 Packets object}}
\sphinxlineitem{\sphinxstylestrong{output: nexoclom2 Output object}}
\end{description}

\sphinxlineitem{Returns}\begin{description}
\sphinxlineitem{Ionization rate}
\end{description}

\end{description}\end{quote}


\begin{sphinxseealso}{See also:}
\begin{description}
\sphinxlineitem{{\hyperref[\detokenize{autoapi/nexoclom2/particle_tracking/packets/index:nexoclom2.particle_tracking.packets.Packets}]{\sphinxcrossref{\sphinxcode{\sphinxupquote{nexoclom2.particle\_tracking.packets.Packets}}}}}}
\sphinxlineitem{{\hyperref[\detokenize{autoapi/nexoclom2/particle_tracking/Output/index:nexoclom2.particle_tracking.Output.Output}]{\sphinxcrossref{\sphinxcode{\sphinxupquote{nexoclom2.particle\_tracking.Output.Output}}}}}}
\end{description}


\end{sphinxseealso}

\subsubsection*{Notes}

\sphinxAtStartPar
Charge exchange is not included yet.

\end{fulllineitems}


\sphinxstepscope


\subparagraph{nexoclom2.atomicdata.reformat\_gvalues}
\label{\detokenize{autoapi/nexoclom2/atomicdata/reformat_gvalues/index:module-nexoclom2.atomicdata.reformat_gvalues}}\label{\detokenize{autoapi/nexoclom2/atomicdata/reformat_gvalues/index:nexoclom2-atomicdata-reformat-gvalues}}\label{\detokenize{autoapi/nexoclom2/atomicdata/reformat_gvalues/index::doc}}\index{module@\spxentry{module}!nexoclom2.atomicdata.reformat\_gvalues@\spxentry{nexoclom2.atomicdata.reformat\_gvalues}}\index{nexoclom2.atomicdata.reformat\_gvalues@\spxentry{nexoclom2.atomicdata.reformat\_gvalues}!module@\spxentry{module}}
\sphinxAtStartPar
Used once to convert gvalue text files to csv


\subparagraph{Attributes}
\label{\detokenize{autoapi/nexoclom2/atomicdata/reformat_gvalues/index:attributes}}

\begin{savenotes}\sphinxattablestart
\sphinxthistablewithglobalstyle
\sphinxthistablewithnovlinesstyle
\centering
\begin{tabulary}{\linewidth}[t]{\X{1}{2}\X{1}{2}}
\sphinxtoprule
\sphinxtableatstartofbodyhook
\sphinxAtStartPar
{\hyperref[\detokenize{autoapi/nexoclom2/atomicdata/reformat_gvalues/index:nexoclom2.atomicdata.reformat_gvalues.species}]{\sphinxcrossref{\sphinxcode{\sphinxupquote{species}}}}}
&
\sphinxAtStartPar

\\
\sphinxbottomrule
\end{tabulary}
\sphinxtableafterendhook\par
\sphinxattableend\end{savenotes}


\subparagraph{Functions}
\label{\detokenize{autoapi/nexoclom2/atomicdata/reformat_gvalues/index:functions}}

\begin{savenotes}\sphinxattablestart
\sphinxthistablewithglobalstyle
\sphinxthistablewithnovlinesstyle
\centering
\begin{tabulary}{\linewidth}[t]{\X{1}{2}\X{1}{2}}
\sphinxtoprule
\sphinxtableatstartofbodyhook
\sphinxAtStartPar
{\hyperref[\detokenize{autoapi/nexoclom2/atomicdata/reformat_gvalues/index:nexoclom2.atomicdata.reformat_gvalues.reformat_gvalues}]{\sphinxcrossref{\sphinxcode{\sphinxupquote{reformat\_gvalues}}}}}(species)
&
\sphinxAtStartPar

\\
\sphinxbottomrule
\end{tabulary}
\sphinxtableafterendhook\par
\sphinxattableend\end{savenotes}


\subparagraph{Module Contents}
\label{\detokenize{autoapi/nexoclom2/atomicdata/reformat_gvalues/index:module-contents}}\index{reformat\_gvalues() (in module nexoclom2.atomicdata.reformat\_gvalues)@\spxentry{reformat\_gvalues()}\spxextra{in module nexoclom2.atomicdata.reformat\_gvalues}}

\begin{fulllineitems}
\phantomsection\label{\detokenize{autoapi/nexoclom2/atomicdata/reformat_gvalues/index:nexoclom2.atomicdata.reformat_gvalues.reformat_gvalues}}
\pysigstartsignatures
\pysiglinewithargsret
{\sphinxcode{\sphinxupquote{nexoclom2.atomicdata.reformat\_gvalues.}}\sphinxbfcode{\sphinxupquote{reformat\_gvalues}}}
{\sphinxparam{\DUrole{n}{species}}}
{}
\pysigstopsignatures
\end{fulllineitems}

\index{species (in module nexoclom2.atomicdata.reformat\_gvalues)@\spxentry{species}\spxextra{in module nexoclom2.atomicdata.reformat\_gvalues}}

\begin{fulllineitems}
\phantomsection\label{\detokenize{autoapi/nexoclom2/atomicdata/reformat_gvalues/index:nexoclom2.atomicdata.reformat_gvalues.species}}
\pysigstartsignatures
\pysigline
{\sphinxcode{\sphinxupquote{nexoclom2.atomicdata.reformat\_gvalues.}}\sphinxbfcode{\sphinxupquote{species}}\sphinxbfcode{\sphinxupquote{\DUrole{w}{ }\DUrole{p}{=}\DUrole{w}{ }{[}\textquotesingle{}Al\textquotesingle{}, \textquotesingle{}Ca\textquotesingle{}, \textquotesingle{}CaII\textquotesingle{}, \textquotesingle{}H\textquotesingle{}, \textquotesingle{}He\textquotesingle{}, \textquotesingle{}K\textquotesingle{}, \textquotesingle{}Mg\textquotesingle{}, \textquotesingle{}MgII\textquotesingle{}, \textquotesingle{}Mn\textquotesingle{}, \textquotesingle{}Na\textquotesingle{}, \textquotesingle{}O\textquotesingle{}, \textquotesingle{}S\textquotesingle{}{]}}}}
\pysigstopsignatures
\end{fulllineitems}



\subparagraph{Classes}
\label{\detokenize{autoapi/nexoclom2/atomicdata/index:classes}}

\begin{savenotes}\sphinxattablestart
\sphinxthistablewithglobalstyle
\sphinxthistablewithnovlinesstyle
\centering
\begin{tabulary}{\linewidth}[t]{\X{1}{2}\X{1}{2}}
\sphinxtoprule
\sphinxtableatstartofbodyhook
\sphinxAtStartPar
{\hyperref[\detokenize{autoapi/nexoclom2/atomicdata/index:nexoclom2.atomicdata.gValue}]{\sphinxcrossref{\sphinxcode{\sphinxupquote{gValue}}}}}
&
\sphinxAtStartPar
Class to compute g\sphinxhyphen{}values and radiation acceleration
\\
\sphinxhline
\sphinxAtStartPar
{\hyperref[\detokenize{autoapi/nexoclom2/atomicdata/index:nexoclom2.atomicdata.Atom}]{\sphinxcrossref{\sphinxcode{\sphinxupquote{Atom}}}}}
&
\sphinxAtStartPar
Class containing all useful atomic data for a neutral or ionic species.
\\
\sphinxbottomrule
\end{tabulary}
\sphinxtableafterendhook\par
\sphinxattableend\end{savenotes}


\subparagraph{Functions}
\label{\detokenize{autoapi/nexoclom2/atomicdata/index:functions}}

\begin{savenotes}\sphinxattablestart
\sphinxthistablewithglobalstyle
\sphinxthistablewithnovlinesstyle
\centering
\begin{tabulary}{\linewidth}[t]{\X{1}{2}\X{1}{2}}
\sphinxtoprule
\sphinxtableatstartofbodyhook
\sphinxAtStartPar
{\hyperref[\detokenize{autoapi/nexoclom2/atomicdata/lossrate/index:module-nexoclom2.atomicdata.lossrate}]{\sphinxcrossref{\sphinxcode{\sphinxupquote{lossrate}}}}}(packets, output)
&
\sphinxAtStartPar
Calculate the loss rates due to photons, electron impacts, and charge\sphinxhyphen{}exchange
\\
\sphinxbottomrule
\end{tabulary}
\sphinxtableafterendhook\par
\sphinxattableend\end{savenotes}


\subparagraph{Package Contents}
\label{\detokenize{autoapi/nexoclom2/atomicdata/index:package-contents}}\index{gValue (class in nexoclom2.atomicdata)@\spxentry{gValue}\spxextra{class in nexoclom2.atomicdata}}

\begin{fulllineitems}
\phantomsection\label{\detokenize{autoapi/nexoclom2/atomicdata/index:nexoclom2.atomicdata.gValue}}
\pysigstartsignatures
\pysiglinewithargsret
{\sphinxbfcode{\sphinxupquote{class\DUrole{w}{ }}}\sphinxcode{\sphinxupquote{nexoclom2.atomicdata.}}\sphinxbfcode{\sphinxupquote{gValue}}}
{\sphinxparam{\DUrole{n}{species}}}
{}
\pysigstopsignatures
\sphinxAtStartPar
Class to compute g\sphinxhyphen{}values and radiation acceleration

\sphinxAtStartPar
The g\sphinxhyphen{}value is the product of the solar flux at the dopler\sphinxhyphen{}shifted
emission wavelength and the scattering probability per atom. See
\sphinxhref{https://doi.org/10.3847/1538-4365/ac9eab}{Killen, R.M. et al., Ap. J. Supp., 2022} for details on calculating
g\sphinxhyphen{}values for important species in Mercury’s atmosphere.

\sphinxAtStartPar
These g\sphinxhyphen{}values have been calculated for this solarsystem at a reference
distance of 0.352 AU.

\sphinxAtStartPar
Methods are provided to return the \sphinxstyleemphasis{g}\sphinxhyphen{}values for each resonant wavelength
and radiation acceleration as functions of distance from the Sun and
radial velocity relative to the Sun.

\sphinxAtStartPar
Radiation acceleration \(a_r\) is computed by:
\begin{equation*}
\begin{split}a_r = \sum_i g_i(v_r) \times \frac{h}{m \lambda_i}\end{split}
\end{equation*}
\sphinxAtStartPar
where \(g_i\) is \sphinxstyleemphasis{g} at wavelength \(\lambda_i\), \(h\) is
Planck’s constant, \(m\) is the mass of the species in question, and the
sum is over all resonant wavelengths.
\begin{quote}\begin{description}
\sphinxlineitem{Parameters}\begin{description}
\sphinxlineitem{\sphinxstylestrong{species}}{[}str{]}
\sphinxAtStartPar
atomic species

\end{description}

\sphinxlineitem{Attributes}\begin{description}
\sphinxlineitem{\sphinxstylestrong{wavelengths}}{[}astropy Quantity array{]}
\sphinxAtStartPar
Wavelengths are rounded to the nearest Angstrom.

\sphinxlineitem{\sphinxstylestrong{velocity}}{[}astropy Quantity array{]}
\sphinxAtStartPar
Radial velocity relative to the Sun

\end{description}

\end{description}\end{quote}
\subsubsection*{Methods}


\begin{savenotes}\sphinxattablestart
\sphinxthistablewithglobalstyle
\centering
\begin{tabulary}{\linewidth}[t]{TT}
\sphinxtoprule
\sphinxtableatstartofbodyhook
\sphinxAtStartPar
\sphinxstylestrong{gvalue(drdt, r=1*u.au)}
&
\sphinxAtStartPar
Dict with \sphinxstyleemphasis{g} as function of \sphinxcode{\sphinxupquote{drdt}} and \sphinxcode{\sphinxupquote{r}} for each resonant wavelength
\\
\sphinxhline
\sphinxAtStartPar
\sphinxstylestrong{radaccel(drdt, r=1*u.au)}
&
\sphinxAtStartPar
Radiation acceleration as function of \sphinxcode{\sphinxupquote{drdt}} at the reference distance from the Sun
\\
\sphinxbottomrule
\end{tabulary}
\sphinxtableafterendhook\par
\sphinxattableend\end{savenotes}
\index{species (nexoclom2.atomicdata.gValue attribute)@\spxentry{species}\spxextra{nexoclom2.atomicdata.gValue attribute}}

\begin{fulllineitems}
\phantomsection\label{\detokenize{autoapi/nexoclom2/atomicdata/index:nexoclom2.atomicdata.gValue.species}}
\pysigstartsignatures
\pysigline
{\sphinxbfcode{\sphinxupquote{species}}}
\pysigstopsignatures
\end{fulllineitems}

\index{gvalue() (nexoclom2.atomicdata.gValue method)@\spxentry{gvalue()}\spxextra{nexoclom2.atomicdata.gValue method}}

\begin{fulllineitems}
\phantomsection\label{\detokenize{autoapi/nexoclom2/atomicdata/index:nexoclom2.atomicdata.gValue.gvalue}}
\pysigstartsignatures
\pysiglinewithargsret
{\sphinxbfcode{\sphinxupquote{gvalue}}}
{\sphinxparam{\DUrole{n}{drdt}}\sphinxparamcomma \sphinxparam{\DUrole{n}{r}\DUrole{o}{=}\DUrole{default_value}{1.0 * u.au}}}
{}
\pysigstopsignatures
\sphinxAtStartPar
\sphinxstyleemphasis{g} as function of radial velocity and solar distance
\begin{quote}\begin{description}
\sphinxlineitem{Parameters}\begin{description}
\sphinxlineitem{\sphinxstylestrong{drdt: astropy Quantity array}}
\sphinxAtStartPar
Radial velocity relative to the Sun

\sphinxlineitem{\sphinxstylestrong{r: astropy Quantity array}}
\sphinxAtStartPar
Distance from the Sun. Default = 1 AU

\end{description}

\sphinxlineitem{Returns}\begin{description}
\sphinxlineitem{Dictionary with \sphinxstyleemphasis{g}\sphinxhyphen{}values at each point for each wavelength as function}
\sphinxlineitem{of \sphinxcode{\sphinxupquote{drdt}} and \sphinxcode{\sphinxupquote{r}}}
\end{description}

\end{description}\end{quote}

\end{fulllineitems}

\index{radaccel() (nexoclom2.atomicdata.gValue method)@\spxentry{radaccel()}\spxextra{nexoclom2.atomicdata.gValue method}}

\begin{fulllineitems}
\phantomsection\label{\detokenize{autoapi/nexoclom2/atomicdata/index:nexoclom2.atomicdata.gValue.radaccel}}
\pysigstartsignatures
\pysiglinewithargsret
{\sphinxbfcode{\sphinxupquote{radaccel}}}
{\sphinxparam{\DUrole{n}{drdt}}\sphinxparamcomma \sphinxparam{\DUrole{n}{r}\DUrole{o}{=}\DUrole{default_value}{1.0 * u.au}}}
{}
\pysigstopsignatures
\sphinxAtStartPar
radial acceleration as function of radial velocity and solar distance
\begin{quote}\begin{description}
\sphinxlineitem{Parameters}\begin{description}
\sphinxlineitem{\sphinxstylestrong{drdt: astropy Quantity array}}
\sphinxAtStartPar
Radial velocity relative to the Sun

\sphinxlineitem{\sphinxstylestrong{r: astropy Quantity array}}
\sphinxAtStartPar
Distance from the Sun. Default = 1 AU

\end{description}

\sphinxlineitem{Returns}\begin{description}
\sphinxlineitem{radiation acceleration at each point as function of \sphinxcode{\sphinxupquote{drdt}} and \sphinxcode{\sphinxupquote{r}}}
\end{description}

\end{description}\end{quote}

\end{fulllineitems}

\index{\_\_eq\_\_() (nexoclom2.atomicdata.gValue method)@\spxentry{\_\_eq\_\_()}\spxextra{nexoclom2.atomicdata.gValue method}}

\begin{fulllineitems}
\phantomsection\label{\detokenize{autoapi/nexoclom2/atomicdata/index:nexoclom2.atomicdata.gValue.__eq__}}
\pysigstartsignatures
\pysiglinewithargsret
{\sphinxbfcode{\sphinxupquote{\_\_eq\_\_}}}
{\sphinxparam{\DUrole{n}{other}}}
{}
\pysigstopsignatures
\end{fulllineitems}

\index{\_\_repr\_\_() (nexoclom2.atomicdata.gValue method)@\spxentry{\_\_repr\_\_()}\spxextra{nexoclom2.atomicdata.gValue method}}

\begin{fulllineitems}
\phantomsection\label{\detokenize{autoapi/nexoclom2/atomicdata/index:nexoclom2.atomicdata.gValue.__repr__}}
\pysigstartsignatures
\pysiglinewithargsret
{\sphinxbfcode{\sphinxupquote{\_\_repr\_\_}}}
{}
{}
\pysigstopsignatures
\end{fulllineitems}

\index{\_\_str\_\_() (nexoclom2.atomicdata.gValue method)@\spxentry{\_\_str\_\_()}\spxextra{nexoclom2.atomicdata.gValue method}}

\begin{fulllineitems}
\phantomsection\label{\detokenize{autoapi/nexoclom2/atomicdata/index:nexoclom2.atomicdata.gValue.__str__}}
\pysigstartsignatures
\pysiglinewithargsret
{\sphinxbfcode{\sphinxupquote{\_\_str\_\_}}}
{}
{}
\pysigstopsignatures
\end{fulllineitems}


\end{fulllineitems}

\index{Atom (class in nexoclom2.atomicdata)@\spxentry{Atom}\spxextra{class in nexoclom2.atomicdata}}

\begin{fulllineitems}
\phantomsection\label{\detokenize{autoapi/nexoclom2/atomicdata/index:nexoclom2.atomicdata.Atom}}
\pysigstartsignatures
\pysiglinewithargsret
{\sphinxbfcode{\sphinxupquote{class\DUrole{w}{ }}}\sphinxcode{\sphinxupquote{nexoclom2.atomicdata.}}\sphinxbfcode{\sphinxupquote{Atom}}}
{\sphinxparam{\DUrole{n}{species}\DUrole{p}{:}\DUrole{w}{ }\DUrole{n}{str}}}
{}
\pysigstopsignatures
\sphinxAtStartPar
Class containing all useful atomic data for a neutral or ionic species.
\begin{quote}\begin{description}
\sphinxlineitem{Parameters}\begin{description}
\sphinxlineitem{\sphinxstylestrong{species: str}}
\sphinxAtStartPar
Atomic species. Charge is indicated by the number of ‘+’ or ‘\sphinxhyphen{}’
appended to the chemical symbol

\end{description}

\sphinxlineitem{Attributes}\begin{description}
\sphinxlineitem{\sphinxstylestrong{charge: int}}
\sphinxAtStartPar
Electric charge

\sphinxlineitem{\sphinxstylestrong{symbol: str}}
\sphinxAtStartPar
Chemical symbol

\sphinxlineitem{\sphinxstylestrong{name: str}}
\sphinxlineitem{\sphinxstylestrong{number: int}}
\sphinxlineitem{\sphinxstylestrong{mass: astropy Quantity}}
\sphinxlineitem{\sphinxstylestrong{gvalues: nexoclom2 gValue object}}
\sphinxlineitem{\sphinxstylestrong{photo\_refpt: astropy Quantity}}
\sphinxAtStartPar
Reference point for the photoionziation rate (generally 1 AU)

\sphinxlineitem{\sphinxstylestrong{photo\_rate: astropy Quantity}}
\sphinxAtStartPar
Photoionization rate at the reference point

\sphinxlineitem{\sphinxstylestrong{photo\_reactions: list}}
\sphinxAtStartPar
List of tuples containing each photoreaction included and its rate at
the reference point

\sphinxlineitem{\sphinxstylestrong{wavelengths: tuple}}
\sphinxAtStartPar
List of wavelengths for which there are photon scattering rate
coefficients (\sphinxstyleemphasis{g}\sphinxhyphen{}values) or electron impact excitation rate
coefficients.

\sphinxlineitem{\sphinxstylestrong{\_atom: Element}}
\sphinxAtStartPar
Object from
\sphinxhref{https://periodictable.readthedocs.io/en/latest/periodictable}{periodictable}
saved in case there is additional infomation the user might need.

\sphinxlineitem{\sphinxstylestrong{eimp\_ionization: nexoclom2 EimpIonizationCoef object}}
\sphinxlineitem{\sphinxstylestrong{eimp\_emission: nexoclom2 EimpEmissionCoef object}}
\end{description}

\end{description}\end{quote}
\index{\_atom (nexoclom2.atomicdata.Atom attribute)@\spxentry{\_atom}\spxextra{nexoclom2.atomicdata.Atom attribute}}

\begin{fulllineitems}
\phantomsection\label{\detokenize{autoapi/nexoclom2/atomicdata/index:nexoclom2.atomicdata.Atom._atom}}
\pysigstartsignatures
\pysigline
{\sphinxbfcode{\sphinxupquote{\_atom}}}
\pysigstopsignatures
\end{fulllineitems}

\index{charge (nexoclom2.atomicdata.Atom attribute)@\spxentry{charge}\spxextra{nexoclom2.atomicdata.Atom attribute}}

\begin{fulllineitems}
\phantomsection\label{\detokenize{autoapi/nexoclom2/atomicdata/index:nexoclom2.atomicdata.Atom.charge}}
\pysigstartsignatures
\pysigline
{\sphinxbfcode{\sphinxupquote{charge}}}
\pysigstopsignatures
\end{fulllineitems}

\index{symbol (nexoclom2.atomicdata.Atom attribute)@\spxentry{symbol}\spxextra{nexoclom2.atomicdata.Atom attribute}}

\begin{fulllineitems}
\phantomsection\label{\detokenize{autoapi/nexoclom2/atomicdata/index:nexoclom2.atomicdata.Atom.symbol}}
\pysigstartsignatures
\pysigline
{\sphinxbfcode{\sphinxupquote{symbol}}}
\pysigstopsignatures
\end{fulllineitems}

\index{name (nexoclom2.atomicdata.Atom attribute)@\spxentry{name}\spxextra{nexoclom2.atomicdata.Atom attribute}}

\begin{fulllineitems}
\phantomsection\label{\detokenize{autoapi/nexoclom2/atomicdata/index:nexoclom2.atomicdata.Atom.name}}
\pysigstartsignatures
\pysigline
{\sphinxbfcode{\sphinxupquote{name}}}
\pysigstopsignatures
\end{fulllineitems}

\index{number (nexoclom2.atomicdata.Atom attribute)@\spxentry{number}\spxextra{nexoclom2.atomicdata.Atom attribute}}

\begin{fulllineitems}
\phantomsection\label{\detokenize{autoapi/nexoclom2/atomicdata/index:nexoclom2.atomicdata.Atom.number}}
\pysigstartsignatures
\pysigline
{\sphinxbfcode{\sphinxupquote{number}}}
\pysigstopsignatures
\end{fulllineitems}

\index{mass (nexoclom2.atomicdata.Atom attribute)@\spxentry{mass}\spxextra{nexoclom2.atomicdata.Atom attribute}}

\begin{fulllineitems}
\phantomsection\label{\detokenize{autoapi/nexoclom2/atomicdata/index:nexoclom2.atomicdata.Atom.mass}}
\pysigstartsignatures
\pysigline
{\sphinxbfcode{\sphinxupquote{mass}}}
\pysigstopsignatures
\end{fulllineitems}

\index{gvalues (nexoclom2.atomicdata.Atom attribute)@\spxentry{gvalues}\spxextra{nexoclom2.atomicdata.Atom attribute}}

\begin{fulllineitems}
\phantomsection\label{\detokenize{autoapi/nexoclom2/atomicdata/index:nexoclom2.atomicdata.Atom.gvalues}}
\pysigstartsignatures
\pysigline
{\sphinxbfcode{\sphinxupquote{gvalues}}}
\pysigstopsignatures
\end{fulllineitems}

\index{photo\_refpt (nexoclom2.atomicdata.Atom attribute)@\spxentry{photo\_refpt}\spxextra{nexoclom2.atomicdata.Atom attribute}}

\begin{fulllineitems}
\phantomsection\label{\detokenize{autoapi/nexoclom2/atomicdata/index:nexoclom2.atomicdata.Atom.photo_refpt}}
\pysigstartsignatures
\pysigline
{\sphinxbfcode{\sphinxupquote{photo\_refpt}}}
\pysigstopsignatures
\end{fulllineitems}

\index{eimp\_ionization (nexoclom2.atomicdata.Atom attribute)@\spxentry{eimp\_ionization}\spxextra{nexoclom2.atomicdata.Atom attribute}}

\begin{fulllineitems}
\phantomsection\label{\detokenize{autoapi/nexoclom2/atomicdata/index:nexoclom2.atomicdata.Atom.eimp_ionization}}
\pysigstartsignatures
\pysigline
{\sphinxbfcode{\sphinxupquote{eimp\_ionization}}}
\pysigstopsignatures
\end{fulllineitems}

\index{eimp\_emission (nexoclom2.atomicdata.Atom attribute)@\spxentry{eimp\_emission}\spxextra{nexoclom2.atomicdata.Atom attribute}}

\begin{fulllineitems}
\phantomsection\label{\detokenize{autoapi/nexoclom2/atomicdata/index:nexoclom2.atomicdata.Atom.eimp_emission}}
\pysigstartsignatures
\pysigline
{\sphinxbfcode{\sphinxupquote{eimp\_emission}}}
\pysigstopsignatures
\end{fulllineitems}

\index{wavelengths (nexoclom2.atomicdata.Atom attribute)@\spxentry{wavelengths}\spxextra{nexoclom2.atomicdata.Atom attribute}}

\begin{fulllineitems}
\phantomsection\label{\detokenize{autoapi/nexoclom2/atomicdata/index:nexoclom2.atomicdata.Atom.wavelengths}}
\pysigstartsignatures
\pysigline
{\sphinxbfcode{\sphinxupquote{wavelengths}}}
\pysigstopsignatures
\end{fulllineitems}

\index{charge\_exchange (nexoclom2.atomicdata.Atom attribute)@\spxentry{charge\_exchange}\spxextra{nexoclom2.atomicdata.Atom attribute}}

\begin{fulllineitems}
\phantomsection\label{\detokenize{autoapi/nexoclom2/atomicdata/index:nexoclom2.atomicdata.Atom.charge_exchange}}
\pysigstartsignatures
\pysigline
{\sphinxbfcode{\sphinxupquote{charge\_exchange}}}
\pysigstopsignatures
\end{fulllineitems}

\index{\_\_str\_\_() (nexoclom2.atomicdata.Atom method)@\spxentry{\_\_str\_\_()}\spxextra{nexoclom2.atomicdata.Atom method}}

\begin{fulllineitems}
\phantomsection\label{\detokenize{autoapi/nexoclom2/atomicdata/index:nexoclom2.atomicdata.Atom.__str__}}
\pysigstartsignatures
\pysiglinewithargsret
{\sphinxbfcode{\sphinxupquote{\_\_str\_\_}}}
{}
{}
\pysigstopsignatures
\end{fulllineitems}

\index{\_\_repr\_\_() (nexoclom2.atomicdata.Atom method)@\spxentry{\_\_repr\_\_()}\spxextra{nexoclom2.atomicdata.Atom method}}

\begin{fulllineitems}
\phantomsection\label{\detokenize{autoapi/nexoclom2/atomicdata/index:nexoclom2.atomicdata.Atom.__repr__}}
\pysigstartsignatures
\pysiglinewithargsret
{\sphinxbfcode{\sphinxupquote{\_\_repr\_\_}}}
{}
{}
\pysigstopsignatures
\end{fulllineitems}

\index{\_\_eq\_\_() (nexoclom2.atomicdata.Atom method)@\spxentry{\_\_eq\_\_()}\spxextra{nexoclom2.atomicdata.Atom method}}

\begin{fulllineitems}
\phantomsection\label{\detokenize{autoapi/nexoclom2/atomicdata/index:nexoclom2.atomicdata.Atom.__eq__}}
\pysigstartsignatures
\pysiglinewithargsret
{\sphinxbfcode{\sphinxupquote{\_\_eq\_\_}}}
{\sphinxparam{\DUrole{n}{other}}}
{}
\pysigstopsignatures
\end{fulllineitems}


\end{fulllineitems}

\index{lossrate() (in module nexoclom2.atomicdata)@\spxentry{lossrate()}\spxextra{in module nexoclom2.atomicdata}}

\begin{fulllineitems}
\phantomsection\label{\detokenize{autoapi/nexoclom2/atomicdata/index:nexoclom2.atomicdata.lossrate}}
\pysigstartsignatures
\pysiglinewithargsret
{\sphinxcode{\sphinxupquote{nexoclom2.atomicdata.}}\sphinxbfcode{\sphinxupquote{lossrate}}}
{\sphinxparam{\DUrole{n}{packets}}\sphinxparamcomma \sphinxparam{\DUrole{n}{output}}}
{}
\pysigstopsignatures
\sphinxAtStartPar
Calculate the loss rates due to photons, electron impacts, and charge\sphinxhyphen{}exchange
with optional constant loss rate.
\begin{quote}\begin{description}
\sphinxlineitem{Parameters}\begin{description}
\sphinxlineitem{\sphinxstylestrong{packets: nexoclom2 Packets object}}
\sphinxlineitem{\sphinxstylestrong{output: nexoclom2 Output object}}
\end{description}

\sphinxlineitem{Returns}\begin{description}
\sphinxlineitem{Ionization rate}
\end{description}

\end{description}\end{quote}


\begin{sphinxseealso}{See also:}
\begin{description}
\sphinxlineitem{{\hyperref[\detokenize{autoapi/nexoclom2/particle_tracking/packets/index:nexoclom2.particle_tracking.packets.Packets}]{\sphinxcrossref{\sphinxcode{\sphinxupquote{nexoclom2.particle\_tracking.packets.Packets}}}}}}
\sphinxlineitem{{\hyperref[\detokenize{autoapi/nexoclom2/particle_tracking/Output/index:nexoclom2.particle_tracking.Output.Output}]{\sphinxcrossref{\sphinxcode{\sphinxupquote{nexoclom2.particle\_tracking.Output.Output}}}}}}
\end{description}


\end{sphinxseealso}

\subsubsection*{Notes}

\sphinxAtStartPar
Charge exchange is not included yet.

\end{fulllineitems}


\sphinxstepscope


\paragraph{nexoclom2.data}
\label{\detokenize{autoapi/nexoclom2/data/index:module-nexoclom2.data}}\label{\detokenize{autoapi/nexoclom2/data/index:nexoclom2-data}}\label{\detokenize{autoapi/nexoclom2/data/index::doc}}\index{module@\spxentry{module}!nexoclom2.data@\spxentry{nexoclom2.data}}\index{nexoclom2.data@\spxentry{nexoclom2.data}!module@\spxentry{module}}
\sphinxAtStartPar
nexoclom.inputs package

\sphinxstepscope


\paragraph{nexoclom2.data\_simulation}
\label{\detokenize{autoapi/nexoclom2/data_simulation/index:module-nexoclom2.data_simulation}}\label{\detokenize{autoapi/nexoclom2/data_simulation/index:nexoclom2-data-simulation}}\label{\detokenize{autoapi/nexoclom2/data_simulation/index::doc}}\index{module@\spxentry{module}!nexoclom2.data\_simulation@\spxentry{nexoclom2.data\_simulation}}\index{nexoclom2.data\_simulation@\spxentry{nexoclom2.data\_simulation}!module@\spxentry{module}}
\sphinxAtStartPar
nexoclom.inputs package


\subparagraph{Submodules}
\label{\detokenize{autoapi/nexoclom2/data_simulation/index:submodules}}
\sphinxstepscope


\subparagraph{nexoclom2.data\_simulation.ModelImage}
\label{\detokenize{autoapi/nexoclom2/data_simulation/ModelImage/index:module-nexoclom2.data_simulation.ModelImage}}\label{\detokenize{autoapi/nexoclom2/data_simulation/ModelImage/index:nexoclom2-data-simulation-modelimage}}\label{\detokenize{autoapi/nexoclom2/data_simulation/ModelImage/index::doc}}\index{module@\spxentry{module}!nexoclom2.data\_simulation.ModelImage@\spxentry{nexoclom2.data\_simulation.ModelImage}}\index{nexoclom2.data\_simulation.ModelImage@\spxentry{nexoclom2.data\_simulation.ModelImage}!module@\spxentry{module}}

\subparagraph{Classes}
\label{\detokenize{autoapi/nexoclom2/data_simulation/ModelImage/index:classes}}

\begin{savenotes}\sphinxattablestart
\sphinxthistablewithglobalstyle
\sphinxthistablewithnovlinesstyle
\centering
\begin{tabulary}{\linewidth}[t]{\X{1}{2}\X{1}{2}}
\sphinxtoprule
\sphinxtableatstartofbodyhook
\sphinxAtStartPar
{\hyperref[\detokenize{autoapi/nexoclom2/data_simulation/ModelImage/index:nexoclom2.data_simulation.ModelImage.ModelImage}]{\sphinxcrossref{\sphinxcode{\sphinxupquote{ModelImage}}}}}
&
\sphinxAtStartPar

\\
\sphinxbottomrule
\end{tabulary}
\sphinxtableafterendhook\par
\sphinxattableend\end{savenotes}


\subparagraph{Module Contents}
\label{\detokenize{autoapi/nexoclom2/data_simulation/ModelImage/index:module-contents}}\index{ModelImage (class in nexoclom2.data\_simulation.ModelImage)@\spxentry{ModelImage}\spxextra{class in nexoclom2.data\_simulation.ModelImage}}

\begin{fulllineitems}
\phantomsection\label{\detokenize{autoapi/nexoclom2/data_simulation/ModelImage/index:nexoclom2.data_simulation.ModelImage.ModelImage}}
\pysigstartsignatures
\pysiglinewithargsret
{\sphinxbfcode{\sphinxupquote{class\DUrole{w}{ }}}\sphinxcode{\sphinxupquote{nexoclom2.data\_simulation.ModelImage.}}\sphinxbfcode{\sphinxupquote{ModelImage}}}
{\sphinxparam{\DUrole{n}{output}}\sphinxparamcomma \sphinxparam{\DUrole{n}{params}}\sphinxparamcomma \sphinxparam{\DUrole{n}{overwrite}\DUrole{o}{=}\DUrole{default_value}{False}}\sphinxparamcomma \sphinxparam{\DUrole{n}{chunksize}\DUrole{o}{=}\DUrole{default_value}{1000000}}}
{}
\pysigstopsignatures
\sphinxAtStartPar
Bases: {\hyperref[\detokenize{autoapi/nexoclom2/data_simulation/ModelResult/index:nexoclom2.data_simulation.ModelResult.ModelResult}]{\sphinxcrossref{\sphinxcode{\sphinxupquote{nexoclom2.data\_simulation.ModelResult.ModelResult}}}}}
\index{dimensions (nexoclom2.data\_simulation.ModelImage.ModelImage attribute)@\spxentry{dimensions}\spxextra{nexoclom2.data\_simulation.ModelImage.ModelImage attribute}}

\begin{fulllineitems}
\phantomsection\label{\detokenize{autoapi/nexoclom2/data_simulation/ModelImage/index:nexoclom2.data_simulation.ModelImage.ModelImage.dimensions}}
\pysigstartsignatures
\pysigline
{\sphinxbfcode{\sphinxupquote{dimensions}}}
\pysigstopsignatures
\end{fulllineitems}

\index{yrange (nexoclom2.data\_simulation.ModelImage.ModelImage attribute)@\spxentry{yrange}\spxextra{nexoclom2.data\_simulation.ModelImage.ModelImage attribute}}

\begin{fulllineitems}
\phantomsection\label{\detokenize{autoapi/nexoclom2/data_simulation/ModelImage/index:nexoclom2.data_simulation.ModelImage.ModelImage.yrange}}
\pysigstartsignatures
\pysigline
{\sphinxbfcode{\sphinxupquote{yrange}}}
\pysigstopsignatures
\end{fulllineitems}

\index{zrange (nexoclom2.data\_simulation.ModelImage.ModelImage attribute)@\spxentry{zrange}\spxextra{nexoclom2.data\_simulation.ModelImage.ModelImage attribute}}

\begin{fulllineitems}
\phantomsection\label{\detokenize{autoapi/nexoclom2/data_simulation/ModelImage/index:nexoclom2.data_simulation.ModelImage.ModelImage.zrange}}
\pysigstartsignatures
\pysigline
{\sphinxbfcode{\sphinxupquote{zrange}}}
\pysigstopsignatures
\end{fulllineitems}

\index{subobs\_longitude (nexoclom2.data\_simulation.ModelImage.ModelImage attribute)@\spxentry{subobs\_longitude}\spxextra{nexoclom2.data\_simulation.ModelImage.ModelImage attribute}}

\begin{fulllineitems}
\phantomsection\label{\detokenize{autoapi/nexoclom2/data_simulation/ModelImage/index:nexoclom2.data_simulation.ModelImage.ModelImage.subobs_longitude}}
\pysigstartsignatures
\pysigline
{\sphinxbfcode{\sphinxupquote{subobs\_longitude}}}
\pysigstopsignatures
\end{fulllineitems}

\index{subobs\_latitude (nexoclom2.data\_simulation.ModelImage.ModelImage attribute)@\spxentry{subobs\_latitude}\spxextra{nexoclom2.data\_simulation.ModelImage.ModelImage attribute}}

\begin{fulllineitems}
\phantomsection\label{\detokenize{autoapi/nexoclom2/data_simulation/ModelImage/index:nexoclom2.data_simulation.ModelImage.ModelImage.subobs_latitude}}
\pysigstartsignatures
\pysigline
{\sphinxbfcode{\sphinxupquote{subobs\_latitude}}}
\pysigstopsignatures
\end{fulllineitems}

\index{packet\_image (nexoclom2.data\_simulation.ModelImage.ModelImage attribute)@\spxentry{packet\_image}\spxextra{nexoclom2.data\_simulation.ModelImage.ModelImage attribute}}

\begin{fulllineitems}
\phantomsection\label{\detokenize{autoapi/nexoclom2/data_simulation/ModelImage/index:nexoclom2.data_simulation.ModelImage.ModelImage.packet_image}}
\pysigstartsignatures
\pysigline
{\sphinxbfcode{\sphinxupquote{packet\_image}}}
\pysigstopsignatures
\end{fulllineitems}

\index{Apix (nexoclom2.data\_simulation.ModelImage.ModelImage attribute)@\spxentry{Apix}\spxextra{nexoclom2.data\_simulation.ModelImage.ModelImage attribute}}

\begin{fulllineitems}
\phantomsection\label{\detokenize{autoapi/nexoclom2/data_simulation/ModelImage/index:nexoclom2.data_simulation.ModelImage.ModelImage.Apix}}
\pysigstartsignatures
\pysigline
{\sphinxbfcode{\sphinxupquote{Apix}}}
\pysigstopsignatures
\end{fulllineitems}

\index{rotation (nexoclom2.data\_simulation.ModelImage.ModelImage attribute)@\spxentry{rotation}\spxextra{nexoclom2.data\_simulation.ModelImage.ModelImage attribute}}

\begin{fulllineitems}
\phantomsection\label{\detokenize{autoapi/nexoclom2/data_simulation/ModelImage/index:nexoclom2.data_simulation.ModelImage.ModelImage.rotation}}
\pysigstartsignatures
\pysigline
{\sphinxbfcode{\sphinxupquote{rotation}}}
\pysigstopsignatures
\end{fulllineitems}

\index{centers (nexoclom2.data\_simulation.ModelImage.ModelImage attribute)@\spxentry{centers}\spxextra{nexoclom2.data\_simulation.ModelImage.ModelImage attribute}}

\begin{fulllineitems}
\phantomsection\label{\detokenize{autoapi/nexoclom2/data_simulation/ModelImage/index:nexoclom2.data_simulation.ModelImage.ModelImage.centers}}
\pysigstartsignatures
\pysigline
{\sphinxbfcode{\sphinxupquote{centers}}}
\pysigstopsignatures
\end{fulllineitems}


\end{fulllineitems}


\sphinxstepscope


\subparagraph{nexoclom2.data\_simulation.ModelResult}
\label{\detokenize{autoapi/nexoclom2/data_simulation/ModelResult/index:module-nexoclom2.data_simulation.ModelResult}}\label{\detokenize{autoapi/nexoclom2/data_simulation/ModelResult/index:nexoclom2-data-simulation-modelresult}}\label{\detokenize{autoapi/nexoclom2/data_simulation/ModelResult/index::doc}}\index{module@\spxentry{module}!nexoclom2.data\_simulation.ModelResult@\spxentry{nexoclom2.data\_simulation.ModelResult}}\index{nexoclom2.data\_simulation.ModelResult@\spxentry{nexoclom2.data\_simulation.ModelResult}!module@\spxentry{module}}

\subparagraph{Classes}
\label{\detokenize{autoapi/nexoclom2/data_simulation/ModelResult/index:classes}}

\begin{savenotes}\sphinxattablestart
\sphinxthistablewithglobalstyle
\sphinxthistablewithnovlinesstyle
\centering
\begin{tabulary}{\linewidth}[t]{\X{1}{2}\X{1}{2}}
\sphinxtoprule
\sphinxtableatstartofbodyhook
\sphinxAtStartPar
{\hyperref[\detokenize{autoapi/nexoclom2/data_simulation/ModelResult/index:nexoclom2.data_simulation.ModelResult.ResultPacket}]{\sphinxcrossref{\sphinxcode{\sphinxupquote{ResultPacket}}}}}
&
\sphinxAtStartPar

\\
\sphinxhline
\sphinxAtStartPar
{\hyperref[\detokenize{autoapi/nexoclom2/data_simulation/ModelResult/index:nexoclom2.data_simulation.ModelResult.ModelResult}]{\sphinxcrossref{\sphinxcode{\sphinxupquote{ModelResult}}}}}
&
\sphinxAtStartPar

\\
\sphinxbottomrule
\end{tabulary}
\sphinxtableafterendhook\par
\sphinxattableend\end{savenotes}


\subparagraph{Module Contents}
\label{\detokenize{autoapi/nexoclom2/data_simulation/ModelResult/index:module-contents}}\index{ResultPacket (class in nexoclom2.data\_simulation.ModelResult)@\spxentry{ResultPacket}\spxextra{class in nexoclom2.data\_simulation.ModelResult}}

\begin{fulllineitems}
\phantomsection\label{\detokenize{autoapi/nexoclom2/data_simulation/ModelResult/index:nexoclom2.data_simulation.ModelResult.ResultPacket}}
\pysigstartsignatures
\pysiglinewithargsret
{\sphinxbfcode{\sphinxupquote{class\DUrole{w}{ }}}\sphinxcode{\sphinxupquote{nexoclom2.data\_simulation.ModelResult.}}\sphinxbfcode{\sphinxupquote{ResultPacket}}}
{\sphinxparam{\DUrole{n}{X}}}
{}
\pysigstopsignatures\index{time (nexoclom2.data\_simulation.ModelResult.ResultPacket attribute)@\spxentry{time}\spxextra{nexoclom2.data\_simulation.ModelResult.ResultPacket attribute}}

\begin{fulllineitems}
\phantomsection\label{\detokenize{autoapi/nexoclom2/data_simulation/ModelResult/index:nexoclom2.data_simulation.ModelResult.ResultPacket.time}}
\pysigstartsignatures
\pysigline
{\sphinxbfcode{\sphinxupquote{time}}}
\pysigstopsignatures
\end{fulllineitems}

\index{X (nexoclom2.data\_simulation.ModelResult.ResultPacket attribute)@\spxentry{X}\spxextra{nexoclom2.data\_simulation.ModelResult.ResultPacket attribute}}

\begin{fulllineitems}
\phantomsection\label{\detokenize{autoapi/nexoclom2/data_simulation/ModelResult/index:nexoclom2.data_simulation.ModelResult.ResultPacket.X}}
\pysigstartsignatures
\pysigline
{\sphinxbfcode{\sphinxupquote{X}}}
\pysigstopsignatures
\end{fulllineitems}


\end{fulllineitems}

\index{ModelResult (class in nexoclom2.data\_simulation.ModelResult)@\spxentry{ModelResult}\spxextra{class in nexoclom2.data\_simulation.ModelResult}}

\begin{fulllineitems}
\phantomsection\label{\detokenize{autoapi/nexoclom2/data_simulation/ModelResult/index:nexoclom2.data_simulation.ModelResult.ModelResult}}
\pysigstartsignatures
\pysiglinewithargsret
{\sphinxbfcode{\sphinxupquote{class\DUrole{w}{ }}}\sphinxcode{\sphinxupquote{nexoclom2.data\_simulation.ModelResult.}}\sphinxbfcode{\sphinxupquote{ModelResult}}}
{\sphinxparam{\DUrole{n}{output}}\sphinxparamcomma \sphinxparam{\DUrole{n}{params}}}
{}
\pysigstopsignatures\index{species (nexoclom2.data\_simulation.ModelResult.ModelResult attribute)@\spxentry{species}\spxextra{nexoclom2.data\_simulation.ModelResult.ModelResult attribute}}

\begin{fulllineitems}
\phantomsection\label{\detokenize{autoapi/nexoclom2/data_simulation/ModelResult/index:nexoclom2.data_simulation.ModelResult.ModelResult.species}}
\pysigstartsignatures
\pysigline
{\sphinxbfcode{\sphinxupquote{species}}}
\pysigstopsignatures
\end{fulllineitems}

\index{origin (nexoclom2.data\_simulation.ModelResult.ModelResult attribute)@\spxentry{origin}\spxextra{nexoclom2.data\_simulation.ModelResult.ModelResult attribute}}

\begin{fulllineitems}
\phantomsection\label{\detokenize{autoapi/nexoclom2/data_simulation/ModelResult/index:nexoclom2.data_simulation.ModelResult.ModelResult.origin}}
\pysigstartsignatures
\pysigline
{\sphinxbfcode{\sphinxupquote{origin}}}
\pysigstopsignatures
\end{fulllineitems}

\index{quantity (nexoclom2.data\_simulation.ModelResult.ModelResult attribute)@\spxentry{quantity}\spxextra{nexoclom2.data\_simulation.ModelResult.ModelResult attribute}}

\begin{fulllineitems}
\phantomsection\label{\detokenize{autoapi/nexoclom2/data_simulation/ModelResult/index:nexoclom2.data_simulation.ModelResult.ModelResult.quantity}}
\pysigstartsignatures
\pysigline
{\sphinxbfcode{\sphinxupquote{quantity}}}
\pysigstopsignatures
\end{fulllineitems}

\index{radiance\_per\_atom() (nexoclom2.data\_simulation.ModelResult.ModelResult method)@\spxentry{radiance\_per\_atom()}\spxextra{nexoclom2.data\_simulation.ModelResult.ModelResult method}}

\begin{fulllineitems}
\phantomsection\label{\detokenize{autoapi/nexoclom2/data_simulation/ModelResult/index:nexoclom2.data_simulation.ModelResult.ModelResult.radiance_per_atom}}
\pysigstartsignatures
\pysiglinewithargsret
{\sphinxbfcode{\sphinxupquote{radiance\_per\_atom}}}
{\sphinxparam{\DUrole{n}{X}}\sphinxparamcomma \sphinxparam{\DUrole{n}{V}}\sphinxparamcomma \sphinxparam{\DUrole{n}{output}}}
{}
\pysigstopsignatures
\end{fulllineitems}

\index{choose\_wavelengths() (nexoclom2.data\_simulation.ModelResult.ModelResult method)@\spxentry{choose\_wavelengths()}\spxextra{nexoclom2.data\_simulation.ModelResult.ModelResult method}}

\begin{fulllineitems}
\phantomsection\label{\detokenize{autoapi/nexoclom2/data_simulation/ModelResult/index:nexoclom2.data_simulation.ModelResult.ModelResult.choose_wavelengths}}
\pysigstartsignatures
\pysiglinewithargsret
{\sphinxbfcode{\sphinxupquote{choose\_wavelengths}}}
{\sphinxparam{\DUrole{n}{wavelengths}}}
{}
\pysigstopsignatures
\end{fulllineitems}


\end{fulllineitems}


\sphinxstepscope


\paragraph{nexoclom2.initial\_state}
\label{\detokenize{autoapi/nexoclom2/initial_state/index:module-nexoclom2.initial_state}}\label{\detokenize{autoapi/nexoclom2/initial_state/index:nexoclom2-initial-state}}\label{\detokenize{autoapi/nexoclom2/initial_state/index::doc}}\index{module@\spxentry{module}!nexoclom2.initial\_state@\spxentry{nexoclom2.initial\_state}}\index{nexoclom2.initial\_state@\spxentry{nexoclom2.initial\_state}!module@\spxentry{module}}
\sphinxAtStartPar
nexoclom2.initial\_state package


\subparagraph{Submodules}
\label{\detokenize{autoapi/nexoclom2/initial_state/index:submodules}}
\sphinxstepscope


\subparagraph{nexoclom2.initial\_state.Forces}
\label{\detokenize{autoapi/nexoclom2/initial_state/Forces/index:module-nexoclom2.initial_state.Forces}}\label{\detokenize{autoapi/nexoclom2/initial_state/Forces/index:nexoclom2-initial-state-forces}}\label{\detokenize{autoapi/nexoclom2/initial_state/Forces/index::doc}}\index{module@\spxentry{module}!nexoclom2.initial\_state.Forces@\spxentry{nexoclom2.initial\_state.Forces}}\index{nexoclom2.initial\_state.Forces@\spxentry{nexoclom2.initial\_state.Forces}!module@\spxentry{module}}

\subparagraph{Classes}
\label{\detokenize{autoapi/nexoclom2/initial_state/Forces/index:classes}}

\begin{savenotes}\sphinxattablestart
\sphinxthistablewithglobalstyle
\sphinxthistablewithnovlinesstyle
\centering
\begin{tabulary}{\linewidth}[t]{\X{1}{2}\X{1}{2}}
\sphinxtoprule
\sphinxtableatstartofbodyhook
\sphinxAtStartPar
{\hyperref[\detokenize{autoapi/nexoclom2/initial_state/Forces/index:nexoclom2.initial_state.Forces.Forces}]{\sphinxcrossref{\sphinxcode{\sphinxupquote{Forces}}}}}
&
\sphinxAtStartPar
Specify what forces to include in the model simulation
\\
\sphinxbottomrule
\end{tabulary}
\sphinxtableafterendhook\par
\sphinxattableend\end{savenotes}


\subparagraph{Module Contents}
\label{\detokenize{autoapi/nexoclom2/initial_state/Forces/index:module-contents}}\index{Forces (class in nexoclom2.initial\_state.Forces)@\spxentry{Forces}\spxextra{class in nexoclom2.initial\_state.Forces}}

\begin{fulllineitems}
\phantomsection\label{\detokenize{autoapi/nexoclom2/initial_state/Forces/index:nexoclom2.initial_state.Forces.Forces}}
\pysigstartsignatures
\pysiglinewithargsret
{\sphinxbfcode{\sphinxupquote{class\DUrole{w}{ }}}\sphinxcode{\sphinxupquote{nexoclom2.initial\_state.Forces.}}\sphinxbfcode{\sphinxupquote{Forces}}}
{\sphinxparam{\DUrole{n}{fparam}\DUrole{p}{:}\DUrole{w}{ }\DUrole{n}{dict\DUrole{p}{,}\DUrole{w}{ }tinydb.table.Document}}}
{}
\pysigstopsignatures
\sphinxAtStartPar
Bases: {\hyperref[\detokenize{autoapi/nexoclom2/initial_state/InputClass/index:nexoclom2.initial_state.InputClass.InputClass}]{\sphinxcrossref{\sphinxcode{\sphinxupquote{nexoclom2.initial\_state.InputClass.InputClass}}}}}

\sphinxAtStartPar
Specify what forces to include in the model simulation

\sphinxAtStartPar
Specify whether to include gravitational and radiation pressure forces.
Default is to include both. See {\hyperref[\detokenize{nexoclom2/inputfiles:forces}]{\sphinxcrossref{\DUrole{std}{\DUrole{std-ref}{Forces}}}}} for more information.
\begin{quote}\begin{description}
\sphinxlineitem{Parameters}\begin{description}
\sphinxlineitem{\sphinxstylestrong{fparams}}{[}dict, tinydb.table.Document{]}
\sphinxAtStartPar
keys, values for indicating what forces to include. If a tinydb
Document, no checks are performed since it is assumed to be a record
from the database

\end{description}

\sphinxlineitem{Attributes}\begin{description}
\sphinxlineitem{\sphinxstylestrong{gravity}}{[}bool, default = True{]}
\sphinxlineitem{\sphinxstylestrong{radpres}}{[}bool, default = True{]}
\end{description}

\end{description}\end{quote}
\index{\_\_name\_\_ (nexoclom2.initial\_state.Forces.Forces attribute)@\spxentry{\_\_name\_\_}\spxextra{nexoclom2.initial\_state.Forces.Forces attribute}}

\begin{fulllineitems}
\phantomsection\label{\detokenize{autoapi/nexoclom2/initial_state/Forces/index:nexoclom2.initial_state.Forces.Forces.__name__}}
\pysigstartsignatures
\pysigline
{\sphinxbfcode{\sphinxupquote{\_\_name\_\_}}\sphinxbfcode{\sphinxupquote{\DUrole{w}{ }\DUrole{p}{=}\DUrole{w}{ }\textquotesingle{}Forces\textquotesingle{}}}}
\pysigstopsignatures
\end{fulllineitems}


\end{fulllineitems}


\sphinxstepscope


\subparagraph{nexoclom2.initial\_state.Input}
\label{\detokenize{autoapi/nexoclom2/initial_state/Input/index:module-nexoclom2.initial_state.Input}}\label{\detokenize{autoapi/nexoclom2/initial_state/Input/index:nexoclom2-initial-state-input}}\label{\detokenize{autoapi/nexoclom2/initial_state/Input/index::doc}}\index{module@\spxentry{module}!nexoclom2.initial\_state.Input@\spxentry{nexoclom2.initial\_state.Input}}\index{nexoclom2.initial\_state.Input@\spxentry{nexoclom2.initial\_state.Input}!module@\spxentry{module}}

\subparagraph{Classes}
\label{\detokenize{autoapi/nexoclom2/initial_state/Input/index:classes}}

\begin{savenotes}\sphinxattablestart
\sphinxthistablewithglobalstyle
\sphinxthistablewithnovlinesstyle
\centering
\begin{tabulary}{\linewidth}[t]{\X{1}{2}\X{1}{2}}
\sphinxtoprule
\sphinxtableatstartofbodyhook
\sphinxAtStartPar
{\hyperref[\detokenize{autoapi/nexoclom2/initial_state/Input/index:nexoclom2.initial_state.Input.Input}]{\sphinxcrossref{\sphinxcode{\sphinxupquote{Input}}}}}
&
\sphinxAtStartPar
Class defining all input parameters for a NEXOCLOM2 model run.
\\
\sphinxbottomrule
\end{tabulary}
\sphinxtableafterendhook\par
\sphinxattableend\end{savenotes}


\subparagraph{Module Contents}
\label{\detokenize{autoapi/nexoclom2/initial_state/Input/index:module-contents}}\index{Input (class in nexoclom2.initial\_state.Input)@\spxentry{Input}\spxextra{class in nexoclom2.initial\_state.Input}}

\begin{fulllineitems}
\phantomsection\label{\detokenize{autoapi/nexoclom2/initial_state/Input/index:nexoclom2.initial_state.Input.Input}}
\pysigstartsignatures
\pysiglinewithargsret
{\sphinxbfcode{\sphinxupquote{class\DUrole{w}{ }}}\sphinxcode{\sphinxupquote{nexoclom2.initial\_state.Input.}}\sphinxbfcode{\sphinxupquote{Input}}}
{\sphinxparam{\DUrole{n}{infile}\DUrole{p}{:}\DUrole{w}{ }\DUrole{n}{str}}}
{}
\pysigstopsignatures
\sphinxAtStartPar
Class defining all input parameters for a NEXOCLOM2 model run.
\begin{quote}\begin{description}
\sphinxlineitem{Parameters}\begin{description}
\sphinxlineitem{\sphinxstylestrong{infile}}{[}str{]}
\sphinxAtStartPar
plain text file containing model input parameters. See inputfiles for
a description of the input file format

\end{description}

\sphinxlineitem{Attributes}\begin{description}
\sphinxlineitem{\sphinxstylestrong{geometry}}{[}Geometry{]}
\sphinxlineitem{\sphinxstylestrong{forces}}{[}Forces{]}
\sphinxlineitem{\sphinxstylestrong{surfaceinteraction}}{[}ConstantSurfaceInteraction, etc{]}
\sphinxlineitem{\sphinxstylestrong{spatialdist}}{[}UniformSpatialDist, etc.{]}
\sphinxlineitem{\sphinxstylestrong{speeddist}}{[}GaussianSpeedDist, etc.{]}
\sphinxlineitem{\sphinxstylestrong{angulardist}}{[}RadialAngularDist, IsotropicAngularDist{]}
\sphinxlineitem{\sphinxstylestrong{lossinfo}}{[}LossInformation{]}
\sphinxlineitem{\sphinxstylestrong{options}}{[}Options{]}
\end{description}

\end{description}\end{quote}
\index{\_inputfile (nexoclom2.initial\_state.Input.Input attribute)@\spxentry{\_inputfile}\spxextra{nexoclom2.initial\_state.Input.Input attribute}}

\begin{fulllineitems}
\phantomsection\label{\detokenize{autoapi/nexoclom2/initial_state/Input/index:nexoclom2.initial_state.Input.Input._inputfile}}
\pysigstartsignatures
\pysigline
{\sphinxbfcode{\sphinxupquote{\_inputfile}}}
\pysigstopsignatures
\end{fulllineitems}

\index{\_classes (nexoclom2.initial\_state.Input.Input attribute)@\spxentry{\_classes}\spxextra{nexoclom2.initial\_state.Input.Input attribute}}

\begin{fulllineitems}
\phantomsection\label{\detokenize{autoapi/nexoclom2/initial_state/Input/index:nexoclom2.initial_state.Input.Input._classes}}
\pysigstartsignatures
\pysigline
{\sphinxbfcode{\sphinxupquote{\_classes}}\sphinxbfcode{\sphinxupquote{\DUrole{w}{ }\DUrole{p}{=}\DUrole{w}{ }{[}\textquotesingle{}geometry\textquotesingle{}, \textquotesingle{}surfaceinteraction\textquotesingle{}, \textquotesingle{}forces\textquotesingle{}, \textquotesingle{}spatialdist\textquotesingle{}, \textquotesingle{}speeddist\textquotesingle{}, \textquotesingle{}angulardist\textquotesingle{},...}}}
\pysigstopsignatures
\end{fulllineitems}

\index{config (nexoclom2.initial\_state.Input.Input attribute)@\spxentry{config}\spxextra{nexoclom2.initial\_state.Input.Input attribute}}

\begin{fulllineitems}
\phantomsection\label{\detokenize{autoapi/nexoclom2/initial_state/Input/index:nexoclom2.initial_state.Input.Input.config}}
\pysigstartsignatures
\pysigline
{\sphinxbfcode{\sphinxupquote{config}}}
\pysigstopsignatures
\end{fulllineitems}

\index{forces (nexoclom2.initial\_state.Input.Input attribute)@\spxentry{forces}\spxextra{nexoclom2.initial\_state.Input.Input attribute}}

\begin{fulllineitems}
\phantomsection\label{\detokenize{autoapi/nexoclom2/initial_state/Input/index:nexoclom2.initial_state.Input.Input.forces}}
\pysigstartsignatures
\pysigline
{\sphinxbfcode{\sphinxupquote{forces}}}
\pysigstopsignatures
\end{fulllineitems}

\index{lossinfo (nexoclom2.initial\_state.Input.Input attribute)@\spxentry{lossinfo}\spxextra{nexoclom2.initial\_state.Input.Input attribute}}

\begin{fulllineitems}
\phantomsection\label{\detokenize{autoapi/nexoclom2/initial_state/Input/index:nexoclom2.initial_state.Input.Input.lossinfo}}
\pysigstartsignatures
\pysigline
{\sphinxbfcode{\sphinxupquote{lossinfo}}}
\pysigstopsignatures
\end{fulllineitems}

\index{options (nexoclom2.initial\_state.Input.Input attribute)@\spxentry{options}\spxextra{nexoclom2.initial\_state.Input.Input attribute}}

\begin{fulllineitems}
\phantomsection\label{\detokenize{autoapi/nexoclom2/initial_state/Input/index:nexoclom2.initial_state.Input.Input.options}}
\pysigstartsignatures
\pysigline
{\sphinxbfcode{\sphinxupquote{options}}}
\pysigstopsignatures
\end{fulllineitems}

\index{\_\_str\_\_() (nexoclom2.initial\_state.Input.Input method)@\spxentry{\_\_str\_\_()}\spxextra{nexoclom2.initial\_state.Input.Input method}}

\begin{fulllineitems}
\phantomsection\label{\detokenize{autoapi/nexoclom2/initial_state/Input/index:nexoclom2.initial_state.Input.Input.__str__}}
\pysigstartsignatures
\pysiglinewithargsret
{\sphinxbfcode{\sphinxupquote{\_\_str\_\_}}}
{}
{}
\pysigstopsignatures
\end{fulllineitems}

\index{\_\_repr\_\_() (nexoclom2.initial\_state.Input.Input method)@\spxentry{\_\_repr\_\_()}\spxextra{nexoclom2.initial\_state.Input.Input method}}

\begin{fulllineitems}
\phantomsection\label{\detokenize{autoapi/nexoclom2/initial_state/Input/index:nexoclom2.initial_state.Input.Input.__repr__}}
\pysigstartsignatures
\pysiglinewithargsret
{\sphinxbfcode{\sphinxupquote{\_\_repr\_\_}}}
{}
{}
\pysigstopsignatures
\end{fulllineitems}

\index{\_\_eq\_\_() (nexoclom2.initial\_state.Input.Input method)@\spxentry{\_\_eq\_\_()}\spxextra{nexoclom2.initial\_state.Input.Input method}}

\begin{fulllineitems}
\phantomsection\label{\detokenize{autoapi/nexoclom2/initial_state/Input/index:nexoclom2.initial_state.Input.Input.__eq__}}
\pysigstartsignatures
\pysiglinewithargsret
{\sphinxbfcode{\sphinxupquote{\_\_eq\_\_}}}
{\sphinxparam{\DUrole{n}{other}}}
{}
\pysigstopsignatures
\end{fulllineitems}

\index{\_read\_params() (nexoclom2.initial\_state.Input.Input method)@\spxentry{\_read\_params()}\spxextra{nexoclom2.initial\_state.Input.Input method}}

\begin{fulllineitems}
\phantomsection\label{\detokenize{autoapi/nexoclom2/initial_state/Input/index:nexoclom2.initial_state.Input.Input._read_params}}
\pysigstartsignatures
\pysiglinewithargsret
{\sphinxbfcode{\sphinxupquote{\_read\_params}}}
{}
{}
\pysigstopsignatures
\end{fulllineitems}

\index{search() (nexoclom2.initial\_state.Input.Input method)@\spxentry{search()}\spxextra{nexoclom2.initial\_state.Input.Input method}}

\begin{fulllineitems}
\phantomsection\label{\detokenize{autoapi/nexoclom2/initial_state/Input/index:nexoclom2.initial_state.Input.Input.search}}
\pysigstartsignatures
\pysiglinewithargsret
{\sphinxbfcode{\sphinxupquote{search}}}
{}
{}
\pysigstopsignatures
\sphinxAtStartPar
This method allows users to search for inputs without having to
instantiate the database themselves.
\begin{quote}\begin{description}
\sphinxlineitem{Returns}\begin{description}
\sphinxlineitem{Database document ID associated with inputs}
\end{description}

\end{description}\end{quote}

\end{fulllineitems}

\index{make\_savefile() (nexoclom2.initial\_state.Input.Input method)@\spxentry{make\_savefile()}\spxextra{nexoclom2.initial\_state.Input.Input method}}

\begin{fulllineitems}
\phantomsection\label{\detokenize{autoapi/nexoclom2/initial_state/Input/index:nexoclom2.initial_state.Input.Input.make_savefile}}
\pysigstartsignatures
\pysiglinewithargsret
{\sphinxbfcode{\sphinxupquote{make\_savefile}}}
{\sphinxparam{\DUrole{n}{doc\_id}}}
{}
\pysigstopsignatures
\end{fulllineitems}


\end{fulllineitems}


\sphinxstepscope


\subparagraph{nexoclom2.initial\_state.InputClass}
\label{\detokenize{autoapi/nexoclom2/initial_state/InputClass/index:module-nexoclom2.initial_state.InputClass}}\label{\detokenize{autoapi/nexoclom2/initial_state/InputClass/index:nexoclom2-initial-state-inputclass}}\label{\detokenize{autoapi/nexoclom2/initial_state/InputClass/index::doc}}\index{module@\spxentry{module}!nexoclom2.initial\_state.InputClass@\spxentry{nexoclom2.initial\_state.InputClass}}\index{nexoclom2.initial\_state.InputClass@\spxentry{nexoclom2.initial\_state.InputClass}!module@\spxentry{module}}

\subparagraph{Classes}
\label{\detokenize{autoapi/nexoclom2/initial_state/InputClass/index:classes}}

\begin{savenotes}\sphinxattablestart
\sphinxthistablewithglobalstyle
\sphinxthistablewithnovlinesstyle
\centering
\begin{tabulary}{\linewidth}[t]{\X{1}{2}\X{1}{2}}
\sphinxtoprule
\sphinxtableatstartofbodyhook
\sphinxAtStartPar
{\hyperref[\detokenize{autoapi/nexoclom2/initial_state/InputClass/index:nexoclom2.initial_state.InputClass.InputClass}]{\sphinxcrossref{\sphinxcode{\sphinxupquote{InputClass}}}}}
&
\sphinxAtStartPar
Base class for Input subclasses.
\\
\sphinxbottomrule
\end{tabulary}
\sphinxtableafterendhook\par
\sphinxattableend\end{savenotes}


\subparagraph{Module Contents}
\label{\detokenize{autoapi/nexoclom2/initial_state/InputClass/index:module-contents}}\index{InputClass (class in nexoclom2.initial\_state.InputClass)@\spxentry{InputClass}\spxextra{class in nexoclom2.initial\_state.InputClass}}

\begin{fulllineitems}
\phantomsection\label{\detokenize{autoapi/nexoclom2/initial_state/InputClass/index:nexoclom2.initial_state.InputClass.InputClass}}
\pysigstartsignatures
\pysiglinewithargsret
{\sphinxbfcode{\sphinxupquote{class\DUrole{w}{ }}}\sphinxcode{\sphinxupquote{nexoclom2.initial\_state.InputClass.}}\sphinxbfcode{\sphinxupquote{InputClass}}}
{\sphinxparam{\DUrole{n}{sparam}\DUrole{p}{:}\DUrole{w}{ }\DUrole{n}{dict\DUrole{p}{,}\DUrole{w}{ }tinydb.table.Document}}}
{}
\pysigstopsignatures
\sphinxAtStartPar
Base class for Input subclasses.
\index{\_\_eq\_\_() (nexoclom2.initial\_state.InputClass.InputClass method)@\spxentry{\_\_eq\_\_()}\spxextra{nexoclom2.initial\_state.InputClass.InputClass method}}

\begin{fulllineitems}
\phantomsection\label{\detokenize{autoapi/nexoclom2/initial_state/InputClass/index:nexoclom2.initial_state.InputClass.InputClass.__eq__}}
\pysigstartsignatures
\pysiglinewithargsret
{\sphinxbfcode{\sphinxupquote{\_\_eq\_\_}}}
{\sphinxparam{\DUrole{n}{other}}}
{}
\pysigstopsignatures
\end{fulllineitems}

\index{\_check\_value() (nexoclom2.initial\_state.InputClass.InputClass method)@\spxentry{\_check\_value()}\spxextra{nexoclom2.initial\_state.InputClass.InputClass method}}

\begin{fulllineitems}
\phantomsection\label{\detokenize{autoapi/nexoclom2/initial_state/InputClass/index:nexoclom2.initial_state.InputClass.InputClass._check_value}}
\pysigstartsignatures
\pysiglinewithargsret
{\sphinxbfcode{\sphinxupquote{\_check\_value}}}
{\sphinxparam{\DUrole{n}{value}}\sphinxparamcomma \sphinxparam{\DUrole{n}{rng}}\sphinxparamcomma \sphinxparam{\DUrole{n}{include\_min}\DUrole{o}{=}\DUrole{default_value}{True}}\sphinxparamcomma \sphinxparam{\DUrole{n}{include\_max}\DUrole{o}{=}\DUrole{default_value}{True}}}
{}
\pysigstopsignatures
\sphinxAtStartPar
Verify a value is in the proper range

\end{fulllineitems}

\index{\_\_str\_\_() (nexoclom2.initial\_state.InputClass.InputClass method)@\spxentry{\_\_str\_\_()}\spxextra{nexoclom2.initial\_state.InputClass.InputClass method}}

\begin{fulllineitems}
\phantomsection\label{\detokenize{autoapi/nexoclom2/initial_state/InputClass/index:nexoclom2.initial_state.InputClass.InputClass.__str__}}
\pysigstartsignatures
\pysiglinewithargsret
{\sphinxbfcode{\sphinxupquote{\_\_str\_\_}}}
{}
{}
\pysigstopsignatures
\end{fulllineitems}

\index{\_\_repr\_\_() (nexoclom2.initial\_state.InputClass.InputClass method)@\spxentry{\_\_repr\_\_()}\spxextra{nexoclom2.initial\_state.InputClass.InputClass method}}

\begin{fulllineitems}
\phantomsection\label{\detokenize{autoapi/nexoclom2/initial_state/InputClass/index:nexoclom2.initial_state.InputClass.InputClass.__repr__}}
\pysigstartsignatures
\pysiglinewithargsret
{\sphinxbfcode{\sphinxupquote{\_\_repr\_\_}}}
{}
{}
\pysigstopsignatures
\end{fulllineitems}

\index{pdf() (nexoclom2.initial\_state.InputClass.InputClass method)@\spxentry{pdf()}\spxextra{nexoclom2.initial\_state.InputClass.InputClass method}}

\begin{fulllineitems}
\phantomsection\label{\detokenize{autoapi/nexoclom2/initial_state/InputClass/index:nexoclom2.initial_state.InputClass.InputClass.pdf}}
\pysigstartsignatures
\pysiglinewithargsret
{\sphinxbfcode{\sphinxupquote{pdf}}}
{\sphinxparam{\DUrole{n}{v}}}
{}
\pysigstopsignatures
\end{fulllineitems}

\index{support() (nexoclom2.initial\_state.InputClass.InputClass method)@\spxentry{support()}\spxextra{nexoclom2.initial\_state.InputClass.InputClass method}}

\begin{fulllineitems}
\phantomsection\label{\detokenize{autoapi/nexoclom2/initial_state/InputClass/index:nexoclom2.initial_state.InputClass.InputClass.support}}
\pysigstartsignatures
\pysiglinewithargsret
{\sphinxbfcode{\sphinxupquote{support}}}
{}
{}
\pysigstopsignatures
\end{fulllineitems}

\index{cdf() (nexoclom2.initial\_state.InputClass.InputClass method)@\spxentry{cdf()}\spxextra{nexoclom2.initial\_state.InputClass.InputClass method}}

\begin{fulllineitems}
\phantomsection\label{\detokenize{autoapi/nexoclom2/initial_state/InputClass/index:nexoclom2.initial_state.InputClass.InputClass.cdf}}
\pysigstartsignatures
\pysiglinewithargsret
{\sphinxbfcode{\sphinxupquote{cdf}}}
{\sphinxparam{\DUrole{n}{x}}}
{}
\pysigstopsignatures
\sphinxAtStartPar
Cumulative Distribution Function

\sphinxAtStartPar
pdf must be defined in the distribution if you want to use the cdf

\end{fulllineitems}

\index{generate1d() (nexoclom2.initial\_state.InputClass.InputClass method)@\spxentry{generate1d()}\spxextra{nexoclom2.initial\_state.InputClass.InputClass method}}

\begin{fulllineitems}
\phantomsection\label{\detokenize{autoapi/nexoclom2/initial_state/InputClass/index:nexoclom2.initial_state.InputClass.InputClass.generate1d}}
\pysigstartsignatures
\pysiglinewithargsret
{\sphinxbfcode{\sphinxupquote{generate1d}}}
{\sphinxparam{\DUrole{n}{n\_packets}}\sphinxparamcomma \sphinxparam{\DUrole{n}{randgen}\DUrole{o}{=}\DUrole{default_value}{None}}}
{}
\pysigstopsignatures
\sphinxAtStartPar
Compute random deviates from arbitrary 1D distribution.
f\_x does not need to integrate to 1. The function normalizes the
distribution. Uses Transformation method (Numerical Recipes, 7.3.2)
\begin{quote}\begin{description}
\sphinxlineitem{Parameters}\begin{description}
\sphinxlineitem{\sphinxstylestrong{n\_packets}}{[}int{]}
\sphinxAtStartPar
The number of random deviates to compute

\sphinxlineitem{\sphinxstylestrong{randgen}}{[}numpy.random.\_generator.Generator{]}
\end{description}

\sphinxlineitem{Returns}\begin{description}
\sphinxlineitem{numpy array of length num chosen from the distribution f\_x.}
\end{description}

\end{description}\end{quote}

\end{fulllineitems}

\index{generate\_sphere() (nexoclom2.initial\_state.InputClass.InputClass method)@\spxentry{generate\_sphere()}\spxextra{nexoclom2.initial\_state.InputClass.InputClass method}}

\begin{fulllineitems}
\phantomsection\label{\detokenize{autoapi/nexoclom2/initial_state/InputClass/index:nexoclom2.initial_state.InputClass.InputClass.generate_sphere}}
\pysigstartsignatures
\pysiglinewithargsret
{\sphinxbfcode{\sphinxupquote{generate\_sphere}}}
{\sphinxparam{\DUrole{n}{n\_packets}}\sphinxparamcomma \sphinxparam{\DUrole{n}{randgen}}}
{}
\pysigstopsignatures
\sphinxAtStartPar
Create random deviates (longitude, latitude) from a distribution on a sphere

\sphinxAtStartPar
Uses the rejection mechanism to choose random points on the surface
according to a specified distribution. Note that the distribution needs
to be defined in longitude and sin(latitude) for proper determination on
a sphere. Also, the probability distribution function (pdf) in the
distribution class must vary between 0 and 1

\end{fulllineitems}

\index{query() (nexoclom2.initial\_state.InputClass.InputClass method)@\spxentry{query()}\spxextra{nexoclom2.initial\_state.InputClass.InputClass method}}

\begin{fulllineitems}
\phantomsection\label{\detokenize{autoapi/nexoclom2/initial_state/InputClass/index:nexoclom2.initial_state.InputClass.InputClass.query}}
\pysigstartsignatures
\pysiglinewithargsret
{\sphinxbfcode{\sphinxupquote{query}}}
{}
{}
\pysigstopsignatures
\sphinxAtStartPar
Find matching records in the database
\begin{quote}\begin{description}
\sphinxlineitem{Returns}\begin{description}
\sphinxlineitem{Tuple of matching doc\_ids.}
\end{description}

\end{description}\end{quote}

\end{fulllineitems}


\end{fulllineitems}


\sphinxstepscope


\subparagraph{nexoclom2.initial\_state.LossInformation}
\label{\detokenize{autoapi/nexoclom2/initial_state/LossInformation/index:module-nexoclom2.initial_state.LossInformation}}\label{\detokenize{autoapi/nexoclom2/initial_state/LossInformation/index:nexoclom2-initial-state-lossinformation}}\label{\detokenize{autoapi/nexoclom2/initial_state/LossInformation/index::doc}}\index{module@\spxentry{module}!nexoclom2.initial\_state.LossInformation@\spxentry{nexoclom2.initial\_state.LossInformation}}\index{nexoclom2.initial\_state.LossInformation@\spxentry{nexoclom2.initial\_state.LossInformation}!module@\spxentry{module}}

\subparagraph{Classes}
\label{\detokenize{autoapi/nexoclom2/initial_state/LossInformation/index:classes}}

\begin{savenotes}\sphinxattablestart
\sphinxthistablewithglobalstyle
\sphinxthistablewithnovlinesstyle
\centering
\begin{tabulary}{\linewidth}[t]{\X{1}{2}\X{1}{2}}
\sphinxtoprule
\sphinxtableatstartofbodyhook
\sphinxAtStartPar
{\hyperref[\detokenize{autoapi/nexoclom2/initial_state/LossInformation/index:nexoclom2.initial_state.LossInformation.LossInformation}]{\sphinxcrossref{\sphinxcode{\sphinxupquote{LossInformation}}}}}
&
\sphinxAtStartPar
Determines how loss should be calculated.
\\
\sphinxbottomrule
\end{tabulary}
\sphinxtableafterendhook\par
\sphinxattableend\end{savenotes}


\subparagraph{Module Contents}
\label{\detokenize{autoapi/nexoclom2/initial_state/LossInformation/index:module-contents}}\index{LossInformation (class in nexoclom2.initial\_state.LossInformation)@\spxentry{LossInformation}\spxextra{class in nexoclom2.initial\_state.LossInformation}}

\begin{fulllineitems}
\phantomsection\label{\detokenize{autoapi/nexoclom2/initial_state/LossInformation/index:nexoclom2.initial_state.LossInformation.LossInformation}}
\pysigstartsignatures
\pysiglinewithargsret
{\sphinxbfcode{\sphinxupquote{class\DUrole{w}{ }}}\sphinxcode{\sphinxupquote{nexoclom2.initial\_state.LossInformation.}}\sphinxbfcode{\sphinxupquote{LossInformation}}}
{\sphinxparam{\DUrole{n}{lparam}\DUrole{p}{:}\DUrole{w}{ }\DUrole{n}{dict\DUrole{p}{,}\DUrole{w}{ }tinydb.table.Document}}\sphinxparamcomma \sphinxparam{\DUrole{n}{center}\DUrole{o}{=}\DUrole{default_value}{None}}\sphinxparamcomma \sphinxparam{\DUrole{n}{startpt}\DUrole{o}{=}\DUrole{default_value}{None}}}
{}
\pysigstopsignatures
\sphinxAtStartPar
Bases: {\hyperref[\detokenize{autoapi/nexoclom2/initial_state/InputClass/index:nexoclom2.initial_state.InputClass.InputClass}]{\sphinxcrossref{\sphinxcode{\sphinxupquote{nexoclom2.initial\_state.InputClass.InputClass}}}}}

\sphinxAtStartPar
Determines how loss should be calculated.

\sphinxAtStartPar
Configures the loss rates due to processes other than collisions with the
surface. See {\hyperref[\detokenize{nexoclom2/inputfiles:lossinfo}]{\sphinxcrossref{\DUrole{std}{\DUrole{std-ref}{Loss Information}}}}} for more information.
\begin{quote}\begin{description}
\sphinxlineitem{Parameters}\begin{description}
\sphinxlineitem{\sphinxstylestrong{lparam}}{[}dict{]}
\end{description}

\sphinxlineitem{Attributes}\begin{description}
\sphinxlineitem{\sphinxstylestrong{constant\_lifetime: astropy Time or False}}
\sphinxAtStartPar
Constant lifetime in seconds to use everywhere in the system.

\sphinxlineitem{\sphinxstylestrong{photoionization: bool}}
\sphinxAtStartPar
Determines whether to include photoionization.

\sphinxlineitem{\sphinxstylestrong{photoionization\_lifetime: astropy Time quantity}}
\sphinxAtStartPar
If 0, uses the measured photoionization rate. If \textgreater{}0, sets the
photoionization lifetime to that value.

\sphinxlineitem{\sphinxstylestrong{photo\_factor: float}}
\sphinxAtStartPar
Factor by which to modify photo\sphinxhyphen{}loss rate.

\sphinxlineitem{\sphinxstylestrong{electron\_impact: bool}}
\sphinxAtStartPar
Determines whether to include electron impact processes.

\sphinxlineitem{\sphinxstylestrong{eimp\_factor: float}}
\sphinxAtStartPar
Factor by which to modify electron impact rate.

\sphinxlineitem{\sphinxstylestrong{charge\_exchange: bool}}
\sphinxAtStartPar
Determines whether to include charge exchange.

\sphinxlineitem{\sphinxstylestrong{chx\_factor: float}}
\sphinxAtStartPar
Factor by which to modify charge exchange rate

\end{description}

\end{description}\end{quote}
\index{\_\_name\_\_ (nexoclom2.initial\_state.LossInformation.LossInformation attribute)@\spxentry{\_\_name\_\_}\spxextra{nexoclom2.initial\_state.LossInformation.LossInformation attribute}}

\begin{fulllineitems}
\phantomsection\label{\detokenize{autoapi/nexoclom2/initial_state/LossInformation/index:nexoclom2.initial_state.LossInformation.LossInformation.__name__}}
\pysigstartsignatures
\pysigline
{\sphinxbfcode{\sphinxupquote{\_\_name\_\_}}\sphinxbfcode{\sphinxupquote{\DUrole{w}{ }\DUrole{p}{=}\DUrole{w}{ }\textquotesingle{}LossInformation\textquotesingle{}}}}
\pysigstopsignatures
\end{fulllineitems}


\end{fulllineitems}


\sphinxstepscope


\subparagraph{nexoclom2.initial\_state.Options}
\label{\detokenize{autoapi/nexoclom2/initial_state/Options/index:module-nexoclom2.initial_state.Options}}\label{\detokenize{autoapi/nexoclom2/initial_state/Options/index:nexoclom2-initial-state-options}}\label{\detokenize{autoapi/nexoclom2/initial_state/Options/index::doc}}\index{module@\spxentry{module}!nexoclom2.initial\_state.Options@\spxentry{nexoclom2.initial\_state.Options}}\index{nexoclom2.initial\_state.Options@\spxentry{nexoclom2.initial\_state.Options}!module@\spxentry{module}}

\subparagraph{Classes}
\label{\detokenize{autoapi/nexoclom2/initial_state/Options/index:classes}}

\begin{savenotes}\sphinxattablestart
\sphinxthistablewithglobalstyle
\sphinxthistablewithnovlinesstyle
\centering
\begin{tabulary}{\linewidth}[t]{\X{1}{2}\X{1}{2}}
\sphinxtoprule
\sphinxtableatstartofbodyhook
\sphinxAtStartPar
{\hyperref[\detokenize{autoapi/nexoclom2/initial_state/Options/index:nexoclom2.initial_state.Options.Options}]{\sphinxcrossref{\sphinxcode{\sphinxupquote{Options}}}}}
&
\sphinxAtStartPar
Sets general run options
\\
\sphinxbottomrule
\end{tabulary}
\sphinxtableafterendhook\par
\sphinxattableend\end{savenotes}


\subparagraph{Module Contents}
\label{\detokenize{autoapi/nexoclom2/initial_state/Options/index:module-contents}}\index{Options (class in nexoclom2.initial\_state.Options)@\spxentry{Options}\spxextra{class in nexoclom2.initial\_state.Options}}

\begin{fulllineitems}
\phantomsection\label{\detokenize{autoapi/nexoclom2/initial_state/Options/index:nexoclom2.initial_state.Options.Options}}
\pysigstartsignatures
\pysiglinewithargsret
{\sphinxbfcode{\sphinxupquote{class\DUrole{w}{ }}}\sphinxcode{\sphinxupquote{nexoclom2.initial\_state.Options.}}\sphinxbfcode{\sphinxupquote{Options}}}
{\sphinxparam{\DUrole{n}{options}}}
{}
\pysigstopsignatures
\sphinxAtStartPar
Bases: {\hyperref[\detokenize{autoapi/nexoclom2/initial_state/InputClass/index:nexoclom2.initial_state.InputClass.InputClass}]{\sphinxcrossref{\sphinxcode{\sphinxupquote{nexoclom2.initial\_state.InputClass.InputClass}}}}}

\sphinxAtStartPar
Sets general run options

\sphinxAtStartPar
See ref:\sphinxtitleref{options} for more information.
\begin{quote}\begin{description}
\sphinxlineitem{Parameters}\begin{description}
\sphinxlineitem{\sphinxstylestrong{options}}{[}dict, TinyDB Document{]}
\sphinxAtStartPar
Key, value for defining the options

\end{description}

\sphinxlineitem{Attributes}\begin{description}
\sphinxlineitem{\sphinxstylestrong{runtime}}{[}astropy quantity{]}
\sphinxAtStartPar
Total runtime for the simulation in seconds

\sphinxlineitem{\sphinxstylestrong{species}}{[}str{]}
\sphinxlineitem{\sphinxstylestrong{outer\_edge}}{[}float, Default = 10**30 (infinite){]}
\sphinxAtStartPar
Outer edge of simulation in central body radii

\sphinxlineitem{\sphinxstylestrong{step\_size}}{[}astropy quantity, Default = 0 sec{]}
\sphinxAtStartPar
Time step. If 0, uses adaptive step size integrator.

\sphinxlineitem{\sphinxstylestrong{resolution}}{[}float, Default = 10**\sphinxhyphen{}5{]}
\sphinxAtStartPar
Precision necessary in adaptive step size integrator. Not used with
constant step size integrator.

\sphinxlineitem{\sphinxstylestrong{start\_together}}{[}bool, optional, Default = False{]}
\sphinxAtStartPar
If step\_size = 0, tells integrator to start all packets at the same
time. Good for seeing cloud evolution and particle trajectories

\sphinxlineitem{\sphinxstylestrong{random\_seed}}{[}int \textgreater{}= 0, None, Default = None{]}
\sphinxAtStartPar
seed for random generator

\end{description}

\end{description}\end{quote}
\index{\_\_name\_\_ (nexoclom2.initial\_state.Options.Options attribute)@\spxentry{\_\_name\_\_}\spxextra{nexoclom2.initial\_state.Options.Options attribute}}

\begin{fulllineitems}
\phantomsection\label{\detokenize{autoapi/nexoclom2/initial_state/Options/index:nexoclom2.initial_state.Options.Options.__name__}}
\pysigstartsignatures
\pysigline
{\sphinxbfcode{\sphinxupquote{\_\_name\_\_}}\sphinxbfcode{\sphinxupquote{\DUrole{w}{ }\DUrole{p}{=}\DUrole{w}{ }\textquotesingle{}Options\textquotesingle{}}}}
\pysigstopsignatures
\end{fulllineitems}


\end{fulllineitems}



\subparagraph{Classes}
\label{\detokenize{autoapi/nexoclom2/initial_state/index:classes}}

\begin{savenotes}\sphinxattablestart
\sphinxthistablewithglobalstyle
\sphinxthistablewithnovlinesstyle
\centering
\begin{tabulary}{\linewidth}[t]{\X{1}{2}\X{1}{2}}
\sphinxtoprule
\sphinxtableatstartofbodyhook
\sphinxAtStartPar
{\hyperref[\detokenize{autoapi/nexoclom2/initial_state/index:nexoclom2.initial_state.InputClass}]{\sphinxcrossref{\sphinxcode{\sphinxupquote{InputClass}}}}}
&
\sphinxAtStartPar
Base class for Input subclasses.
\\
\sphinxhline
\sphinxAtStartPar
{\hyperref[\detokenize{autoapi/nexoclom2/initial_state/index:nexoclom2.initial_state.Forces}]{\sphinxcrossref{\sphinxcode{\sphinxupquote{Forces}}}}}
&
\sphinxAtStartPar
Specify what forces to include in the model simulation
\\
\sphinxhline
\sphinxAtStartPar
{\hyperref[\detokenize{autoapi/nexoclom2/initial_state/index:nexoclom2.initial_state.LossInformation}]{\sphinxcrossref{\sphinxcode{\sphinxupquote{LossInformation}}}}}
&
\sphinxAtStartPar
Determines how loss should be calculated.
\\
\sphinxhline
\sphinxAtStartPar
{\hyperref[\detokenize{autoapi/nexoclom2/initial_state/index:nexoclom2.initial_state.Options}]{\sphinxcrossref{\sphinxcode{\sphinxupquote{Options}}}}}
&
\sphinxAtStartPar
Sets general run options
\\
\sphinxbottomrule
\end{tabulary}
\sphinxtableafterendhook\par
\sphinxattableend\end{savenotes}


\subparagraph{Package Contents}
\label{\detokenize{autoapi/nexoclom2/initial_state/index:package-contents}}\index{InputClass (class in nexoclom2.initial\_state)@\spxentry{InputClass}\spxextra{class in nexoclom2.initial\_state}}

\begin{fulllineitems}
\phantomsection\label{\detokenize{autoapi/nexoclom2/initial_state/index:nexoclom2.initial_state.InputClass}}
\pysigstartsignatures
\pysiglinewithargsret
{\sphinxbfcode{\sphinxupquote{class\DUrole{w}{ }}}\sphinxcode{\sphinxupquote{nexoclom2.initial\_state.}}\sphinxbfcode{\sphinxupquote{InputClass}}}
{\sphinxparam{\DUrole{n}{sparam}\DUrole{p}{:}\DUrole{w}{ }\DUrole{n}{dict\DUrole{p}{,}\DUrole{w}{ }tinydb.table.Document}}}
{}
\pysigstopsignatures
\sphinxAtStartPar
Base class for Input subclasses.
\index{\_\_eq\_\_() (nexoclom2.initial\_state.InputClass method)@\spxentry{\_\_eq\_\_()}\spxextra{nexoclom2.initial\_state.InputClass method}}

\begin{fulllineitems}
\phantomsection\label{\detokenize{autoapi/nexoclom2/initial_state/index:nexoclom2.initial_state.InputClass.__eq__}}
\pysigstartsignatures
\pysiglinewithargsret
{\sphinxbfcode{\sphinxupquote{\_\_eq\_\_}}}
{\sphinxparam{\DUrole{n}{other}}}
{}
\pysigstopsignatures
\end{fulllineitems}

\index{\_check\_value() (nexoclom2.initial\_state.InputClass method)@\spxentry{\_check\_value()}\spxextra{nexoclom2.initial\_state.InputClass method}}

\begin{fulllineitems}
\phantomsection\label{\detokenize{autoapi/nexoclom2/initial_state/index:nexoclom2.initial_state.InputClass._check_value}}
\pysigstartsignatures
\pysiglinewithargsret
{\sphinxbfcode{\sphinxupquote{\_check\_value}}}
{\sphinxparam{\DUrole{n}{value}}\sphinxparamcomma \sphinxparam{\DUrole{n}{rng}}\sphinxparamcomma \sphinxparam{\DUrole{n}{include\_min}\DUrole{o}{=}\DUrole{default_value}{True}}\sphinxparamcomma \sphinxparam{\DUrole{n}{include\_max}\DUrole{o}{=}\DUrole{default_value}{True}}}
{}
\pysigstopsignatures
\sphinxAtStartPar
Verify a value is in the proper range

\end{fulllineitems}

\index{\_\_str\_\_() (nexoclom2.initial\_state.InputClass method)@\spxentry{\_\_str\_\_()}\spxextra{nexoclom2.initial\_state.InputClass method}}

\begin{fulllineitems}
\phantomsection\label{\detokenize{autoapi/nexoclom2/initial_state/index:nexoclom2.initial_state.InputClass.__str__}}
\pysigstartsignatures
\pysiglinewithargsret
{\sphinxbfcode{\sphinxupquote{\_\_str\_\_}}}
{}
{}
\pysigstopsignatures
\end{fulllineitems}

\index{\_\_repr\_\_() (nexoclom2.initial\_state.InputClass method)@\spxentry{\_\_repr\_\_()}\spxextra{nexoclom2.initial\_state.InputClass method}}

\begin{fulllineitems}
\phantomsection\label{\detokenize{autoapi/nexoclom2/initial_state/index:nexoclom2.initial_state.InputClass.__repr__}}
\pysigstartsignatures
\pysiglinewithargsret
{\sphinxbfcode{\sphinxupquote{\_\_repr\_\_}}}
{}
{}
\pysigstopsignatures
\end{fulllineitems}

\index{pdf() (nexoclom2.initial\_state.InputClass method)@\spxentry{pdf()}\spxextra{nexoclom2.initial\_state.InputClass method}}

\begin{fulllineitems}
\phantomsection\label{\detokenize{autoapi/nexoclom2/initial_state/index:nexoclom2.initial_state.InputClass.pdf}}
\pysigstartsignatures
\pysiglinewithargsret
{\sphinxbfcode{\sphinxupquote{pdf}}}
{\sphinxparam{\DUrole{n}{v}}}
{}
\pysigstopsignatures
\end{fulllineitems}

\index{support() (nexoclom2.initial\_state.InputClass method)@\spxentry{support()}\spxextra{nexoclom2.initial\_state.InputClass method}}

\begin{fulllineitems}
\phantomsection\label{\detokenize{autoapi/nexoclom2/initial_state/index:nexoclom2.initial_state.InputClass.support}}
\pysigstartsignatures
\pysiglinewithargsret
{\sphinxbfcode{\sphinxupquote{support}}}
{}
{}
\pysigstopsignatures
\end{fulllineitems}

\index{cdf() (nexoclom2.initial\_state.InputClass method)@\spxentry{cdf()}\spxextra{nexoclom2.initial\_state.InputClass method}}

\begin{fulllineitems}
\phantomsection\label{\detokenize{autoapi/nexoclom2/initial_state/index:nexoclom2.initial_state.InputClass.cdf}}
\pysigstartsignatures
\pysiglinewithargsret
{\sphinxbfcode{\sphinxupquote{cdf}}}
{\sphinxparam{\DUrole{n}{x}}}
{}
\pysigstopsignatures
\sphinxAtStartPar
Cumulative Distribution Function

\sphinxAtStartPar
pdf must be defined in the distribution if you want to use the cdf

\end{fulllineitems}

\index{generate1d() (nexoclom2.initial\_state.InputClass method)@\spxentry{generate1d()}\spxextra{nexoclom2.initial\_state.InputClass method}}

\begin{fulllineitems}
\phantomsection\label{\detokenize{autoapi/nexoclom2/initial_state/index:nexoclom2.initial_state.InputClass.generate1d}}
\pysigstartsignatures
\pysiglinewithargsret
{\sphinxbfcode{\sphinxupquote{generate1d}}}
{\sphinxparam{\DUrole{n}{n\_packets}}\sphinxparamcomma \sphinxparam{\DUrole{n}{randgen}\DUrole{o}{=}\DUrole{default_value}{None}}}
{}
\pysigstopsignatures
\sphinxAtStartPar
Compute random deviates from arbitrary 1D distribution.
f\_x does not need to integrate to 1. The function normalizes the
distribution. Uses Transformation method (Numerical Recipes, 7.3.2)
\begin{quote}\begin{description}
\sphinxlineitem{Parameters}\begin{description}
\sphinxlineitem{\sphinxstylestrong{n\_packets}}{[}int{]}
\sphinxAtStartPar
The number of random deviates to compute

\sphinxlineitem{\sphinxstylestrong{randgen}}{[}numpy.random.\_generator.Generator{]}
\end{description}

\sphinxlineitem{Returns}\begin{description}
\sphinxlineitem{numpy array of length num chosen from the distribution f\_x.}
\end{description}

\end{description}\end{quote}

\end{fulllineitems}

\index{generate\_sphere() (nexoclom2.initial\_state.InputClass method)@\spxentry{generate\_sphere()}\spxextra{nexoclom2.initial\_state.InputClass method}}

\begin{fulllineitems}
\phantomsection\label{\detokenize{autoapi/nexoclom2/initial_state/index:nexoclom2.initial_state.InputClass.generate_sphere}}
\pysigstartsignatures
\pysiglinewithargsret
{\sphinxbfcode{\sphinxupquote{generate\_sphere}}}
{\sphinxparam{\DUrole{n}{n\_packets}}\sphinxparamcomma \sphinxparam{\DUrole{n}{randgen}}}
{}
\pysigstopsignatures
\sphinxAtStartPar
Create random deviates (longitude, latitude) from a distribution on a sphere

\sphinxAtStartPar
Uses the rejection mechanism to choose random points on the surface
according to a specified distribution. Note that the distribution needs
to be defined in longitude and sin(latitude) for proper determination on
a sphere. Also, the probability distribution function (pdf) in the
distribution class must vary between 0 and 1

\end{fulllineitems}

\index{query() (nexoclom2.initial\_state.InputClass method)@\spxentry{query()}\spxextra{nexoclom2.initial\_state.InputClass method}}

\begin{fulllineitems}
\phantomsection\label{\detokenize{autoapi/nexoclom2/initial_state/index:nexoclom2.initial_state.InputClass.query}}
\pysigstartsignatures
\pysiglinewithargsret
{\sphinxbfcode{\sphinxupquote{query}}}
{}
{}
\pysigstopsignatures
\sphinxAtStartPar
Find matching records in the database
\begin{quote}\begin{description}
\sphinxlineitem{Returns}\begin{description}
\sphinxlineitem{Tuple of matching doc\_ids.}
\end{description}

\end{description}\end{quote}

\end{fulllineitems}


\end{fulllineitems}

\index{Forces (class in nexoclom2.initial\_state)@\spxentry{Forces}\spxextra{class in nexoclom2.initial\_state}}

\begin{fulllineitems}
\phantomsection\label{\detokenize{autoapi/nexoclom2/initial_state/index:nexoclom2.initial_state.Forces}}
\pysigstartsignatures
\pysiglinewithargsret
{\sphinxbfcode{\sphinxupquote{class\DUrole{w}{ }}}\sphinxcode{\sphinxupquote{nexoclom2.initial\_state.}}\sphinxbfcode{\sphinxupquote{Forces}}}
{\sphinxparam{\DUrole{n}{fparam}\DUrole{p}{:}\DUrole{w}{ }\DUrole{n}{dict\DUrole{p}{,}\DUrole{w}{ }tinydb.table.Document}}}
{}
\pysigstopsignatures
\sphinxAtStartPar
Bases: {\hyperref[\detokenize{autoapi/nexoclom2/initial_state/InputClass/index:nexoclom2.initial_state.InputClass.InputClass}]{\sphinxcrossref{\sphinxcode{\sphinxupquote{nexoclom2.initial\_state.InputClass.InputClass}}}}}

\sphinxAtStartPar
Specify what forces to include in the model simulation

\sphinxAtStartPar
Specify whether to include gravitational and radiation pressure forces.
Default is to include both. See {\hyperref[\detokenize{nexoclom2/inputfiles:forces}]{\sphinxcrossref{\DUrole{std}{\DUrole{std-ref}{Forces}}}}} for more information.
\begin{quote}\begin{description}
\sphinxlineitem{Parameters}\begin{description}
\sphinxlineitem{\sphinxstylestrong{fparams}}{[}dict, tinydb.table.Document{]}
\sphinxAtStartPar
keys, values for indicating what forces to include. If a tinydb
Document, no checks are performed since it is assumed to be a record
from the database

\end{description}

\sphinxlineitem{Attributes}\begin{description}
\sphinxlineitem{\sphinxstylestrong{gravity}}{[}bool, default = True{]}
\sphinxlineitem{\sphinxstylestrong{radpres}}{[}bool, default = True{]}
\end{description}

\end{description}\end{quote}
\index{\_\_name\_\_ (nexoclom2.initial\_state.Forces attribute)@\spxentry{\_\_name\_\_}\spxextra{nexoclom2.initial\_state.Forces attribute}}

\begin{fulllineitems}
\phantomsection\label{\detokenize{autoapi/nexoclom2/initial_state/index:nexoclom2.initial_state.Forces.__name__}}
\pysigstartsignatures
\pysigline
{\sphinxbfcode{\sphinxupquote{\_\_name\_\_}}\sphinxbfcode{\sphinxupquote{\DUrole{w}{ }\DUrole{p}{=}\DUrole{w}{ }\textquotesingle{}Forces\textquotesingle{}}}}
\pysigstopsignatures
\end{fulllineitems}


\end{fulllineitems}

\index{LossInformation (class in nexoclom2.initial\_state)@\spxentry{LossInformation}\spxextra{class in nexoclom2.initial\_state}}

\begin{fulllineitems}
\phantomsection\label{\detokenize{autoapi/nexoclom2/initial_state/index:nexoclom2.initial_state.LossInformation}}
\pysigstartsignatures
\pysiglinewithargsret
{\sphinxbfcode{\sphinxupquote{class\DUrole{w}{ }}}\sphinxcode{\sphinxupquote{nexoclom2.initial\_state.}}\sphinxbfcode{\sphinxupquote{LossInformation}}}
{\sphinxparam{\DUrole{n}{lparam}\DUrole{p}{:}\DUrole{w}{ }\DUrole{n}{dict\DUrole{p}{,}\DUrole{w}{ }tinydb.table.Document}}\sphinxparamcomma \sphinxparam{\DUrole{n}{center}\DUrole{o}{=}\DUrole{default_value}{None}}\sphinxparamcomma \sphinxparam{\DUrole{n}{startpt}\DUrole{o}{=}\DUrole{default_value}{None}}}
{}
\pysigstopsignatures
\sphinxAtStartPar
Bases: {\hyperref[\detokenize{autoapi/nexoclom2/initial_state/InputClass/index:nexoclom2.initial_state.InputClass.InputClass}]{\sphinxcrossref{\sphinxcode{\sphinxupquote{nexoclom2.initial\_state.InputClass.InputClass}}}}}

\sphinxAtStartPar
Determines how loss should be calculated.

\sphinxAtStartPar
Configures the loss rates due to processes other than collisions with the
surface. See {\hyperref[\detokenize{nexoclom2/inputfiles:lossinfo}]{\sphinxcrossref{\DUrole{std}{\DUrole{std-ref}{Loss Information}}}}} for more information.
\begin{quote}\begin{description}
\sphinxlineitem{Parameters}\begin{description}
\sphinxlineitem{\sphinxstylestrong{lparam}}{[}dict{]}
\end{description}

\sphinxlineitem{Attributes}\begin{description}
\sphinxlineitem{\sphinxstylestrong{constant\_lifetime: astropy Time or False}}
\sphinxAtStartPar
Constant lifetime in seconds to use everywhere in the system.

\sphinxlineitem{\sphinxstylestrong{photoionization: bool}}
\sphinxAtStartPar
Determines whether to include photoionization.

\sphinxlineitem{\sphinxstylestrong{photoionization\_lifetime: astropy Time quantity}}
\sphinxAtStartPar
If 0, uses the measured photoionization rate. If \textgreater{}0, sets the
photoionization lifetime to that value.

\sphinxlineitem{\sphinxstylestrong{photo\_factor: float}}
\sphinxAtStartPar
Factor by which to modify photo\sphinxhyphen{}loss rate.

\sphinxlineitem{\sphinxstylestrong{electron\_impact: bool}}
\sphinxAtStartPar
Determines whether to include electron impact processes.

\sphinxlineitem{\sphinxstylestrong{eimp\_factor: float}}
\sphinxAtStartPar
Factor by which to modify electron impact rate.

\sphinxlineitem{\sphinxstylestrong{charge\_exchange: bool}}
\sphinxAtStartPar
Determines whether to include charge exchange.

\sphinxlineitem{\sphinxstylestrong{chx\_factor: float}}
\sphinxAtStartPar
Factor by which to modify charge exchange rate

\end{description}

\end{description}\end{quote}
\index{\_\_name\_\_ (nexoclom2.initial\_state.LossInformation attribute)@\spxentry{\_\_name\_\_}\spxextra{nexoclom2.initial\_state.LossInformation attribute}}

\begin{fulllineitems}
\phantomsection\label{\detokenize{autoapi/nexoclom2/initial_state/index:nexoclom2.initial_state.LossInformation.__name__}}
\pysigstartsignatures
\pysigline
{\sphinxbfcode{\sphinxupquote{\_\_name\_\_}}\sphinxbfcode{\sphinxupquote{\DUrole{w}{ }\DUrole{p}{=}\DUrole{w}{ }\textquotesingle{}LossInformation\textquotesingle{}}}}
\pysigstopsignatures
\end{fulllineitems}


\end{fulllineitems}

\index{Options (class in nexoclom2.initial\_state)@\spxentry{Options}\spxextra{class in nexoclom2.initial\_state}}

\begin{fulllineitems}
\phantomsection\label{\detokenize{autoapi/nexoclom2/initial_state/index:nexoclom2.initial_state.Options}}
\pysigstartsignatures
\pysiglinewithargsret
{\sphinxbfcode{\sphinxupquote{class\DUrole{w}{ }}}\sphinxcode{\sphinxupquote{nexoclom2.initial\_state.}}\sphinxbfcode{\sphinxupquote{Options}}}
{\sphinxparam{\DUrole{n}{options}}}
{}
\pysigstopsignatures
\sphinxAtStartPar
Bases: {\hyperref[\detokenize{autoapi/nexoclom2/initial_state/InputClass/index:nexoclom2.initial_state.InputClass.InputClass}]{\sphinxcrossref{\sphinxcode{\sphinxupquote{nexoclom2.initial\_state.InputClass.InputClass}}}}}

\sphinxAtStartPar
Sets general run options

\sphinxAtStartPar
See ref:\sphinxtitleref{options} for more information.
\begin{quote}\begin{description}
\sphinxlineitem{Parameters}\begin{description}
\sphinxlineitem{\sphinxstylestrong{options}}{[}dict, TinyDB Document{]}
\sphinxAtStartPar
Key, value for defining the options

\end{description}

\sphinxlineitem{Attributes}\begin{description}
\sphinxlineitem{\sphinxstylestrong{runtime}}{[}astropy quantity{]}
\sphinxAtStartPar
Total runtime for the simulation in seconds

\sphinxlineitem{\sphinxstylestrong{species}}{[}str{]}
\sphinxlineitem{\sphinxstylestrong{outer\_edge}}{[}float, Default = 10**30 (infinite){]}
\sphinxAtStartPar
Outer edge of simulation in central body radii

\sphinxlineitem{\sphinxstylestrong{step\_size}}{[}astropy quantity, Default = 0 sec{]}
\sphinxAtStartPar
Time step. If 0, uses adaptive step size integrator.

\sphinxlineitem{\sphinxstylestrong{resolution}}{[}float, Default = 10**\sphinxhyphen{}5{]}
\sphinxAtStartPar
Precision necessary in adaptive step size integrator. Not used with
constant step size integrator.

\sphinxlineitem{\sphinxstylestrong{start\_together}}{[}bool, optional, Default = False{]}
\sphinxAtStartPar
If step\_size = 0, tells integrator to start all packets at the same
time. Good for seeing cloud evolution and particle trajectories

\sphinxlineitem{\sphinxstylestrong{random\_seed}}{[}int \textgreater{}= 0, None, Default = None{]}
\sphinxAtStartPar
seed for random generator

\end{description}

\end{description}\end{quote}
\index{\_\_name\_\_ (nexoclom2.initial\_state.Options attribute)@\spxentry{\_\_name\_\_}\spxextra{nexoclom2.initial\_state.Options attribute}}

\begin{fulllineitems}
\phantomsection\label{\detokenize{autoapi/nexoclom2/initial_state/index:nexoclom2.initial_state.Options.__name__}}
\pysigstartsignatures
\pysigline
{\sphinxbfcode{\sphinxupquote{\_\_name\_\_}}\sphinxbfcode{\sphinxupquote{\DUrole{w}{ }\DUrole{p}{=}\DUrole{w}{ }\textquotesingle{}Options\textquotesingle{}}}}
\pysigstopsignatures
\end{fulllineitems}


\end{fulllineitems}


\sphinxstepscope


\paragraph{nexoclom2.math}
\label{\detokenize{autoapi/nexoclom2/math/index:module-nexoclom2.math}}\label{\detokenize{autoapi/nexoclom2/math/index:nexoclom2-math}}\label{\detokenize{autoapi/nexoclom2/math/index::doc}}\index{module@\spxentry{module}!nexoclom2.math@\spxentry{nexoclom2.math}}\index{nexoclom2.math@\spxentry{nexoclom2.math}!module@\spxentry{module}}
\sphinxAtStartPar
nexoclom.math package


\subparagraph{Submodules}
\label{\detokenize{autoapi/nexoclom2/math/index:submodules}}
\sphinxstepscope


\subparagraph{nexoclom2.math.histogram}
\label{\detokenize{autoapi/nexoclom2/math/histogram/index:module-nexoclom2.math.histogram}}\label{\detokenize{autoapi/nexoclom2/math/histogram/index:nexoclom2-math-histogram}}\label{\detokenize{autoapi/nexoclom2/math/histogram/index::doc}}\index{module@\spxentry{module}!nexoclom2.math.histogram@\spxentry{nexoclom2.math.histogram}}\index{nexoclom2.math.histogram@\spxentry{nexoclom2.math.histogram}!module@\spxentry{module}}
\sphinxAtStartPar
Wrapper classes for numpy.histogram and numpy.histogram2d


\subparagraph{Classes}
\label{\detokenize{autoapi/nexoclom2/math/histogram/index:classes}}

\begin{savenotes}\sphinxattablestart
\sphinxthistablewithglobalstyle
\sphinxthistablewithnovlinesstyle
\centering
\begin{tabulary}{\linewidth}[t]{\X{1}{2}\X{1}{2}}
\sphinxtoprule
\sphinxtableatstartofbodyhook
\sphinxAtStartPar
{\hyperref[\detokenize{autoapi/nexoclom2/math/histogram/index:nexoclom2.math.histogram.Histogram}]{\sphinxcrossref{\sphinxcode{\sphinxupquote{Histogram}}}}}
&
\sphinxAtStartPar
Wrapper for np.histogram that makes the x\sphinxhyphen{}axis the center of each bin.
\\
\sphinxhline
\sphinxAtStartPar
{\hyperref[\detokenize{autoapi/nexoclom2/math/histogram/index:nexoclom2.math.histogram.Histogram2d}]{\sphinxcrossref{\sphinxcode{\sphinxupquote{Histogram2d}}}}}
&
\sphinxAtStartPar
Wrapper for np.histogram2d that makes the x,y axes the centers of each bin.
\\
\sphinxbottomrule
\end{tabulary}
\sphinxtableafterendhook\par
\sphinxattableend\end{savenotes}


\subparagraph{Module Contents}
\label{\detokenize{autoapi/nexoclom2/math/histogram/index:module-contents}}\index{Histogram (class in nexoclom2.math.histogram)@\spxentry{Histogram}\spxextra{class in nexoclom2.math.histogram}}

\begin{fulllineitems}
\phantomsection\label{\detokenize{autoapi/nexoclom2/math/histogram/index:nexoclom2.math.histogram.Histogram}}
\pysigstartsignatures
\pysiglinewithargsret
{\sphinxbfcode{\sphinxupquote{class\DUrole{w}{ }}}\sphinxcode{\sphinxupquote{nexoclom2.math.histogram.}}\sphinxbfcode{\sphinxupquote{Histogram}}}
{\sphinxparam{\DUrole{n}{a}}\sphinxparamcomma \sphinxparam{\DUrole{n}{bins}\DUrole{o}{=}\DUrole{default_value}{10}}\sphinxparamcomma \sphinxparam{\DUrole{n}{range}\DUrole{o}{=}\DUrole{default_value}{None}}\sphinxparamcomma \sphinxparam{\DUrole{n}{weights}\DUrole{o}{=}\DUrole{default_value}{None}}\sphinxparamcomma \sphinxparam{\DUrole{n}{density}\DUrole{o}{=}\DUrole{default_value}{None}}}
{}
\pysigstopsignatures
\sphinxAtStartPar
Wrapper for np.histogram that makes the x\sphinxhyphen{}axis the center of each bin.
Returns a class with everything self\sphinxhyphen{}contained.
\index{histogram (nexoclom2.math.histogram.Histogram attribute)@\spxentry{histogram}\spxextra{nexoclom2.math.histogram.Histogram attribute}}

\begin{fulllineitems}
\phantomsection\label{\detokenize{autoapi/nexoclom2/math/histogram/index:nexoclom2.math.histogram.Histogram.histogram}}
\pysigstartsignatures
\pysigline
{\sphinxbfcode{\sphinxupquote{histogram}}}
\pysigstopsignatures
\end{fulllineitems}

\index{dx (nexoclom2.math.histogram.Histogram attribute)@\spxentry{dx}\spxextra{nexoclom2.math.histogram.Histogram attribute}}

\begin{fulllineitems}
\phantomsection\label{\detokenize{autoapi/nexoclom2/math/histogram/index:nexoclom2.math.histogram.Histogram.dx}}
\pysigstartsignatures
\pysigline
{\sphinxbfcode{\sphinxupquote{dx}}}
\pysigstopsignatures
\end{fulllineitems}

\index{x (nexoclom2.math.histogram.Histogram attribute)@\spxentry{x}\spxextra{nexoclom2.math.histogram.Histogram attribute}}

\begin{fulllineitems}
\phantomsection\label{\detokenize{autoapi/nexoclom2/math/histogram/index:nexoclom2.math.histogram.Histogram.x}}
\pysigstartsignatures
\pysigline
{\sphinxbfcode{\sphinxupquote{x}}}
\pysigstopsignatures
\end{fulllineitems}

\index{\_\_repr\_\_() (nexoclom2.math.histogram.Histogram method)@\spxentry{\_\_repr\_\_()}\spxextra{nexoclom2.math.histogram.Histogram method}}

\begin{fulllineitems}
\phantomsection\label{\detokenize{autoapi/nexoclom2/math/histogram/index:nexoclom2.math.histogram.Histogram.__repr__}}
\pysigstartsignatures
\pysiglinewithargsret
{\sphinxbfcode{\sphinxupquote{\_\_repr\_\_}}}
{}
{}
\pysigstopsignatures
\end{fulllineitems}

\index{\_\_str\_\_() (nexoclom2.math.histogram.Histogram method)@\spxentry{\_\_str\_\_()}\spxextra{nexoclom2.math.histogram.Histogram method}}

\begin{fulllineitems}
\phantomsection\label{\detokenize{autoapi/nexoclom2/math/histogram/index:nexoclom2.math.histogram.Histogram.__str__}}
\pysigstartsignatures
\pysiglinewithargsret
{\sphinxbfcode{\sphinxupquote{\_\_str\_\_}}}
{}
{}
\pysigstopsignatures
\end{fulllineitems}


\end{fulllineitems}

\index{Histogram2d (class in nexoclom2.math.histogram)@\spxentry{Histogram2d}\spxextra{class in nexoclom2.math.histogram}}

\begin{fulllineitems}
\phantomsection\label{\detokenize{autoapi/nexoclom2/math/histogram/index:nexoclom2.math.histogram.Histogram2d}}
\pysigstartsignatures
\pysiglinewithargsret
{\sphinxbfcode{\sphinxupquote{class\DUrole{w}{ }}}\sphinxcode{\sphinxupquote{nexoclom2.math.histogram.}}\sphinxbfcode{\sphinxupquote{Histogram2d}}}
{\sphinxparam{\DUrole{n}{ptsx}}\sphinxparamcomma \sphinxparam{\DUrole{n}{ptsy}}\sphinxparamcomma \sphinxparam{\DUrole{n}{bins}\DUrole{o}{=}\DUrole{default_value}{10}}\sphinxparamcomma \sphinxparam{\DUrole{n}{range}\DUrole{o}{=}\DUrole{default_value}{None}}\sphinxparamcomma \sphinxparam{\DUrole{n}{weights}\DUrole{o}{=}\DUrole{default_value}{None}}\sphinxparamcomma \sphinxparam{\DUrole{n}{density}\DUrole{o}{=}\DUrole{default_value}{None}}}
{}
\pysigstopsignatures
\sphinxAtStartPar
Wrapper for np.histogram2d that makes the x,y axes the centers of each bin.
Returns a class with everything self\sphinxhyphen{}contained.
\index{histogram (nexoclom2.math.histogram.Histogram2d attribute)@\spxentry{histogram}\spxextra{nexoclom2.math.histogram.Histogram2d attribute}}

\begin{fulllineitems}
\phantomsection\label{\detokenize{autoapi/nexoclom2/math/histogram/index:nexoclom2.math.histogram.Histogram2d.histogram}}
\pysigstartsignatures
\pysigline
{\sphinxbfcode{\sphinxupquote{histogram}}}
\pysigstopsignatures
\end{fulllineitems}

\index{x (nexoclom2.math.histogram.Histogram2d attribute)@\spxentry{x}\spxextra{nexoclom2.math.histogram.Histogram2d attribute}}

\begin{fulllineitems}
\phantomsection\label{\detokenize{autoapi/nexoclom2/math/histogram/index:nexoclom2.math.histogram.Histogram2d.x}}
\pysigstartsignatures
\pysigline
{\sphinxbfcode{\sphinxupquote{x}}}
\pysigstopsignatures
\end{fulllineitems}

\index{y (nexoclom2.math.histogram.Histogram2d attribute)@\spxentry{y}\spxextra{nexoclom2.math.histogram.Histogram2d attribute}}

\begin{fulllineitems}
\phantomsection\label{\detokenize{autoapi/nexoclom2/math/histogram/index:nexoclom2.math.histogram.Histogram2d.y}}
\pysigstartsignatures
\pysigline
{\sphinxbfcode{\sphinxupquote{y}}}
\pysigstopsignatures
\end{fulllineitems}


\end{fulllineitems}


\sphinxstepscope


\subparagraph{nexoclom2.math.ks\_test}
\label{\detokenize{autoapi/nexoclom2/math/ks_test/index:module-nexoclom2.math.ks_test}}\label{\detokenize{autoapi/nexoclom2/math/ks_test/index:nexoclom2-math-ks-test}}\label{\detokenize{autoapi/nexoclom2/math/ks_test/index::doc}}\index{module@\spxentry{module}!nexoclom2.math.ks\_test@\spxentry{nexoclom2.math.ks\_test}}\index{nexoclom2.math.ks\_test@\spxentry{nexoclom2.math.ks\_test}!module@\spxentry{module}}

\subparagraph{Functions}
\label{\detokenize{autoapi/nexoclom2/math/ks_test/index:functions}}

\begin{savenotes}\sphinxattablestart
\sphinxthistablewithglobalstyle
\sphinxthistablewithnovlinesstyle
\centering
\begin{tabulary}{\linewidth}[t]{\X{1}{2}\X{1}{2}}
\sphinxtoprule
\sphinxtableatstartofbodyhook
\sphinxAtStartPar
{\hyperref[\detokenize{autoapi/nexoclom2/math/ks_test/index:nexoclom2.math.ks_test.Q_KS}]{\sphinxcrossref{\sphinxcode{\sphinxupquote{Q\_KS}}}}}(z)
&
\sphinxAtStartPar

\\
\sphinxhline
\sphinxAtStartPar
{\hyperref[\detokenize{autoapi/nexoclom2/math/ks_test/index:nexoclom2.math.ks_test.ks_test}]{\sphinxcrossref{\sphinxcode{\sphinxupquote{ks\_test}}}}}(test\_data, speeddist)
&
\sphinxAtStartPar

\\
\sphinxhline
\sphinxAtStartPar
{\hyperref[\detokenize{autoapi/nexoclom2/math/ks_test/index:nexoclom2.math.ks_test.ks_d}]{\sphinxcrossref{\sphinxcode{\sphinxupquote{ks\_d}}}}}(test\_data, cdf)
&
\sphinxAtStartPar

\\
\sphinxbottomrule
\end{tabulary}
\sphinxtableafterendhook\par
\sphinxattableend\end{savenotes}


\subparagraph{Module Contents}
\label{\detokenize{autoapi/nexoclom2/math/ks_test/index:module-contents}}\index{Q\_KS() (in module nexoclom2.math.ks\_test)@\spxentry{Q\_KS()}\spxextra{in module nexoclom2.math.ks\_test}}

\begin{fulllineitems}
\phantomsection\label{\detokenize{autoapi/nexoclom2/math/ks_test/index:nexoclom2.math.ks_test.Q_KS}}
\pysigstartsignatures
\pysiglinewithargsret
{\sphinxcode{\sphinxupquote{nexoclom2.math.ks\_test.}}\sphinxbfcode{\sphinxupquote{Q\_KS}}}
{\sphinxparam{\DUrole{n}{z}}}
{}
\pysigstopsignatures
\end{fulllineitems}

\index{ks\_test() (in module nexoclom2.math.ks\_test)@\spxentry{ks\_test()}\spxextra{in module nexoclom2.math.ks\_test}}

\begin{fulllineitems}
\phantomsection\label{\detokenize{autoapi/nexoclom2/math/ks_test/index:nexoclom2.math.ks_test.ks_test}}
\pysigstartsignatures
\pysiglinewithargsret
{\sphinxcode{\sphinxupquote{nexoclom2.math.ks\_test.}}\sphinxbfcode{\sphinxupquote{ks\_test}}}
{\sphinxparam{\DUrole{n}{test\_data}}\sphinxparamcomma \sphinxparam{\DUrole{n}{speeddist}}}
{}
\pysigstopsignatures
\end{fulllineitems}

\index{ks\_d() (in module nexoclom2.math.ks\_test)@\spxentry{ks\_d()}\spxextra{in module nexoclom2.math.ks\_test}}

\begin{fulllineitems}
\phantomsection\label{\detokenize{autoapi/nexoclom2/math/ks_test/index:nexoclom2.math.ks_test.ks_d}}
\pysigstartsignatures
\pysiglinewithargsret
{\sphinxcode{\sphinxupquote{nexoclom2.math.ks\_test.}}\sphinxbfcode{\sphinxupquote{ks\_d}}}
{\sphinxparam{\DUrole{n}{test\_data}}\sphinxparamcomma \sphinxparam{\DUrole{n}{cdf}}}
{}
\pysigstopsignatures
\end{fulllineitems}


\sphinxstepscope


\subparagraph{nexoclom2.math.mod\_close}
\label{\detokenize{autoapi/nexoclom2/math/mod_close/index:module-nexoclom2.math.mod_close}}\label{\detokenize{autoapi/nexoclom2/math/mod_close/index:nexoclom2-math-mod-close}}\label{\detokenize{autoapi/nexoclom2/math/mod_close/index::doc}}\index{module@\spxentry{module}!nexoclom2.math.mod\_close@\spxentry{nexoclom2.math.mod\_close}}\index{nexoclom2.math.mod\_close@\spxentry{nexoclom2.math.mod\_close}!module@\spxentry{module}}
\sphinxAtStartPar
Determines whether two values are close in a modular system


\subparagraph{Functions}
\label{\detokenize{autoapi/nexoclom2/math/mod_close/index:functions}}

\begin{savenotes}\sphinxattablestart
\sphinxthistablewithglobalstyle
\sphinxthistablewithnovlinesstyle
\centering
\begin{tabulary}{\linewidth}[t]{\X{1}{2}\X{1}{2}}
\sphinxtoprule
\sphinxtableatstartofbodyhook
\sphinxAtStartPar
{\hyperref[\detokenize{autoapi/nexoclom2/math/mod_close/index:nexoclom2.math.mod_close.mod_close}]{\sphinxcrossref{\sphinxcode{\sphinxupquote{mod\_close}}}}}(a, b{[}, period, atol{]})
&
\sphinxAtStartPar
Wrapper for np.isclose for values close to the periodic boundary
\\
\sphinxbottomrule
\end{tabulary}
\sphinxtableafterendhook\par
\sphinxattableend\end{savenotes}


\subparagraph{Module Contents}
\label{\detokenize{autoapi/nexoclom2/math/mod_close/index:module-contents}}\index{mod\_close() (in module nexoclom2.math.mod\_close)@\spxentry{mod\_close()}\spxextra{in module nexoclom2.math.mod\_close}}

\begin{fulllineitems}
\phantomsection\label{\detokenize{autoapi/nexoclom2/math/mod_close/index:nexoclom2.math.mod_close.mod_close}}
\pysigstartsignatures
\pysiglinewithargsret
{\sphinxcode{\sphinxupquote{nexoclom2.math.mod\_close.}}\sphinxbfcode{\sphinxupquote{mod\_close}}}
{\sphinxparam{\DUrole{n}{a}}\sphinxparamcomma \sphinxparam{\DUrole{n}{b}}\sphinxparamcomma \sphinxparam{\DUrole{n}{period}\DUrole{o}{=}\DUrole{default_value}{2 * np.pi}}\sphinxparamcomma \sphinxparam{\DUrole{n}{atol}\DUrole{o}{=}\DUrole{default_value}{1e\sphinxhyphen{}08}}}
{}
\pysigstopsignatures
\sphinxAtStartPar
Wrapper for np.isclose for values close to the periodic boundary
Decision Chart


\begin{savenotes}\sphinxattablestart
\sphinxthistablewithglobalstyle
\centering
\begin{tabulary}{\linewidth}[t]{TTT}
\sphinxtoprule
\sphinxstyletheadfamily 
\sphinxAtStartPar
A
&\sphinxstyletheadfamily 
\sphinxAtStartPar
B
&\sphinxstyletheadfamily 
\sphinxAtStartPar
Comparison
\\
\sphinxmidrule
\sphinxtableatstartofbodyhook
\sphinxAtStartPar
low
&
\sphinxAtStartPar
low
&
\sphinxAtStartPar
\(|A - B| \leq \delta\)
\\
\sphinxhline
\sphinxAtStartPar
low
&
\sphinxAtStartPar
mid
&
\sphinxAtStartPar
\(|A - B| \leq \delta\)
\\
\sphinxhline
\sphinxAtStartPar
low
&
\sphinxAtStartPar
high
&
\sphinxAtStartPar
\(|(A + P) - B| \leq \delta\)
\\
\sphinxhline
\sphinxAtStartPar
mid
&
\sphinxAtStartPar
low
&
\sphinxAtStartPar
\(|A - B| \leq \delta\)
\\
\sphinxhline
\sphinxAtStartPar
mid
&
\sphinxAtStartPar
mid
&
\sphinxAtStartPar
\(|A - B| \leq \delta\)
\\
\sphinxhline
\sphinxAtStartPar
mid
&
\sphinxAtStartPar
high
&
\sphinxAtStartPar
\(|A - B| \leq \delta\)
\\
\sphinxhline
\sphinxAtStartPar
high
&
\sphinxAtStartPar
low
&
\sphinxAtStartPar
\(|A - (B - P)| \leq \delta\)
\\
\sphinxhline
\sphinxAtStartPar
high
&
\sphinxAtStartPar
mid
&
\sphinxAtStartPar
\(|A - B| \leq \delta\)
\\
\sphinxhline
\sphinxAtStartPar
high
&
\sphinxAtStartPar
high
&
\sphinxAtStartPar
\(|A - B| \leq \delta\)
\\
\sphinxbottomrule
\end{tabulary}
\sphinxtableafterendhook\par
\sphinxattableend\end{savenotes}

\sphinxAtStartPar
where low = \(x \leq \delta\), mid = \(\delta \leq x < P-\delta\) and
high = \(x \geq P - \delta\).
\begin{quote}\begin{description}
\sphinxlineitem{Parameters}\begin{itemize}
\item {} 
\sphinxAtStartPar
\sphinxstyleliteralstrong{\sphinxupquote{a}} (\sphinxstyleliteralemphasis{\sphinxupquote{int}}\sphinxstyleliteralemphasis{\sphinxupquote{, }}\sphinxstyleliteralemphasis{\sphinxupquote{float}}) \textendash{} Inputs to compare. Must be single valued.

\item {} 
\sphinxAtStartPar
\sphinxstyleliteralstrong{\sphinxupquote{b}} (\sphinxstyleliteralemphasis{\sphinxupquote{int}}\sphinxstyleliteralemphasis{\sphinxupquote{, }}\sphinxstyleliteralemphasis{\sphinxupquote{float}}) \textendash{} Inputs to compare. Must be single valued.

\item {} 
\sphinxAtStartPar
\sphinxstyleliteralstrong{\sphinxupquote{period}} (\sphinxstyleliteralemphasis{\sphinxupquote{float}}\sphinxstyleliteralemphasis{\sphinxupquote{, }}\sphinxstyleliteralemphasis{\sphinxupquote{Default = 2π}})

\item {} 
\sphinxAtStartPar
\sphinxstyleliteralstrong{\sphinxupquote{atol}} (\sphinxstyleliteralemphasis{\sphinxupquote{float}}) \textendash{} The absolute tolerance parameter

\end{itemize}

\sphinxlineitem{Returns}\begin{description}
\sphinxlineitem{Bool}
\end{description}

\end{description}\end{quote}
\subsubsection*{Notes}

\sphinxAtStartPar
Uses \sphinxhref{https://docutils.sourceforge.io/rst.html}{numpy.isclose()} with
default relative tolerance and abolute tolerance defined by \sphinxtitleref{atol} parameter.

\end{fulllineitems}


\sphinxstepscope


\subparagraph{nexoclom2.math.rotation\_matrix}
\label{\detokenize{autoapi/nexoclom2/math/rotation_matrix/index:module-nexoclom2.math.rotation_matrix}}\label{\detokenize{autoapi/nexoclom2/math/rotation_matrix/index:nexoclom2-math-rotation-matrix}}\label{\detokenize{autoapi/nexoclom2/math/rotation_matrix/index::doc}}\index{module@\spxentry{module}!nexoclom2.math.rotation\_matrix@\spxentry{nexoclom2.math.rotation\_matrix}}\index{nexoclom2.math.rotation\_matrix@\spxentry{nexoclom2.math.rotation\_matrix}!module@\spxentry{module}}

\subparagraph{Functions}
\label{\detokenize{autoapi/nexoclom2/math/rotation_matrix/index:functions}}

\begin{savenotes}\sphinxattablestart
\sphinxthistablewithglobalstyle
\sphinxthistablewithnovlinesstyle
\centering
\begin{tabulary}{\linewidth}[t]{\X{1}{2}\X{1}{2}}
\sphinxtoprule
\sphinxtableatstartofbodyhook
\sphinxAtStartPar
{\hyperref[\detokenize{autoapi/nexoclom2/math/rotation_matrix/index:nexoclom2.math.rotation_matrix.rotation_matrix}]{\sphinxcrossref{\sphinxcode{\sphinxupquote{rotation\_matrix}}}}}(theta, axis)
&
\sphinxAtStartPar
Compute the rotation matrix for a rotation of theta about axis.
\\
\sphinxbottomrule
\end{tabulary}
\sphinxtableafterendhook\par
\sphinxattableend\end{savenotes}


\subparagraph{Module Contents}
\label{\detokenize{autoapi/nexoclom2/math/rotation_matrix/index:module-contents}}\index{rotation\_matrix() (in module nexoclom2.math.rotation\_matrix)@\spxentry{rotation\_matrix()}\spxextra{in module nexoclom2.math.rotation\_matrix}}

\begin{fulllineitems}
\phantomsection\label{\detokenize{autoapi/nexoclom2/math/rotation_matrix/index:nexoclom2.math.rotation_matrix.rotation_matrix}}
\pysigstartsignatures
\pysiglinewithargsret
{\sphinxcode{\sphinxupquote{nexoclom2.math.rotation\_matrix.}}\sphinxbfcode{\sphinxupquote{rotation\_matrix}}}
{\sphinxparam{\DUrole{n}{theta}}\sphinxparamcomma \sphinxparam{\DUrole{n}{axis}}}
{}
\pysigstopsignatures
\sphinxAtStartPar
Compute the rotation matrix for a rotation of theta about axis.

\end{fulllineitems}



\subparagraph{Classes}
\label{\detokenize{autoapi/nexoclom2/math/index:classes}}

\begin{savenotes}\sphinxattablestart
\sphinxthistablewithglobalstyle
\sphinxthistablewithnovlinesstyle
\centering
\begin{tabulary}{\linewidth}[t]{\X{1}{2}\X{1}{2}}
\sphinxtoprule
\sphinxtableatstartofbodyhook
\sphinxAtStartPar
{\hyperref[\detokenize{autoapi/nexoclom2/math/index:nexoclom2.math.Histogram}]{\sphinxcrossref{\sphinxcode{\sphinxupquote{Histogram}}}}}
&
\sphinxAtStartPar
Wrapper for np.histogram that makes the x\sphinxhyphen{}axis the center of each bin.
\\
\sphinxhline
\sphinxAtStartPar
{\hyperref[\detokenize{autoapi/nexoclom2/math/index:nexoclom2.math.Histogram2d}]{\sphinxcrossref{\sphinxcode{\sphinxupquote{Histogram2d}}}}}
&
\sphinxAtStartPar
Wrapper for np.histogram2d that makes the x,y axes the centers of each bin.
\\
\sphinxbottomrule
\end{tabulary}
\sphinxtableafterendhook\par
\sphinxattableend\end{savenotes}


\subparagraph{Functions}
\label{\detokenize{autoapi/nexoclom2/math/index:functions}}

\begin{savenotes}\sphinxattablestart
\sphinxthistablewithglobalstyle
\sphinxthistablewithnovlinesstyle
\centering
\begin{tabulary}{\linewidth}[t]{\X{1}{2}\X{1}{2}}
\sphinxtoprule
\sphinxtableatstartofbodyhook
\sphinxAtStartPar
{\hyperref[\detokenize{autoapi/nexoclom2/math/rotation_matrix/index:module-nexoclom2.math.rotation_matrix}]{\sphinxcrossref{\sphinxcode{\sphinxupquote{rotation\_matrix}}}}}(theta, axis)
&
\sphinxAtStartPar
Compute the rotation matrix for a rotation of theta about axis.
\\
\sphinxbottomrule
\end{tabulary}
\sphinxtableafterendhook\par
\sphinxattableend\end{savenotes}


\subparagraph{Package Contents}
\label{\detokenize{autoapi/nexoclom2/math/index:package-contents}}\index{Histogram (class in nexoclom2.math)@\spxentry{Histogram}\spxextra{class in nexoclom2.math}}

\begin{fulllineitems}
\phantomsection\label{\detokenize{autoapi/nexoclom2/math/index:nexoclom2.math.Histogram}}
\pysigstartsignatures
\pysiglinewithargsret
{\sphinxbfcode{\sphinxupquote{class\DUrole{w}{ }}}\sphinxcode{\sphinxupquote{nexoclom2.math.}}\sphinxbfcode{\sphinxupquote{Histogram}}}
{\sphinxparam{\DUrole{n}{a}}\sphinxparamcomma \sphinxparam{\DUrole{n}{bins}\DUrole{o}{=}\DUrole{default_value}{10}}\sphinxparamcomma \sphinxparam{\DUrole{n}{range}\DUrole{o}{=}\DUrole{default_value}{None}}\sphinxparamcomma \sphinxparam{\DUrole{n}{weights}\DUrole{o}{=}\DUrole{default_value}{None}}\sphinxparamcomma \sphinxparam{\DUrole{n}{density}\DUrole{o}{=}\DUrole{default_value}{None}}}
{}
\pysigstopsignatures
\sphinxAtStartPar
Wrapper for np.histogram that makes the x\sphinxhyphen{}axis the center of each bin.
Returns a class with everything self\sphinxhyphen{}contained.
\index{histogram (nexoclom2.math.Histogram attribute)@\spxentry{histogram}\spxextra{nexoclom2.math.Histogram attribute}}

\begin{fulllineitems}
\phantomsection\label{\detokenize{autoapi/nexoclom2/math/index:nexoclom2.math.Histogram.histogram}}
\pysigstartsignatures
\pysigline
{\sphinxbfcode{\sphinxupquote{histogram}}}
\pysigstopsignatures
\end{fulllineitems}

\index{dx (nexoclom2.math.Histogram attribute)@\spxentry{dx}\spxextra{nexoclom2.math.Histogram attribute}}

\begin{fulllineitems}
\phantomsection\label{\detokenize{autoapi/nexoclom2/math/index:nexoclom2.math.Histogram.dx}}
\pysigstartsignatures
\pysigline
{\sphinxbfcode{\sphinxupquote{dx}}}
\pysigstopsignatures
\end{fulllineitems}

\index{x (nexoclom2.math.Histogram attribute)@\spxentry{x}\spxextra{nexoclom2.math.Histogram attribute}}

\begin{fulllineitems}
\phantomsection\label{\detokenize{autoapi/nexoclom2/math/index:nexoclom2.math.Histogram.x}}
\pysigstartsignatures
\pysigline
{\sphinxbfcode{\sphinxupquote{x}}}
\pysigstopsignatures
\end{fulllineitems}

\index{\_\_repr\_\_() (nexoclom2.math.Histogram method)@\spxentry{\_\_repr\_\_()}\spxextra{nexoclom2.math.Histogram method}}

\begin{fulllineitems}
\phantomsection\label{\detokenize{autoapi/nexoclom2/math/index:nexoclom2.math.Histogram.__repr__}}
\pysigstartsignatures
\pysiglinewithargsret
{\sphinxbfcode{\sphinxupquote{\_\_repr\_\_}}}
{}
{}
\pysigstopsignatures
\end{fulllineitems}

\index{\_\_str\_\_() (nexoclom2.math.Histogram method)@\spxentry{\_\_str\_\_()}\spxextra{nexoclom2.math.Histogram method}}

\begin{fulllineitems}
\phantomsection\label{\detokenize{autoapi/nexoclom2/math/index:nexoclom2.math.Histogram.__str__}}
\pysigstartsignatures
\pysiglinewithargsret
{\sphinxbfcode{\sphinxupquote{\_\_str\_\_}}}
{}
{}
\pysigstopsignatures
\end{fulllineitems}


\end{fulllineitems}

\index{Histogram2d (class in nexoclom2.math)@\spxentry{Histogram2d}\spxextra{class in nexoclom2.math}}

\begin{fulllineitems}
\phantomsection\label{\detokenize{autoapi/nexoclom2/math/index:nexoclom2.math.Histogram2d}}
\pysigstartsignatures
\pysiglinewithargsret
{\sphinxbfcode{\sphinxupquote{class\DUrole{w}{ }}}\sphinxcode{\sphinxupquote{nexoclom2.math.}}\sphinxbfcode{\sphinxupquote{Histogram2d}}}
{\sphinxparam{\DUrole{n}{ptsx}}\sphinxparamcomma \sphinxparam{\DUrole{n}{ptsy}}\sphinxparamcomma \sphinxparam{\DUrole{n}{bins}\DUrole{o}{=}\DUrole{default_value}{10}}\sphinxparamcomma \sphinxparam{\DUrole{n}{range}\DUrole{o}{=}\DUrole{default_value}{None}}\sphinxparamcomma \sphinxparam{\DUrole{n}{weights}\DUrole{o}{=}\DUrole{default_value}{None}}\sphinxparamcomma \sphinxparam{\DUrole{n}{density}\DUrole{o}{=}\DUrole{default_value}{None}}}
{}
\pysigstopsignatures
\sphinxAtStartPar
Wrapper for np.histogram2d that makes the x,y axes the centers of each bin.
Returns a class with everything self\sphinxhyphen{}contained.
\index{histogram (nexoclom2.math.Histogram2d attribute)@\spxentry{histogram}\spxextra{nexoclom2.math.Histogram2d attribute}}

\begin{fulllineitems}
\phantomsection\label{\detokenize{autoapi/nexoclom2/math/index:nexoclom2.math.Histogram2d.histogram}}
\pysigstartsignatures
\pysigline
{\sphinxbfcode{\sphinxupquote{histogram}}}
\pysigstopsignatures
\end{fulllineitems}

\index{x (nexoclom2.math.Histogram2d attribute)@\spxentry{x}\spxextra{nexoclom2.math.Histogram2d attribute}}

\begin{fulllineitems}
\phantomsection\label{\detokenize{autoapi/nexoclom2/math/index:nexoclom2.math.Histogram2d.x}}
\pysigstartsignatures
\pysigline
{\sphinxbfcode{\sphinxupquote{x}}}
\pysigstopsignatures
\end{fulllineitems}

\index{y (nexoclom2.math.Histogram2d attribute)@\spxentry{y}\spxextra{nexoclom2.math.Histogram2d attribute}}

\begin{fulllineitems}
\phantomsection\label{\detokenize{autoapi/nexoclom2/math/index:nexoclom2.math.Histogram2d.y}}
\pysigstartsignatures
\pysigline
{\sphinxbfcode{\sphinxupquote{y}}}
\pysigstopsignatures
\end{fulllineitems}


\end{fulllineitems}

\index{rotation\_matrix() (in module nexoclom2.math)@\spxentry{rotation\_matrix()}\spxextra{in module nexoclom2.math}}

\begin{fulllineitems}
\phantomsection\label{\detokenize{autoapi/nexoclom2/math/index:nexoclom2.math.rotation_matrix}}
\pysigstartsignatures
\pysiglinewithargsret
{\sphinxcode{\sphinxupquote{nexoclom2.math.}}\sphinxbfcode{\sphinxupquote{rotation\_matrix}}}
{\sphinxparam{\DUrole{n}{theta}}\sphinxparamcomma \sphinxparam{\DUrole{n}{axis}}}
{}
\pysigstopsignatures
\sphinxAtStartPar
Compute the rotation matrix for a rotation of theta about axis.

\end{fulllineitems}


\sphinxstepscope


\paragraph{nexoclom2.particle\_tracking}
\label{\detokenize{autoapi/nexoclom2/particle_tracking/index:module-nexoclom2.particle_tracking}}\label{\detokenize{autoapi/nexoclom2/particle_tracking/index:nexoclom2-particle-tracking}}\label{\detokenize{autoapi/nexoclom2/particle_tracking/index::doc}}\index{module@\spxentry{module}!nexoclom2.particle\_tracking@\spxentry{nexoclom2.particle\_tracking}}\index{nexoclom2.particle\_tracking@\spxentry{nexoclom2.particle\_tracking}!module@\spxentry{module}}
\sphinxAtStartPar
nexoclom.inputs package


\subparagraph{Submodules}
\label{\detokenize{autoapi/nexoclom2/particle_tracking/index:submodules}}
\sphinxstepscope


\subparagraph{nexoclom2.particle\_tracking.ConstantIntegrator}
\label{\detokenize{autoapi/nexoclom2/particle_tracking/ConstantIntegrator/index:module-nexoclom2.particle_tracking.ConstantIntegrator}}\label{\detokenize{autoapi/nexoclom2/particle_tracking/ConstantIntegrator/index:nexoclom2-particle-tracking-constantintegrator}}\label{\detokenize{autoapi/nexoclom2/particle_tracking/ConstantIntegrator/index::doc}}\index{module@\spxentry{module}!nexoclom2.particle\_tracking.ConstantIntegrator@\spxentry{nexoclom2.particle\_tracking.ConstantIntegrator}}\index{nexoclom2.particle\_tracking.ConstantIntegrator@\spxentry{nexoclom2.particle\_tracking.ConstantIntegrator}!module@\spxentry{module}}

\subparagraph{Classes}
\label{\detokenize{autoapi/nexoclom2/particle_tracking/ConstantIntegrator/index:classes}}

\begin{savenotes}\sphinxattablestart
\sphinxthistablewithglobalstyle
\sphinxthistablewithnovlinesstyle
\centering
\begin{tabulary}{\linewidth}[t]{\X{1}{2}\X{1}{2}}
\sphinxtoprule
\sphinxtableatstartofbodyhook
\sphinxAtStartPar
{\hyperref[\detokenize{autoapi/nexoclom2/particle_tracking/ConstantIntegrator/index:nexoclom2.particle_tracking.ConstantIntegrator.ConstantIntegrator}]{\sphinxcrossref{\sphinxcode{\sphinxupquote{ConstantIntegrator}}}}}
&
\sphinxAtStartPar
Constant step size integrator
\\
\sphinxbottomrule
\end{tabulary}
\sphinxtableafterendhook\par
\sphinxattableend\end{savenotes}


\subparagraph{Module Contents}
\label{\detokenize{autoapi/nexoclom2/particle_tracking/ConstantIntegrator/index:module-contents}}\index{ConstantIntegrator (class in nexoclom2.particle\_tracking.ConstantIntegrator)@\spxentry{ConstantIntegrator}\spxextra{class in nexoclom2.particle\_tracking.ConstantIntegrator}}

\begin{fulllineitems}
\phantomsection\label{\detokenize{autoapi/nexoclom2/particle_tracking/ConstantIntegrator/index:nexoclom2.particle_tracking.ConstantIntegrator.ConstantIntegrator}}
\pysigstartsignatures
\pysiglinewithargsret
{\sphinxbfcode{\sphinxupquote{class\DUrole{w}{ }}}\sphinxcode{\sphinxupquote{nexoclom2.particle\_tracking.ConstantIntegrator.}}\sphinxbfcode{\sphinxupquote{ConstantIntegrator}}}
{\sphinxparam{\DUrole{n}{output}}\sphinxparamcomma \sphinxparam{\DUrole{n}{state}}\sphinxparamcomma \sphinxparam{\DUrole{n}{method}\DUrole{o}{=}\DUrole{default_value}{\textquotesingle{}rk5\textquotesingle{}}}}
{}
\pysigstopsignatures
\sphinxAtStartPar
Constant step size integrator
:param output: nexoclom2 Output clas
:type output: Output
\begin{quote}\begin{description}
\sphinxlineitem{Returns}\begin{description}
\sphinxlineitem{Final state of the system}
\end{description}

\end{description}\end{quote}

\end{fulllineitems}


\sphinxstepscope


\subparagraph{nexoclom2.particle\_tracking.Output}
\label{\detokenize{autoapi/nexoclom2/particle_tracking/Output/index:module-nexoclom2.particle_tracking.Output}}\label{\detokenize{autoapi/nexoclom2/particle_tracking/Output/index:nexoclom2-particle-tracking-output}}\label{\detokenize{autoapi/nexoclom2/particle_tracking/Output/index::doc}}\index{module@\spxentry{module}!nexoclom2.particle\_tracking.Output@\spxentry{nexoclom2.particle\_tracking.Output}}\index{nexoclom2.particle\_tracking.Output@\spxentry{nexoclom2.particle\_tracking.Output}!module@\spxentry{module}}

\subparagraph{Classes}
\label{\detokenize{autoapi/nexoclom2/particle_tracking/Output/index:classes}}

\begin{savenotes}\sphinxattablestart
\sphinxthistablewithglobalstyle
\sphinxthistablewithnovlinesstyle
\centering
\begin{tabulary}{\linewidth}[t]{\X{1}{2}\X{1}{2}}
\sphinxtoprule
\sphinxtableatstartofbodyhook
\sphinxAtStartPar
{\hyperref[\detokenize{autoapi/nexoclom2/particle_tracking/Output/index:nexoclom2.particle_tracking.Output.Output}]{\sphinxcrossref{\sphinxcode{\sphinxupquote{Output}}}}}
&
\sphinxAtStartPar
Class to store compute particle trajectories and store the results.
\\
\sphinxbottomrule
\end{tabulary}
\sphinxtableafterendhook\par
\sphinxattableend\end{savenotes}


\subparagraph{Module Contents}
\label{\detokenize{autoapi/nexoclom2/particle_tracking/Output/index:module-contents}}\index{Output (class in nexoclom2.particle\_tracking.Output)@\spxentry{Output}\spxextra{class in nexoclom2.particle\_tracking.Output}}

\begin{fulllineitems}
\phantomsection\label{\detokenize{autoapi/nexoclom2/particle_tracking/Output/index:nexoclom2.particle_tracking.Output.Output}}
\pysigstartsignatures
\pysiglinewithargsret
{\sphinxbfcode{\sphinxupquote{class\DUrole{w}{ }}}\sphinxcode{\sphinxupquote{nexoclom2.particle\_tracking.Output.}}\sphinxbfcode{\sphinxupquote{Output}}}
{\sphinxparam{\DUrole{n}{inputs}}\sphinxparamcomma \sphinxparam{\DUrole{n}{n\_packets}\DUrole{o}{=}\DUrole{default_value}{0}}\sphinxparamcomma \sphinxparam{\DUrole{n}{n\_iterations}\DUrole{o}{=}\DUrole{default_value}{1}}\sphinxparamcomma \sphinxparam{\DUrole{n}{compress}\DUrole{o}{=}\DUrole{default_value}{True}}\sphinxparamcomma \sphinxparam{\DUrole{n}{overwrite}\DUrole{o}{=}\DUrole{default_value}{False}}}
{}
\pysigstopsignatures
\sphinxAtStartPar
Class to store compute particle trajectories and store the results.
\begin{quote}\begin{description}
\sphinxlineitem{Parameters}\begin{description}
\sphinxlineitem{\sphinxstylestrong{inputs}}{[}Input{]}
\sphinxlineitem{\sphinxstylestrong{n\_packets}}{[}int{]}
\sphinxlineitem{\sphinxstylestrong{compress}}{[}bool, Default=True{]}
\end{description}

\sphinxlineitem{Attributes}\begin{description}
\sphinxlineitem{\sphinxstylestrong{inputs: Input}}
\sphinxAtStartPar
The inputs used in this model run.

\sphinxlineitem{\sphinxstylestrong{n\_packets: int, float}}
\sphinxAtStartPar
Total number of packets to run

\sphinxlineitem{\sphinxstylestrong{compress: Bool}}
\sphinxAtStartPar
If True removes packets with frac=0 from the saved output, Default = True

\sphinxlineitem{\sphinxstylestrong{starting\_point: ndarray}}
\sphinxAtStartPar
Initial state relative to startpoint with standard units. Columns are
time (s), x (km), y (km), z (km), r (km), vx (km/s), vy (km/s),
(km/s), v (km/s), frac, longitude (rad), latitude (rad),
local\_time (hr), altitude (rad), azimuth (rad)

\sphinxlineitem{\sphinxstylestrong{final\_state: ndarray}}
\end{description}

\end{description}\end{quote}
\subsubsection*{Notes}

\sphinxAtStartPar
The user will not generally call this directly but will instead use
\sphinxcode{\sphinxupquote{inputs.run()}}.

\sphinxAtStartPar
All calculations will be done in units of s, km, km/s, rad and converted
to more appropriate units at the end.
\index{inputs (nexoclom2.particle\_tracking.Output.Output attribute)@\spxentry{inputs}\spxextra{nexoclom2.particle\_tracking.Output.Output attribute}}

\begin{fulllineitems}
\phantomsection\label{\detokenize{autoapi/nexoclom2/particle_tracking/Output/index:nexoclom2.particle_tracking.Output.Output.inputs}}
\pysigstartsignatures
\pysigline
{\sphinxbfcode{\sphinxupquote{inputs}}}
\pysigstopsignatures
\end{fulllineitems}

\index{compress (nexoclom2.particle\_tracking.Output.Output attribute)@\spxentry{compress}\spxextra{nexoclom2.particle\_tracking.Output.Output attribute}}

\begin{fulllineitems}
\phantomsection\label{\detokenize{autoapi/nexoclom2/particle_tracking/Output/index:nexoclom2.particle_tracking.Output.Output.compress}}
\pysigstartsignatures
\pysigline
{\sphinxbfcode{\sphinxupquote{compress}}\sphinxbfcode{\sphinxupquote{\DUrole{w}{ }\DUrole{p}{=}\DUrole{w}{ }True}}}
\pysigstopsignatures
\end{fulllineitems}

\index{randgen (nexoclom2.particle\_tracking.Output.Output attribute)@\spxentry{randgen}\spxextra{nexoclom2.particle\_tracking.Output.Output attribute}}

\begin{fulllineitems}
\phantomsection\label{\detokenize{autoapi/nexoclom2/particle_tracking/Output/index:nexoclom2.particle_tracking.Output.Output.randgen}}
\pysigstartsignatures
\pysigline
{\sphinxbfcode{\sphinxupquote{randgen}}}
\pysigstopsignatures
\end{fulllineitems}

\index{center (nexoclom2.particle\_tracking.Output.Output attribute)@\spxentry{center}\spxextra{nexoclom2.particle\_tracking.Output.Output attribute}}

\begin{fulllineitems}
\phantomsection\label{\detokenize{autoapi/nexoclom2/particle_tracking/Output/index:nexoclom2.particle_tracking.Output.Output.center}}
\pysigstartsignatures
\pysigline
{\sphinxbfcode{\sphinxupquote{center}}}
\pysigstopsignatures
\end{fulllineitems}

\index{startpoint (nexoclom2.particle\_tracking.Output.Output attribute)@\spxentry{startpoint}\spxextra{nexoclom2.particle\_tracking.Output.Output attribute}}

\begin{fulllineitems}
\phantomsection\label{\detokenize{autoapi/nexoclom2/particle_tracking/Output/index:nexoclom2.particle_tracking.Output.Output.startpoint}}
\pysigstartsignatures
\pysigline
{\sphinxbfcode{\sphinxupquote{startpoint}}}
\pysigstopsignatures
\end{fulllineitems}

\index{objects (nexoclom2.particle\_tracking.Output.Output attribute)@\spxentry{objects}\spxextra{nexoclom2.particle\_tracking.Output.Output attribute}}

\begin{fulllineitems}
\phantomsection\label{\detokenize{autoapi/nexoclom2/particle_tracking/Output/index:nexoclom2.particle_tracking.Output.Output.objects}}
\pysigstartsignatures
\pysigline
{\sphinxbfcode{\sphinxupquote{objects}}}
\pysigstopsignatures
\end{fulllineitems}

\index{positions (nexoclom2.particle\_tracking.Output.Output attribute)@\spxentry{positions}\spxextra{nexoclom2.particle\_tracking.Output.Output attribute}}

\begin{fulllineitems}
\phantomsection\label{\detokenize{autoapi/nexoclom2/particle_tracking/Output/index:nexoclom2.particle_tracking.Output.Output.positions}}
\pysigstartsignatures
\pysigline
{\sphinxbfcode{\sphinxupquote{positions}}}
\pysigstopsignatures
\end{fulllineitems}

\index{unit (nexoclom2.particle\_tracking.Output.Output attribute)@\spxentry{unit}\spxextra{nexoclom2.particle\_tracking.Output.Output attribute}}

\begin{fulllineitems}
\phantomsection\label{\detokenize{autoapi/nexoclom2/particle_tracking/Output/index:nexoclom2.particle_tracking.Output.Output.unit}}
\pysigstartsignatures
\pysigline
{\sphinxbfcode{\sphinxupquote{unit}}\sphinxbfcode{\sphinxupquote{\DUrole{w}{ }\DUrole{p}{=}\DUrole{w}{ }None}}}
\pysigstopsignatures
\end{fulllineitems}

\index{species (nexoclom2.particle\_tracking.Output.Output attribute)@\spxentry{species}\spxextra{nexoclom2.particle\_tracking.Output.Output attribute}}

\begin{fulllineitems}
\phantomsection\label{\detokenize{autoapi/nexoclom2/particle_tracking/Output/index:nexoclom2.particle_tracking.Output.Output.species}}
\pysigstartsignatures
\pysigline
{\sphinxbfcode{\sphinxupquote{species}}}
\pysigstopsignatures
\end{fulllineitems}

\index{initialize\_objects() (nexoclom2.particle\_tracking.Output.Output method)@\spxentry{initialize\_objects()}\spxextra{nexoclom2.particle\_tracking.Output.Output method}}

\begin{fulllineitems}
\phantomsection\label{\detokenize{autoapi/nexoclom2/particle_tracking/Output/index:nexoclom2.particle_tracking.Output.Output.initialize_objects}}
\pysigstartsignatures
\pysiglinewithargsret
{\sphinxbfcode{\sphinxupquote{initialize\_objects}}}
{}
{}
\pysigstopsignatures
\end{fulllineitems}

\index{\_remove() (nexoclom2.particle\_tracking.Output.Output method)@\spxentry{\_remove()}\spxextra{nexoclom2.particle\_tracking.Output.Output method}}

\begin{fulllineitems}
\phantomsection\label{\detokenize{autoapi/nexoclom2/particle_tracking/Output/index:nexoclom2.particle_tracking.Output.Output._remove}}
\pysigstartsignatures
\pysiglinewithargsret
{\sphinxbfcode{\sphinxupquote{\_remove}}}
{}
{}
\pysigstopsignatures
\end{fulllineitems}

\index{\_start\_outputfile() (nexoclom2.particle\_tracking.Output.Output method)@\spxentry{\_start\_outputfile()}\spxextra{nexoclom2.particle\_tracking.Output.Output method}}

\begin{fulllineitems}
\phantomsection\label{\detokenize{autoapi/nexoclom2/particle_tracking/Output/index:nexoclom2.particle_tracking.Output.Output._start_outputfile}}
\pysigstartsignatures
\pysiglinewithargsret
{\sphinxbfcode{\sphinxupquote{\_start\_outputfile}}}
{}
{}
\pysigstopsignatures
\end{fulllineitems}

\index{\_save\_starting\_point() (nexoclom2.particle\_tracking.Output.Output method)@\spxentry{\_save\_starting\_point()}\spxextra{nexoclom2.particle\_tracking.Output.Output method}}

\begin{fulllineitems}
\phantomsection\label{\detokenize{autoapi/nexoclom2/particle_tracking/Output/index:nexoclom2.particle_tracking.Output.Output._save_starting_point}}
\pysigstartsignatures
\pysiglinewithargsret
{\sphinxbfcode{\sphinxupquote{\_save\_starting\_point}}}
{\sphinxparam{\DUrole{n}{starting\_point}}}
{}
\pysigstopsignatures
\end{fulllineitems}

\index{save\_final\_state() (nexoclom2.particle\_tracking.Output.Output method)@\spxentry{save\_final\_state()}\spxextra{nexoclom2.particle\_tracking.Output.Output method}}

\begin{fulllineitems}
\phantomsection\label{\detokenize{autoapi/nexoclom2/particle_tracking/Output/index:nexoclom2.particle_tracking.Output.Output.save_final_state}}
\pysigstartsignatures
\pysiglinewithargsret
{\sphinxbfcode{\sphinxupquote{save\_final\_state}}}
{\sphinxparam{\DUrole{n}{final\_state}}}
{}
\pysigstopsignatures
\end{fulllineitems}

\index{starting\_point() (nexoclom2.particle\_tracking.Output.Output method)@\spxentry{starting\_point()}\spxextra{nexoclom2.particle\_tracking.Output.Output method}}

\begin{fulllineitems}
\phantomsection\label{\detokenize{autoapi/nexoclom2/particle_tracking/Output/index:nexoclom2.particle_tracking.Output.Output.starting_point}}
\pysigstartsignatures
\pysiglinewithargsret
{\sphinxbfcode{\sphinxupquote{starting\_point}}}
{\sphinxparam{\DUrole{n}{iteration}\DUrole{o}{=}\DUrole{default_value}{None}}\sphinxparamcomma \sphinxparam{\DUrole{n}{n\_packets}\DUrole{o}{=}\DUrole{default_value}{None}}}
{}
\pysigstopsignatures\begin{quote}\begin{description}
\sphinxlineitem{Parameters}\begin{description}
\sphinxlineitem{\sphinxstylestrong{iteration}}
\sphinxlineitem{\sphinxstylestrong{n\_packets}}
\end{description}

\sphinxlineitem{Returns}\begin{description}
\sphinxlineitem{StartingPoint}
\end{description}

\end{description}\end{quote}
\subsubsection*{Notes}

\sphinxAtStartPar
If iteration and n\_packets are both given, iteration number takes
precedence.

\end{fulllineitems}

\index{initial\_state() (nexoclom2.particle\_tracking.Output.Output method)@\spxentry{initial\_state()}\spxextra{nexoclom2.particle\_tracking.Output.Output method}}

\begin{fulllineitems}
\phantomsection\label{\detokenize{autoapi/nexoclom2/particle_tracking/Output/index:nexoclom2.particle_tracking.Output.Output.initial_state}}
\pysigstartsignatures
\pysiglinewithargsret
{\sphinxbfcode{\sphinxupquote{initial\_state}}}
{\sphinxparam{\DUrole{n}{iteration}\DUrole{o}{=}\DUrole{default_value}{None}}\sphinxparamcomma \sphinxparam{\DUrole{n}{n\_packets}\DUrole{o}{=}\DUrole{default_value}{None}}}
{}
\pysigstopsignatures
\end{fulllineitems}

\index{to\_planet\_coords() (nexoclom2.particle\_tracking.Output.Output method)@\spxentry{to\_planet\_coords()}\spxextra{nexoclom2.particle\_tracking.Output.Output method}}

\begin{fulllineitems}
\phantomsection\label{\detokenize{autoapi/nexoclom2/particle_tracking/Output/index:nexoclom2.particle_tracking.Output.Output.to_planet_coords}}
\pysigstartsignatures
\pysiglinewithargsret
{\sphinxbfcode{\sphinxupquote{to\_planet\_coords}}}
{\sphinxparam{\DUrole{n}{time}}\sphinxparamcomma \sphinxparam{\DUrole{n}{X}}\sphinxparamcomma \sphinxparam{\DUrole{n}{V}}}
{}
\pysigstopsignatures
\end{fulllineitems}

\index{final\_state() (nexoclom2.particle\_tracking.Output.Output method)@\spxentry{final\_state()}\spxextra{nexoclom2.particle\_tracking.Output.Output method}}

\begin{fulllineitems}
\phantomsection\label{\detokenize{autoapi/nexoclom2/particle_tracking/Output/index:nexoclom2.particle_tracking.Output.Output.final_state}}
\pysigstartsignatures
\pysiglinewithargsret
{\sphinxbfcode{\sphinxupquote{final\_state}}}
{\sphinxparam{\DUrole{n}{which}\DUrole{o}{=}\DUrole{default_value}{None}}\sphinxparamcomma \sphinxparam{\DUrole{n}{to\_planet\_coords}\DUrole{o}{=}\DUrole{default_value}{False}}}
{}
\pysigstopsignatures
\end{fulllineitems}


\end{fulllineitems}


\sphinxstepscope


\subparagraph{nexoclom2.particle\_tracking.VariableIntegrator}
\label{\detokenize{autoapi/nexoclom2/particle_tracking/VariableIntegrator/index:module-nexoclom2.particle_tracking.VariableIntegrator}}\label{\detokenize{autoapi/nexoclom2/particle_tracking/VariableIntegrator/index:nexoclom2-particle-tracking-variableintegrator}}\label{\detokenize{autoapi/nexoclom2/particle_tracking/VariableIntegrator/index::doc}}\index{module@\spxentry{module}!nexoclom2.particle\_tracking.VariableIntegrator@\spxentry{nexoclom2.particle\_tracking.VariableIntegrator}}\index{nexoclom2.particle\_tracking.VariableIntegrator@\spxentry{nexoclom2.particle\_tracking.VariableIntegrator}!module@\spxentry{module}}

\subparagraph{Classes}
\label{\detokenize{autoapi/nexoclom2/particle_tracking/VariableIntegrator/index:classes}}

\begin{savenotes}\sphinxattablestart
\sphinxthistablewithglobalstyle
\sphinxthistablewithnovlinesstyle
\centering
\begin{tabulary}{\linewidth}[t]{\X{1}{2}\X{1}{2}}
\sphinxtoprule
\sphinxtableatstartofbodyhook
\sphinxAtStartPar
{\hyperref[\detokenize{autoapi/nexoclom2/particle_tracking/VariableIntegrator/index:nexoclom2.particle_tracking.VariableIntegrator.VariableIntegrator}]{\sphinxcrossref{\sphinxcode{\sphinxupquote{VariableIntegrator}}}}}
&
\sphinxAtStartPar
Runge Kutta integrator
\\
\sphinxbottomrule
\end{tabulary}
\sphinxtableafterendhook\par
\sphinxattableend\end{savenotes}


\subparagraph{Module Contents}
\label{\detokenize{autoapi/nexoclom2/particle_tracking/VariableIntegrator/index:module-contents}}\index{VariableIntegrator (class in nexoclom2.particle\_tracking.VariableIntegrator)@\spxentry{VariableIntegrator}\spxextra{class in nexoclom2.particle\_tracking.VariableIntegrator}}

\begin{fulllineitems}
\phantomsection\label{\detokenize{autoapi/nexoclom2/particle_tracking/VariableIntegrator/index:nexoclom2.particle_tracking.VariableIntegrator.VariableIntegrator}}
\pysigstartsignatures
\pysiglinewithargsret
{\sphinxbfcode{\sphinxupquote{class\DUrole{w}{ }}}\sphinxcode{\sphinxupquote{nexoclom2.particle\_tracking.VariableIntegrator.}}\sphinxbfcode{\sphinxupquote{VariableIntegrator}}}
{\sphinxparam{\DUrole{n}{output}}\sphinxparamcomma \sphinxparam{\DUrole{n}{state}}\sphinxparamcomma \sphinxparam{\DUrole{n}{method}\DUrole{o}{=}\DUrole{default_value}{\textquotesingle{}rk5\textquotesingle{}}}}
{}
\pysigstopsignatures
\sphinxAtStartPar
Runge Kutta integrator

\sphinxAtStartPar
Can run with either constant or variable step size.
\begin{quote}\begin{description}
\sphinxlineitem{Parameters}\begin{description}
\sphinxlineitem{\sphinxstylestrong{output}}{[}Output{]}
\sphinxAtStartPar
nexoclom2 Output clas

\end{description}

\sphinxlineitem{Returns}\begin{description}
\sphinxlineitem{Final state of the system}
\end{description}

\end{description}\end{quote}

\end{fulllineitems}


\sphinxstepscope


\subparagraph{nexoclom2.particle\_tracking.compute\_accel}
\label{\detokenize{autoapi/nexoclom2/particle_tracking/compute_accel/index:module-nexoclom2.particle_tracking.compute_accel}}\label{\detokenize{autoapi/nexoclom2/particle_tracking/compute_accel/index:nexoclom2-particle-tracking-compute-accel}}\label{\detokenize{autoapi/nexoclom2/particle_tracking/compute_accel/index::doc}}\index{module@\spxentry{module}!nexoclom2.particle\_tracking.compute\_accel@\spxentry{nexoclom2.particle\_tracking.compute\_accel}}\index{nexoclom2.particle\_tracking.compute\_accel@\spxentry{nexoclom2.particle\_tracking.compute\_accel}!module@\spxentry{module}}

\subparagraph{Functions}
\label{\detokenize{autoapi/nexoclom2/particle_tracking/compute_accel/index:functions}}

\begin{savenotes}\sphinxattablestart
\sphinxthistablewithglobalstyle
\sphinxthistablewithnovlinesstyle
\centering
\begin{tabulary}{\linewidth}[t]{\X{1}{2}\X{1}{2}}
\sphinxtoprule
\sphinxtableatstartofbodyhook
\sphinxAtStartPar
{\hyperref[\detokenize{autoapi/nexoclom2/particle_tracking/compute_accel/index:nexoclom2.particle_tracking.compute_accel.compute_accel}]{\sphinxcrossref{\sphinxcode{\sphinxupquote{compute\_accel}}}}}(packets, output)
&
\sphinxAtStartPar
Compute acceleration due to gravity and radiation pressure
\\
\sphinxbottomrule
\end{tabulary}
\sphinxtableafterendhook\par
\sphinxattableend\end{savenotes}


\subparagraph{Module Contents}
\label{\detokenize{autoapi/nexoclom2/particle_tracking/compute_accel/index:module-contents}}\index{compute\_accel() (in module nexoclom2.particle\_tracking.compute\_accel)@\spxentry{compute\_accel()}\spxextra{in module nexoclom2.particle\_tracking.compute\_accel}}

\begin{fulllineitems}
\phantomsection\label{\detokenize{autoapi/nexoclom2/particle_tracking/compute_accel/index:nexoclom2.particle_tracking.compute_accel.compute_accel}}
\pysigstartsignatures
\pysiglinewithargsret
{\sphinxcode{\sphinxupquote{nexoclom2.particle\_tracking.compute\_accel.}}\sphinxbfcode{\sphinxupquote{compute\_accel}}}
{\sphinxparam{\DUrole{n}{packets}}\sphinxparamcomma \sphinxparam{\DUrole{n}{output}}}
{}
\pysigstopsignatures
\sphinxAtStartPar
Compute acceleration due to gravity and radiation pressure
\begin{quote}\begin{description}
\sphinxlineitem{Parameters}\begin{description}
\sphinxlineitem{\sphinxstylestrong{packets: nexoclom2 Packets object}}
\sphinxlineitem{\sphinxstylestrong{output: nexoclom2 Output object}}
\end{description}

\sphinxlineitem{Returns}\begin{description}
\sphinxlineitem{Numpy array with 3 components of the acceleration.}
\end{description}

\end{description}\end{quote}


\begin{sphinxseealso}{See also:}
\begin{description}
\sphinxlineitem{{\hyperref[\detokenize{autoapi/nexoclom2/particle_tracking/packets/index:nexoclom2.particle_tracking.packets.Packets}]{\sphinxcrossref{\sphinxcode{\sphinxupquote{nexoclom2.particle\_tracking.packets.Packets}}}}}}
\sphinxlineitem{{\hyperref[\detokenize{autoapi/nexoclom2/particle_tracking/Output/index:nexoclom2.particle_tracking.Output.Output}]{\sphinxcrossref{\sphinxcode{\sphinxupquote{nexoclom2.particle\_tracking.Output.Output}}}}}}
\end{description}


\end{sphinxseealso}


\end{fulllineitems}


\sphinxstepscope


\subparagraph{nexoclom2.particle\_tracking.final\_state}
\label{\detokenize{autoapi/nexoclom2/particle_tracking/final_state/index:module-nexoclom2.particle_tracking.final_state}}\label{\detokenize{autoapi/nexoclom2/particle_tracking/final_state/index:nexoclom2-particle-tracking-final-state}}\label{\detokenize{autoapi/nexoclom2/particle_tracking/final_state/index::doc}}\index{module@\spxentry{module}!nexoclom2.particle\_tracking.final\_state@\spxentry{nexoclom2.particle\_tracking.final\_state}}\index{nexoclom2.particle\_tracking.final\_state@\spxentry{nexoclom2.particle\_tracking.final\_state}!module@\spxentry{module}}

\subparagraph{Classes}
\label{\detokenize{autoapi/nexoclom2/particle_tracking/final_state/index:classes}}

\begin{savenotes}\sphinxattablestart
\sphinxthistablewithglobalstyle
\sphinxthistablewithnovlinesstyle
\centering
\begin{tabulary}{\linewidth}[t]{\X{1}{2}\X{1}{2}}
\sphinxtoprule
\sphinxtableatstartofbodyhook
\sphinxAtStartPar
{\hyperref[\detokenize{autoapi/nexoclom2/particle_tracking/final_state/index:nexoclom2.particle_tracking.final_state.FinalState}]{\sphinxcrossref{\sphinxcode{\sphinxupquote{FinalState}}}}}
&
\sphinxAtStartPar

\\
\sphinxbottomrule
\end{tabulary}
\sphinxtableafterendhook\par
\sphinxattableend\end{savenotes}


\subparagraph{Module Contents}
\label{\detokenize{autoapi/nexoclom2/particle_tracking/final_state/index:module-contents}}\index{FinalState (class in nexoclom2.particle\_tracking.final\_state)@\spxentry{FinalState}\spxextra{class in nexoclom2.particle\_tracking.final\_state}}

\begin{fulllineitems}
\phantomsection\label{\detokenize{autoapi/nexoclom2/particle_tracking/final_state/index:nexoclom2.particle_tracking.final_state.FinalState}}
\pysigstartsignatures
\pysiglinewithargsret
{\sphinxbfcode{\sphinxupquote{class\DUrole{w}{ }}}\sphinxcode{\sphinxupquote{nexoclom2.particle\_tracking.final\_state.}}\sphinxbfcode{\sphinxupquote{FinalState}}}
{\sphinxparam{\DUrole{n}{output}}\sphinxparamcomma \sphinxparam{\DUrole{n}{which}\DUrole{o}{=}\DUrole{default_value}{None}}}
{}
\pysigstopsignatures\index{\_\_getitem\_\_() (nexoclom2.particle\_tracking.final\_state.FinalState method)@\spxentry{\_\_getitem\_\_()}\spxextra{nexoclom2.particle\_tracking.final\_state.FinalState method}}

\begin{fulllineitems}
\phantomsection\label{\detokenize{autoapi/nexoclom2/particle_tracking/final_state/index:nexoclom2.particle_tracking.final_state.FinalState.__getitem__}}
\pysigstartsignatures
\pysiglinewithargsret
{\sphinxbfcode{\sphinxupquote{\_\_getitem\_\_}}}
{\sphinxparam{\DUrole{n}{q}}}
{}
\pysigstopsignatures
\end{fulllineitems}

\index{\_\_len\_\_() (nexoclom2.particle\_tracking.final\_state.FinalState method)@\spxentry{\_\_len\_\_()}\spxextra{nexoclom2.particle\_tracking.final\_state.FinalState method}}

\begin{fulllineitems}
\phantomsection\label{\detokenize{autoapi/nexoclom2/particle_tracking/final_state/index:nexoclom2.particle_tracking.final_state.FinalState.__len__}}
\pysigstartsignatures
\pysiglinewithargsret
{\sphinxbfcode{\sphinxupquote{\_\_len\_\_}}}
{}
{}
\pysigstopsignatures
\end{fulllineitems}

\index{concatenate() (nexoclom2.particle\_tracking.final\_state.FinalState method)@\spxentry{concatenate()}\spxextra{nexoclom2.particle\_tracking.final\_state.FinalState method}}

\begin{fulllineitems}
\phantomsection\label{\detokenize{autoapi/nexoclom2/particle_tracking/final_state/index:nexoclom2.particle_tracking.final_state.FinalState.concatenate}}
\pysigstartsignatures
\pysiglinewithargsret
{\sphinxbfcode{\sphinxupquote{concatenate}}}
{\sphinxparam{\DUrole{n}{new}}}
{}
\pysigstopsignatures
\end{fulllineitems}


\end{fulllineitems}


\sphinxstepscope


\subparagraph{nexoclom2.particle\_tracking.packets}
\label{\detokenize{autoapi/nexoclom2/particle_tracking/packets/index:module-nexoclom2.particle_tracking.packets}}\label{\detokenize{autoapi/nexoclom2/particle_tracking/packets/index:nexoclom2-particle-tracking-packets}}\label{\detokenize{autoapi/nexoclom2/particle_tracking/packets/index::doc}}\index{module@\spxentry{module}!nexoclom2.particle\_tracking.packets@\spxentry{nexoclom2.particle\_tracking.packets}}\index{nexoclom2.particle\_tracking.packets@\spxentry{nexoclom2.particle\_tracking.packets}!module@\spxentry{module}}

\subparagraph{Classes}
\label{\detokenize{autoapi/nexoclom2/particle_tracking/packets/index:classes}}

\begin{savenotes}\sphinxattablestart
\sphinxthistablewithglobalstyle
\sphinxthistablewithnovlinesstyle
\centering
\begin{tabulary}{\linewidth}[t]{\X{1}{2}\X{1}{2}}
\sphinxtoprule
\sphinxtableatstartofbodyhook
\sphinxAtStartPar
{\hyperref[\detokenize{autoapi/nexoclom2/particle_tracking/packets/index:nexoclom2.particle_tracking.packets.Packets}]{\sphinxcrossref{\sphinxcode{\sphinxupquote{Packets}}}}}
&
\sphinxAtStartPar
Class to hold components needed for an rkstep
\\
\sphinxbottomrule
\end{tabulary}
\sphinxtableafterendhook\par
\sphinxattableend\end{savenotes}


\subparagraph{Module Contents}
\label{\detokenize{autoapi/nexoclom2/particle_tracking/packets/index:module-contents}}\index{Packets (class in nexoclom2.particle\_tracking.packets)@\spxentry{Packets}\spxextra{class in nexoclom2.particle\_tracking.packets}}

\begin{fulllineitems}
\phantomsection\label{\detokenize{autoapi/nexoclom2/particle_tracking/packets/index:nexoclom2.particle_tracking.packets.Packets}}
\pysigstartsignatures
\pysiglinewithargsret
{\sphinxbfcode{\sphinxupquote{class\DUrole{w}{ }}}\sphinxcode{\sphinxupquote{nexoclom2.particle\_tracking.packets.}}\sphinxbfcode{\sphinxupquote{Packets}}}
{\sphinxparam{\DUrole{n}{prev}}}
{}
\pysigstopsignatures
\sphinxAtStartPar
Class to hold components needed for an rkstep

\sphinxAtStartPar
See Numerical Recipes, 3rd edition, chapter 17.2
\index{time (nexoclom2.particle\_tracking.packets.Packets attribute)@\spxentry{time}\spxextra{nexoclom2.particle\_tracking.packets.Packets attribute}}

\begin{fulllineitems}
\phantomsection\label{\detokenize{autoapi/nexoclom2/particle_tracking/packets/index:nexoclom2.particle_tracking.packets.Packets.time}}
\pysigstartsignatures
\pysigline
{\sphinxbfcode{\sphinxupquote{time}}}
\pysigstopsignatures
\end{fulllineitems}

\index{X (nexoclom2.particle\_tracking.packets.Packets attribute)@\spxentry{X}\spxextra{nexoclom2.particle\_tracking.packets.Packets attribute}}

\begin{fulllineitems}
\phantomsection\label{\detokenize{autoapi/nexoclom2/particle_tracking/packets/index:nexoclom2.particle_tracking.packets.Packets.X}}
\pysigstartsignatures
\pysigline
{\sphinxbfcode{\sphinxupquote{X}}}
\pysigstopsignatures
\end{fulllineitems}

\index{V (nexoclom2.particle\_tracking.packets.Packets attribute)@\spxentry{V}\spxextra{nexoclom2.particle\_tracking.packets.Packets attribute}}

\begin{fulllineitems}
\phantomsection\label{\detokenize{autoapi/nexoclom2/particle_tracking/packets/index:nexoclom2.particle_tracking.packets.Packets.V}}
\pysigstartsignatures
\pysigline
{\sphinxbfcode{\sphinxupquote{V}}}
\pysigstopsignatures
\end{fulllineitems}

\index{frac (nexoclom2.particle\_tracking.packets.Packets attribute)@\spxentry{frac}\spxextra{nexoclom2.particle\_tracking.packets.Packets attribute}}

\begin{fulllineitems}
\phantomsection\label{\detokenize{autoapi/nexoclom2/particle_tracking/packets/index:nexoclom2.particle_tracking.packets.Packets.frac}}
\pysigstartsignatures
\pysigline
{\sphinxbfcode{\sphinxupquote{frac}}}
\pysigstopsignatures
\end{fulllineitems}

\index{accel (nexoclom2.particle\_tracking.packets.Packets attribute)@\spxentry{accel}\spxextra{nexoclom2.particle\_tracking.packets.Packets attribute}}

\begin{fulllineitems}
\phantomsection\label{\detokenize{autoapi/nexoclom2/particle_tracking/packets/index:nexoclom2.particle_tracking.packets.Packets.accel}}
\pysigstartsignatures
\pysigline
{\sphinxbfcode{\sphinxupquote{accel}}}
\pysigstopsignatures
\end{fulllineitems}

\index{ioniz (nexoclom2.particle\_tracking.packets.Packets attribute)@\spxentry{ioniz}\spxextra{nexoclom2.particle\_tracking.packets.Packets attribute}}

\begin{fulllineitems}
\phantomsection\label{\detokenize{autoapi/nexoclom2/particle_tracking/packets/index:nexoclom2.particle_tracking.packets.Packets.ioniz}}
\pysigstartsignatures
\pysigline
{\sphinxbfcode{\sphinxupquote{ioniz}}}
\pysigstopsignatures
\end{fulllineitems}

\index{\_\_getitem\_\_() (nexoclom2.particle\_tracking.packets.Packets method)@\spxentry{\_\_getitem\_\_()}\spxextra{nexoclom2.particle\_tracking.packets.Packets method}}

\begin{fulllineitems}
\phantomsection\label{\detokenize{autoapi/nexoclom2/particle_tracking/packets/index:nexoclom2.particle_tracking.packets.Packets.__getitem__}}
\pysigstartsignatures
\pysiglinewithargsret
{\sphinxbfcode{\sphinxupquote{\_\_getitem\_\_}}}
{\sphinxparam{\DUrole{n}{q}}}
{}
\pysigstopsignatures
\end{fulllineitems}

\index{\_\_len\_\_() (nexoclom2.particle\_tracking.packets.Packets method)@\spxentry{\_\_len\_\_()}\spxextra{nexoclom2.particle\_tracking.packets.Packets method}}

\begin{fulllineitems}
\phantomsection\label{\detokenize{autoapi/nexoclom2/particle_tracking/packets/index:nexoclom2.particle_tracking.packets.Packets.__len__}}
\pysigstartsignatures
\pysiglinewithargsret
{\sphinxbfcode{\sphinxupquote{\_\_len\_\_}}}
{}
{}
\pysigstopsignatures
\end{fulllineitems}

\index{add\_deriv() (nexoclom2.particle\_tracking.packets.Packets method)@\spxentry{add\_deriv()}\spxextra{nexoclom2.particle\_tracking.packets.Packets method}}

\begin{fulllineitems}
\phantomsection\label{\detokenize{autoapi/nexoclom2/particle_tracking/packets/index:nexoclom2.particle_tracking.packets.Packets.add_deriv}}
\pysigstartsignatures
\pysiglinewithargsret
{\sphinxbfcode{\sphinxupquote{add\_deriv}}}
{\sphinxparam{\DUrole{n}{d}}}
{}
\pysigstopsignatures
\end{fulllineitems}


\end{fulllineitems}


\sphinxstepscope


\subparagraph{nexoclom2.particle\_tracking.rk5\_integrator}
\label{\detokenize{autoapi/nexoclom2/particle_tracking/rk5_integrator/index:module-nexoclom2.particle_tracking.rk5_integrator}}\label{\detokenize{autoapi/nexoclom2/particle_tracking/rk5_integrator/index:nexoclom2-particle-tracking-rk5-integrator}}\label{\detokenize{autoapi/nexoclom2/particle_tracking/rk5_integrator/index::doc}}\index{module@\spxentry{module}!nexoclom2.particle\_tracking.rk5\_integrator@\spxentry{nexoclom2.particle\_tracking.rk5\_integrator}}\index{nexoclom2.particle\_tracking.rk5\_integrator@\spxentry{nexoclom2.particle\_tracking.rk5\_integrator}!module@\spxentry{module}}

\subparagraph{Attributes}
\label{\detokenize{autoapi/nexoclom2/particle_tracking/rk5_integrator/index:attributes}}

\begin{savenotes}\sphinxattablestart
\sphinxthistablewithglobalstyle
\sphinxthistablewithnovlinesstyle
\centering
\begin{tabulary}{\linewidth}[t]{\X{1}{2}\X{1}{2}}
\sphinxtoprule
\sphinxtableatstartofbodyhook
\sphinxAtStartPar
{\hyperref[\detokenize{autoapi/nexoclom2/particle_tracking/rk5_integrator/index:nexoclom2.particle_tracking.rk5_integrator.c}]{\sphinxcrossref{\sphinxcode{\sphinxupquote{c}}}}}
&
\sphinxAtStartPar

\\
\sphinxhline
\sphinxAtStartPar
{\hyperref[\detokenize{autoapi/nexoclom2/particle_tracking/rk5_integrator/index:nexoclom2.particle_tracking.rk5_integrator.b}]{\sphinxcrossref{\sphinxcode{\sphinxupquote{b}}}}}
&
\sphinxAtStartPar

\\
\sphinxhline
\sphinxAtStartPar
{\hyperref[\detokenize{autoapi/nexoclom2/particle_tracking/rk5_integrator/index:nexoclom2.particle_tracking.rk5_integrator.bs}]{\sphinxcrossref{\sphinxcode{\sphinxupquote{bs}}}}}
&
\sphinxAtStartPar

\\
\sphinxhline
\sphinxAtStartPar
{\hyperref[\detokenize{autoapi/nexoclom2/particle_tracking/rk5_integrator/index:nexoclom2.particle_tracking.rk5_integrator.bd}]{\sphinxcrossref{\sphinxcode{\sphinxupquote{bd}}}}}
&
\sphinxAtStartPar

\\
\sphinxhline
\sphinxAtStartPar
{\hyperref[\detokenize{autoapi/nexoclom2/particle_tracking/rk5_integrator/index:nexoclom2.particle_tracking.rk5_integrator.a}]{\sphinxcrossref{\sphinxcode{\sphinxupquote{a}}}}}
&
\sphinxAtStartPar

\\
\sphinxhline
\sphinxAtStartPar
{\hyperref[\detokenize{autoapi/nexoclom2/particle_tracking/rk5_integrator/index:nexoclom2.particle_tracking.rk5_integrator.C}]{\sphinxcrossref{\sphinxcode{\sphinxupquote{C}}}}}
&
\sphinxAtStartPar

\\
\sphinxhline
\sphinxAtStartPar
{\hyperref[\detokenize{autoapi/nexoclom2/particle_tracking/rk5_integrator/index:nexoclom2.particle_tracking.rk5_integrator.A}]{\sphinxcrossref{\sphinxcode{\sphinxupquote{A}}}}}
&
\sphinxAtStartPar

\\
\sphinxhline
\sphinxAtStartPar
{\hyperref[\detokenize{autoapi/nexoclom2/particle_tracking/rk5_integrator/index:nexoclom2.particle_tracking.rk5_integrator.B}]{\sphinxcrossref{\sphinxcode{\sphinxupquote{B}}}}}
&
\sphinxAtStartPar

\\
\sphinxhline
\sphinxAtStartPar
{\hyperref[\detokenize{autoapi/nexoclom2/particle_tracking/rk5_integrator/index:nexoclom2.particle_tracking.rk5_integrator.E}]{\sphinxcrossref{\sphinxcode{\sphinxupquote{E}}}}}
&
\sphinxAtStartPar

\\
\sphinxbottomrule
\end{tabulary}
\sphinxtableafterendhook\par
\sphinxattableend\end{savenotes}


\subparagraph{Classes}
\label{\detokenize{autoapi/nexoclom2/particle_tracking/rk5_integrator/index:classes}}

\begin{savenotes}\sphinxattablestart
\sphinxthistablewithglobalstyle
\sphinxthistablewithnovlinesstyle
\centering
\begin{tabulary}{\linewidth}[t]{\X{1}{2}\X{1}{2}}
\sphinxtoprule
\sphinxtableatstartofbodyhook
\sphinxAtStartPar
{\hyperref[\detokenize{autoapi/nexoclom2/particle_tracking/rk5_integrator/index:nexoclom2.particle_tracking.rk5_integrator.Delta}]{\sphinxcrossref{\sphinxcode{\sphinxupquote{Delta}}}}}
&
\sphinxAtStartPar

\\
\sphinxhline
\sphinxAtStartPar
{\hyperref[\detokenize{autoapi/nexoclom2/particle_tracking/rk5_integrator/index:nexoclom2.particle_tracking.rk5_integrator.rk5Integrator}]{\sphinxcrossref{\sphinxcode{\sphinxupquote{rk5Integrator}}}}}
&
\sphinxAtStartPar
Runge Kutta integrator
\\
\sphinxbottomrule
\end{tabulary}
\sphinxtableafterendhook\par
\sphinxattableend\end{savenotes}


\subparagraph{Module Contents}
\label{\detokenize{autoapi/nexoclom2/particle_tracking/rk5_integrator/index:module-contents}}\index{c (in module nexoclom2.particle\_tracking.rk5\_integrator)@\spxentry{c}\spxextra{in module nexoclom2.particle\_tracking.rk5\_integrator}}

\begin{fulllineitems}
\phantomsection\label{\detokenize{autoapi/nexoclom2/particle_tracking/rk5_integrator/index:nexoclom2.particle_tracking.rk5_integrator.c}}
\pysigstartsignatures
\pysigline
{\sphinxcode{\sphinxupquote{nexoclom2.particle\_tracking.rk5\_integrator.}}\sphinxbfcode{\sphinxupquote{c}}}
\pysigstopsignatures
\end{fulllineitems}

\index{b (in module nexoclom2.particle\_tracking.rk5\_integrator)@\spxentry{b}\spxextra{in module nexoclom2.particle\_tracking.rk5\_integrator}}

\begin{fulllineitems}
\phantomsection\label{\detokenize{autoapi/nexoclom2/particle_tracking/rk5_integrator/index:nexoclom2.particle_tracking.rk5_integrator.b}}
\pysigstartsignatures
\pysigline
{\sphinxcode{\sphinxupquote{nexoclom2.particle\_tracking.rk5\_integrator.}}\sphinxbfcode{\sphinxupquote{b}}}
\pysigstopsignatures
\end{fulllineitems}

\index{bs (in module nexoclom2.particle\_tracking.rk5\_integrator)@\spxentry{bs}\spxextra{in module nexoclom2.particle\_tracking.rk5\_integrator}}

\begin{fulllineitems}
\phantomsection\label{\detokenize{autoapi/nexoclom2/particle_tracking/rk5_integrator/index:nexoclom2.particle_tracking.rk5_integrator.bs}}
\pysigstartsignatures
\pysigline
{\sphinxcode{\sphinxupquote{nexoclom2.particle\_tracking.rk5\_integrator.}}\sphinxbfcode{\sphinxupquote{bs}}}
\pysigstopsignatures
\end{fulllineitems}

\index{bd (in module nexoclom2.particle\_tracking.rk5\_integrator)@\spxentry{bd}\spxextra{in module nexoclom2.particle\_tracking.rk5\_integrator}}

\begin{fulllineitems}
\phantomsection\label{\detokenize{autoapi/nexoclom2/particle_tracking/rk5_integrator/index:nexoclom2.particle_tracking.rk5_integrator.bd}}
\pysigstartsignatures
\pysigline
{\sphinxcode{\sphinxupquote{nexoclom2.particle\_tracking.rk5\_integrator.}}\sphinxbfcode{\sphinxupquote{bd}}}
\pysigstopsignatures
\end{fulllineitems}

\index{a (in module nexoclom2.particle\_tracking.rk5\_integrator)@\spxentry{a}\spxextra{in module nexoclom2.particle\_tracking.rk5\_integrator}}

\begin{fulllineitems}
\phantomsection\label{\detokenize{autoapi/nexoclom2/particle_tracking/rk5_integrator/index:nexoclom2.particle_tracking.rk5_integrator.a}}
\pysigstartsignatures
\pysigline
{\sphinxcode{\sphinxupquote{nexoclom2.particle\_tracking.rk5\_integrator.}}\sphinxbfcode{\sphinxupquote{a}}}
\pysigstopsignatures
\end{fulllineitems}

\index{C (in module nexoclom2.particle\_tracking.rk5\_integrator)@\spxentry{C}\spxextra{in module nexoclom2.particle\_tracking.rk5\_integrator}}

\begin{fulllineitems}
\phantomsection\label{\detokenize{autoapi/nexoclom2/particle_tracking/rk5_integrator/index:nexoclom2.particle_tracking.rk5_integrator.C}}
\pysigstartsignatures
\pysigline
{\sphinxcode{\sphinxupquote{nexoclom2.particle\_tracking.rk5\_integrator.}}\sphinxbfcode{\sphinxupquote{C}}}
\pysigstopsignatures
\end{fulllineitems}

\index{A (in module nexoclom2.particle\_tracking.rk5\_integrator)@\spxentry{A}\spxextra{in module nexoclom2.particle\_tracking.rk5\_integrator}}

\begin{fulllineitems}
\phantomsection\label{\detokenize{autoapi/nexoclom2/particle_tracking/rk5_integrator/index:nexoclom2.particle_tracking.rk5_integrator.A}}
\pysigstartsignatures
\pysigline
{\sphinxcode{\sphinxupquote{nexoclom2.particle\_tracking.rk5\_integrator.}}\sphinxbfcode{\sphinxupquote{A}}}
\pysigstopsignatures
\end{fulllineitems}

\index{B (in module nexoclom2.particle\_tracking.rk5\_integrator)@\spxentry{B}\spxextra{in module nexoclom2.particle\_tracking.rk5\_integrator}}

\begin{fulllineitems}
\phantomsection\label{\detokenize{autoapi/nexoclom2/particle_tracking/rk5_integrator/index:nexoclom2.particle_tracking.rk5_integrator.B}}
\pysigstartsignatures
\pysigline
{\sphinxcode{\sphinxupquote{nexoclom2.particle\_tracking.rk5\_integrator.}}\sphinxbfcode{\sphinxupquote{B}}}
\pysigstopsignatures
\end{fulllineitems}

\index{E (in module nexoclom2.particle\_tracking.rk5\_integrator)@\spxentry{E}\spxextra{in module nexoclom2.particle\_tracking.rk5\_integrator}}

\begin{fulllineitems}
\phantomsection\label{\detokenize{autoapi/nexoclom2/particle_tracking/rk5_integrator/index:nexoclom2.particle_tracking.rk5_integrator.E}}
\pysigstartsignatures
\pysigline
{\sphinxcode{\sphinxupquote{nexoclom2.particle\_tracking.rk5\_integrator.}}\sphinxbfcode{\sphinxupquote{E}}}
\pysigstopsignatures
\end{fulllineitems}

\index{Delta (class in nexoclom2.particle\_tracking.rk5\_integrator)@\spxentry{Delta}\spxextra{class in nexoclom2.particle\_tracking.rk5\_integrator}}

\begin{fulllineitems}
\phantomsection\label{\detokenize{autoapi/nexoclom2/particle_tracking/rk5_integrator/index:nexoclom2.particle_tracking.rk5_integrator.Delta}}
\pysigstartsignatures
\pysiglinewithargsret
{\sphinxbfcode{\sphinxupquote{class\DUrole{w}{ }}}\sphinxcode{\sphinxupquote{nexoclom2.particle\_tracking.rk5\_integrator.}}\sphinxbfcode{\sphinxupquote{Delta}}}
{\sphinxparam{\DUrole{n}{prev}}}
{}
\pysigstopsignatures\index{X (nexoclom2.particle\_tracking.rk5\_integrator.Delta attribute)@\spxentry{X}\spxextra{nexoclom2.particle\_tracking.rk5\_integrator.Delta attribute}}

\begin{fulllineitems}
\phantomsection\label{\detokenize{autoapi/nexoclom2/particle_tracking/rk5_integrator/index:nexoclom2.particle_tracking.rk5_integrator.Delta.X}}
\pysigstartsignatures
\pysigline
{\sphinxbfcode{\sphinxupquote{X}}}
\pysigstopsignatures
\end{fulllineitems}

\index{V (nexoclom2.particle\_tracking.rk5\_integrator.Delta attribute)@\spxentry{V}\spxextra{nexoclom2.particle\_tracking.rk5\_integrator.Delta attribute}}

\begin{fulllineitems}
\phantomsection\label{\detokenize{autoapi/nexoclom2/particle_tracking/rk5_integrator/index:nexoclom2.particle_tracking.rk5_integrator.Delta.V}}
\pysigstartsignatures
\pysigline
{\sphinxbfcode{\sphinxupquote{V}}}
\pysigstopsignatures
\end{fulllineitems}

\index{frac (nexoclom2.particle\_tracking.rk5\_integrator.Delta attribute)@\spxentry{frac}\spxextra{nexoclom2.particle\_tracking.rk5\_integrator.Delta attribute}}

\begin{fulllineitems}
\phantomsection\label{\detokenize{autoapi/nexoclom2/particle_tracking/rk5_integrator/index:nexoclom2.particle_tracking.rk5_integrator.Delta.frac}}
\pysigstartsignatures
\pysigline
{\sphinxbfcode{\sphinxupquote{frac}}}
\pysigstopsignatures
\end{fulllineitems}

\index{\_\_len\_\_() (nexoclom2.particle\_tracking.rk5\_integrator.Delta method)@\spxentry{\_\_len\_\_()}\spxextra{nexoclom2.particle\_tracking.rk5\_integrator.Delta method}}

\begin{fulllineitems}
\phantomsection\label{\detokenize{autoapi/nexoclom2/particle_tracking/rk5_integrator/index:nexoclom2.particle_tracking.rk5_integrator.Delta.__len__}}
\pysigstartsignatures
\pysiglinewithargsret
{\sphinxbfcode{\sphinxupquote{\_\_len\_\_}}}
{}
{}
\pysigstopsignatures
\end{fulllineitems}

\index{max() (nexoclom2.particle\_tracking.rk5\_integrator.Delta method)@\spxentry{max()}\spxextra{nexoclom2.particle\_tracking.rk5\_integrator.Delta method}}

\begin{fulllineitems}
\phantomsection\label{\detokenize{autoapi/nexoclom2/particle_tracking/rk5_integrator/index:nexoclom2.particle_tracking.rk5_integrator.Delta.max}}
\pysigstartsignatures
\pysiglinewithargsret
{\sphinxbfcode{\sphinxupquote{max}}}
{}
{}
\pysigstopsignatures
\end{fulllineitems}


\end{fulllineitems}

\index{rk5Integrator (class in nexoclom2.particle\_tracking.rk5\_integrator)@\spxentry{rk5Integrator}\spxextra{class in nexoclom2.particle\_tracking.rk5\_integrator}}

\begin{fulllineitems}
\phantomsection\label{\detokenize{autoapi/nexoclom2/particle_tracking/rk5_integrator/index:nexoclom2.particle_tracking.rk5_integrator.rk5Integrator}}
\pysigstartsignatures
\pysigline
{\sphinxbfcode{\sphinxupquote{class\DUrole{w}{ }}}\sphinxcode{\sphinxupquote{nexoclom2.particle\_tracking.rk5\_integrator.}}\sphinxbfcode{\sphinxupquote{rk5Integrator}}}
\pysigstopsignatures
\sphinxAtStartPar
Runge Kutta integrator

\sphinxAtStartPar
Can run with either constant or variable step size.

\sphinxAtStartPar
Equations of motion:

\sphinxAtStartPar
t\_n+1 = t+n + h
x\_n+1 = x\_n + h*v\_n
v\_n+1 = v\_n + j\_a\_n
frac\_n+1 = frac\_n**(h/tau\_n)
\begin{quote}\begin{description}
\sphinxlineitem{Parameters}\begin{description}
\sphinxlineitem{\sphinxstylestrong{output}}{[}Output{]}
\sphinxAtStartPar
nexoclom2 Output clas

\end{description}

\sphinxlineitem{Returns}\begin{description}
\sphinxlineitem{Final state of the system}
\end{description}

\end{description}\end{quote}
\index{step() (nexoclom2.particle\_tracking.rk5\_integrator.rk5Integrator method)@\spxentry{step()}\spxextra{nexoclom2.particle\_tracking.rk5\_integrator.rk5Integrator method}}

\begin{fulllineitems}
\phantomsection\label{\detokenize{autoapi/nexoclom2/particle_tracking/rk5_integrator/index:nexoclom2.particle_tracking.rk5_integrator.rk5Integrator.step}}
\pysigstartsignatures
\pysiglinewithargsret
{\sphinxbfcode{\sphinxupquote{step}}}
{\sphinxparam{\DUrole{n}{prev\_step}}\sphinxparamcomma \sphinxparam{\DUrole{n}{output}}\sphinxparamcomma \sphinxparam{\DUrole{n}{h}}}
{}
\pysigstopsignatures
\sphinxAtStartPar
Perform a single rk5 step.

\sphinxAtStartPar
See Numerical Recipes, 3rd edition, chapter 17.2

\end{fulllineitems}


\end{fulllineitems}


\sphinxstepscope


\subparagraph{nexoclom2.particle\_tracking.starting\_point}
\label{\detokenize{autoapi/nexoclom2/particle_tracking/starting_point/index:module-nexoclom2.particle_tracking.starting_point}}\label{\detokenize{autoapi/nexoclom2/particle_tracking/starting_point/index:nexoclom2-particle-tracking-starting-point}}\label{\detokenize{autoapi/nexoclom2/particle_tracking/starting_point/index::doc}}\index{module@\spxentry{module}!nexoclom2.particle\_tracking.starting\_point@\spxentry{nexoclom2.particle\_tracking.starting\_point}}\index{nexoclom2.particle\_tracking.starting\_point@\spxentry{nexoclom2.particle\_tracking.starting\_point}!module@\spxentry{module}}

\subparagraph{Classes}
\label{\detokenize{autoapi/nexoclom2/particle_tracking/starting_point/index:classes}}

\begin{savenotes}\sphinxattablestart
\sphinxthistablewithglobalstyle
\sphinxthistablewithnovlinesstyle
\centering
\begin{tabulary}{\linewidth}[t]{\X{1}{2}\X{1}{2}}
\sphinxtoprule
\sphinxtableatstartofbodyhook
\sphinxAtStartPar
{\hyperref[\detokenize{autoapi/nexoclom2/particle_tracking/starting_point/index:nexoclom2.particle_tracking.starting_point.StartingPoint}]{\sphinxcrossref{\sphinxcode{\sphinxupquote{StartingPoint}}}}}
&
\sphinxAtStartPar

\\
\sphinxbottomrule
\end{tabulary}
\sphinxtableafterendhook\par
\sphinxattableend\end{savenotes}


\subparagraph{Module Contents}
\label{\detokenize{autoapi/nexoclom2/particle_tracking/starting_point/index:module-contents}}\index{StartingPoint (class in nexoclom2.particle\_tracking.starting\_point)@\spxentry{StartingPoint}\spxextra{class in nexoclom2.particle\_tracking.starting\_point}}

\begin{fulllineitems}
\phantomsection\label{\detokenize{autoapi/nexoclom2/particle_tracking/starting_point/index:nexoclom2.particle_tracking.starting_point.StartingPoint}}
\pysigstartsignatures
\pysiglinewithargsret
{\sphinxbfcode{\sphinxupquote{class\DUrole{w}{ }}}\sphinxcode{\sphinxupquote{nexoclom2.particle\_tracking.starting\_point.}}\sphinxbfcode{\sphinxupquote{StartingPoint}}}
{\sphinxparam{\DUrole{n}{output}}\sphinxparamcomma \sphinxparam{\DUrole{n}{iteration}\DUrole{o}{=}\DUrole{default_value}{None}}\sphinxparamcomma \sphinxparam{\DUrole{n}{n\_packets}\DUrole{o}{=}\DUrole{default_value}{None}}\sphinxparamcomma \sphinxparam{\DUrole{n}{existing}\DUrole{o}{=}\DUrole{default_value}{False}}}
{}
\pysigstopsignatures\index{\_\_len\_\_() (nexoclom2.particle\_tracking.starting\_point.StartingPoint method)@\spxentry{\_\_len\_\_()}\spxextra{nexoclom2.particle\_tracking.starting\_point.StartingPoint method}}

\begin{fulllineitems}
\phantomsection\label{\detokenize{autoapi/nexoclom2/particle_tracking/starting_point/index:nexoclom2.particle_tracking.starting_point.StartingPoint.__len__}}
\pysigstartsignatures
\pysiglinewithargsret
{\sphinxbfcode{\sphinxupquote{\_\_len\_\_}}}
{}
{}
\pysigstopsignatures
\end{fulllineitems}


\end{fulllineitems}


\sphinxstepscope


\subparagraph{nexoclom2.particle\_tracking.state\_vectors}
\label{\detokenize{autoapi/nexoclom2/particle_tracking/state_vectors/index:module-nexoclom2.particle_tracking.state_vectors}}\label{\detokenize{autoapi/nexoclom2/particle_tracking/state_vectors/index:nexoclom2-particle-tracking-state-vectors}}\label{\detokenize{autoapi/nexoclom2/particle_tracking/state_vectors/index::doc}}\index{module@\spxentry{module}!nexoclom2.particle\_tracking.state\_vectors@\spxentry{nexoclom2.particle\_tracking.state\_vectors}}\index{nexoclom2.particle\_tracking.state\_vectors@\spxentry{nexoclom2.particle\_tracking.state\_vectors}!module@\spxentry{module}}

\subparagraph{Classes}
\label{\detokenize{autoapi/nexoclom2/particle_tracking/state_vectors/index:classes}}

\begin{savenotes}\sphinxattablestart
\sphinxthistablewithglobalstyle
\sphinxthistablewithnovlinesstyle
\centering
\begin{tabulary}{\linewidth}[t]{\X{1}{2}\X{1}{2}}
\sphinxtoprule
\sphinxtableatstartofbodyhook
\sphinxAtStartPar
{\hyperref[\detokenize{autoapi/nexoclom2/particle_tracking/state_vectors/index:nexoclom2.particle_tracking.state_vectors.StateVector}]{\sphinxcrossref{\sphinxcode{\sphinxupquote{StateVector}}}}}
&
\sphinxAtStartPar

\\
\sphinxbottomrule
\end{tabulary}
\sphinxtableafterendhook\par
\sphinxattableend\end{savenotes}


\subparagraph{Module Contents}
\label{\detokenize{autoapi/nexoclom2/particle_tracking/state_vectors/index:module-contents}}\index{StateVector (class in nexoclom2.particle\_tracking.state\_vectors)@\spxentry{StateVector}\spxextra{class in nexoclom2.particle\_tracking.state\_vectors}}

\begin{fulllineitems}
\phantomsection\label{\detokenize{autoapi/nexoclom2/particle_tracking/state_vectors/index:nexoclom2.particle_tracking.state_vectors.StateVector}}
\pysigstartsignatures
\pysiglinewithargsret
{\sphinxbfcode{\sphinxupquote{class\DUrole{w}{ }}}\sphinxcode{\sphinxupquote{nexoclom2.particle\_tracking.state\_vectors.}}\sphinxbfcode{\sphinxupquote{StateVector}}}
{\sphinxparam{\DUrole{n}{output}}\sphinxparamcomma \sphinxparam{\DUrole{n}{starting\_point}}}
{}
\pysigstopsignatures\index{time (nexoclom2.particle\_tracking.state\_vectors.StateVector attribute)@\spxentry{time}\spxextra{nexoclom2.particle\_tracking.state\_vectors.StateVector attribute}}

\begin{fulllineitems}
\phantomsection\label{\detokenize{autoapi/nexoclom2/particle_tracking/state_vectors/index:nexoclom2.particle_tracking.state_vectors.StateVector.time}}
\pysigstartsignatures
\pysigline
{\sphinxbfcode{\sphinxupquote{time}}}
\pysigstopsignatures
\end{fulllineitems}

\index{X (nexoclom2.particle\_tracking.state\_vectors.StateVector attribute)@\spxentry{X}\spxextra{nexoclom2.particle\_tracking.state\_vectors.StateVector attribute}}

\begin{fulllineitems}
\phantomsection\label{\detokenize{autoapi/nexoclom2/particle_tracking/state_vectors/index:nexoclom2.particle_tracking.state_vectors.StateVector.X}}
\pysigstartsignatures
\pysigline
{\sphinxbfcode{\sphinxupquote{X}}}
\pysigstopsignatures
\end{fulllineitems}

\index{V (nexoclom2.particle\_tracking.state\_vectors.StateVector attribute)@\spxentry{V}\spxextra{nexoclom2.particle\_tracking.state\_vectors.StateVector attribute}}

\begin{fulllineitems}
\phantomsection\label{\detokenize{autoapi/nexoclom2/particle_tracking/state_vectors/index:nexoclom2.particle_tracking.state_vectors.StateVector.V}}
\pysigstartsignatures
\pysigline
{\sphinxbfcode{\sphinxupquote{V}}}
\pysigstopsignatures
\end{fulllineitems}

\index{frac (nexoclom2.particle\_tracking.state\_vectors.StateVector attribute)@\spxentry{frac}\spxextra{nexoclom2.particle\_tracking.state\_vectors.StateVector attribute}}

\begin{fulllineitems}
\phantomsection\label{\detokenize{autoapi/nexoclom2/particle_tracking/state_vectors/index:nexoclom2.particle_tracking.state_vectors.StateVector.frac}}
\pysigstartsignatures
\pysigline
{\sphinxbfcode{\sphinxupquote{frac}}}
\pysigstopsignatures
\end{fulllineitems}

\index{escaped (nexoclom2.particle\_tracking.state\_vectors.StateVector attribute)@\spxentry{escaped}\spxextra{nexoclom2.particle\_tracking.state\_vectors.StateVector attribute}}

\begin{fulllineitems}
\phantomsection\label{\detokenize{autoapi/nexoclom2/particle_tracking/state_vectors/index:nexoclom2.particle_tracking.state_vectors.StateVector.escaped}}
\pysigstartsignatures
\pysigline
{\sphinxbfcode{\sphinxupquote{escaped}}}
\pysigstopsignatures
\end{fulllineitems}

\index{hit (nexoclom2.particle\_tracking.state\_vectors.StateVector attribute)@\spxentry{hit}\spxextra{nexoclom2.particle\_tracking.state\_vectors.StateVector attribute}}

\begin{fulllineitems}
\phantomsection\label{\detokenize{autoapi/nexoclom2/particle_tracking/state_vectors/index:nexoclom2.particle_tracking.state_vectors.StateVector.hit}}
\pysigstartsignatures
\pysigline
{\sphinxbfcode{\sphinxupquote{hit}}}
\pysigstopsignatures
\end{fulllineitems}

\index{ionized (nexoclom2.particle\_tracking.state\_vectors.StateVector attribute)@\spxentry{ionized}\spxextra{nexoclom2.particle\_tracking.state\_vectors.StateVector attribute}}

\begin{fulllineitems}
\phantomsection\label{\detokenize{autoapi/nexoclom2/particle_tracking/state_vectors/index:nexoclom2.particle_tracking.state_vectors.StateVector.ionized}}
\pysigstartsignatures
\pysigline
{\sphinxbfcode{\sphinxupquote{ionized}}}
\pysigstopsignatures
\end{fulllineitems}

\index{packet\_number (nexoclom2.particle\_tracking.state\_vectors.StateVector attribute)@\spxentry{packet\_number}\spxextra{nexoclom2.particle\_tracking.state\_vectors.StateVector attribute}}

\begin{fulllineitems}
\phantomsection\label{\detokenize{autoapi/nexoclom2/particle_tracking/state_vectors/index:nexoclom2.particle_tracking.state_vectors.StateVector.packet_number}}
\pysigstartsignatures
\pysigline
{\sphinxbfcode{\sphinxupquote{packet\_number}}}
\pysigstopsignatures
\end{fulllineitems}

\index{iteration (nexoclom2.particle\_tracking.state\_vectors.StateVector attribute)@\spxentry{iteration}\spxextra{nexoclom2.particle\_tracking.state\_vectors.StateVector attribute}}

\begin{fulllineitems}
\phantomsection\label{\detokenize{autoapi/nexoclom2/particle_tracking/state_vectors/index:nexoclom2.particle_tracking.state_vectors.StateVector.iteration}}
\pysigstartsignatures
\pysigline
{\sphinxbfcode{\sphinxupquote{iteration}}}
\pysigstopsignatures
\end{fulllineitems}

\index{\_\_getitem\_\_() (nexoclom2.particle\_tracking.state\_vectors.StateVector method)@\spxentry{\_\_getitem\_\_()}\spxextra{nexoclom2.particle\_tracking.state\_vectors.StateVector method}}

\begin{fulllineitems}
\phantomsection\label{\detokenize{autoapi/nexoclom2/particle_tracking/state_vectors/index:nexoclom2.particle_tracking.state_vectors.StateVector.__getitem__}}
\pysigstartsignatures
\pysiglinewithargsret
{\sphinxbfcode{\sphinxupquote{\_\_getitem\_\_}}}
{\sphinxparam{\DUrole{n}{q}}}
{}
\pysigstopsignatures
\end{fulllineitems}

\index{\_\_setitem\_\_() (nexoclom2.particle\_tracking.state\_vectors.StateVector method)@\spxentry{\_\_setitem\_\_()}\spxextra{nexoclom2.particle\_tracking.state\_vectors.StateVector method}}

\begin{fulllineitems}
\phantomsection\label{\detokenize{autoapi/nexoclom2/particle_tracking/state_vectors/index:nexoclom2.particle_tracking.state_vectors.StateVector.__setitem__}}
\pysigstartsignatures
\pysiglinewithargsret
{\sphinxbfcode{\sphinxupquote{\_\_setitem\_\_}}}
{\sphinxparam{\DUrole{n}{q}}\sphinxparamcomma \sphinxparam{\DUrole{n}{new}}}
{}
\pysigstopsignatures
\end{fulllineitems}

\index{\_\_len\_\_() (nexoclom2.particle\_tracking.state\_vectors.StateVector method)@\spxentry{\_\_len\_\_()}\spxextra{nexoclom2.particle\_tracking.state\_vectors.StateVector method}}

\begin{fulllineitems}
\phantomsection\label{\detokenize{autoapi/nexoclom2/particle_tracking/state_vectors/index:nexoclom2.particle_tracking.state_vectors.StateVector.__len__}}
\pysigstartsignatures
\pysiglinewithargsret
{\sphinxbfcode{\sphinxupquote{\_\_len\_\_}}}
{}
{}
\pysigstopsignatures
\end{fulllineitems}

\index{surface\_interaction() (nexoclom2.particle\_tracking.state\_vectors.StateVector method)@\spxentry{surface\_interaction()}\spxextra{nexoclom2.particle\_tracking.state\_vectors.StateVector method}}

\begin{fulllineitems}
\phantomsection\label{\detokenize{autoapi/nexoclom2/particle_tracking/state_vectors/index:nexoclom2.particle_tracking.state_vectors.StateVector.surface_interaction}}
\pysigstartsignatures
\pysiglinewithargsret
{\sphinxbfcode{\sphinxupquote{surface\_interaction}}}
{\sphinxparam{\DUrole{n}{output}}}
{}
\pysigstopsignatures
\end{fulllineitems}

\index{check\_escape() (nexoclom2.particle\_tracking.state\_vectors.StateVector method)@\spxentry{check\_escape()}\spxextra{nexoclom2.particle\_tracking.state\_vectors.StateVector method}}

\begin{fulllineitems}
\phantomsection\label{\detokenize{autoapi/nexoclom2/particle_tracking/state_vectors/index:nexoclom2.particle_tracking.state_vectors.StateVector.check_escape}}
\pysigstartsignatures
\pysiglinewithargsret
{\sphinxbfcode{\sphinxupquote{check\_escape}}}
{\sphinxparam{\DUrole{n}{output}}}
{}
\pysigstopsignatures
\end{fulllineitems}

\index{vecmul() (nexoclom2.particle\_tracking.state\_vectors.StateVector method)@\spxentry{vecmul()}\spxextra{nexoclom2.particle\_tracking.state\_vectors.StateVector method}}

\begin{fulllineitems}
\phantomsection\label{\detokenize{autoapi/nexoclom2/particle_tracking/state_vectors/index:nexoclom2.particle_tracking.state_vectors.StateVector.vecmul}}
\pysigstartsignatures
\pysiglinewithargsret
{\sphinxbfcode{\sphinxupquote{vecmul}}}
{\sphinxparam{\DUrole{n}{vec}}}
{}
\pysigstopsignatures
\end{fulllineitems}

\index{constmul() (nexoclom2.particle\_tracking.state\_vectors.StateVector method)@\spxentry{constmul()}\spxextra{nexoclom2.particle\_tracking.state\_vectors.StateVector method}}

\begin{fulllineitems}
\phantomsection\label{\detokenize{autoapi/nexoclom2/particle_tracking/state_vectors/index:nexoclom2.particle_tracking.state_vectors.StateVector.constmul}}
\pysigstartsignatures
\pysiglinewithargsret
{\sphinxbfcode{\sphinxupquote{constmul}}}
{\sphinxparam{\DUrole{n}{const}}}
{}
\pysigstopsignatures
\end{fulllineitems}

\index{\_\_add\_\_() (nexoclom2.particle\_tracking.state\_vectors.StateVector method)@\spxentry{\_\_add\_\_()}\spxextra{nexoclom2.particle\_tracking.state\_vectors.StateVector method}}

\begin{fulllineitems}
\phantomsection\label{\detokenize{autoapi/nexoclom2/particle_tracking/state_vectors/index:nexoclom2.particle_tracking.state_vectors.StateVector.__add__}}
\pysigstartsignatures
\pysiglinewithargsret
{\sphinxbfcode{\sphinxupquote{\_\_add\_\_}}}
{\sphinxparam{\DUrole{n}{other}}}
{}
\pysigstopsignatures
\end{fulllineitems}

\index{\_\_radd\_\_() (nexoclom2.particle\_tracking.state\_vectors.StateVector method)@\spxentry{\_\_radd\_\_()}\spxextra{nexoclom2.particle\_tracking.state\_vectors.StateVector method}}

\begin{fulllineitems}
\phantomsection\label{\detokenize{autoapi/nexoclom2/particle_tracking/state_vectors/index:nexoclom2.particle_tracking.state_vectors.StateVector.__radd__}}
\pysigstartsignatures
\pysiglinewithargsret
{\sphinxbfcode{\sphinxupquote{\_\_radd\_\_}}}
{\sphinxparam{\DUrole{n}{const}}}
{}
\pysigstopsignatures
\end{fulllineitems}

\index{max() (nexoclom2.particle\_tracking.state\_vectors.StateVector method)@\spxentry{max()}\spxextra{nexoclom2.particle\_tracking.state\_vectors.StateVector method}}

\begin{fulllineitems}
\phantomsection\label{\detokenize{autoapi/nexoclom2/particle_tracking/state_vectors/index:nexoclom2.particle_tracking.state_vectors.StateVector.max}}
\pysigstartsignatures
\pysiglinewithargsret
{\sphinxbfcode{\sphinxupquote{max}}}
{}
{}
\pysigstopsignatures
\end{fulllineitems}


\end{fulllineitems}



\subparagraph{Classes}
\label{\detokenize{autoapi/nexoclom2/particle_tracking/index:classes}}

\begin{savenotes}\sphinxattablestart
\sphinxthistablewithglobalstyle
\sphinxthistablewithnovlinesstyle
\centering
\begin{tabulary}{\linewidth}[t]{\X{1}{2}\X{1}{2}}
\sphinxtoprule
\sphinxtableatstartofbodyhook
\sphinxAtStartPar
{\hyperref[\detokenize{autoapi/nexoclom2/particle_tracking/index:nexoclom2.particle_tracking.ConstantIntegrator}]{\sphinxcrossref{\sphinxcode{\sphinxupquote{ConstantIntegrator}}}}}
&
\sphinxAtStartPar
Constant step size integrator
\\
\sphinxhline
\sphinxAtStartPar
{\hyperref[\detokenize{autoapi/nexoclom2/particle_tracking/index:nexoclom2.particle_tracking.VariableIntegrator}]{\sphinxcrossref{\sphinxcode{\sphinxupquote{VariableIntegrator}}}}}
&
\sphinxAtStartPar
Runge Kutta integrator
\\
\sphinxbottomrule
\end{tabulary}
\sphinxtableafterendhook\par
\sphinxattableend\end{savenotes}


\subparagraph{Package Contents}
\label{\detokenize{autoapi/nexoclom2/particle_tracking/index:package-contents}}\index{ConstantIntegrator (class in nexoclom2.particle\_tracking)@\spxentry{ConstantIntegrator}\spxextra{class in nexoclom2.particle\_tracking}}

\begin{fulllineitems}
\phantomsection\label{\detokenize{autoapi/nexoclom2/particle_tracking/index:nexoclom2.particle_tracking.ConstantIntegrator}}
\pysigstartsignatures
\pysiglinewithargsret
{\sphinxbfcode{\sphinxupquote{class\DUrole{w}{ }}}\sphinxcode{\sphinxupquote{nexoclom2.particle\_tracking.}}\sphinxbfcode{\sphinxupquote{ConstantIntegrator}}}
{\sphinxparam{\DUrole{n}{output}}\sphinxparamcomma \sphinxparam{\DUrole{n}{state}}\sphinxparamcomma \sphinxparam{\DUrole{n}{method}\DUrole{o}{=}\DUrole{default_value}{\textquotesingle{}rk5\textquotesingle{}}}}
{}
\pysigstopsignatures
\sphinxAtStartPar
Constant step size integrator
:param output: nexoclom2 Output clas
:type output: Output
\begin{quote}\begin{description}
\sphinxlineitem{Returns}\begin{description}
\sphinxlineitem{Final state of the system}
\end{description}

\end{description}\end{quote}

\end{fulllineitems}

\index{VariableIntegrator (class in nexoclom2.particle\_tracking)@\spxentry{VariableIntegrator}\spxextra{class in nexoclom2.particle\_tracking}}

\begin{fulllineitems}
\phantomsection\label{\detokenize{autoapi/nexoclom2/particle_tracking/index:nexoclom2.particle_tracking.VariableIntegrator}}
\pysigstartsignatures
\pysiglinewithargsret
{\sphinxbfcode{\sphinxupquote{class\DUrole{w}{ }}}\sphinxcode{\sphinxupquote{nexoclom2.particle\_tracking.}}\sphinxbfcode{\sphinxupquote{VariableIntegrator}}}
{\sphinxparam{\DUrole{n}{output}}\sphinxparamcomma \sphinxparam{\DUrole{n}{state}}\sphinxparamcomma \sphinxparam{\DUrole{n}{method}\DUrole{o}{=}\DUrole{default_value}{\textquotesingle{}rk5\textquotesingle{}}}}
{}
\pysigstopsignatures
\sphinxAtStartPar
Runge Kutta integrator

\sphinxAtStartPar
Can run with either constant or variable step size.
\begin{quote}\begin{description}
\sphinxlineitem{Parameters}\begin{description}
\sphinxlineitem{\sphinxstylestrong{output}}{[}Output{]}
\sphinxAtStartPar
nexoclom2 Output clas

\end{description}

\sphinxlineitem{Returns}\begin{description}
\sphinxlineitem{Final state of the system}
\end{description}

\end{description}\end{quote}

\end{fulllineitems}


\sphinxstepscope


\paragraph{nexoclom2.solarsystem}
\label{\detokenize{autoapi/nexoclom2/solarsystem/index:module-nexoclom2.solarsystem}}\label{\detokenize{autoapi/nexoclom2/solarsystem/index:nexoclom2-solarsystem}}\label{\detokenize{autoapi/nexoclom2/solarsystem/index::doc}}\index{module@\spxentry{module}!nexoclom2.solarsystem@\spxentry{nexoclom2.solarsystem}}\index{nexoclom2.solarsystem@\spxentry{nexoclom2.solarsystem}!module@\spxentry{module}}
\sphinxAtStartPar
Classes and functions for working with solar system data


\subparagraph{Submodules}
\label{\detokenize{autoapi/nexoclom2/solarsystem/index:submodules}}
\sphinxstepscope


\subparagraph{nexoclom2.solarsystem.IoTorus}
\label{\detokenize{autoapi/nexoclom2/solarsystem/IoTorus/index:module-nexoclom2.solarsystem.IoTorus}}\label{\detokenize{autoapi/nexoclom2/solarsystem/IoTorus/index:nexoclom2-solarsystem-iotorus}}\label{\detokenize{autoapi/nexoclom2/solarsystem/IoTorus/index::doc}}\index{module@\spxentry{module}!nexoclom2.solarsystem.IoTorus@\spxentry{nexoclom2.solarsystem.IoTorus}}\index{nexoclom2.solarsystem.IoTorus@\spxentry{nexoclom2.solarsystem.IoTorus}!module@\spxentry{module}}

\subparagraph{Classes}
\label{\detokenize{autoapi/nexoclom2/solarsystem/IoTorus/index:classes}}

\begin{savenotes}\sphinxattablestart
\sphinxthistablewithglobalstyle
\sphinxthistablewithnovlinesstyle
\centering
\begin{tabulary}{\linewidth}[t]{\X{1}{2}\X{1}{2}}
\sphinxtoprule
\sphinxtableatstartofbodyhook
\sphinxAtStartPar
{\hyperref[\detokenize{autoapi/nexoclom2/solarsystem/IoTorus/index:nexoclom2.solarsystem.IoTorus.IoTorus}]{\sphinxcrossref{\sphinxcode{\sphinxupquote{IoTorus}}}}}
&
\sphinxAtStartPar

\\
\sphinxbottomrule
\end{tabulary}
\sphinxtableafterendhook\par
\sphinxattableend\end{savenotes}


\subparagraph{Module Contents}
\label{\detokenize{autoapi/nexoclom2/solarsystem/IoTorus/index:module-contents}}\index{IoTorus (class in nexoclom2.solarsystem.IoTorus)@\spxentry{IoTorus}\spxextra{class in nexoclom2.solarsystem.IoTorus}}

\begin{fulllineitems}
\phantomsection\label{\detokenize{autoapi/nexoclom2/solarsystem/IoTorus/index:nexoclom2.solarsystem.IoTorus.IoTorus}}
\pysigstartsignatures
\pysiglinewithargsret
{\sphinxbfcode{\sphinxupquote{class\DUrole{w}{ }}}\sphinxcode{\sphinxupquote{nexoclom2.solarsystem.IoTorus.}}\sphinxbfcode{\sphinxupquote{IoTorus}}}
{\sphinxparam{\DUrole{n}{source}\DUrole{o}{=}\DUrole{default_value}{\textquotesingle{}Voyager\textquotesingle{}}}}
{}
\pysigstopsignatures\index{source (nexoclom2.solarsystem.IoTorus.IoTorus attribute)@\spxentry{source}\spxextra{nexoclom2.solarsystem.IoTorus.IoTorus attribute}}

\begin{fulllineitems}
\phantomsection\label{\detokenize{autoapi/nexoclom2/solarsystem/IoTorus/index:nexoclom2.solarsystem.IoTorus.IoTorus.source}}
\pysigstartsignatures
\pysigline
{\sphinxbfcode{\sphinxupquote{source}}\sphinxbfcode{\sphinxupquote{\DUrole{w}{ }\DUrole{p}{=}\DUrole{w}{ }\textquotesingle{}Voyager\textquotesingle{}}}}
\pysigstopsignatures
\end{fulllineitems}

\index{M (nexoclom2.solarsystem.IoTorus.IoTorus attribute)@\spxentry{M}\spxextra{nexoclom2.solarsystem.IoTorus.IoTorus attribute}}

\begin{fulllineitems}
\phantomsection\label{\detokenize{autoapi/nexoclom2/solarsystem/IoTorus/index:nexoclom2.solarsystem.IoTorus.IoTorus.M}}
\pysigstartsignatures
\pysigline
{\sphinxbfcode{\sphinxupquote{M}}}
\pysigstopsignatures
\end{fulllineitems}

\index{n\_e (nexoclom2.solarsystem.IoTorus.IoTorus attribute)@\spxentry{n\_e}\spxextra{nexoclom2.solarsystem.IoTorus.IoTorus attribute}}

\begin{fulllineitems}
\phantomsection\label{\detokenize{autoapi/nexoclom2/solarsystem/IoTorus/index:nexoclom2.solarsystem.IoTorus.IoTorus.n_e}}
\pysigstartsignatures
\pysigline
{\sphinxbfcode{\sphinxupquote{n\_e}}}
\pysigstopsignatures
\end{fulllineitems}

\index{T\_e (nexoclom2.solarsystem.IoTorus.IoTorus attribute)@\spxentry{T\_e}\spxextra{nexoclom2.solarsystem.IoTorus.IoTorus attribute}}

\begin{fulllineitems}
\phantomsection\label{\detokenize{autoapi/nexoclom2/solarsystem/IoTorus/index:nexoclom2.solarsystem.IoTorus.IoTorus.T_e}}
\pysigstartsignatures
\pysigline
{\sphinxbfcode{\sphinxupquote{T\_e}}}
\pysigstopsignatures
\end{fulllineitems}

\index{H\_e (nexoclom2.solarsystem.IoTorus.IoTorus attribute)@\spxentry{H\_e}\spxextra{nexoclom2.solarsystem.IoTorus.IoTorus attribute}}

\begin{fulllineitems}
\phantomsection\label{\detokenize{autoapi/nexoclom2/solarsystem/IoTorus/index:nexoclom2.solarsystem.IoTorus.IoTorus.H_e}}
\pysigstartsignatures
\pysigline
{\sphinxbfcode{\sphinxupquote{H\_e}}}
\pysigstopsignatures
\end{fulllineitems}

\index{T\_i (nexoclom2.solarsystem.IoTorus.IoTorus attribute)@\spxentry{T\_i}\spxextra{nexoclom2.solarsystem.IoTorus.IoTorus attribute}}

\begin{fulllineitems}
\phantomsection\label{\detokenize{autoapi/nexoclom2/solarsystem/IoTorus/index:nexoclom2.solarsystem.IoTorus.IoTorus.T_i}}
\pysigstartsignatures
\pysigline
{\sphinxbfcode{\sphinxupquote{T\_i}}}
\pysigstopsignatures
\end{fulllineitems}

\index{ions (nexoclom2.solarsystem.IoTorus.IoTorus attribute)@\spxentry{ions}\spxextra{nexoclom2.solarsystem.IoTorus.IoTorus attribute}}

\begin{fulllineitems}
\phantomsection\label{\detokenize{autoapi/nexoclom2/solarsystem/IoTorus/index:nexoclom2.solarsystem.IoTorus.IoTorus.ions}}
\pysigstartsignatures
\pysigline
{\sphinxbfcode{\sphinxupquote{ions}}}
\pysigstopsignatures
\end{fulllineitems}

\index{n\_i (nexoclom2.solarsystem.IoTorus.IoTorus attribute)@\spxentry{n\_i}\spxextra{nexoclom2.solarsystem.IoTorus.IoTorus attribute}}

\begin{fulllineitems}
\phantomsection\label{\detokenize{autoapi/nexoclom2/solarsystem/IoTorus/index:nexoclom2.solarsystem.IoTorus.IoTorus.n_i}}
\pysigstartsignatures
\pysigline
{\sphinxbfcode{\sphinxupquote{n\_i}}}
\pysigstopsignatures
\end{fulllineitems}

\index{H\_i (nexoclom2.solarsystem.IoTorus.IoTorus attribute)@\spxentry{H\_i}\spxextra{nexoclom2.solarsystem.IoTorus.IoTorus attribute}}

\begin{fulllineitems}
\phantomsection\label{\detokenize{autoapi/nexoclom2/solarsystem/IoTorus/index:nexoclom2.solarsystem.IoTorus.IoTorus.H_i}}
\pysigstartsignatures
\pysigline
{\sphinxbfcode{\sphinxupquote{H\_i}}}
\pysigstopsignatures
\end{fulllineitems}

\index{planet (nexoclom2.solarsystem.IoTorus.IoTorus attribute)@\spxentry{planet}\spxextra{nexoclom2.solarsystem.IoTorus.IoTorus attribute}}

\begin{fulllineitems}
\phantomsection\label{\detokenize{autoapi/nexoclom2/solarsystem/IoTorus/index:nexoclom2.solarsystem.IoTorus.IoTorus.planet}}
\pysigstartsignatures
\pysigline
{\sphinxbfcode{\sphinxupquote{planet}}}
\pysigstopsignatures
\end{fulllineitems}

\index{xyz\_to\_Mzeta() (nexoclom2.solarsystem.IoTorus.IoTorus method)@\spxentry{xyz\_to\_Mzeta()}\spxextra{nexoclom2.solarsystem.IoTorus.IoTorus method}}

\begin{fulllineitems}
\phantomsection\label{\detokenize{autoapi/nexoclom2/solarsystem/IoTorus/index:nexoclom2.solarsystem.IoTorus.IoTorus.xyz_to_Mzeta}}
\pysigstartsignatures
\pysiglinewithargsret
{\sphinxbfcode{\sphinxupquote{xyz\_to\_Mzeta}}}
{\sphinxparam{\DUrole{n}{x}}\sphinxparamcomma \sphinxparam{\DUrole{n}{y}}\sphinxparamcomma \sphinxparam{\DUrole{n}{z}}\sphinxparamcomma \sphinxparam{\DUrole{n}{cml}}}
{}
\pysigstopsignatures
\end{fulllineitems}

\index{n\_and\_T() (nexoclom2.solarsystem.IoTorus.IoTorus method)@\spxentry{n\_and\_T()}\spxextra{nexoclom2.solarsystem.IoTorus.IoTorus method}}

\begin{fulllineitems}
\phantomsection\label{\detokenize{autoapi/nexoclom2/solarsystem/IoTorus/index:nexoclom2.solarsystem.IoTorus.IoTorus.n_and_T}}
\pysigstartsignatures
\pysiglinewithargsret
{\sphinxbfcode{\sphinxupquote{n\_and\_T}}}
{\sphinxparam{\DUrole{n}{species}}\sphinxparamcomma \sphinxparam{\DUrole{n}{x}}\sphinxparamcomma \sphinxparam{\DUrole{n}{y}}\sphinxparamcomma \sphinxparam{\DUrole{n}{z}}\sphinxparamcomma \sphinxparam{\DUrole{n}{cml}}}
{}
\pysigstopsignatures
\end{fulllineitems}

\index{distance\_along\_field\_line() (nexoclom2.solarsystem.IoTorus.IoTorus method)@\spxentry{distance\_along\_field\_line()}\spxextra{nexoclom2.solarsystem.IoTorus.IoTorus method}}

\begin{fulllineitems}
\phantomsection\label{\detokenize{autoapi/nexoclom2/solarsystem/IoTorus/index:nexoclom2.solarsystem.IoTorus.IoTorus.distance_along_field_line}}
\pysigstartsignatures
\pysiglinewithargsret
{\sphinxbfcode{\sphinxupquote{distance\_along\_field\_line}}}
{\sphinxparam{\DUrole{n}{L}}\sphinxparamcomma \sphinxparam{\DUrole{n}{theta\_c}}\sphinxparamcomma \sphinxparam{\DUrole{n}{theta\_p}}}
{}
\pysigstopsignatures
\sphinxAtStartPar
Calculate the distance along the dipole field line
\begin{quote}\begin{description}
\sphinxlineitem{Parameters}\begin{description}
\sphinxlineitem{\sphinxstylestrong{L: astropy Quantity}}
\sphinxAtStartPar
L (or modified L) \sphinxhyphen{} Field line through magnetic equator

\sphinxlineitem{\sphinxstylestrong{theta\_c: astropy Quantity}}
\sphinxAtStartPar
Angle from magnetic equator to centrifugal equator

\sphinxlineitem{\sphinxstylestrong{theta\_p: astropy Quantity}}
\sphinxAtStartPar
Angle from magnetic equator to packet

\end{description}

\sphinxlineitem{Returns}\begin{description}
\sphinxlineitem{Distance along field line}
\end{description}

\end{description}\end{quote}

\end{fulllineitems}


\end{fulllineitems}


\sphinxstepscope


\subparagraph{nexoclom2.solarsystem.IoTorus\_bak}
\label{\detokenize{autoapi/nexoclom2/solarsystem/IoTorus_bak/index:module-nexoclom2.solarsystem.IoTorus_bak}}\label{\detokenize{autoapi/nexoclom2/solarsystem/IoTorus_bak/index:nexoclom2-solarsystem-iotorus-bak}}\label{\detokenize{autoapi/nexoclom2/solarsystem/IoTorus_bak/index::doc}}\index{module@\spxentry{module}!nexoclom2.solarsystem.IoTorus\_bak@\spxentry{nexoclom2.solarsystem.IoTorus\_bak}}\index{nexoclom2.solarsystem.IoTorus\_bak@\spxentry{nexoclom2.solarsystem.IoTorus\_bak}!module@\spxentry{module}}

\subparagraph{Classes}
\label{\detokenize{autoapi/nexoclom2/solarsystem/IoTorus_bak/index:classes}}

\begin{savenotes}\sphinxattablestart
\sphinxthistablewithglobalstyle
\sphinxthistablewithnovlinesstyle
\centering
\begin{tabulary}{\linewidth}[t]{\X{1}{2}\X{1}{2}}
\sphinxtoprule
\sphinxtableatstartofbodyhook
\sphinxAtStartPar
{\hyperref[\detokenize{autoapi/nexoclom2/solarsystem/IoTorus_bak/index:nexoclom2.solarsystem.IoTorus_bak.IoTorus}]{\sphinxcrossref{\sphinxcode{\sphinxupquote{IoTorus}}}}}
&
\sphinxAtStartPar

\\
\sphinxbottomrule
\end{tabulary}
\sphinxtableafterendhook\par
\sphinxattableend\end{savenotes}


\subparagraph{Module Contents}
\label{\detokenize{autoapi/nexoclom2/solarsystem/IoTorus_bak/index:module-contents}}\index{IoTorus (class in nexoclom2.solarsystem.IoTorus\_bak)@\spxentry{IoTorus}\spxextra{class in nexoclom2.solarsystem.IoTorus\_bak}}

\begin{fulllineitems}
\phantomsection\label{\detokenize{autoapi/nexoclom2/solarsystem/IoTorus_bak/index:nexoclom2.solarsystem.IoTorus_bak.IoTorus}}
\pysigstartsignatures
\pysiglinewithargsret
{\sphinxbfcode{\sphinxupquote{class\DUrole{w}{ }}}\sphinxcode{\sphinxupquote{nexoclom2.solarsystem.IoTorus\_bak.}}\sphinxbfcode{\sphinxupquote{IoTorus}}}
{\sphinxparam{\DUrole{n}{source}\DUrole{o}{=}\DUrole{default_value}{\textquotesingle{}Voyager\textquotesingle{}}}}
{}
\pysigstopsignatures\index{source (nexoclom2.solarsystem.IoTorus\_bak.IoTorus attribute)@\spxentry{source}\spxextra{nexoclom2.solarsystem.IoTorus\_bak.IoTorus attribute}}

\begin{fulllineitems}
\phantomsection\label{\detokenize{autoapi/nexoclom2/solarsystem/IoTorus_bak/index:nexoclom2.solarsystem.IoTorus_bak.IoTorus.source}}
\pysigstartsignatures
\pysigline
{\sphinxbfcode{\sphinxupquote{source}}\sphinxbfcode{\sphinxupquote{\DUrole{w}{ }\DUrole{p}{=}\DUrole{w}{ }\textquotesingle{}Voyager\textquotesingle{}}}}
\pysigstopsignatures
\end{fulllineitems}

\index{M (nexoclom2.solarsystem.IoTorus\_bak.IoTorus attribute)@\spxentry{M}\spxextra{nexoclom2.solarsystem.IoTorus\_bak.IoTorus attribute}}

\begin{fulllineitems}
\phantomsection\label{\detokenize{autoapi/nexoclom2/solarsystem/IoTorus_bak/index:nexoclom2.solarsystem.IoTorus_bak.IoTorus.M}}
\pysigstartsignatures
\pysigline
{\sphinxbfcode{\sphinxupquote{M}}}
\pysigstopsignatures
\end{fulllineitems}

\index{n\_e (nexoclom2.solarsystem.IoTorus\_bak.IoTorus attribute)@\spxentry{n\_e}\spxextra{nexoclom2.solarsystem.IoTorus\_bak.IoTorus attribute}}

\begin{fulllineitems}
\phantomsection\label{\detokenize{autoapi/nexoclom2/solarsystem/IoTorus_bak/index:nexoclom2.solarsystem.IoTorus_bak.IoTorus.n_e}}
\pysigstartsignatures
\pysigline
{\sphinxbfcode{\sphinxupquote{n\_e}}}
\pysigstopsignatures
\end{fulllineitems}

\index{T\_e (nexoclom2.solarsystem.IoTorus\_bak.IoTorus attribute)@\spxentry{T\_e}\spxextra{nexoclom2.solarsystem.IoTorus\_bak.IoTorus attribute}}

\begin{fulllineitems}
\phantomsection\label{\detokenize{autoapi/nexoclom2/solarsystem/IoTorus_bak/index:nexoclom2.solarsystem.IoTorus_bak.IoTorus.T_e}}
\pysigstartsignatures
\pysigline
{\sphinxbfcode{\sphinxupquote{T\_e}}}
\pysigstopsignatures
\end{fulllineitems}

\index{H\_e (nexoclom2.solarsystem.IoTorus\_bak.IoTorus attribute)@\spxentry{H\_e}\spxextra{nexoclom2.solarsystem.IoTorus\_bak.IoTorus attribute}}

\begin{fulllineitems}
\phantomsection\label{\detokenize{autoapi/nexoclom2/solarsystem/IoTorus_bak/index:nexoclom2.solarsystem.IoTorus_bak.IoTorus.H_e}}
\pysigstartsignatures
\pysigline
{\sphinxbfcode{\sphinxupquote{H\_e}}}
\pysigstopsignatures
\end{fulllineitems}

\index{T\_i (nexoclom2.solarsystem.IoTorus\_bak.IoTorus attribute)@\spxentry{T\_i}\spxextra{nexoclom2.solarsystem.IoTorus\_bak.IoTorus attribute}}

\begin{fulllineitems}
\phantomsection\label{\detokenize{autoapi/nexoclom2/solarsystem/IoTorus_bak/index:nexoclom2.solarsystem.IoTorus_bak.IoTorus.T_i}}
\pysigstartsignatures
\pysigline
{\sphinxbfcode{\sphinxupquote{T\_i}}}
\pysigstopsignatures
\end{fulllineitems}

\index{ions (nexoclom2.solarsystem.IoTorus\_bak.IoTorus attribute)@\spxentry{ions}\spxextra{nexoclom2.solarsystem.IoTorus\_bak.IoTorus attribute}}

\begin{fulllineitems}
\phantomsection\label{\detokenize{autoapi/nexoclom2/solarsystem/IoTorus_bak/index:nexoclom2.solarsystem.IoTorus_bak.IoTorus.ions}}
\pysigstartsignatures
\pysigline
{\sphinxbfcode{\sphinxupquote{ions}}}
\pysigstopsignatures
\end{fulllineitems}

\index{jupiter (nexoclom2.solarsystem.IoTorus\_bak.IoTorus attribute)@\spxentry{jupiter}\spxextra{nexoclom2.solarsystem.IoTorus\_bak.IoTorus attribute}}

\begin{fulllineitems}
\phantomsection\label{\detokenize{autoapi/nexoclom2/solarsystem/IoTorus_bak/index:nexoclom2.solarsystem.IoTorus_bak.IoTorus.jupiter}}
\pysigstartsignatures
\pysigline
{\sphinxbfcode{\sphinxupquote{jupiter}}}
\pysigstopsignatures
\end{fulllineitems}

\index{xyz\_to\_Mzeta() (nexoclom2.solarsystem.IoTorus\_bak.IoTorus method)@\spxentry{xyz\_to\_Mzeta()}\spxextra{nexoclom2.solarsystem.IoTorus\_bak.IoTorus method}}

\begin{fulllineitems}
\phantomsection\label{\detokenize{autoapi/nexoclom2/solarsystem/IoTorus_bak/index:nexoclom2.solarsystem.IoTorus_bak.IoTorus.xyz_to_Mzeta}}
\pysigstartsignatures
\pysiglinewithargsret
{\sphinxbfcode{\sphinxupquote{xyz\_to\_Mzeta}}}
{\sphinxparam{\DUrole{n}{x}}\sphinxparamcomma \sphinxparam{\DUrole{n}{y}}\sphinxparamcomma \sphinxparam{\DUrole{n}{z}}\sphinxparamcomma \sphinxparam{\DUrole{n}{cml}}}
{}
\pysigstopsignatures
\end{fulllineitems}

\index{n\_and\_T() (nexoclom2.solarsystem.IoTorus\_bak.IoTorus method)@\spxentry{n\_and\_T()}\spxextra{nexoclom2.solarsystem.IoTorus\_bak.IoTorus method}}

\begin{fulllineitems}
\phantomsection\label{\detokenize{autoapi/nexoclom2/solarsystem/IoTorus_bak/index:nexoclom2.solarsystem.IoTorus_bak.IoTorus.n_and_T}}
\pysigstartsignatures
\pysiglinewithargsret
{\sphinxbfcode{\sphinxupquote{n\_and\_T}}}
{\sphinxparam{\DUrole{n}{species}}\sphinxparamcomma \sphinxparam{\DUrole{n}{x}}\sphinxparamcomma \sphinxparam{\DUrole{n}{y}}\sphinxparamcomma \sphinxparam{\DUrole{n}{z}}\sphinxparamcomma \sphinxparam{\DUrole{n}{cml}}}
{}
\pysigstopsignatures
\end{fulllineitems}


\end{fulllineitems}


\sphinxstepscope


\subparagraph{nexoclom2.solarsystem.SSObject}
\label{\detokenize{autoapi/nexoclom2/solarsystem/SSObject/index:module-nexoclom2.solarsystem.SSObject}}\label{\detokenize{autoapi/nexoclom2/solarsystem/SSObject/index:nexoclom2-solarsystem-ssobject}}\label{\detokenize{autoapi/nexoclom2/solarsystem/SSObject/index::doc}}\index{module@\spxentry{module}!nexoclom2.solarsystem.SSObject@\spxentry{nexoclom2.solarsystem.SSObject}}\index{nexoclom2.solarsystem.SSObject@\spxentry{nexoclom2.solarsystem.SSObject}!module@\spxentry{module}}

\subparagraph{Classes}
\label{\detokenize{autoapi/nexoclom2/solarsystem/SSObject/index:classes}}

\begin{savenotes}\sphinxattablestart
\sphinxthistablewithglobalstyle
\sphinxthistablewithnovlinesstyle
\centering
\begin{tabulary}{\linewidth}[t]{\X{1}{2}\X{1}{2}}
\sphinxtoprule
\sphinxtableatstartofbodyhook
\sphinxAtStartPar
{\hyperref[\detokenize{autoapi/nexoclom2/solarsystem/SSObject/index:nexoclom2.solarsystem.SSObject.SSObject}]{\sphinxcrossref{\sphinxcode{\sphinxupquote{SSObject}}}}}
&
\sphinxAtStartPar
Physical data for solar system bodies.
\\
\sphinxbottomrule
\end{tabulary}
\sphinxtableafterendhook\par
\sphinxattableend\end{savenotes}


\subparagraph{Module Contents}
\label{\detokenize{autoapi/nexoclom2/solarsystem/SSObject/index:module-contents}}\index{SSObject (class in nexoclom2.solarsystem.SSObject)@\spxentry{SSObject}\spxextra{class in nexoclom2.solarsystem.SSObject}}

\begin{fulllineitems}
\phantomsection\label{\detokenize{autoapi/nexoclom2/solarsystem/SSObject/index:nexoclom2.solarsystem.SSObject.SSObject}}
\pysigstartsignatures
\pysiglinewithargsret
{\sphinxbfcode{\sphinxupquote{class\DUrole{w}{ }}}\sphinxcode{\sphinxupquote{nexoclom2.solarsystem.SSObject.}}\sphinxbfcode{\sphinxupquote{SSObject}}}
{\sphinxparam{\DUrole{n}{obj}\DUrole{p}{:}\DUrole{w}{ }\DUrole{n}{str}}}
{}
\pysigstopsignatures\begin{description}
\sphinxlineitem{Physical data for solar system bodies.}
\sphinxAtStartPar
Object containing all the necessary physical data for solar system objects.
Data is stored in a table included with the package. A separate table
contains the NAIF IDs. If the object is not found in the data table, returns
an object with just the object name, type = Unknown, and if possible the
NAIF ID.

\end{description}
\begin{quote}\begin{description}
\sphinxlineitem{Parameters}\begin{description}
\sphinxlineitem{\sphinxstylestrong{obj}}{[}str{]}
\sphinxAtStartPar
Name of the solar system object to gather data for.

\end{description}

\sphinxlineitem{Attributes}\begin{description}
\sphinxlineitem{\sphinxstylestrong{object: str}}\begin{quote}

\sphinxAtStartPar
Name of solar system body. Source: input parameter
\end{quote}
\begin{description}
\sphinxlineitem{orbits: str}
\sphinxAtStartPar
Object the body orbits. Source: PlanetaryConstants.csv

\sphinxlineitem{radius}{[}distance quantity{]}
\sphinxAtStartPar
Object radius. Source: SPICE

\sphinxlineitem{unit: astropy unit}
\sphinxAtStartPar
Named: R\_\textless{}object\textgreater{}

\sphinxlineitem{GM: Quantity}
\sphinxAtStartPar
Mass times gravitational constant. Source: SPICE

\sphinxlineitem{mass: mass quantity}
\sphinxAtStartPar
Object mass in kg. Source: GM from SPICE

\sphinxlineitem{a: distance quantity}
\sphinxAtStartPar
Object semi\sphinxhyphen{}major axis. Source: SPICE

\sphinxlineitem{e: float}
\sphinxAtStartPar
Orbital eccentricity. For planets: Source SPICE. For moons: Set to 0.
This only affects calculations when a modeltime is not specified and
is a small affect.

\sphinxlineitem{tilt: angle quantity}
\sphinxAtStartPar
Tilt of rotation axis relative to ecliptic in degrees.
Source: PlanetaryConstants.csv

\sphinxlineitem{rotperiod: time quantity}
\sphinxAtStartPar
Siderial rotational period in hours. Source: PlanetaryConstants.csv

\sphinxlineitem{orbperiod: time quantity}
\sphinxAtStartPar
Sideral orbital period. Source: SPICE

\sphinxlineitem{orbvel: velocity quantity}
\sphinxAtStartPar
:math:{\color{red}\bfseries{}\textasciigrave{}}v\_\{orb\} =

\end{description}

\sphinxlineitem{\sphinxstylestrong{rac\{2 pi a\}\{orbperiod\}\textasciigrave{}}}\begin{description}
\sphinxlineitem{satellites: list of str or None}
\sphinxAtStartPar
List of satellites of the body. Source: PlanetaryConstants.csv

\sphinxlineitem{type}{[}\{‘Star’, ‘Planet’, or ‘Moon’\}{]}
\sphinxAtStartPar
Source: PlanetaryConstants.csv

\sphinxlineitem{naifid}{[}int{]}
\sphinxAtStartPar
Source: naifids.csv

\end{description}

\end{description}

\end{description}\end{quote}
\subsubsection*{Examples}

\begin{sphinxVerbatim}[commandchars=\\\{\}]
\PYG{g+gp}{\PYGZgt{}\PYGZgt{}\PYGZgt{} }\PYG{k+kn}{from}\PYG{+w}{ }\PYG{n+nn}{nexoclom2}\PYG{n+nn}{.}\PYG{n+nn}{solarsystem}\PYG{+w}{ }\PYG{k+kn}{import} \PYG{n}{SSObject}
\PYG{g+gp}{\PYGZgt{}\PYGZgt{}\PYGZgt{} }\PYG{n}{jupiter} \PYG{o}{=} \PYG{n}{SSObject}\PYG{p}{(}\PYG{l+s+s1}{\PYGZsq{}}\PYG{l+s+s1}{Jupiter}\PYG{l+s+s1}{\PYGZsq{}}\PYG{p}{)}
\PYG{g+gp}{\PYGZgt{}\PYGZgt{}\PYGZgt{} }\PYG{n+nb}{print}\PYG{p}{(}\PYG{n}{jupiter}\PYG{p}{)}
\PYG{g+go}{Object: Jupiter}
\PYG{g+go}{Type = Planet}
\PYG{g+go}{Orbits Sun}
\PYG{g+go}{Satellites: Io, Europa, Ganymede, Callisto}
\PYG{g+go}{Radius = 71492.00 km}
\PYG{g+go}{Mass = 1.90e+27 kg}
\PYG{g+go}{a = 5.20 AU}
\PYG{g+go}{Eccentricity = 0.05}
\PYG{g+go}{Tilt = 3.08 deg}
\PYG{g+go}{Rotation Period = 9.93 h}
\PYG{g+go}{Orbital Period = 4333.00 d}
\PYG{g+go}{GM = \PYGZhy{}1.27e+17 m3 / s2}
\PYG{g+go}{NAIFID = 599}
\PYG{g+gp}{\PYGZgt{}\PYGZgt{}\PYGZgt{} }\PYG{n+nb}{print}\PYG{p}{(}\PYG{n+nb}{len}\PYG{p}{(}\PYG{n}{jupiter}\PYG{p}{)}\PYG{p}{)}
\PYG{g+go}{5}
\PYG{g+gp}{\PYGZgt{}\PYGZgt{}\PYGZgt{} }\PYG{n}{hst} \PYG{o}{=} \PYG{n}{SSObject}\PYG{p}{(}\PYG{l+s+s1}{\PYGZsq{}}\PYG{l+s+s1}{HST}\PYG{l+s+s1}{\PYGZsq{}}\PYG{p}{)}
\PYG{g+gp}{\PYGZgt{}\PYGZgt{}\PYGZgt{} }\PYG{n+nb}{print}\PYG{p}{(}\PYG{n}{hst}\PYG{p}{)}
\PYG{g+go}{Object: Hst}
\PYG{g+go}{Type = Unknown}
\PYG{g+go}{NAIFID = \PYGZhy{}48}
\PYG{g+gp}{\PYGZgt{}\PYGZgt{}\PYGZgt{} }\PYG{n+nb}{print}\PYG{p}{(}\PYG{n}{jupiter} \PYG{o}{==} \PYG{n}{hst}\PYG{p}{)}
\PYG{g+go}{False}
\end{sphinxVerbatim}
\begin{quote}\begin{description}
\sphinxlineitem{Authors}
\sphinxAtStartPar
Matthew Burger

\end{description}\end{quote}
\index{object (nexoclom2.solarsystem.SSObject.SSObject attribute)@\spxentry{object}\spxextra{nexoclom2.solarsystem.SSObject.SSObject attribute}}

\begin{fulllineitems}
\phantomsection\label{\detokenize{autoapi/nexoclom2/solarsystem/SSObject/index:nexoclom2.solarsystem.SSObject.SSObject.object}}
\pysigstartsignatures
\pysigline
{\sphinxbfcode{\sphinxupquote{object}}}
\pysigstopsignatures
\end{fulllineitems}

\index{\_\_eq\_\_() (nexoclom2.solarsystem.SSObject.SSObject method)@\spxentry{\_\_eq\_\_()}\spxextra{nexoclom2.solarsystem.SSObject.SSObject method}}

\begin{fulllineitems}
\phantomsection\label{\detokenize{autoapi/nexoclom2/solarsystem/SSObject/index:nexoclom2.solarsystem.SSObject.SSObject.__eq__}}
\pysigstartsignatures
\pysiglinewithargsret
{\sphinxbfcode{\sphinxupquote{\_\_eq\_\_}}}
{\sphinxparam{\DUrole{n}{other}}}
{}
\pysigstopsignatures
\end{fulllineitems}

\index{\_\_len\_\_() (nexoclom2.solarsystem.SSObject.SSObject method)@\spxentry{\_\_len\_\_()}\spxextra{nexoclom2.solarsystem.SSObject.SSObject method}}

\begin{fulllineitems}
\phantomsection\label{\detokenize{autoapi/nexoclom2/solarsystem/SSObject/index:nexoclom2.solarsystem.SSObject.SSObject.__len__}}
\pysigstartsignatures
\pysiglinewithargsret
{\sphinxbfcode{\sphinxupquote{\_\_len\_\_}}}
{}
{}
\pysigstopsignatures
\sphinxAtStartPar
Returns number of satellites + 1

\end{fulllineitems}

\index{\_\_repr\_\_() (nexoclom2.solarsystem.SSObject.SSObject method)@\spxentry{\_\_repr\_\_()}\spxextra{nexoclom2.solarsystem.SSObject.SSObject method}}

\begin{fulllineitems}
\phantomsection\label{\detokenize{autoapi/nexoclom2/solarsystem/SSObject/index:nexoclom2.solarsystem.SSObject.SSObject.__repr__}}
\pysigstartsignatures
\pysiglinewithargsret
{\sphinxbfcode{\sphinxupquote{\_\_repr\_\_}}}
{}
{}
\pysigstopsignatures
\end{fulllineitems}

\index{\_\_str\_\_() (nexoclom2.solarsystem.SSObject.SSObject method)@\spxentry{\_\_str\_\_()}\spxextra{nexoclom2.solarsystem.SSObject.SSObject method}}

\begin{fulllineitems}
\phantomsection\label{\detokenize{autoapi/nexoclom2/solarsystem/SSObject/index:nexoclom2.solarsystem.SSObject.SSObject.__str__}}
\pysigstartsignatures
\pysiglinewithargsret
{\sphinxbfcode{\sphinxupquote{\_\_str\_\_}}}
{}
{}
\pysigstopsignatures
\end{fulllineitems}


\end{fulllineitems}


\sphinxstepscope


\subparagraph{nexoclom2.solarsystem.SSObject\_bak}
\label{\detokenize{autoapi/nexoclom2/solarsystem/SSObject_bak/index:module-nexoclom2.solarsystem.SSObject_bak}}\label{\detokenize{autoapi/nexoclom2/solarsystem/SSObject_bak/index:nexoclom2-solarsystem-ssobject-bak}}\label{\detokenize{autoapi/nexoclom2/solarsystem/SSObject_bak/index::doc}}\index{module@\spxentry{module}!nexoclom2.solarsystem.SSObject\_bak@\spxentry{nexoclom2.solarsystem.SSObject\_bak}}\index{nexoclom2.solarsystem.SSObject\_bak@\spxentry{nexoclom2.solarsystem.SSObject\_bak}!module@\spxentry{module}}

\subparagraph{Classes}
\label{\detokenize{autoapi/nexoclom2/solarsystem/SSObject_bak/index:classes}}

\begin{savenotes}\sphinxattablestart
\sphinxthistablewithglobalstyle
\sphinxthistablewithnovlinesstyle
\centering
\begin{tabulary}{\linewidth}[t]{\X{1}{2}\X{1}{2}}
\sphinxtoprule
\sphinxtableatstartofbodyhook
\sphinxAtStartPar
{\hyperref[\detokenize{autoapi/nexoclom2/solarsystem/SSObject_bak/index:nexoclom2.solarsystem.SSObject_bak.SSObject}]{\sphinxcrossref{\sphinxcode{\sphinxupquote{SSObject}}}}}
&
\sphinxAtStartPar
Physical data for solar system bodies.
\\
\sphinxbottomrule
\end{tabulary}
\sphinxtableafterendhook\par
\sphinxattableend\end{savenotes}


\subparagraph{Module Contents}
\label{\detokenize{autoapi/nexoclom2/solarsystem/SSObject_bak/index:module-contents}}\index{SSObject (class in nexoclom2.solarsystem.SSObject\_bak)@\spxentry{SSObject}\spxextra{class in nexoclom2.solarsystem.SSObject\_bak}}

\begin{fulllineitems}
\phantomsection\label{\detokenize{autoapi/nexoclom2/solarsystem/SSObject_bak/index:nexoclom2.solarsystem.SSObject_bak.SSObject}}
\pysigstartsignatures
\pysiglinewithargsret
{\sphinxbfcode{\sphinxupquote{class\DUrole{w}{ }}}\sphinxcode{\sphinxupquote{nexoclom2.solarsystem.SSObject\_bak.}}\sphinxbfcode{\sphinxupquote{SSObject}}}
{\sphinxparam{\DUrole{n}{obj}\DUrole{p}{:}\DUrole{w}{ }\DUrole{n}{str}}}
{}
\pysigstopsignatures
\sphinxAtStartPar
Physical data for solar system bodies.
Object containing all the necessary physical data for solar system objects.
Data is stored in a table included with the package. A separate table
contains the NAIF IDs. If the object is not found in the data table, returns
an object with just the object name, type = Unknown, and if possible the
NAIF ID.
\begin{quote}\begin{description}
\sphinxlineitem{Parameters}\begin{description}
\sphinxlineitem{\sphinxstylestrong{obj}}{[}str{]}
\sphinxAtStartPar
Name of the solar system object to gather data for.

\end{description}

\sphinxlineitem{Attributes}\begin{description}
\sphinxlineitem{\sphinxstylestrong{object}}{[}str{]}
\sphinxAtStartPar
Name of solar system body.

\sphinxlineitem{\sphinxstylestrong{orbits}}{[}str{]}
\sphinxAtStartPar
Object the body orbits.

\sphinxlineitem{\sphinxstylestrong{radius}}{[}distance quantity{]}
\sphinxAtStartPar
Object radius in km.

\sphinxlineitem{\sphinxstylestrong{mass}}{[}mass quantity{]}
\sphinxAtStartPar
Object mass in kg

\sphinxlineitem{\sphinxstylestrong{a}}{[}distance quantity{]}
\sphinxAtStartPar
Object semi\sphinxhyphen{}major axis in km for Sun and moons; au for planets

\sphinxlineitem{\sphinxstylestrong{e}}{[}float{]}
\sphinxAtStartPar
Orbital eccentricity

\sphinxlineitem{\sphinxstylestrong{tilt}}{[}angle quantity{]}
\sphinxAtStartPar
Tilt of rotation axis relative to ecliptic in degrees

\sphinxlineitem{\sphinxstylestrong{rotperiod}}{[}time quantity{]}
\sphinxAtStartPar
Siderial rotational period in hours

\sphinxlineitem{\sphinxstylestrong{orbperiod}}{[}time quantity{]}
\sphinxAtStartPar
Sideral orbital period (days)

\sphinxlineitem{\sphinxstylestrong{GM}}{[}Quantity{]}
\sphinxAtStartPar
Mass time gravitational constant in km\textasciicircum{}3 s\textasciicircum{}\sphinxhyphen{}2.

\sphinxlineitem{\sphinxstylestrong{satellites}}{[}list of str or None{]}
\sphinxAtStartPar
List of satellites of the body

\sphinxlineitem{\sphinxstylestrong{type}}{[}\{‘Star’, ‘Planet’, or ‘Moon’\}{]}
\end{description}

\end{description}\end{quote}
\subsubsection*{Methods}


\begin{savenotes}\sphinxattablestart
\sphinxthistablewithglobalstyle
\centering
\begin{tabulary}{\linewidth}[t]{TT}
\sphinxtoprule
\sphinxtableatstartofbodyhook&\\
\sphinxbottomrule
\end{tabulary}
\sphinxtableafterendhook\par
\sphinxattableend\end{savenotes}
\subsubsection*{Notes}
\begin{itemize}
\item {} 
\sphinxAtStartPar
subsolar\_longitude and subsolar\_latitude for moons refers to the planet’s

\end{itemize}

\sphinxAtStartPar
subsolar longitude and latitude. If, for example running an Io model
centered at Io, need to know Jupiter’s CML.

\sphinxAtStartPar
References for physical data (Last Verified TBD):

\sphinxAtStartPar
\sphinxstylestrong{TO DO}
\begin{itemize}
\item {} 
\sphinxAtStartPar
Verify data and cite references

\item {} 
\sphinxAtStartPar
Support for objects not included in the predefined table. Extract

\end{itemize}

\sphinxAtStartPar
information from the SPICE kernels if possible.
\begin{itemize}
\item {} 
\sphinxAtStartPar
Could move gravitational acceleration calculation here

\end{itemize}

\sphinxAtStartPar
NAIF IDS found at JPL’s \sphinxhref{https://naif.jpl.nasa.gov/pub/naif/toolkit\_docs/C/req/naif\_ids.html}{Navigation and Ancillary Information
Facility}.
\subsubsection*{Examples}

\begin{sphinxVerbatim}[commandchars=\\\{\}]
\PYG{g+gp}{\PYGZgt{}\PYGZgt{}\PYGZgt{} }\PYG{k+kn}{from}\PYG{+w}{ }\PYG{n+nn}{nexoclom2}\PYG{n+nn}{.}\PYG{n+nn}{solarsystem}\PYG{+w}{ }\PYG{k+kn}{import} \PYG{n}{SSObject}
\PYG{g+gp}{\PYGZgt{}\PYGZgt{}\PYGZgt{} }\PYG{n}{jupiter} \PYG{o}{=} \PYG{n}{SSObject}\PYG{p}{(}\PYG{l+s+s1}{\PYGZsq{}}\PYG{l+s+s1}{Jupiter}\PYG{l+s+s1}{\PYGZsq{}}\PYG{p}{)}
\PYG{g+gp}{\PYGZgt{}\PYGZgt{}\PYGZgt{} }\PYG{n+nb}{print}\PYG{p}{(}\PYG{n}{jupiter}\PYG{p}{)}
\PYG{g+go}{Object: Jupiter}
\PYG{g+go}{Type = Planet}
\PYG{g+go}{Orbits Sun}
\PYG{g+go}{Satellites: Io, Europa, Ganymede, Callisto}
\PYG{g+go}{Radius = 71492.00 km}
\PYG{g+go}{Mass = 1.90e+27 kg}
\PYG{g+go}{a = 5.20 AU}
\PYG{g+go}{Eccentricity = 0.05}
\PYG{g+go}{Tilt = 3.08 deg}
\PYG{g+go}{Rotation Period = 9.93 h}
\PYG{g+go}{Orbital Period = 4333.00 d}
\PYG{g+go}{GM = \PYGZhy{}1.27e+17 m3 / s2}
\PYG{g+go}{NAIFID = 599}
\PYG{g+gp}{\PYGZgt{}\PYGZgt{}\PYGZgt{} }\PYG{n+nb}{print}\PYG{p}{(}\PYG{n+nb}{len}\PYG{p}{(}\PYG{n}{jupiter}\PYG{p}{)}\PYG{p}{)}
\PYG{g+go}{5}
\PYG{g+gp}{\PYGZgt{}\PYGZgt{}\PYGZgt{} }\PYG{n}{hst} \PYG{o}{=} \PYG{n}{SSObject}\PYG{p}{(}\PYG{l+s+s1}{\PYGZsq{}}\PYG{l+s+s1}{HST}\PYG{l+s+s1}{\PYGZsq{}}\PYG{p}{)}
\PYG{g+gp}{\PYGZgt{}\PYGZgt{}\PYGZgt{} }\PYG{n+nb}{print}\PYG{p}{(}\PYG{n}{hst}\PYG{p}{)}
\PYG{g+go}{Object: Hst}
\PYG{g+go}{Type = Unknown}
\PYG{g+go}{NAIFID = \PYGZhy{}48}
\PYG{g+gp}{\PYGZgt{}\PYGZgt{}\PYGZgt{} }\PYG{n+nb}{print}\PYG{p}{(}\PYG{n}{jupiter} \PYG{o}{==} \PYG{n}{hst}\PYG{p}{)}
\PYG{g+go}{False}
\end{sphinxVerbatim}
\begin{quote}\begin{description}
\sphinxlineitem{Authors}
\sphinxAtStartPar
Matthew Burger

\end{description}\end{quote}
\index{object (nexoclom2.solarsystem.SSObject\_bak.SSObject attribute)@\spxentry{object}\spxextra{nexoclom2.solarsystem.SSObject\_bak.SSObject attribute}}

\begin{fulllineitems}
\phantomsection\label{\detokenize{autoapi/nexoclom2/solarsystem/SSObject_bak/index:nexoclom2.solarsystem.SSObject_bak.SSObject.object}}
\pysigstartsignatures
\pysigline
{\sphinxbfcode{\sphinxupquote{object}}}
\pysigstopsignatures
\end{fulllineitems}

\index{\_\_eq\_\_() (nexoclom2.solarsystem.SSObject\_bak.SSObject method)@\spxentry{\_\_eq\_\_()}\spxextra{nexoclom2.solarsystem.SSObject\_bak.SSObject method}}

\begin{fulllineitems}
\phantomsection\label{\detokenize{autoapi/nexoclom2/solarsystem/SSObject_bak/index:nexoclom2.solarsystem.SSObject_bak.SSObject.__eq__}}
\pysigstartsignatures
\pysiglinewithargsret
{\sphinxbfcode{\sphinxupquote{\_\_eq\_\_}}}
{\sphinxparam{\DUrole{n}{other}}}
{}
\pysigstopsignatures
\end{fulllineitems}

\index{\_\_len\_\_() (nexoclom2.solarsystem.SSObject\_bak.SSObject method)@\spxentry{\_\_len\_\_()}\spxextra{nexoclom2.solarsystem.SSObject\_bak.SSObject method}}

\begin{fulllineitems}
\phantomsection\label{\detokenize{autoapi/nexoclom2/solarsystem/SSObject_bak/index:nexoclom2.solarsystem.SSObject_bak.SSObject.__len__}}
\pysigstartsignatures
\pysiglinewithargsret
{\sphinxbfcode{\sphinxupquote{\_\_len\_\_}}}
{}
{}
\pysigstopsignatures
\sphinxAtStartPar
Returns number of satellites + 1

\end{fulllineitems}

\index{\_\_repr\_\_() (nexoclom2.solarsystem.SSObject\_bak.SSObject method)@\spxentry{\_\_repr\_\_()}\spxextra{nexoclom2.solarsystem.SSObject\_bak.SSObject method}}

\begin{fulllineitems}
\phantomsection\label{\detokenize{autoapi/nexoclom2/solarsystem/SSObject_bak/index:nexoclom2.solarsystem.SSObject_bak.SSObject.__repr__}}
\pysigstartsignatures
\pysiglinewithargsret
{\sphinxbfcode{\sphinxupquote{\_\_repr\_\_}}}
{}
{}
\pysigstopsignatures
\end{fulllineitems}

\index{\_\_str\_\_() (nexoclom2.solarsystem.SSObject\_bak.SSObject method)@\spxentry{\_\_str\_\_()}\spxextra{nexoclom2.solarsystem.SSObject\_bak.SSObject method}}

\begin{fulllineitems}
\phantomsection\label{\detokenize{autoapi/nexoclom2/solarsystem/SSObject_bak/index:nexoclom2.solarsystem.SSObject_bak.SSObject.__str__}}
\pysigstartsignatures
\pysiglinewithargsret
{\sphinxbfcode{\sphinxupquote{\_\_str\_\_}}}
{}
{}
\pysigstopsignatures
\end{fulllineitems}

\index{get\_geometry() (nexoclom2.solarsystem.SSObject\_bak.SSObject method)@\spxentry{get\_geometry()}\spxextra{nexoclom2.solarsystem.SSObject\_bak.SSObject method}}

\begin{fulllineitems}
\phantomsection\label{\detokenize{autoapi/nexoclom2/solarsystem/SSObject_bak/index:nexoclom2.solarsystem.SSObject_bak.SSObject.get_geometry}}
\pysigstartsignatures
\pysiglinewithargsret
{\sphinxbfcode{\sphinxupquote{get\_geometry}}}
{\sphinxparam{\DUrole{n}{geometry}}\sphinxparamcomma \sphinxparam{\DUrole{n}{runtime}}}
{}
\pysigstopsignatures
\sphinxAtStartPar
Determines the proper coordinates and solar direction, etc.
Cases:
\begin{enumerate}
\sphinxsetlistlabels{\arabic}{enumi}{enumii}{}{.}%
\item {} 
\sphinxAtStartPar
Star

\item {} 
\sphinxAtStartPar
GeometryTime, self.type == planet, geometry.center == self.orbits

\item {} 
\sphinxAtStartPar
GeometryTime, self.type == planet, geometry.center == self.object

\item {} 
\sphinxAtStartPar
GeometryTime, self.type == moon, geometry.center == self.orbits

\item {} 
\sphinxAtStartPar
GeometryTime, self.type == moon, geometry.center == self.object

\item {} 
\sphinxAtStartPar
GeometryNoTime, self.type == planet, geometry.center == self.orbits

\item {} 
\sphinxAtStartPar
GeometryNoTime, self.type == planet, geometry.center == self.object

\item {} 
\sphinxAtStartPar
GeometryNoTime, self.type == moon, geometry.center == self.orbits

\item {} 
\sphinxAtStartPar
GeometryNoTime, self.type == moon, geometry.center == self.object

\end{enumerate}
\subsubsection*{Notes}

\sphinxAtStartPar
Radial velocity for determining g\sphinxhyphen{}value is based on the center object.
Moons need to include radial velocity including orbital motion around
planet.

\end{fulllineitems}

\index{\_geometry\_star() (nexoclom2.solarsystem.SSObject\_bak.SSObject method)@\spxentry{\_geometry\_star()}\spxextra{nexoclom2.solarsystem.SSObject\_bak.SSObject method}}

\begin{fulllineitems}
\phantomsection\label{\detokenize{autoapi/nexoclom2/solarsystem/SSObject_bak/index:nexoclom2.solarsystem.SSObject_bak.SSObject._geometry_star}}
\pysigstartsignatures
\pysiglinewithargsret
{\sphinxbfcode{\sphinxupquote{\_geometry\_star}}}
{\sphinxparam{\DUrole{n}{geometry}}}
{}
\pysigstopsignatures
\end{fulllineitems}

\index{\_geometry\_modeltime() (nexoclom2.solarsystem.SSObject\_bak.SSObject method)@\spxentry{\_geometry\_modeltime()}\spxextra{nexoclom2.solarsystem.SSObject\_bak.SSObject method}}

\begin{fulllineitems}
\phantomsection\label{\detokenize{autoapi/nexoclom2/solarsystem/SSObject_bak/index:nexoclom2.solarsystem.SSObject_bak.SSObject._geometry_modeltime}}
\pysigstartsignatures
\pysiglinewithargsret
{\sphinxbfcode{\sphinxupquote{\_geometry\_modeltime}}}
{\sphinxparam{\DUrole{n}{geometry}}}
{}
\pysigstopsignatures
\end{fulllineitems}

\index{\_geometry\_notime() (nexoclom2.solarsystem.SSObject\_bak.SSObject method)@\spxentry{\_geometry\_notime()}\spxextra{nexoclom2.solarsystem.SSObject\_bak.SSObject method}}

\begin{fulllineitems}
\phantomsection\label{\detokenize{autoapi/nexoclom2/solarsystem/SSObject_bak/index:nexoclom2.solarsystem.SSObject_bak.SSObject._geometry_notime}}
\pysigstartsignatures
\pysiglinewithargsret
{\sphinxbfcode{\sphinxupquote{\_geometry\_notime}}}
{\sphinxparam{\DUrole{n}{geometry}}}
{}
\pysigstopsignatures
\end{fulllineitems}

\index{out\_of\_shadow() (nexoclom2.solarsystem.SSObject\_bak.SSObject method)@\spxentry{out\_of\_shadow()}\spxextra{nexoclom2.solarsystem.SSObject\_bak.SSObject method}}

\begin{fulllineitems}
\phantomsection\label{\detokenize{autoapi/nexoclom2/solarsystem/SSObject_bak/index:nexoclom2.solarsystem.SSObject_bak.SSObject.out_of_shadow}}
\pysigstartsignatures
\pysiglinewithargsret
{\sphinxbfcode{\sphinxupquote{out\_of\_shadow}}}
{\sphinxparam{\DUrole{n}{packets}}}
{}
\pysigstopsignatures
\end{fulllineitems}


\end{fulllineitems}


\sphinxstepscope


\subparagraph{nexoclom2.solarsystem.SSPosition}
\label{\detokenize{autoapi/nexoclom2/solarsystem/SSPosition/index:module-nexoclom2.solarsystem.SSPosition}}\label{\detokenize{autoapi/nexoclom2/solarsystem/SSPosition/index:nexoclom2-solarsystem-ssposition}}\label{\detokenize{autoapi/nexoclom2/solarsystem/SSPosition/index::doc}}\index{module@\spxentry{module}!nexoclom2.solarsystem.SSPosition@\spxentry{nexoclom2.solarsystem.SSPosition}}\index{nexoclom2.solarsystem.SSPosition@\spxentry{nexoclom2.solarsystem.SSPosition}!module@\spxentry{module}}

\subparagraph{Classes}
\label{\detokenize{autoapi/nexoclom2/solarsystem/SSPosition/index:classes}}

\begin{savenotes}\sphinxattablestart
\sphinxthistablewithglobalstyle
\sphinxthistablewithnovlinesstyle
\centering
\begin{tabulary}{\linewidth}[t]{\X{1}{2}\X{1}{2}}
\sphinxtoprule
\sphinxtableatstartofbodyhook
\sphinxAtStartPar
{\hyperref[\detokenize{autoapi/nexoclom2/solarsystem/SSPosition/index:nexoclom2.solarsystem.SSPosition.SSPosition}]{\sphinxcrossref{\sphinxcode{\sphinxupquote{SSPosition}}}}}
&
\sphinxAtStartPar
Baseclass for Solar System Positions
\\
\sphinxbottomrule
\end{tabulary}
\sphinxtableafterendhook\par
\sphinxattableend\end{savenotes}


\subparagraph{Module Contents}
\label{\detokenize{autoapi/nexoclom2/solarsystem/SSPosition/index:module-contents}}\index{SSPosition (class in nexoclom2.solarsystem.SSPosition)@\spxentry{SSPosition}\spxextra{class in nexoclom2.solarsystem.SSPosition}}

\begin{fulllineitems}
\phantomsection\label{\detokenize{autoapi/nexoclom2/solarsystem/SSPosition/index:nexoclom2.solarsystem.SSPosition.SSPosition}}
\pysigstartsignatures
\pysiglinewithargsret
{\sphinxbfcode{\sphinxupquote{class\DUrole{w}{ }}}\sphinxcode{\sphinxupquote{nexoclom2.solarsystem.SSPosition.}}\sphinxbfcode{\sphinxupquote{SSPosition}}}
{\sphinxparam{\DUrole{n}{objname}}\sphinxparamcomma \sphinxparam{\DUrole{n}{geometry}}\sphinxparamcomma \sphinxparam{\DUrole{n}{runtime}}}
{}
\pysigstopsignatures
\sphinxAtStartPar
Baseclass for Solar System Positions
\begin{quote}\begin{description}
\sphinxlineitem{Parameters}\begin{description}
\sphinxlineitem{\sphinxstylestrong{ssobject: SSObject}}
\sphinxlineitem{\sphinxstylestrong{geometry: Geometry object}}
\sphinxlineitem{\sphinxstylestrong{runtime: time Quantity}}
\sphinxAtStartPar
Total model simulation time

\end{description}

\end{description}\end{quote}
\index{object (nexoclom2.solarsystem.SSPosition.SSPosition attribute)@\spxentry{object}\spxextra{nexoclom2.solarsystem.SSPosition.SSPosition attribute}}

\begin{fulllineitems}
\phantomsection\label{\detokenize{autoapi/nexoclom2/solarsystem/SSPosition/index:nexoclom2.solarsystem.SSPosition.SSPosition.object}}
\pysigstartsignatures
\pysigline
{\sphinxbfcode{\sphinxupquote{object}}}
\pysigstopsignatures
\end{fulllineitems}

\index{runtime (nexoclom2.solarsystem.SSPosition.SSPosition attribute)@\spxentry{runtime}\spxextra{nexoclom2.solarsystem.SSPosition.SSPosition attribute}}

\begin{fulllineitems}
\phantomsection\label{\detokenize{autoapi/nexoclom2/solarsystem/SSPosition/index:nexoclom2.solarsystem.SSPosition.SSPosition.runtime}}
\pysigstartsignatures
\pysigline
{\sphinxbfcode{\sphinxupquote{runtime}}}
\pysigstopsignatures
\end{fulllineitems}

\index{unit (nexoclom2.solarsystem.SSPosition.SSPosition attribute)@\spxentry{unit}\spxextra{nexoclom2.solarsystem.SSPosition.SSPosition attribute}}

\begin{fulllineitems}
\phantomsection\label{\detokenize{autoapi/nexoclom2/solarsystem/SSPosition/index:nexoclom2.solarsystem.SSPosition.SSPosition.unit}}
\pysigstartsignatures
\pysigline
{\sphinxbfcode{\sphinxupquote{unit}}\sphinxbfcode{\sphinxupquote{\DUrole{w}{ }\DUrole{p}{=}\DUrole{w}{ }None}}}
\pysigstopsignatures
\end{fulllineitems}

\index{taa (nexoclom2.solarsystem.SSPosition.SSPosition attribute)@\spxentry{taa}\spxextra{nexoclom2.solarsystem.SSPosition.SSPosition attribute}}

\begin{fulllineitems}
\phantomsection\label{\detokenize{autoapi/nexoclom2/solarsystem/SSPosition/index:nexoclom2.solarsystem.SSPosition.SSPosition.taa}}
\pysigstartsignatures
\pysigline
{\sphinxbfcode{\sphinxupquote{taa}}}
\pysigstopsignatures
\end{fulllineitems}

\index{phi (nexoclom2.solarsystem.SSPosition.SSPosition attribute)@\spxentry{phi}\spxextra{nexoclom2.solarsystem.SSPosition.SSPosition attribute}}

\begin{fulllineitems}
\phantomsection\label{\detokenize{autoapi/nexoclom2/solarsystem/SSPosition/index:nexoclom2.solarsystem.SSPosition.SSPosition.phi}}
\pysigstartsignatures
\pysigline
{\sphinxbfcode{\sphinxupquote{phi}}}
\pysigstopsignatures
\end{fulllineitems}

\index{x (nexoclom2.solarsystem.SSPosition.SSPosition attribute)@\spxentry{x}\spxextra{nexoclom2.solarsystem.SSPosition.SSPosition attribute}}

\begin{fulllineitems}
\phantomsection\label{\detokenize{autoapi/nexoclom2/solarsystem/SSPosition/index:nexoclom2.solarsystem.SSPosition.SSPosition.x}}
\pysigstartsignatures
\pysigline
{\sphinxbfcode{\sphinxupquote{x}}}
\pysigstopsignatures
\end{fulllineitems}

\index{y (nexoclom2.solarsystem.SSPosition.SSPosition attribute)@\spxentry{y}\spxextra{nexoclom2.solarsystem.SSPosition.SSPosition attribute}}

\begin{fulllineitems}
\phantomsection\label{\detokenize{autoapi/nexoclom2/solarsystem/SSPosition/index:nexoclom2.solarsystem.SSPosition.SSPosition.y}}
\pysigstartsignatures
\pysigline
{\sphinxbfcode{\sphinxupquote{y}}}
\pysigstopsignatures
\end{fulllineitems}

\index{z (nexoclom2.solarsystem.SSPosition.SSPosition attribute)@\spxentry{z}\spxextra{nexoclom2.solarsystem.SSPosition.SSPosition attribute}}

\begin{fulllineitems}
\phantomsection\label{\detokenize{autoapi/nexoclom2/solarsystem/SSPosition/index:nexoclom2.solarsystem.SSPosition.SSPosition.z}}
\pysigstartsignatures
\pysigline
{\sphinxbfcode{\sphinxupquote{z}}}
\pysigstopsignatures
\end{fulllineitems}

\index{r (nexoclom2.solarsystem.SSPosition.SSPosition attribute)@\spxentry{r}\spxextra{nexoclom2.solarsystem.SSPosition.SSPosition attribute}}

\begin{fulllineitems}
\phantomsection\label{\detokenize{autoapi/nexoclom2/solarsystem/SSPosition/index:nexoclom2.solarsystem.SSPosition.SSPosition.r}}
\pysigstartsignatures
\pysigline
{\sphinxbfcode{\sphinxupquote{r}}}
\pysigstopsignatures
\end{fulllineitems}

\index{vx (nexoclom2.solarsystem.SSPosition.SSPosition attribute)@\spxentry{vx}\spxextra{nexoclom2.solarsystem.SSPosition.SSPosition attribute}}

\begin{fulllineitems}
\phantomsection\label{\detokenize{autoapi/nexoclom2/solarsystem/SSPosition/index:nexoclom2.solarsystem.SSPosition.SSPosition.vx}}
\pysigstartsignatures
\pysigline
{\sphinxbfcode{\sphinxupquote{vx}}}
\pysigstopsignatures
\end{fulllineitems}

\index{vy (nexoclom2.solarsystem.SSPosition.SSPosition attribute)@\spxentry{vy}\spxextra{nexoclom2.solarsystem.SSPosition.SSPosition attribute}}

\begin{fulllineitems}
\phantomsection\label{\detokenize{autoapi/nexoclom2/solarsystem/SSPosition/index:nexoclom2.solarsystem.SSPosition.SSPosition.vy}}
\pysigstartsignatures
\pysigline
{\sphinxbfcode{\sphinxupquote{vy}}}
\pysigstopsignatures
\end{fulllineitems}

\index{vz (nexoclom2.solarsystem.SSPosition.SSPosition attribute)@\spxentry{vz}\spxextra{nexoclom2.solarsystem.SSPosition.SSPosition attribute}}

\begin{fulllineitems}
\phantomsection\label{\detokenize{autoapi/nexoclom2/solarsystem/SSPosition/index:nexoclom2.solarsystem.SSPosition.SSPosition.vz}}
\pysigstartsignatures
\pysigline
{\sphinxbfcode{\sphinxupquote{vz}}}
\pysigstopsignatures
\end{fulllineitems}

\index{v (nexoclom2.solarsystem.SSPosition.SSPosition attribute)@\spxentry{v}\spxextra{nexoclom2.solarsystem.SSPosition.SSPosition attribute}}

\begin{fulllineitems}
\phantomsection\label{\detokenize{autoapi/nexoclom2/solarsystem/SSPosition/index:nexoclom2.solarsystem.SSPosition.SSPosition.v}}
\pysigstartsignatures
\pysigline
{\sphinxbfcode{\sphinxupquote{v}}}
\pysigstopsignatures
\end{fulllineitems}

\index{r\_sun (nexoclom2.solarsystem.SSPosition.SSPosition attribute)@\spxentry{r\_sun}\spxextra{nexoclom2.solarsystem.SSPosition.SSPosition attribute}}

\begin{fulllineitems}
\phantomsection\label{\detokenize{autoapi/nexoclom2/solarsystem/SSPosition/index:nexoclom2.solarsystem.SSPosition.SSPosition.r_sun}}
\pysigstartsignatures
\pysigline
{\sphinxbfcode{\sphinxupquote{r\_sun}}}
\pysigstopsignatures
\end{fulllineitems}

\index{drdt\_sun (nexoclom2.solarsystem.SSPosition.SSPosition attribute)@\spxentry{drdt\_sun}\spxextra{nexoclom2.solarsystem.SSPosition.SSPosition attribute}}

\begin{fulllineitems}
\phantomsection\label{\detokenize{autoapi/nexoclom2/solarsystem/SSPosition/index:nexoclom2.solarsystem.SSPosition.SSPosition.drdt_sun}}
\pysigstartsignatures
\pysigline
{\sphinxbfcode{\sphinxupquote{drdt\_sun}}}
\pysigstopsignatures
\end{fulllineitems}

\index{sun\_dir\_x (nexoclom2.solarsystem.SSPosition.SSPosition attribute)@\spxentry{sun\_dir\_x}\spxextra{nexoclom2.solarsystem.SSPosition.SSPosition attribute}}

\begin{fulllineitems}
\phantomsection\label{\detokenize{autoapi/nexoclom2/solarsystem/SSPosition/index:nexoclom2.solarsystem.SSPosition.SSPosition.sun_dir_x}}
\pysigstartsignatures
\pysigline
{\sphinxbfcode{\sphinxupquote{sun\_dir\_x}}}
\pysigstopsignatures
\end{fulllineitems}

\index{sun\_dir\_y (nexoclom2.solarsystem.SSPosition.SSPosition attribute)@\spxentry{sun\_dir\_y}\spxextra{nexoclom2.solarsystem.SSPosition.SSPosition attribute}}

\begin{fulllineitems}
\phantomsection\label{\detokenize{autoapi/nexoclom2/solarsystem/SSPosition/index:nexoclom2.solarsystem.SSPosition.SSPosition.sun_dir_y}}
\pysigstartsignatures
\pysigline
{\sphinxbfcode{\sphinxupquote{sun\_dir\_y}}}
\pysigstopsignatures
\end{fulllineitems}

\index{sun\_dir\_z (nexoclom2.solarsystem.SSPosition.SSPosition attribute)@\spxentry{sun\_dir\_z}\spxextra{nexoclom2.solarsystem.SSPosition.SSPosition attribute}}

\begin{fulllineitems}
\phantomsection\label{\detokenize{autoapi/nexoclom2/solarsystem/SSPosition/index:nexoclom2.solarsystem.SSPosition.SSPosition.sun_dir_z}}
\pysigstartsignatures
\pysigline
{\sphinxbfcode{\sphinxupquote{sun\_dir\_z}}}
\pysigstopsignatures
\end{fulllineitems}

\index{subsolar\_longitude (nexoclom2.solarsystem.SSPosition.SSPosition attribute)@\spxentry{subsolar\_longitude}\spxextra{nexoclom2.solarsystem.SSPosition.SSPosition attribute}}

\begin{fulllineitems}
\phantomsection\label{\detokenize{autoapi/nexoclom2/solarsystem/SSPosition/index:nexoclom2.solarsystem.SSPosition.SSPosition.subsolar_longitude}}
\pysigstartsignatures
\pysigline
{\sphinxbfcode{\sphinxupquote{subsolar\_longitude}}}
\pysigstopsignatures
\end{fulllineitems}

\index{subsolar\_latitude (nexoclom2.solarsystem.SSPosition.SSPosition attribute)@\spxentry{subsolar\_latitude}\spxextra{nexoclom2.solarsystem.SSPosition.SSPosition attribute}}

\begin{fulllineitems}
\phantomsection\label{\detokenize{autoapi/nexoclom2/solarsystem/SSPosition/index:nexoclom2.solarsystem.SSPosition.SSPosition.subsolar_latitude}}
\pysigstartsignatures
\pysigline
{\sphinxbfcode{\sphinxupquote{subsolar\_latitude}}}
\pysigstopsignatures
\end{fulllineitems}

\index{X (nexoclom2.solarsystem.SSPosition.SSPosition attribute)@\spxentry{X}\spxextra{nexoclom2.solarsystem.SSPosition.SSPosition attribute}}

\begin{fulllineitems}
\phantomsection\label{\detokenize{autoapi/nexoclom2/solarsystem/SSPosition/index:nexoclom2.solarsystem.SSPosition.SSPosition.X}}
\pysigstartsignatures
\pysigline
{\sphinxbfcode{\sphinxupquote{X}}}
\pysigstopsignatures
\end{fulllineitems}

\index{V (nexoclom2.solarsystem.SSPosition.SSPosition attribute)@\spxentry{V}\spxextra{nexoclom2.solarsystem.SSPosition.SSPosition attribute}}

\begin{fulllineitems}
\phantomsection\label{\detokenize{autoapi/nexoclom2/solarsystem/SSPosition/index:nexoclom2.solarsystem.SSPosition.SSPosition.V}}
\pysigstartsignatures
\pysigline
{\sphinxbfcode{\sphinxupquote{V}}}
\pysigstopsignatures
\end{fulllineitems}

\index{sun\_dir (nexoclom2.solarsystem.SSPosition.SSPosition attribute)@\spxentry{sun\_dir}\spxextra{nexoclom2.solarsystem.SSPosition.SSPosition attribute}}

\begin{fulllineitems}
\phantomsection\label{\detokenize{autoapi/nexoclom2/solarsystem/SSPosition/index:nexoclom2.solarsystem.SSPosition.SSPosition.sun_dir}}
\pysigstartsignatures
\pysigline
{\sphinxbfcode{\sphinxupquote{sun\_dir}}}
\pysigstopsignatures
\end{fulllineitems}

\index{rotmat (nexoclom2.solarsystem.SSPosition.SSPosition attribute)@\spxentry{rotmat}\spxextra{nexoclom2.solarsystem.SSPosition.SSPosition attribute}}

\begin{fulllineitems}
\phantomsection\label{\detokenize{autoapi/nexoclom2/solarsystem/SSPosition/index:nexoclom2.solarsystem.SSPosition.SSPosition.rotmat}}
\pysigstartsignatures
\pysigline
{\sphinxbfcode{\sphinxupquote{rotmat}}}
\pysigstopsignatures
\end{fulllineitems}

\index{to\_solar() (nexoclom2.solarsystem.SSPosition.SSPosition method)@\spxentry{to\_solar()}\spxextra{nexoclom2.solarsystem.SSPosition.SSPosition method}}

\begin{fulllineitems}
\phantomsection\label{\detokenize{autoapi/nexoclom2/solarsystem/SSPosition/index:nexoclom2.solarsystem.SSPosition.SSPosition.to_solar}}
\pysigstartsignatures
\pysiglinewithargsret
{\sphinxbfcode{\sphinxupquote{to\_solar}}}
{\sphinxparam{\DUrole{n}{pts}}\sphinxparamcomma \sphinxparam{\DUrole{n}{t}}}
{}
\pysigstopsignatures
\end{fulllineitems}

\index{zeros() (nexoclom2.solarsystem.SSPosition.SSPosition method)@\spxentry{zeros()}\spxextra{nexoclom2.solarsystem.SSPosition.SSPosition method}}

\begin{fulllineitems}
\phantomsection\label{\detokenize{autoapi/nexoclom2/solarsystem/SSPosition/index:nexoclom2.solarsystem.SSPosition.SSPosition.zeros}}
\pysigstartsignatures
\pysiglinewithargsret
{\sphinxbfcode{\sphinxupquote{zeros}}}
{\sphinxparam{\DUrole{n}{t}}}
{}
\pysigstopsignatures
\end{fulllineitems}

\index{out\_of\_shadow() (nexoclom2.solarsystem.SSPosition.SSPosition method)@\spxentry{out\_of\_shadow()}\spxextra{nexoclom2.solarsystem.SSPosition.SSPosition method}}

\begin{fulllineitems}
\phantomsection\label{\detokenize{autoapi/nexoclom2/solarsystem/SSPosition/index:nexoclom2.solarsystem.SSPosition.SSPosition.out_of_shadow}}
\pysigstartsignatures
\pysiglinewithargsret
{\sphinxbfcode{\sphinxupquote{out\_of\_shadow}}}
{\sphinxparam{\DUrole{n}{obj}}\sphinxparamcomma \sphinxparam{\DUrole{n}{packets}}}
{}
\pysigstopsignatures
\end{fulllineitems}


\end{fulllineitems}


\sphinxstepscope


\subparagraph{nexoclom2.solarsystem.SSPositionTime}
\label{\detokenize{autoapi/nexoclom2/solarsystem/SSPositionTime/index:module-nexoclom2.solarsystem.SSPositionTime}}\label{\detokenize{autoapi/nexoclom2/solarsystem/SSPositionTime/index:nexoclom2-solarsystem-sspositiontime}}\label{\detokenize{autoapi/nexoclom2/solarsystem/SSPositionTime/index::doc}}\index{module@\spxentry{module}!nexoclom2.solarsystem.SSPositionTime@\spxentry{nexoclom2.solarsystem.SSPositionTime}}\index{nexoclom2.solarsystem.SSPositionTime@\spxentry{nexoclom2.solarsystem.SSPositionTime}!module@\spxentry{module}}

\subparagraph{Attributes}
\label{\detokenize{autoapi/nexoclom2/solarsystem/SSPositionTime/index:attributes}}

\begin{savenotes}\sphinxattablestart
\sphinxthistablewithglobalstyle
\sphinxthistablewithnovlinesstyle
\centering
\begin{tabulary}{\linewidth}[t]{\X{1}{2}\X{1}{2}}
\sphinxtoprule
\sphinxtableatstartofbodyhook
\sphinxAtStartPar
{\hyperref[\detokenize{autoapi/nexoclom2/solarsystem/SSPositionTime/index:nexoclom2.solarsystem.SSPositionTime.pi}]{\sphinxcrossref{\sphinxcode{\sphinxupquote{pi}}}}}
&
\sphinxAtStartPar

\\
\sphinxbottomrule
\end{tabulary}
\sphinxtableafterendhook\par
\sphinxattableend\end{savenotes}


\subparagraph{Classes}
\label{\detokenize{autoapi/nexoclom2/solarsystem/SSPositionTime/index:classes}}

\begin{savenotes}\sphinxattablestart
\sphinxthistablewithglobalstyle
\sphinxthistablewithnovlinesstyle
\centering
\begin{tabulary}{\linewidth}[t]{\X{1}{2}\X{1}{2}}
\sphinxtoprule
\sphinxtableatstartofbodyhook
\sphinxAtStartPar
{\hyperref[\detokenize{autoapi/nexoclom2/solarsystem/SSPositionTime/index:nexoclom2.solarsystem.SSPositionTime.SSPositionTime}]{\sphinxcrossref{\sphinxcode{\sphinxupquote{SSPositionTime}}}}}
&
\sphinxAtStartPar

\\
\sphinxbottomrule
\end{tabulary}
\sphinxtableafterendhook\par
\sphinxattableend\end{savenotes}


\subparagraph{Module Contents}
\label{\detokenize{autoapi/nexoclom2/solarsystem/SSPositionTime/index:module-contents}}\index{pi (in module nexoclom2.solarsystem.SSPositionTime)@\spxentry{pi}\spxextra{in module nexoclom2.solarsystem.SSPositionTime}}

\begin{fulllineitems}
\phantomsection\label{\detokenize{autoapi/nexoclom2/solarsystem/SSPositionTime/index:nexoclom2.solarsystem.SSPositionTime.pi}}
\pysigstartsignatures
\pysigline
{\sphinxcode{\sphinxupquote{nexoclom2.solarsystem.SSPositionTime.}}\sphinxbfcode{\sphinxupquote{pi}}}
\pysigstopsignatures
\end{fulllineitems}

\index{SSPositionTime (class in nexoclom2.solarsystem.SSPositionTime)@\spxentry{SSPositionTime}\spxextra{class in nexoclom2.solarsystem.SSPositionTime}}

\begin{fulllineitems}
\phantomsection\label{\detokenize{autoapi/nexoclom2/solarsystem/SSPositionTime/index:nexoclom2.solarsystem.SSPositionTime.SSPositionTime}}
\pysigstartsignatures
\pysiglinewithargsret
{\sphinxbfcode{\sphinxupquote{class\DUrole{w}{ }}}\sphinxcode{\sphinxupquote{nexoclom2.solarsystem.SSPositionTime.}}\sphinxbfcode{\sphinxupquote{SSPositionTime}}}
{\sphinxparam{\DUrole{n}{ssobject}}\sphinxparamcomma \sphinxparam{\DUrole{n}{geometry}}\sphinxparamcomma \sphinxparam{\DUrole{n}{runtime}}\sphinxparamcomma \sphinxparam{\DUrole{n}{ntimes}\DUrole{o}{=}\DUrole{default_value}{1000}}}
{}
\pysigstopsignatures
\sphinxAtStartPar
Bases: {\hyperref[\detokenize{autoapi/nexoclom2/solarsystem/SSPosition/index:nexoclom2.solarsystem.SSPosition.SSPosition}]{\sphinxcrossref{\sphinxcode{\sphinxupquote{nexoclom2.solarsystem.SSPosition.SSPosition}}}}}
\subsubsection*{Notes}
\begin{itemize}
\item {} 
\sphinxAtStartPar
subsolar\_longitude and subsolar\_latitude for moons refers to the planet’s

\end{itemize}

\sphinxAtStartPar
subsolar longitude and latitude. If, for example running an Io model
centered at Io, need to know Jupiter’s CML.
\begin{itemize}
\item {} 
\sphinxAtStartPar
SPICE aberation correction = ‘LT+S’. The observer can be set in the

\end{itemize}

\sphinxAtStartPar
geometry inputs.
\begin{itemize}
\item {} 
\sphinxAtStartPar
Method for calculating sub\sphinxhyphen{}solar point = ‘INTERCEPT/ELLIPSOID’

\item {} 
\sphinxAtStartPar
Body\sphinxhyphen{}fixed latitude/longitude calculated using SPICE native IAU frames.

\item {} 
\sphinxAtStartPar
Solar\sphinxhyphen{}fixed frames based on Jupiter\sphinxhyphen{}De\sphinxhyphen{}Spun\sphinxhyphen{}Sun (JUNO\_JSS) frame (Bagenal \&

\end{itemize}

\sphinxAtStartPar
Wilson 2016). These frame have z\sphinxhyphen{}axis aligned with spin vector, x\sphinxhyphen{}axis
directed toward the Sun, and y = z cross x.

\sphinxAtStartPar
NAIF IDS found at JPL’s \sphinxhref{https://naif.jpl.nasa.gov/pub/naif/toolkit\_docs/C/req/naif\_ids.html}{Navigation and Ancillary Information
Facility}.
\index{endtime (nexoclom2.solarsystem.SSPositionTime.SSPositionTime attribute)@\spxentry{endtime}\spxextra{nexoclom2.solarsystem.SSPositionTime.SSPositionTime attribute}}

\begin{fulllineitems}
\phantomsection\label{\detokenize{autoapi/nexoclom2/solarsystem/SSPositionTime/index:nexoclom2.solarsystem.SSPositionTime.SSPositionTime.endtime}}
\pysigstartsignatures
\pysigline
{\sphinxbfcode{\sphinxupquote{endtime}}}
\pysigstopsignatures
\end{fulllineitems}

\index{starttime (nexoclom2.solarsystem.SSPositionTime.SSPositionTime attribute)@\spxentry{starttime}\spxextra{nexoclom2.solarsystem.SSPositionTime.SSPositionTime attribute}}

\begin{fulllineitems}
\phantomsection\label{\detokenize{autoapi/nexoclom2/solarsystem/SSPositionTime/index:nexoclom2.solarsystem.SSPositionTime.SSPositionTime.starttime}}
\pysigstartsignatures
\pysigline
{\sphinxbfcode{\sphinxupquote{starttime}}}
\pysigstopsignatures
\end{fulllineitems}

\index{abcor (nexoclom2.solarsystem.SSPositionTime.SSPositionTime attribute)@\spxentry{abcor}\spxextra{nexoclom2.solarsystem.SSPositionTime.SSPositionTime attribute}}

\begin{fulllineitems}
\phantomsection\label{\detokenize{autoapi/nexoclom2/solarsystem/SSPositionTime/index:nexoclom2.solarsystem.SSPositionTime.SSPositionTime.abcor}}
\pysigstartsignatures
\pysigline
{\sphinxbfcode{\sphinxupquote{abcor}}\sphinxbfcode{\sphinxupquote{\DUrole{w}{ }\DUrole{p}{=}\DUrole{w}{ }\textquotesingle{}None\textquotesingle{}}}}
\pysigstopsignatures
\end{fulllineitems}


\end{fulllineitems}


\sphinxstepscope


\subparagraph{nexoclom2.solarsystem.TorusImage}
\label{\detokenize{autoapi/nexoclom2/solarsystem/TorusImage/index:module-nexoclom2.solarsystem.TorusImage}}\label{\detokenize{autoapi/nexoclom2/solarsystem/TorusImage/index:nexoclom2-solarsystem-torusimage}}\label{\detokenize{autoapi/nexoclom2/solarsystem/TorusImage/index::doc}}\index{module@\spxentry{module}!nexoclom2.solarsystem.TorusImage@\spxentry{nexoclom2.solarsystem.TorusImage}}\index{nexoclom2.solarsystem.TorusImage@\spxentry{nexoclom2.solarsystem.TorusImage}!module@\spxentry{module}}

\subparagraph{Classes}
\label{\detokenize{autoapi/nexoclom2/solarsystem/TorusImage/index:classes}}

\begin{savenotes}\sphinxattablestart
\sphinxthistablewithglobalstyle
\sphinxthistablewithnovlinesstyle
\centering
\begin{tabulary}{\linewidth}[t]{\X{1}{2}\X{1}{2}}
\sphinxtoprule
\sphinxtableatstartofbodyhook
\sphinxAtStartPar
{\hyperref[\detokenize{autoapi/nexoclom2/solarsystem/TorusImage/index:nexoclom2.solarsystem.TorusImage.IoTorusImage}]{\sphinxcrossref{\sphinxcode{\sphinxupquote{IoTorusImage}}}}}
&
\sphinxAtStartPar

\\
\sphinxbottomrule
\end{tabulary}
\sphinxtableafterendhook\par
\sphinxattableend\end{savenotes}


\subparagraph{Module Contents}
\label{\detokenize{autoapi/nexoclom2/solarsystem/TorusImage/index:module-contents}}\index{IoTorusImage (class in nexoclom2.solarsystem.TorusImage)@\spxentry{IoTorusImage}\spxextra{class in nexoclom2.solarsystem.TorusImage}}

\begin{fulllineitems}
\phantomsection\label{\detokenize{autoapi/nexoclom2/solarsystem/TorusImage/index:nexoclom2.solarsystem.TorusImage.IoTorusImage}}
\pysigstartsignatures
\pysiglinewithargsret
{\sphinxbfcode{\sphinxupquote{class\DUrole{w}{ }}}\sphinxcode{\sphinxupquote{nexoclom2.solarsystem.TorusImage.}}\sphinxbfcode{\sphinxupquote{IoTorusImage}}}
{\sphinxparam{\DUrole{n}{cml}}}
{}
\pysigstopsignatures\index{cml (nexoclom2.solarsystem.TorusImage.IoTorusImage attribute)@\spxentry{cml}\spxextra{nexoclom2.solarsystem.TorusImage.IoTorusImage attribute}}

\begin{fulllineitems}
\phantomsection\label{\detokenize{autoapi/nexoclom2/solarsystem/TorusImage/index:nexoclom2.solarsystem.TorusImage.IoTorusImage.cml}}
\pysigstartsignatures
\pysigline
{\sphinxbfcode{\sphinxupquote{cml}}}
\pysigstopsignatures
\end{fulllineitems}


\end{fulllineitems}


\sphinxstepscope


\subparagraph{nexoclom2.solarsystem.VoyagerTorus}
\label{\detokenize{autoapi/nexoclom2/solarsystem/VoyagerTorus/index:module-nexoclom2.solarsystem.VoyagerTorus}}\label{\detokenize{autoapi/nexoclom2/solarsystem/VoyagerTorus/index:nexoclom2-solarsystem-voyagertorus}}\label{\detokenize{autoapi/nexoclom2/solarsystem/VoyagerTorus/index::doc}}\index{module@\spxentry{module}!nexoclom2.solarsystem.VoyagerTorus@\spxentry{nexoclom2.solarsystem.VoyagerTorus}}\index{nexoclom2.solarsystem.VoyagerTorus@\spxentry{nexoclom2.solarsystem.VoyagerTorus}!module@\spxentry{module}}
\sphinxAtStartPar
Extracts the Voyager Io Torus model from the IDL save file

\sphinxAtStartPar
Not intended to be run by users.


\subparagraph{Attributes}
\label{\detokenize{autoapi/nexoclom2/solarsystem/VoyagerTorus/index:attributes}}

\begin{savenotes}\sphinxattablestart
\sphinxthistablewithglobalstyle
\sphinxthistablewithnovlinesstyle
\centering
\begin{tabulary}{\linewidth}[t]{\X{1}{2}\X{1}{2}}
\sphinxtoprule
\sphinxtableatstartofbodyhook
\sphinxAtStartPar
{\hyperref[\detokenize{autoapi/nexoclom2/solarsystem/VoyagerTorus/index:nexoclom2.solarsystem.VoyagerTorus.jupiter}]{\sphinxcrossref{\sphinxcode{\sphinxupquote{jupiter}}}}}
&
\sphinxAtStartPar

\\
\sphinxhline
\sphinxAtStartPar
{\hyperref[\detokenize{autoapi/nexoclom2/solarsystem/VoyagerTorus/index:nexoclom2.solarsystem.VoyagerTorus.data}]{\sphinxcrossref{\sphinxcode{\sphinxupquote{data}}}}}
&
\sphinxAtStartPar

\\
\sphinxhline
\sphinxAtStartPar
{\hyperref[\detokenize{autoapi/nexoclom2/solarsystem/VoyagerTorus/index:nexoclom2.solarsystem.VoyagerTorus.plasma_idl}]{\sphinxcrossref{\sphinxcode{\sphinxupquote{plasma\_idl}}}}}
&
\sphinxAtStartPar

\\
\sphinxhline
\sphinxAtStartPar
{\hyperref[\detokenize{autoapi/nexoclom2/solarsystem/VoyagerTorus/index:nexoclom2.solarsystem.VoyagerTorus.keys}]{\sphinxcrossref{\sphinxcode{\sphinxupquote{keys}}}}}
&
\sphinxAtStartPar

\\
\sphinxhline
\sphinxAtStartPar
{\hyperref[\detokenize{autoapi/nexoclom2/solarsystem/VoyagerTorus/index:nexoclom2.solarsystem.VoyagerTorus.ions}]{\sphinxcrossref{\sphinxcode{\sphinxupquote{ions}}}}}
&
\sphinxAtStartPar

\\
\sphinxhline
\sphinxAtStartPar
{\hyperref[\detokenize{autoapi/nexoclom2/solarsystem/VoyagerTorus/index:nexoclom2.solarsystem.VoyagerTorus.plasma}]{\sphinxcrossref{\sphinxcode{\sphinxupquote{plasma}}}}}
&
\sphinxAtStartPar

\\
\sphinxhline
\sphinxAtStartPar
{\hyperref[\detokenize{autoapi/nexoclom2/solarsystem/VoyagerTorus/index:nexoclom2.solarsystem.VoyagerTorus.voyagerfile}]{\sphinxcrossref{\sphinxcode{\sphinxupquote{voyagerfile}}}}}
&
\sphinxAtStartPar

\\
\sphinxbottomrule
\end{tabulary}
\sphinxtableafterendhook\par
\sphinxattableend\end{savenotes}


\subparagraph{Module Contents}
\label{\detokenize{autoapi/nexoclom2/solarsystem/VoyagerTorus/index:module-contents}}\index{jupiter (in module nexoclom2.solarsystem.VoyagerTorus)@\spxentry{jupiter}\spxextra{in module nexoclom2.solarsystem.VoyagerTorus}}

\begin{fulllineitems}
\phantomsection\label{\detokenize{autoapi/nexoclom2/solarsystem/VoyagerTorus/index:nexoclom2.solarsystem.VoyagerTorus.jupiter}}
\pysigstartsignatures
\pysigline
{\sphinxcode{\sphinxupquote{nexoclom2.solarsystem.VoyagerTorus.}}\sphinxbfcode{\sphinxupquote{jupiter}}}
\pysigstopsignatures
\end{fulllineitems}

\index{data (in module nexoclom2.solarsystem.VoyagerTorus)@\spxentry{data}\spxextra{in module nexoclom2.solarsystem.VoyagerTorus}}

\begin{fulllineitems}
\phantomsection\label{\detokenize{autoapi/nexoclom2/solarsystem/VoyagerTorus/index:nexoclom2.solarsystem.VoyagerTorus.data}}
\pysigstartsignatures
\pysigline
{\sphinxcode{\sphinxupquote{nexoclom2.solarsystem.VoyagerTorus.}}\sphinxbfcode{\sphinxupquote{data}}\sphinxbfcode{\sphinxupquote{\DUrole{w}{ }\DUrole{p}{=}\DUrole{w}{ }None}}}
\pysigstopsignatures
\end{fulllineitems}

\index{plasma\_idl (in module nexoclom2.solarsystem.VoyagerTorus)@\spxentry{plasma\_idl}\spxextra{in module nexoclom2.solarsystem.VoyagerTorus}}

\begin{fulllineitems}
\phantomsection\label{\detokenize{autoapi/nexoclom2/solarsystem/VoyagerTorus/index:nexoclom2.solarsystem.VoyagerTorus.plasma_idl}}
\pysigstartsignatures
\pysigline
{\sphinxcode{\sphinxupquote{nexoclom2.solarsystem.VoyagerTorus.}}\sphinxbfcode{\sphinxupquote{plasma\_idl}}}
\pysigstopsignatures
\end{fulllineitems}

\index{keys (in module nexoclom2.solarsystem.VoyagerTorus)@\spxentry{keys}\spxextra{in module nexoclom2.solarsystem.VoyagerTorus}}

\begin{fulllineitems}
\phantomsection\label{\detokenize{autoapi/nexoclom2/solarsystem/VoyagerTorus/index:nexoclom2.solarsystem.VoyagerTorus.keys}}
\pysigstartsignatures
\pysigline
{\sphinxcode{\sphinxupquote{nexoclom2.solarsystem.VoyagerTorus.}}\sphinxbfcode{\sphinxupquote{keys}}\sphinxbfcode{\sphinxupquote{\DUrole{w}{ }\DUrole{p}{=}\DUrole{w}{ }(\textquotesingle{}L\textquotesingle{}, \textquotesingle{}n\_e\textquotesingle{}, \textquotesingle{}T\_e\textquotesingle{}, \textquotesingle{}H\_e\textquotesingle{}, \textquotesingle{}ions\textquotesingle{}, \textquotesingle{}n\_i\textquotesingle{}, \textquotesingle{}T\_i\textquotesingle{}, \textquotesingle{}H\_i\textquotesingle{})}}}
\pysigstopsignatures
\end{fulllineitems}

\index{ions (in module nexoclom2.solarsystem.VoyagerTorus)@\spxentry{ions}\spxextra{in module nexoclom2.solarsystem.VoyagerTorus}}

\begin{fulllineitems}
\phantomsection\label{\detokenize{autoapi/nexoclom2/solarsystem/VoyagerTorus/index:nexoclom2.solarsystem.VoyagerTorus.ions}}
\pysigstartsignatures
\pysigline
{\sphinxcode{\sphinxupquote{nexoclom2.solarsystem.VoyagerTorus.}}\sphinxbfcode{\sphinxupquote{ions}}}
\pysigstopsignatures
\end{fulllineitems}

\index{plasma (in module nexoclom2.solarsystem.VoyagerTorus)@\spxentry{plasma}\spxextra{in module nexoclom2.solarsystem.VoyagerTorus}}

\begin{fulllineitems}
\phantomsection\label{\detokenize{autoapi/nexoclom2/solarsystem/VoyagerTorus/index:nexoclom2.solarsystem.VoyagerTorus.plasma}}
\pysigstartsignatures
\pysigline
{\sphinxcode{\sphinxupquote{nexoclom2.solarsystem.VoyagerTorus.}}\sphinxbfcode{\sphinxupquote{plasma}}}
\pysigstopsignatures
\end{fulllineitems}

\index{voyagerfile (in module nexoclom2.solarsystem.VoyagerTorus)@\spxentry{voyagerfile}\spxextra{in module nexoclom2.solarsystem.VoyagerTorus}}

\begin{fulllineitems}
\phantomsection\label{\detokenize{autoapi/nexoclom2/solarsystem/VoyagerTorus/index:nexoclom2.solarsystem.VoyagerTorus.voyagerfile}}
\pysigstartsignatures
\pysigline
{\sphinxcode{\sphinxupquote{nexoclom2.solarsystem.VoyagerTorus.}}\sphinxbfcode{\sphinxupquote{voyagerfile}}}
\pysigstopsignatures
\end{fulllineitems}


\sphinxstepscope


\subparagraph{nexoclom2.solarsystem.load\_kernels}
\label{\detokenize{autoapi/nexoclom2/solarsystem/load_kernels/index:module-nexoclom2.solarsystem.load_kernels}}\label{\detokenize{autoapi/nexoclom2/solarsystem/load_kernels/index:nexoclom2-solarsystem-load-kernels}}\label{\detokenize{autoapi/nexoclom2/solarsystem/load_kernels/index::doc}}\index{module@\spxentry{module}!nexoclom2.solarsystem.load\_kernels@\spxentry{nexoclom2.solarsystem.load\_kernels}}\index{nexoclom2.solarsystem.load\_kernels@\spxentry{nexoclom2.solarsystem.load\_kernels}!module@\spxentry{module}}

\subparagraph{Classes}
\label{\detokenize{autoapi/nexoclom2/solarsystem/load_kernels/index:classes}}

\begin{savenotes}\sphinxattablestart
\sphinxthistablewithglobalstyle
\sphinxthistablewithnovlinesstyle
\centering
\begin{tabulary}{\linewidth}[t]{\X{1}{2}\X{1}{2}}
\sphinxtoprule
\sphinxtableatstartofbodyhook
\sphinxAtStartPar
{\hyperref[\detokenize{autoapi/nexoclom2/solarsystem/load_kernels/index:nexoclom2.solarsystem.load_kernels.SpiceKernels}]{\sphinxcrossref{\sphinxcode{\sphinxupquote{SpiceKernels}}}}}
&
\sphinxAtStartPar

\\
\sphinxbottomrule
\end{tabulary}
\sphinxtableafterendhook\par
\sphinxattableend\end{savenotes}


\subparagraph{Module Contents}
\label{\detokenize{autoapi/nexoclom2/solarsystem/load_kernels/index:module-contents}}\index{SpiceKernels (class in nexoclom2.solarsystem.load\_kernels)@\spxentry{SpiceKernels}\spxextra{class in nexoclom2.solarsystem.load\_kernels}}

\begin{fulllineitems}
\phantomsection\label{\detokenize{autoapi/nexoclom2/solarsystem/load_kernels/index:nexoclom2.solarsystem.load_kernels.SpiceKernels}}
\pysigstartsignatures
\pysiglinewithargsret
{\sphinxbfcode{\sphinxupquote{class\DUrole{w}{ }}}\sphinxcode{\sphinxupquote{nexoclom2.solarsystem.load\_kernels.}}\sphinxbfcode{\sphinxupquote{SpiceKernels}}}
{\sphinxparam{\DUrole{n}{object}}}
{}
\pysigstopsignatures\index{kernels (nexoclom2.solarsystem.load\_kernels.SpiceKernels attribute)@\spxentry{kernels}\spxextra{nexoclom2.solarsystem.load\_kernels.SpiceKernels attribute}}

\begin{fulllineitems}
\phantomsection\label{\detokenize{autoapi/nexoclom2/solarsystem/load_kernels/index:nexoclom2.solarsystem.load_kernels.SpiceKernels.kernels}}
\pysigstartsignatures
\pysigline
{\sphinxbfcode{\sphinxupquote{kernels}}\sphinxbfcode{\sphinxupquote{\DUrole{w}{ }\DUrole{p}{=}\DUrole{w}{ }{[}{]}}}}
\pysigstopsignatures
\end{fulllineitems}

\index{unload() (nexoclom2.solarsystem.load\_kernels.SpiceKernels method)@\spxentry{unload()}\spxextra{nexoclom2.solarsystem.load\_kernels.SpiceKernels method}}

\begin{fulllineitems}
\phantomsection\label{\detokenize{autoapi/nexoclom2/solarsystem/load_kernels/index:nexoclom2.solarsystem.load_kernels.SpiceKernels.unload}}
\pysigstartsignatures
\pysiglinewithargsret
{\sphinxbfcode{\sphinxupquote{unload}}}
{}
{}
\pysigstopsignatures
\end{fulllineitems}

\index{load\_kernels() (nexoclom2.solarsystem.load\_kernels.SpiceKernels method)@\spxentry{load\_kernels()}\spxextra{nexoclom2.solarsystem.load\_kernels.SpiceKernels method}}

\begin{fulllineitems}
\phantomsection\label{\detokenize{autoapi/nexoclom2/solarsystem/load_kernels/index:nexoclom2.solarsystem.load_kernels.SpiceKernels.load_kernels}}
\pysigstartsignatures
\pysiglinewithargsret
{\sphinxbfcode{\sphinxupquote{load\_kernels}}}
{\sphinxparam{\DUrole{n}{kernels}}}
{}
\pysigstopsignatures
\sphinxAtStartPar
Manually load a kernel or kernels

\end{fulllineitems}


\end{fulllineitems}



\subparagraph{Classes}
\label{\detokenize{autoapi/nexoclom2/solarsystem/index:classes}}

\begin{savenotes}\sphinxattablestart
\sphinxthistablewithglobalstyle
\sphinxthistablewithnovlinesstyle
\centering
\begin{tabulary}{\linewidth}[t]{\X{1}{2}\X{1}{2}}
\sphinxtoprule
\sphinxtableatstartofbodyhook
\sphinxAtStartPar
{\hyperref[\detokenize{autoapi/nexoclom2/solarsystem/index:nexoclom2.solarsystem.SSObject}]{\sphinxcrossref{\sphinxcode{\sphinxupquote{SSObject}}}}}
&
\sphinxAtStartPar
Physical data for solar system bodies.
\\
\sphinxhline
\sphinxAtStartPar
{\hyperref[\detokenize{autoapi/nexoclom2/solarsystem/index:nexoclom2.solarsystem.SSPositionTime}]{\sphinxcrossref{\sphinxcode{\sphinxupquote{SSPositionTime}}}}}
&
\sphinxAtStartPar

\\
\sphinxhline
\sphinxAtStartPar
{\hyperref[\detokenize{autoapi/nexoclom2/solarsystem/index:nexoclom2.solarsystem.IoTorus}]{\sphinxcrossref{\sphinxcode{\sphinxupquote{IoTorus}}}}}
&
\sphinxAtStartPar

\\
\sphinxbottomrule
\end{tabulary}
\sphinxtableafterendhook\par
\sphinxattableend\end{savenotes}


\subparagraph{Package Contents}
\label{\detokenize{autoapi/nexoclom2/solarsystem/index:package-contents}}\index{SSObject (class in nexoclom2.solarsystem)@\spxentry{SSObject}\spxextra{class in nexoclom2.solarsystem}}

\begin{fulllineitems}
\phantomsection\label{\detokenize{autoapi/nexoclom2/solarsystem/index:nexoclom2.solarsystem.SSObject}}
\pysigstartsignatures
\pysiglinewithargsret
{\sphinxbfcode{\sphinxupquote{class\DUrole{w}{ }}}\sphinxcode{\sphinxupquote{nexoclom2.solarsystem.}}\sphinxbfcode{\sphinxupquote{SSObject}}}
{\sphinxparam{\DUrole{n}{obj}\DUrole{p}{:}\DUrole{w}{ }\DUrole{n}{str}}}
{}
\pysigstopsignatures\begin{description}
\sphinxlineitem{Physical data for solar system bodies.}
\sphinxAtStartPar
Object containing all the necessary physical data for solar system objects.
Data is stored in a table included with the package. A separate table
contains the NAIF IDs. If the object is not found in the data table, returns
an object with just the object name, type = Unknown, and if possible the
NAIF ID.

\end{description}
\begin{quote}\begin{description}
\sphinxlineitem{Parameters}\begin{description}
\sphinxlineitem{\sphinxstylestrong{obj}}{[}str{]}
\sphinxAtStartPar
Name of the solar system object to gather data for.

\end{description}

\sphinxlineitem{Attributes}\begin{description}
\sphinxlineitem{\sphinxstylestrong{object: str}}\begin{quote}

\sphinxAtStartPar
Name of solar system body. Source: input parameter
\end{quote}
\begin{description}
\sphinxlineitem{orbits: str}
\sphinxAtStartPar
Object the body orbits. Source: PlanetaryConstants.csv

\sphinxlineitem{radius}{[}distance quantity{]}
\sphinxAtStartPar
Object radius. Source: SPICE

\sphinxlineitem{unit: astropy unit}
\sphinxAtStartPar
Named: R\_\textless{}object\textgreater{}

\sphinxlineitem{GM: Quantity}
\sphinxAtStartPar
Mass times gravitational constant. Source: SPICE

\sphinxlineitem{mass: mass quantity}
\sphinxAtStartPar
Object mass in kg. Source: GM from SPICE

\sphinxlineitem{a: distance quantity}
\sphinxAtStartPar
Object semi\sphinxhyphen{}major axis. Source: SPICE

\sphinxlineitem{e: float}
\sphinxAtStartPar
Orbital eccentricity. For planets: Source SPICE. For moons: Set to 0.
This only affects calculations when a modeltime is not specified and
is a small affect.

\sphinxlineitem{tilt: angle quantity}
\sphinxAtStartPar
Tilt of rotation axis relative to ecliptic in degrees.
Source: PlanetaryConstants.csv

\sphinxlineitem{rotperiod: time quantity}
\sphinxAtStartPar
Siderial rotational period in hours. Source: PlanetaryConstants.csv

\sphinxlineitem{orbperiod: time quantity}
\sphinxAtStartPar
Sideral orbital period. Source: SPICE

\sphinxlineitem{orbvel: velocity quantity}
\sphinxAtStartPar
:math:{\color{red}\bfseries{}\textasciigrave{}}v\_\{orb\} =

\end{description}

\sphinxlineitem{\sphinxstylestrong{rac\{2 pi a\}\{orbperiod\}\textasciigrave{}}}\begin{description}
\sphinxlineitem{satellites: list of str or None}
\sphinxAtStartPar
List of satellites of the body. Source: PlanetaryConstants.csv

\sphinxlineitem{type}{[}\{‘Star’, ‘Planet’, or ‘Moon’\}{]}
\sphinxAtStartPar
Source: PlanetaryConstants.csv

\sphinxlineitem{naifid}{[}int{]}
\sphinxAtStartPar
Source: naifids.csv

\end{description}

\end{description}

\end{description}\end{quote}
\subsubsection*{Examples}

\begin{sphinxVerbatim}[commandchars=\\\{\}]
\PYG{g+gp}{\PYGZgt{}\PYGZgt{}\PYGZgt{} }\PYG{k+kn}{from}\PYG{+w}{ }\PYG{n+nn}{nexoclom2}\PYG{n+nn}{.}\PYG{n+nn}{solarsystem}\PYG{+w}{ }\PYG{k+kn}{import} \PYG{n}{SSObject}
\PYG{g+gp}{\PYGZgt{}\PYGZgt{}\PYGZgt{} }\PYG{n}{jupiter} \PYG{o}{=} \PYG{n}{SSObject}\PYG{p}{(}\PYG{l+s+s1}{\PYGZsq{}}\PYG{l+s+s1}{Jupiter}\PYG{l+s+s1}{\PYGZsq{}}\PYG{p}{)}
\PYG{g+gp}{\PYGZgt{}\PYGZgt{}\PYGZgt{} }\PYG{n+nb}{print}\PYG{p}{(}\PYG{n}{jupiter}\PYG{p}{)}
\PYG{g+go}{Object: Jupiter}
\PYG{g+go}{Type = Planet}
\PYG{g+go}{Orbits Sun}
\PYG{g+go}{Satellites: Io, Europa, Ganymede, Callisto}
\PYG{g+go}{Radius = 71492.00 km}
\PYG{g+go}{Mass = 1.90e+27 kg}
\PYG{g+go}{a = 5.20 AU}
\PYG{g+go}{Eccentricity = 0.05}
\PYG{g+go}{Tilt = 3.08 deg}
\PYG{g+go}{Rotation Period = 9.93 h}
\PYG{g+go}{Orbital Period = 4333.00 d}
\PYG{g+go}{GM = \PYGZhy{}1.27e+17 m3 / s2}
\PYG{g+go}{NAIFID = 599}
\PYG{g+gp}{\PYGZgt{}\PYGZgt{}\PYGZgt{} }\PYG{n+nb}{print}\PYG{p}{(}\PYG{n+nb}{len}\PYG{p}{(}\PYG{n}{jupiter}\PYG{p}{)}\PYG{p}{)}
\PYG{g+go}{5}
\PYG{g+gp}{\PYGZgt{}\PYGZgt{}\PYGZgt{} }\PYG{n}{hst} \PYG{o}{=} \PYG{n}{SSObject}\PYG{p}{(}\PYG{l+s+s1}{\PYGZsq{}}\PYG{l+s+s1}{HST}\PYG{l+s+s1}{\PYGZsq{}}\PYG{p}{)}
\PYG{g+gp}{\PYGZgt{}\PYGZgt{}\PYGZgt{} }\PYG{n+nb}{print}\PYG{p}{(}\PYG{n}{hst}\PYG{p}{)}
\PYG{g+go}{Object: Hst}
\PYG{g+go}{Type = Unknown}
\PYG{g+go}{NAIFID = \PYGZhy{}48}
\PYG{g+gp}{\PYGZgt{}\PYGZgt{}\PYGZgt{} }\PYG{n+nb}{print}\PYG{p}{(}\PYG{n}{jupiter} \PYG{o}{==} \PYG{n}{hst}\PYG{p}{)}
\PYG{g+go}{False}
\end{sphinxVerbatim}
\begin{quote}\begin{description}
\sphinxlineitem{Authors}
\sphinxAtStartPar
Matthew Burger

\end{description}\end{quote}
\index{object (nexoclom2.solarsystem.SSObject attribute)@\spxentry{object}\spxextra{nexoclom2.solarsystem.SSObject attribute}}

\begin{fulllineitems}
\phantomsection\label{\detokenize{autoapi/nexoclom2/solarsystem/index:nexoclom2.solarsystem.SSObject.object}}
\pysigstartsignatures
\pysigline
{\sphinxbfcode{\sphinxupquote{object}}}
\pysigstopsignatures
\end{fulllineitems}

\index{\_\_eq\_\_() (nexoclom2.solarsystem.SSObject method)@\spxentry{\_\_eq\_\_()}\spxextra{nexoclom2.solarsystem.SSObject method}}

\begin{fulllineitems}
\phantomsection\label{\detokenize{autoapi/nexoclom2/solarsystem/index:nexoclom2.solarsystem.SSObject.__eq__}}
\pysigstartsignatures
\pysiglinewithargsret
{\sphinxbfcode{\sphinxupquote{\_\_eq\_\_}}}
{\sphinxparam{\DUrole{n}{other}}}
{}
\pysigstopsignatures
\end{fulllineitems}

\index{\_\_len\_\_() (nexoclom2.solarsystem.SSObject method)@\spxentry{\_\_len\_\_()}\spxextra{nexoclom2.solarsystem.SSObject method}}

\begin{fulllineitems}
\phantomsection\label{\detokenize{autoapi/nexoclom2/solarsystem/index:nexoclom2.solarsystem.SSObject.__len__}}
\pysigstartsignatures
\pysiglinewithargsret
{\sphinxbfcode{\sphinxupquote{\_\_len\_\_}}}
{}
{}
\pysigstopsignatures
\sphinxAtStartPar
Returns number of satellites + 1

\end{fulllineitems}

\index{\_\_repr\_\_() (nexoclom2.solarsystem.SSObject method)@\spxentry{\_\_repr\_\_()}\spxextra{nexoclom2.solarsystem.SSObject method}}

\begin{fulllineitems}
\phantomsection\label{\detokenize{autoapi/nexoclom2/solarsystem/index:nexoclom2.solarsystem.SSObject.__repr__}}
\pysigstartsignatures
\pysiglinewithargsret
{\sphinxbfcode{\sphinxupquote{\_\_repr\_\_}}}
{}
{}
\pysigstopsignatures
\end{fulllineitems}

\index{\_\_str\_\_() (nexoclom2.solarsystem.SSObject method)@\spxentry{\_\_str\_\_()}\spxextra{nexoclom2.solarsystem.SSObject method}}

\begin{fulllineitems}
\phantomsection\label{\detokenize{autoapi/nexoclom2/solarsystem/index:nexoclom2.solarsystem.SSObject.__str__}}
\pysigstartsignatures
\pysiglinewithargsret
{\sphinxbfcode{\sphinxupquote{\_\_str\_\_}}}
{}
{}
\pysigstopsignatures
\end{fulllineitems}


\end{fulllineitems}

\index{SSPositionTime (class in nexoclom2.solarsystem)@\spxentry{SSPositionTime}\spxextra{class in nexoclom2.solarsystem}}

\begin{fulllineitems}
\phantomsection\label{\detokenize{autoapi/nexoclom2/solarsystem/index:nexoclom2.solarsystem.SSPositionTime}}
\pysigstartsignatures
\pysiglinewithargsret
{\sphinxbfcode{\sphinxupquote{class\DUrole{w}{ }}}\sphinxcode{\sphinxupquote{nexoclom2.solarsystem.}}\sphinxbfcode{\sphinxupquote{SSPositionTime}}}
{\sphinxparam{\DUrole{n}{ssobject}}\sphinxparamcomma \sphinxparam{\DUrole{n}{geometry}}\sphinxparamcomma \sphinxparam{\DUrole{n}{runtime}}\sphinxparamcomma \sphinxparam{\DUrole{n}{ntimes}\DUrole{o}{=}\DUrole{default_value}{1000}}}
{}
\pysigstopsignatures
\sphinxAtStartPar
Bases: {\hyperref[\detokenize{autoapi/nexoclom2/solarsystem/SSPosition/index:nexoclom2.solarsystem.SSPosition.SSPosition}]{\sphinxcrossref{\sphinxcode{\sphinxupquote{nexoclom2.solarsystem.SSPosition.SSPosition}}}}}
\subsubsection*{Notes}
\begin{itemize}
\item {} 
\sphinxAtStartPar
subsolar\_longitude and subsolar\_latitude for moons refers to the planet’s

\end{itemize}

\sphinxAtStartPar
subsolar longitude and latitude. If, for example running an Io model
centered at Io, need to know Jupiter’s CML.
\begin{itemize}
\item {} 
\sphinxAtStartPar
SPICE aberation correction = ‘LT+S’. The observer can be set in the

\end{itemize}

\sphinxAtStartPar
geometry inputs.
\begin{itemize}
\item {} 
\sphinxAtStartPar
Method for calculating sub\sphinxhyphen{}solar point = ‘INTERCEPT/ELLIPSOID’

\item {} 
\sphinxAtStartPar
Body\sphinxhyphen{}fixed latitude/longitude calculated using SPICE native IAU frames.

\item {} 
\sphinxAtStartPar
Solar\sphinxhyphen{}fixed frames based on Jupiter\sphinxhyphen{}De\sphinxhyphen{}Spun\sphinxhyphen{}Sun (JUNO\_JSS) frame (Bagenal \&

\end{itemize}

\sphinxAtStartPar
Wilson 2016). These frame have z\sphinxhyphen{}axis aligned with spin vector, x\sphinxhyphen{}axis
directed toward the Sun, and y = z cross x.

\sphinxAtStartPar
NAIF IDS found at JPL’s \sphinxhref{https://naif.jpl.nasa.gov/pub/naif/toolkit\_docs/C/req/naif\_ids.html}{Navigation and Ancillary Information
Facility}.
\index{endtime (nexoclom2.solarsystem.SSPositionTime attribute)@\spxentry{endtime}\spxextra{nexoclom2.solarsystem.SSPositionTime attribute}}

\begin{fulllineitems}
\phantomsection\label{\detokenize{autoapi/nexoclom2/solarsystem/index:nexoclom2.solarsystem.SSPositionTime.endtime}}
\pysigstartsignatures
\pysigline
{\sphinxbfcode{\sphinxupquote{endtime}}}
\pysigstopsignatures
\end{fulllineitems}

\index{starttime (nexoclom2.solarsystem.SSPositionTime attribute)@\spxentry{starttime}\spxextra{nexoclom2.solarsystem.SSPositionTime attribute}}

\begin{fulllineitems}
\phantomsection\label{\detokenize{autoapi/nexoclom2/solarsystem/index:nexoclom2.solarsystem.SSPositionTime.starttime}}
\pysigstartsignatures
\pysigline
{\sphinxbfcode{\sphinxupquote{starttime}}}
\pysigstopsignatures
\end{fulllineitems}

\index{abcor (nexoclom2.solarsystem.SSPositionTime attribute)@\spxentry{abcor}\spxextra{nexoclom2.solarsystem.SSPositionTime attribute}}

\begin{fulllineitems}
\phantomsection\label{\detokenize{autoapi/nexoclom2/solarsystem/index:nexoclom2.solarsystem.SSPositionTime.abcor}}
\pysigstartsignatures
\pysigline
{\sphinxbfcode{\sphinxupquote{abcor}}\sphinxbfcode{\sphinxupquote{\DUrole{w}{ }\DUrole{p}{=}\DUrole{w}{ }\textquotesingle{}None\textquotesingle{}}}}
\pysigstopsignatures
\end{fulllineitems}


\end{fulllineitems}

\index{IoTorus (class in nexoclom2.solarsystem)@\spxentry{IoTorus}\spxextra{class in nexoclom2.solarsystem}}

\begin{fulllineitems}
\phantomsection\label{\detokenize{autoapi/nexoclom2/solarsystem/index:nexoclom2.solarsystem.IoTorus}}
\pysigstartsignatures
\pysiglinewithargsret
{\sphinxbfcode{\sphinxupquote{class\DUrole{w}{ }}}\sphinxcode{\sphinxupquote{nexoclom2.solarsystem.}}\sphinxbfcode{\sphinxupquote{IoTorus}}}
{\sphinxparam{\DUrole{n}{source}\DUrole{o}{=}\DUrole{default_value}{\textquotesingle{}Voyager\textquotesingle{}}}}
{}
\pysigstopsignatures\index{source (nexoclom2.solarsystem.IoTorus attribute)@\spxentry{source}\spxextra{nexoclom2.solarsystem.IoTorus attribute}}

\begin{fulllineitems}
\phantomsection\label{\detokenize{autoapi/nexoclom2/solarsystem/index:nexoclom2.solarsystem.IoTorus.source}}
\pysigstartsignatures
\pysigline
{\sphinxbfcode{\sphinxupquote{source}}\sphinxbfcode{\sphinxupquote{\DUrole{w}{ }\DUrole{p}{=}\DUrole{w}{ }\textquotesingle{}Voyager\textquotesingle{}}}}
\pysigstopsignatures
\end{fulllineitems}

\index{M (nexoclom2.solarsystem.IoTorus attribute)@\spxentry{M}\spxextra{nexoclom2.solarsystem.IoTorus attribute}}

\begin{fulllineitems}
\phantomsection\label{\detokenize{autoapi/nexoclom2/solarsystem/index:nexoclom2.solarsystem.IoTorus.M}}
\pysigstartsignatures
\pysigline
{\sphinxbfcode{\sphinxupquote{M}}}
\pysigstopsignatures
\end{fulllineitems}

\index{n\_e (nexoclom2.solarsystem.IoTorus attribute)@\spxentry{n\_e}\spxextra{nexoclom2.solarsystem.IoTorus attribute}}

\begin{fulllineitems}
\phantomsection\label{\detokenize{autoapi/nexoclom2/solarsystem/index:nexoclom2.solarsystem.IoTorus.n_e}}
\pysigstartsignatures
\pysigline
{\sphinxbfcode{\sphinxupquote{n\_e}}}
\pysigstopsignatures
\end{fulllineitems}

\index{T\_e (nexoclom2.solarsystem.IoTorus attribute)@\spxentry{T\_e}\spxextra{nexoclom2.solarsystem.IoTorus attribute}}

\begin{fulllineitems}
\phantomsection\label{\detokenize{autoapi/nexoclom2/solarsystem/index:nexoclom2.solarsystem.IoTorus.T_e}}
\pysigstartsignatures
\pysigline
{\sphinxbfcode{\sphinxupquote{T\_e}}}
\pysigstopsignatures
\end{fulllineitems}

\index{H\_e (nexoclom2.solarsystem.IoTorus attribute)@\spxentry{H\_e}\spxextra{nexoclom2.solarsystem.IoTorus attribute}}

\begin{fulllineitems}
\phantomsection\label{\detokenize{autoapi/nexoclom2/solarsystem/index:nexoclom2.solarsystem.IoTorus.H_e}}
\pysigstartsignatures
\pysigline
{\sphinxbfcode{\sphinxupquote{H\_e}}}
\pysigstopsignatures
\end{fulllineitems}

\index{T\_i (nexoclom2.solarsystem.IoTorus attribute)@\spxentry{T\_i}\spxextra{nexoclom2.solarsystem.IoTorus attribute}}

\begin{fulllineitems}
\phantomsection\label{\detokenize{autoapi/nexoclom2/solarsystem/index:nexoclom2.solarsystem.IoTorus.T_i}}
\pysigstartsignatures
\pysigline
{\sphinxbfcode{\sphinxupquote{T\_i}}}
\pysigstopsignatures
\end{fulllineitems}

\index{ions (nexoclom2.solarsystem.IoTorus attribute)@\spxentry{ions}\spxextra{nexoclom2.solarsystem.IoTorus attribute}}

\begin{fulllineitems}
\phantomsection\label{\detokenize{autoapi/nexoclom2/solarsystem/index:nexoclom2.solarsystem.IoTorus.ions}}
\pysigstartsignatures
\pysigline
{\sphinxbfcode{\sphinxupquote{ions}}}
\pysigstopsignatures
\end{fulllineitems}

\index{n\_i (nexoclom2.solarsystem.IoTorus attribute)@\spxentry{n\_i}\spxextra{nexoclom2.solarsystem.IoTorus attribute}}

\begin{fulllineitems}
\phantomsection\label{\detokenize{autoapi/nexoclom2/solarsystem/index:nexoclom2.solarsystem.IoTorus.n_i}}
\pysigstartsignatures
\pysigline
{\sphinxbfcode{\sphinxupquote{n\_i}}}
\pysigstopsignatures
\end{fulllineitems}

\index{H\_i (nexoclom2.solarsystem.IoTorus attribute)@\spxentry{H\_i}\spxextra{nexoclom2.solarsystem.IoTorus attribute}}

\begin{fulllineitems}
\phantomsection\label{\detokenize{autoapi/nexoclom2/solarsystem/index:nexoclom2.solarsystem.IoTorus.H_i}}
\pysigstartsignatures
\pysigline
{\sphinxbfcode{\sphinxupquote{H\_i}}}
\pysigstopsignatures
\end{fulllineitems}

\index{planet (nexoclom2.solarsystem.IoTorus attribute)@\spxentry{planet}\spxextra{nexoclom2.solarsystem.IoTorus attribute}}

\begin{fulllineitems}
\phantomsection\label{\detokenize{autoapi/nexoclom2/solarsystem/index:nexoclom2.solarsystem.IoTorus.planet}}
\pysigstartsignatures
\pysigline
{\sphinxbfcode{\sphinxupquote{planet}}}
\pysigstopsignatures
\end{fulllineitems}

\index{xyz\_to\_Mzeta() (nexoclom2.solarsystem.IoTorus method)@\spxentry{xyz\_to\_Mzeta()}\spxextra{nexoclom2.solarsystem.IoTorus method}}

\begin{fulllineitems}
\phantomsection\label{\detokenize{autoapi/nexoclom2/solarsystem/index:nexoclom2.solarsystem.IoTorus.xyz_to_Mzeta}}
\pysigstartsignatures
\pysiglinewithargsret
{\sphinxbfcode{\sphinxupquote{xyz\_to\_Mzeta}}}
{\sphinxparam{\DUrole{n}{x}}\sphinxparamcomma \sphinxparam{\DUrole{n}{y}}\sphinxparamcomma \sphinxparam{\DUrole{n}{z}}\sphinxparamcomma \sphinxparam{\DUrole{n}{cml}}}
{}
\pysigstopsignatures
\end{fulllineitems}

\index{n\_and\_T() (nexoclom2.solarsystem.IoTorus method)@\spxentry{n\_and\_T()}\spxextra{nexoclom2.solarsystem.IoTorus method}}

\begin{fulllineitems}
\phantomsection\label{\detokenize{autoapi/nexoclom2/solarsystem/index:nexoclom2.solarsystem.IoTorus.n_and_T}}
\pysigstartsignatures
\pysiglinewithargsret
{\sphinxbfcode{\sphinxupquote{n\_and\_T}}}
{\sphinxparam{\DUrole{n}{species}}\sphinxparamcomma \sphinxparam{\DUrole{n}{x}}\sphinxparamcomma \sphinxparam{\DUrole{n}{y}}\sphinxparamcomma \sphinxparam{\DUrole{n}{z}}\sphinxparamcomma \sphinxparam{\DUrole{n}{cml}}}
{}
\pysigstopsignatures
\end{fulllineitems}

\index{distance\_along\_field\_line() (nexoclom2.solarsystem.IoTorus method)@\spxentry{distance\_along\_field\_line()}\spxextra{nexoclom2.solarsystem.IoTorus method}}

\begin{fulllineitems}
\phantomsection\label{\detokenize{autoapi/nexoclom2/solarsystem/index:nexoclom2.solarsystem.IoTorus.distance_along_field_line}}
\pysigstartsignatures
\pysiglinewithargsret
{\sphinxbfcode{\sphinxupquote{distance\_along\_field\_line}}}
{\sphinxparam{\DUrole{n}{L}}\sphinxparamcomma \sphinxparam{\DUrole{n}{theta\_c}}\sphinxparamcomma \sphinxparam{\DUrole{n}{theta\_p}}}
{}
\pysigstopsignatures
\sphinxAtStartPar
Calculate the distance along the dipole field line
\begin{quote}\begin{description}
\sphinxlineitem{Parameters}\begin{description}
\sphinxlineitem{\sphinxstylestrong{L: astropy Quantity}}
\sphinxAtStartPar
L (or modified L) \sphinxhyphen{} Field line through magnetic equator

\sphinxlineitem{\sphinxstylestrong{theta\_c: astropy Quantity}}
\sphinxAtStartPar
Angle from magnetic equator to centrifugal equator

\sphinxlineitem{\sphinxstylestrong{theta\_p: astropy Quantity}}
\sphinxAtStartPar
Angle from magnetic equator to packet

\end{description}

\sphinxlineitem{Returns}\begin{description}
\sphinxlineitem{Distance along field line}
\end{description}

\end{description}\end{quote}

\end{fulllineitems}


\end{fulllineitems}


\sphinxstepscope


\paragraph{nexoclom2.utilities}
\label{\detokenize{autoapi/nexoclom2/utilities/index:module-nexoclom2.utilities}}\label{\detokenize{autoapi/nexoclom2/utilities/index:nexoclom2-utilities}}\label{\detokenize{autoapi/nexoclom2/utilities/index::doc}}\index{module@\spxentry{module}!nexoclom2.utilities@\spxentry{nexoclom2.utilities}}\index{nexoclom2.utilities@\spxentry{nexoclom2.utilities}!module@\spxentry{module}}
\sphinxAtStartPar
Miscellaneous helper functions used by nexoclom2.


\subparagraph{Submodules}
\label{\detokenize{autoapi/nexoclom2/utilities/index:submodules}}
\sphinxstepscope


\subparagraph{nexoclom2.utilities.NexoclomConfig}
\label{\detokenize{autoapi/nexoclom2/utilities/NexoclomConfig/index:module-nexoclom2.utilities.NexoclomConfig}}\label{\detokenize{autoapi/nexoclom2/utilities/NexoclomConfig/index:nexoclom2-utilities-nexoclomconfig}}\label{\detokenize{autoapi/nexoclom2/utilities/NexoclomConfig/index::doc}}\index{module@\spxentry{module}!nexoclom2.utilities.NexoclomConfig@\spxentry{nexoclom2.utilities.NexoclomConfig}}\index{nexoclom2.utilities.NexoclomConfig@\spxentry{nexoclom2.utilities.NexoclomConfig}!module@\spxentry{module}}

\subparagraph{Classes}
\label{\detokenize{autoapi/nexoclom2/utilities/NexoclomConfig/index:classes}}

\begin{savenotes}\sphinxattablestart
\sphinxthistablewithglobalstyle
\sphinxthistablewithnovlinesstyle
\centering
\begin{tabulary}{\linewidth}[t]{\X{1}{2}\X{1}{2}}
\sphinxtoprule
\sphinxtableatstartofbodyhook
\sphinxAtStartPar
{\hyperref[\detokenize{autoapi/nexoclom2/utilities/NexoclomConfig/index:nexoclom2.utilities.NexoclomConfig.NexoclomConfig}]{\sphinxcrossref{\sphinxcode{\sphinxupquote{NexoclomConfig}}}}}
&
\sphinxAtStartPar
Configuration object based on the nexoclom2 configuration file.
\\
\sphinxbottomrule
\end{tabulary}
\sphinxtableafterendhook\par
\sphinxattableend\end{savenotes}


\subparagraph{Module Contents}
\label{\detokenize{autoapi/nexoclom2/utilities/NexoclomConfig/index:module-contents}}\index{NexoclomConfig (class in nexoclom2.utilities.NexoclomConfig)@\spxentry{NexoclomConfig}\spxextra{class in nexoclom2.utilities.NexoclomConfig}}

\begin{fulllineitems}
\phantomsection\label{\detokenize{autoapi/nexoclom2/utilities/NexoclomConfig/index:nexoclom2.utilities.NexoclomConfig.NexoclomConfig}}
\pysigstartsignatures
\pysigline
{\sphinxbfcode{\sphinxupquote{class\DUrole{w}{ }}}\sphinxcode{\sphinxupquote{nexoclom2.utilities.NexoclomConfig.}}\sphinxbfcode{\sphinxupquote{NexoclomConfig}}}
\pysigstopsignatures
\sphinxAtStartPar
Configuration object based on the nexoclom2 configuration file.

\sphinxAtStartPar
The \sphinxtitleref{NEXCOCLOMCONFIG} environment variable must be set. This is automatically
set to \sphinxcode{\sphinxupquote{\$HOME/.nexoclom}} when the nexoclom2 Python environment is activated,
although the user is free to change it.

\sphinxAtStartPar
Each line in the nexoclom configuration file should be in the form
\sphinxcode{\sphinxupquote{key = value}} where the keys are highlighed below. Configuration
settings for nexoclom2 extensions (such as for working with speficic
instrument data) can also be placed here. All lines in the file with the
proper format are included in the returned object.

\sphinxAtStartPar
\sphinxcode{\sphinxupquote{savepath}}: Top\sphinxhyphen{}level path on disk where all model output is saved (Required)

\sphinxAtStartPar
\sphinxcode{\sphinxupquote{database}}: Name of the TinyDB database file (Optional). Defaults to
\sphinxcode{\sphinxupquote{thesolarsystemmb.db}}.

\sphinxAtStartPar
\sphinxcode{\sphinxupquote{user}}: username (Required if not set as an environment variable).
\begin{quote}\begin{description}
\sphinxlineitem{Parameters}\begin{description}
\sphinxlineitem{\sphinxstylestrong{None}}
\end{description}

\sphinxlineitem{Attributes}\begin{description}
\sphinxlineitem{\sphinxstylestrong{configfile: str}}
\sphinxAtStartPar
Name of configuration file used.

\sphinxlineitem{\sphinxstylestrong{savepath: str}}
\sphinxAtStartPar
Top\sphinxhyphen{}level path on disk where all model output is saved.

\sphinxlineitem{\sphinxstylestrong{database: str}}
\sphinxAtStartPar
Name of the TinyDB database file modelresults are cataloged in.

\sphinxlineitem{\sphinxstylestrong{user: str}}
\sphinxAtStartPar
User’s username on the system.

\sphinxlineitem{\sphinxstylestrong{other: str}}
\sphinxAtStartPar
Other configuration settings can be included.

\end{description}

\sphinxlineitem{Raises}\begin{description}
\sphinxlineitem{ConfigfileError}
\sphinxAtStartPar
If \sphinxcode{\sphinxupquote{NEXOCLOMCONFIG}} environment variable not set or a required
parameter in the configuration file is not given

\sphinxlineitem{FileNotFoundError}
\sphinxAtStartPar
If the configuration file is not found

\end{description}

\end{description}\end{quote}


\begin{sphinxseealso}{See also:}
\begin{description}
\sphinxlineitem{{\hyperref[\detokenize{autoapi/nexoclom2/utilities/exceptions/index:nexoclom2.utilities.exceptions.ConfigfileError}]{\sphinxcrossref{\sphinxcode{\sphinxupquote{nexoclom2.utilities.exceptions.ConfigfileError}}}}}}
\end{description}


\end{sphinxseealso}

\subsubsection*{Examples}

\sphinxAtStartPar
For a configuration file in \sphinxcode{\sphinxupquote{\$HOME/.nexoclom}} containing the following:

\begin{sphinxVerbatim}[commandchars=\\\{\}]
\PYG{n}{savepath} \PYG{o}{=} \PYG{o}{/}\PYG{n}{user}\PYG{o}{/}\PYG{n}{mburger}\PYG{o}{/}\PYG{n}{Data}\PYG{o}{/}\PYG{n}{ModelData}
\PYG{n}{database} \PYG{o}{=} \PYG{n}{thesolarsystemmb}\PYG{o}{.}\PYG{n}{db}
\PYG{n}{mesdatapath} \PYG{o}{=} \PYG{o}{/}\PYG{n}{Users}\PYG{o}{/}\PYG{n}{mburger}\PYG{o}{/}\PYG{n}{Work}\PYG{o}{/}\PYG{n}{Data}\PYG{o}{/}\PYG{n}{MESSENGER}\PYG{o}{/}\PYG{n}{UVVS}
\PYG{n}{mesdatabase} \PYG{o}{=} \PYG{n}{messengeruvvsdb}
\end{sphinxVerbatim}

\sphinxAtStartPar
In Python:

\begin{sphinxVerbatim}[commandchars=\\\{\}]
\PYG{g+gp}{\PYGZgt{}\PYGZgt{}\PYGZgt{} }\PYG{n}{config} \PYG{o}{=} \PYG{n}{NexoclomConfig}\PYG{p}{(}\PYG{p}{)}
\PYG{g+gp}{\PYGZgt{}\PYGZgt{}\PYGZgt{} }\PYG{n+nb}{print}\PYG{p}{(}\PYG{n}{config}\PYG{p}{)}
\PYG{g+go}{configfile = /Users/mburger/.nexoclom2\PYGZus{}dev}
\PYG{g+go}{savepath = /Volumes/nexoclom\PYGZus{}output/modeloutputs2\PYGZus{}dev}
\PYG{g+go}{database = thesolarsystemmb\PYGZus{}dev.db}
\PYG{g+go}{user = mburger}
\end{sphinxVerbatim}
\begin{quote}\begin{description}
\sphinxlineitem{Authors}
\sphinxAtStartPar
Matthew Burger

\end{description}\end{quote}
\index{configfile (nexoclom2.utilities.NexoclomConfig.NexoclomConfig attribute)@\spxentry{configfile}\spxextra{nexoclom2.utilities.NexoclomConfig.NexoclomConfig attribute}}

\begin{fulllineitems}
\phantomsection\label{\detokenize{autoapi/nexoclom2/utilities/NexoclomConfig/index:nexoclom2.utilities.NexoclomConfig.NexoclomConfig.configfile}}
\pysigstartsignatures
\pysigline
{\sphinxbfcode{\sphinxupquote{configfile}}}
\pysigstopsignatures
\end{fulllineitems}

\index{database (nexoclom2.utilities.NexoclomConfig.NexoclomConfig attribute)@\spxentry{database}\spxextra{nexoclom2.utilities.NexoclomConfig.NexoclomConfig attribute}}

\begin{fulllineitems}
\phantomsection\label{\detokenize{autoapi/nexoclom2/utilities/NexoclomConfig/index:nexoclom2.utilities.NexoclomConfig.NexoclomConfig.database}}
\pysigstartsignatures
\pysigline
{\sphinxbfcode{\sphinxupquote{database}}}
\pysigstopsignatures
\end{fulllineitems}

\index{user (nexoclom2.utilities.NexoclomConfig.NexoclomConfig attribute)@\spxentry{user}\spxextra{nexoclom2.utilities.NexoclomConfig.NexoclomConfig attribute}}

\begin{fulllineitems}
\phantomsection\label{\detokenize{autoapi/nexoclom2/utilities/NexoclomConfig/index:nexoclom2.utilities.NexoclomConfig.NexoclomConfig.user}}
\pysigstartsignatures
\pysigline
{\sphinxbfcode{\sphinxupquote{user}}}
\pysigstopsignatures
\end{fulllineitems}

\index{\_\_str\_\_() (nexoclom2.utilities.NexoclomConfig.NexoclomConfig method)@\spxentry{\_\_str\_\_()}\spxextra{nexoclom2.utilities.NexoclomConfig.NexoclomConfig method}}

\begin{fulllineitems}
\phantomsection\label{\detokenize{autoapi/nexoclom2/utilities/NexoclomConfig/index:nexoclom2.utilities.NexoclomConfig.NexoclomConfig.__str__}}
\pysigstartsignatures
\pysiglinewithargsret
{\sphinxbfcode{\sphinxupquote{\_\_str\_\_}}}
{}
{}
\pysigstopsignatures
\end{fulllineitems}

\index{\_\_repr\_\_() (nexoclom2.utilities.NexoclomConfig.NexoclomConfig method)@\spxentry{\_\_repr\_\_()}\spxextra{nexoclom2.utilities.NexoclomConfig.NexoclomConfig method}}

\begin{fulllineitems}
\phantomsection\label{\detokenize{autoapi/nexoclom2/utilities/NexoclomConfig/index:nexoclom2.utilities.NexoclomConfig.NexoclomConfig.__repr__}}
\pysigstartsignatures
\pysiglinewithargsret
{\sphinxbfcode{\sphinxupquote{\_\_repr\_\_}}}
{}
{}
\pysigstopsignatures
\end{fulllineitems}


\end{fulllineitems}


\sphinxstepscope


\subparagraph{nexoclom2.utilities.database\_operations}
\label{\detokenize{autoapi/nexoclom2/utilities/database_operations/index:module-nexoclom2.utilities.database_operations}}\label{\detokenize{autoapi/nexoclom2/utilities/database_operations/index:nexoclom2-utilities-database-operations}}\label{\detokenize{autoapi/nexoclom2/utilities/database_operations/index::doc}}\index{module@\spxentry{module}!nexoclom2.utilities.database\_operations@\spxentry{nexoclom2.utilities.database\_operations}}\index{nexoclom2.utilities.database\_operations@\spxentry{nexoclom2.utilities.database\_operations}!module@\spxentry{module}}

\subparagraph{Classes}
\label{\detokenize{autoapi/nexoclom2/utilities/database_operations/index:classes}}

\begin{savenotes}\sphinxattablestart
\sphinxthistablewithglobalstyle
\sphinxthistablewithnovlinesstyle
\centering
\begin{tabulary}{\linewidth}[t]{\X{1}{2}\X{1}{2}}
\sphinxtoprule
\sphinxtableatstartofbodyhook
\sphinxAtStartPar
{\hyperref[\detokenize{autoapi/nexoclom2/utilities/database_operations/index:nexoclom2.utilities.database_operations.DatabaseOperations}]{\sphinxcrossref{\sphinxcode{\sphinxupquote{DatabaseOperations}}}}}
&
\sphinxAtStartPar
Manage the TinyDB database
\\
\sphinxbottomrule
\end{tabulary}
\sphinxtableafterendhook\par
\sphinxattableend\end{savenotes}


\subparagraph{Module Contents}
\label{\detokenize{autoapi/nexoclom2/utilities/database_operations/index:module-contents}}\index{DatabaseOperations (class in nexoclom2.utilities.database\_operations)@\spxentry{DatabaseOperations}\spxextra{class in nexoclom2.utilities.database\_operations}}

\begin{fulllineitems}
\phantomsection\label{\detokenize{autoapi/nexoclom2/utilities/database_operations/index:nexoclom2.utilities.database_operations.DatabaseOperations}}
\pysigstartsignatures
\pysigline
{\sphinxbfcode{\sphinxupquote{class\DUrole{w}{ }}}\sphinxcode{\sphinxupquote{nexoclom2.utilities.database\_operations.}}\sphinxbfcode{\sphinxupquote{DatabaseOperations}}}
\pysigstopsignatures
\sphinxAtStartPar
Manage the TinyDB database
\begin{quote}\begin{description}
\sphinxlineitem{Parameters}\begin{description}
\sphinxlineitem{\sphinxstylestrong{None}}
\end{description}

\sphinxlineitem{Attributes}\begin{description}
\sphinxlineitem{\sphinxstylestrong{dp\_path}}{[}str{]}
\sphinxAtStartPar
Path to the database file as specified in the nexoclom configuration file.

\end{description}

\end{description}\end{quote}
\index{db\_path (nexoclom2.utilities.database\_operations.DatabaseOperations attribute)@\spxentry{db\_path}\spxextra{nexoclom2.utilities.database\_operations.DatabaseOperations attribute}}

\begin{fulllineitems}
\phantomsection\label{\detokenize{autoapi/nexoclom2/utilities/database_operations/index:nexoclom2.utilities.database_operations.DatabaseOperations.db_path}}
\pysigstartsignatures
\pysigline
{\sphinxbfcode{\sphinxupquote{db\_path}}}
\pysigstopsignatures
\end{fulllineitems}

\index{reset\_database() (nexoclom2.utilities.database\_operations.DatabaseOperations class method)@\spxentry{reset\_database()}\spxextra{nexoclom2.utilities.database\_operations.DatabaseOperations class method}}

\begin{fulllineitems}
\phantomsection\label{\detokenize{autoapi/nexoclom2/utilities/database_operations/index:nexoclom2.utilities.database_operations.DatabaseOperations.reset_database}}
\pysigstartsignatures
\pysiglinewithargsret
{\sphinxbfcode{\sphinxupquote{classmethod\DUrole{w}{ }}}\sphinxbfcode{\sphinxupquote{reset\_database}}}
{}
{}
\pysigstopsignatures
\end{fulllineitems}

\index{make\_acceptable() (nexoclom2.utilities.database\_operations.DatabaseOperations method)@\spxentry{make\_acceptable()}\spxextra{nexoclom2.utilities.database\_operations.DatabaseOperations method}}

\begin{fulllineitems}
\phantomsection\label{\detokenize{autoapi/nexoclom2/utilities/database_operations/index:nexoclom2.utilities.database_operations.DatabaseOperations.make_acceptable}}
\pysigstartsignatures
\pysiglinewithargsret
{\sphinxbfcode{\sphinxupquote{make\_acceptable}}}
{\sphinxparam{\DUrole{n}{inputs}}}
{{ $\rightarrow$ dict}}
\pysigstopsignatures
\sphinxAtStartPar
Takes a Python input\sphinxhyphen{}type class and converts it to a saveable format.

\sphinxAtStartPar
The input classes (Geometry, etc.) have objects that can not be stored in
the TinyDB format (cannot be converted to json). This function cleans
things up.
\begin{quote}\begin{description}
\sphinxlineitem{Returns}\begin{description}
\sphinxlineitem{Dict object that can be inserted ito the database}
\end{description}

\end{description}\end{quote}

\end{fulllineitems}

\index{insert\_inputs() (nexoclom2.utilities.database\_operations.DatabaseOperations method)@\spxentry{insert\_inputs()}\spxextra{nexoclom2.utilities.database\_operations.DatabaseOperations method}}

\begin{fulllineitems}
\phantomsection\label{\detokenize{autoapi/nexoclom2/utilities/database_operations/index:nexoclom2.utilities.database_operations.DatabaseOperations.insert_inputs}}
\pysigstartsignatures
\pysiglinewithargsret
{\sphinxbfcode{\sphinxupquote{insert\_inputs}}}
{\sphinxparam{\DUrole{n}{inputs}}}
{}
\pysigstopsignatures
\sphinxAtStartPar
Inserts inputs object into database
Adds the individual pieces of the inputs object into their respective
database tables and creates a unique record in the inputs table for the
complete set that will be associated with the saved outputs. There
should only be one savedfile for each unique set of inputs
\begin{quote}\begin{description}
\sphinxlineitem{Parameters}\begin{description}
\sphinxlineitem{\sphinxstylestrong{inputs}}{[}Input{]}
\sphinxlineitem{\sphinxstylestrong{kwargs}}
\sphinxAtStartPar
kwargs contains pieces from the output that need to be saved such as
number of packets run and whether the output is compressed. More
items TDB.

\end{description}

\sphinxlineitem{Returns}\begin{description}
\sphinxlineitem{Document id associated with the complete set of inputs}
\end{description}

\end{description}\end{quote}

\end{fulllineitems}

\index{search\_inputs() (nexoclom2.utilities.database\_operations.DatabaseOperations method)@\spxentry{search\_inputs()}\spxextra{nexoclom2.utilities.database\_operations.DatabaseOperations method}}

\begin{fulllineitems}
\phantomsection\label{\detokenize{autoapi/nexoclom2/utilities/database_operations/index:nexoclom2.utilities.database_operations.DatabaseOperations.search_inputs}}
\pysigstartsignatures
\pysiglinewithargsret
{\sphinxbfcode{\sphinxupquote{search\_inputs}}}
{\sphinxparam{\DUrole{n}{inputs}}}
{}
\pysigstopsignatures
\end{fulllineitems}

\index{delete\_inputs() (nexoclom2.utilities.database\_operations.DatabaseOperations method)@\spxentry{delete\_inputs()}\spxextra{nexoclom2.utilities.database\_operations.DatabaseOperations method}}

\begin{fulllineitems}
\phantomsection\label{\detokenize{autoapi/nexoclom2/utilities/database_operations/index:nexoclom2.utilities.database_operations.DatabaseOperations.delete_inputs}}
\pysigstartsignatures
\pysiglinewithargsret
{\sphinxbfcode{\sphinxupquote{delete\_inputs}}}
{\sphinxparam{\DUrole{n}{doc\_id}}}
{}
\pysigstopsignatures
\end{fulllineitems}

\index{return\_table() (nexoclom2.utilities.database\_operations.DatabaseOperations method)@\spxentry{return\_table()}\spxextra{nexoclom2.utilities.database\_operations.DatabaseOperations method}}

\begin{fulllineitems}
\phantomsection\label{\detokenize{autoapi/nexoclom2/utilities/database_operations/index:nexoclom2.utilities.database_operations.DatabaseOperations.return_table}}
\pysigstartsignatures
\pysiglinewithargsret
{\sphinxbfcode{\sphinxupquote{return\_table}}}
{\sphinxparam{\DUrole{n}{tablename}\DUrole{p}{:}\DUrole{w}{ }\DUrole{n}{str}}}
{}
\pysigstopsignatures
\sphinxAtStartPar
Retrieve all records for an input classes (inputs.geometry, etc.)
\begin{quote}\begin{description}
\sphinxlineitem{Parameters}\begin{description}
\sphinxlineitem{\sphinxstylestrong{tablename}}{[}str{]}
\end{description}

\sphinxlineitem{Returns}\begin{description}
\sphinxlineitem{List of TinyDB Documents if there are any; None if not.}
\end{description}

\end{description}\end{quote}

\end{fulllineitems}


\end{fulllineitems}


\sphinxstepscope


\subparagraph{nexoclom2.utilities.exceptions}
\label{\detokenize{autoapi/nexoclom2/utilities/exceptions/index:module-nexoclom2.utilities.exceptions}}\label{\detokenize{autoapi/nexoclom2/utilities/exceptions/index:nexoclom2-utilities-exceptions}}\label{\detokenize{autoapi/nexoclom2/utilities/exceptions/index::doc}}\index{module@\spxentry{module}!nexoclom2.utilities.exceptions@\spxentry{nexoclom2.utilities.exceptions}}\index{nexoclom2.utilities.exceptions@\spxentry{nexoclom2.utilities.exceptions}!module@\spxentry{module}}
\sphinxAtStartPar
Exceptions used in nexoclom


\subparagraph{Exceptions}
\label{\detokenize{autoapi/nexoclom2/utilities/exceptions/index:exceptions}}

\begin{savenotes}\sphinxattablestart
\sphinxthistablewithglobalstyle
\sphinxthistablewithnovlinesstyle
\centering
\begin{tabulary}{\linewidth}[t]{\X{1}{2}\X{1}{2}}
\sphinxtoprule
\sphinxtableatstartofbodyhook
\sphinxAtStartPar
{\hyperref[\detokenize{autoapi/nexoclom2/utilities/exceptions/index:nexoclom2.utilities.exceptions.InputfileError}]{\sphinxcrossref{\sphinxcode{\sphinxupquote{InputfileError}}}}}
&
\sphinxAtStartPar
Raised when a required parameter is not included
\\
\sphinxhline
\sphinxAtStartPar
{\hyperref[\detokenize{autoapi/nexoclom2/utilities/exceptions/index:nexoclom2.utilities.exceptions.ConfigfileError}]{\sphinxcrossref{\sphinxcode{\sphinxupquote{ConfigfileError}}}}}
&
\sphinxAtStartPar
Raised when there is a configuration file problem
\\
\sphinxhline
\sphinxAtStartPar
{\hyperref[\detokenize{autoapi/nexoclom2/utilities/exceptions/index:nexoclom2.utilities.exceptions.OutOfRangeError}]{\sphinxcrossref{\sphinxcode{\sphinxupquote{OutOfRangeError}}}}}
&
\sphinxAtStartPar
Raised when an value in an input file is out of specified range
\\
\sphinxbottomrule
\end{tabulary}
\sphinxtableafterendhook\par
\sphinxattableend\end{savenotes}


\subparagraph{Module Contents}
\label{\detokenize{autoapi/nexoclom2/utilities/exceptions/index:module-contents}}\index{InputfileError@\spxentry{InputfileError}}

\begin{fulllineitems}
\phantomsection\label{\detokenize{autoapi/nexoclom2/utilities/exceptions/index:nexoclom2.utilities.exceptions.InputfileError}}
\pysigstartsignatures
\pysiglinewithargsret
{\sphinxbfcode{\sphinxupquote{exception\DUrole{w}{ }}}\sphinxcode{\sphinxupquote{nexoclom2.utilities.exceptions.}}\sphinxbfcode{\sphinxupquote{InputfileError}}}
{\sphinxparam{\DUrole{n}{expression}}\sphinxparamcomma \sphinxparam{\DUrole{n}{message}}}
{}
\pysigstopsignatures
\sphinxAtStartPar
Bases: \sphinxcode{\sphinxupquote{Exception}}

\sphinxAtStartPar
Raised when a required parameter is not included
\index{expression (nexoclom2.utilities.exceptions.InputfileError attribute)@\spxentry{expression}\spxextra{nexoclom2.utilities.exceptions.InputfileError attribute}}

\begin{fulllineitems}
\phantomsection\label{\detokenize{autoapi/nexoclom2/utilities/exceptions/index:nexoclom2.utilities.exceptions.InputfileError.expression}}
\pysigstartsignatures
\pysigline
{\sphinxbfcode{\sphinxupquote{expression}}}
\pysigstopsignatures
\end{fulllineitems}

\index{message (nexoclom2.utilities.exceptions.InputfileError attribute)@\spxentry{message}\spxextra{nexoclom2.utilities.exceptions.InputfileError attribute}}

\begin{fulllineitems}
\phantomsection\label{\detokenize{autoapi/nexoclom2/utilities/exceptions/index:nexoclom2.utilities.exceptions.InputfileError.message}}
\pysigstartsignatures
\pysigline
{\sphinxbfcode{\sphinxupquote{message}}}
\pysigstopsignatures
\end{fulllineitems}


\end{fulllineitems}

\index{ConfigfileError@\spxentry{ConfigfileError}}

\begin{fulllineitems}
\phantomsection\label{\detokenize{autoapi/nexoclom2/utilities/exceptions/index:nexoclom2.utilities.exceptions.ConfigfileError}}
\pysigstartsignatures
\pysiglinewithargsret
{\sphinxbfcode{\sphinxupquote{exception\DUrole{w}{ }}}\sphinxcode{\sphinxupquote{nexoclom2.utilities.exceptions.}}\sphinxbfcode{\sphinxupquote{ConfigfileError}}}
{\sphinxparam{\DUrole{n}{configfile}}\sphinxparamcomma \sphinxparam{\DUrole{n}{message}}}
{}
\pysigstopsignatures
\sphinxAtStartPar
Bases: \sphinxcode{\sphinxupquote{Exception}}

\sphinxAtStartPar
Raised when there is a configuration file problem
\index{expression (nexoclom2.utilities.exceptions.ConfigfileError attribute)@\spxentry{expression}\spxextra{nexoclom2.utilities.exceptions.ConfigfileError attribute}}

\begin{fulllineitems}
\phantomsection\label{\detokenize{autoapi/nexoclom2/utilities/exceptions/index:nexoclom2.utilities.exceptions.ConfigfileError.expression}}
\pysigstartsignatures
\pysigline
{\sphinxbfcode{\sphinxupquote{expression}}}
\pysigstopsignatures
\end{fulllineitems}

\index{message (nexoclom2.utilities.exceptions.ConfigfileError attribute)@\spxentry{message}\spxextra{nexoclom2.utilities.exceptions.ConfigfileError attribute}}

\begin{fulllineitems}
\phantomsection\label{\detokenize{autoapi/nexoclom2/utilities/exceptions/index:nexoclom2.utilities.exceptions.ConfigfileError.message}}
\pysigstartsignatures
\pysigline
{\sphinxbfcode{\sphinxupquote{message}}}
\pysigstopsignatures
\end{fulllineitems}


\end{fulllineitems}

\index{OutOfRangeError@\spxentry{OutOfRangeError}}

\begin{fulllineitems}
\phantomsection\label{\detokenize{autoapi/nexoclom2/utilities/exceptions/index:nexoclom2.utilities.exceptions.OutOfRangeError}}
\pysigstartsignatures
\pysiglinewithargsret
{\sphinxbfcode{\sphinxupquote{exception\DUrole{w}{ }}}\sphinxcode{\sphinxupquote{nexoclom2.utilities.exceptions.}}\sphinxbfcode{\sphinxupquote{OutOfRangeError}}}
{\sphinxparam{\DUrole{n}{expression}}\sphinxparamcomma \sphinxparam{\DUrole{n}{param}}\sphinxparamcomma \sphinxparam{\DUrole{n}{rng}}\sphinxparamcomma \sphinxparam{\DUrole{n}{include\_min}\DUrole{o}{=}\DUrole{default_value}{True}}\sphinxparamcomma \sphinxparam{\DUrole{n}{include\_max}\DUrole{o}{=}\DUrole{default_value}{True}}}
{}
\pysigstopsignatures
\sphinxAtStartPar
Bases: \sphinxcode{\sphinxupquote{Exception}}

\sphinxAtStartPar
Raised when an value in an input file is out of specified range
\index{expression (nexoclom2.utilities.exceptions.OutOfRangeError attribute)@\spxentry{expression}\spxextra{nexoclom2.utilities.exceptions.OutOfRangeError attribute}}

\begin{fulllineitems}
\phantomsection\label{\detokenize{autoapi/nexoclom2/utilities/exceptions/index:nexoclom2.utilities.exceptions.OutOfRangeError.expression}}
\pysigstartsignatures
\pysigline
{\sphinxbfcode{\sphinxupquote{expression}}}
\pysigstopsignatures
\end{fulllineitems}

\index{format\_number() (nexoclom2.utilities.exceptions.OutOfRangeError method)@\spxentry{format\_number()}\spxextra{nexoclom2.utilities.exceptions.OutOfRangeError method}}

\begin{fulllineitems}
\phantomsection\label{\detokenize{autoapi/nexoclom2/utilities/exceptions/index:nexoclom2.utilities.exceptions.OutOfRangeError.format_number}}
\pysigstartsignatures
\pysiglinewithargsret
{\sphinxbfcode{\sphinxupquote{format\_number}}}
{\sphinxparam{\DUrole{n}{num\_}}}
{}
\pysigstopsignatures
\end{fulllineitems}


\end{fulllineitems}


\sphinxstepscope


\subparagraph{nexoclom2.utilities.sci\_notation}
\label{\detokenize{autoapi/nexoclom2/utilities/sci_notation/index:module-nexoclom2.utilities.sci_notation}}\label{\detokenize{autoapi/nexoclom2/utilities/sci_notation/index:nexoclom2-utilities-sci-notation}}\label{\detokenize{autoapi/nexoclom2/utilities/sci_notation/index::doc}}\index{module@\spxentry{module}!nexoclom2.utilities.sci\_notation@\spxentry{nexoclom2.utilities.sci\_notation}}\index{nexoclom2.utilities.sci\_notation@\spxentry{nexoclom2.utilities.sci\_notation}!module@\spxentry{module}}

\subparagraph{Functions}
\label{\detokenize{autoapi/nexoclom2/utilities/sci_notation/index:functions}}

\begin{savenotes}\sphinxattablestart
\sphinxthistablewithglobalstyle
\sphinxthistablewithnovlinesstyle
\centering
\begin{tabulary}{\linewidth}[t]{\X{1}{2}\X{1}{2}}
\sphinxtoprule
\sphinxtableatstartofbodyhook
\sphinxAtStartPar
{\hyperref[\detokenize{autoapi/nexoclom2/utilities/sci_notation/index:nexoclom2.utilities.sci_notation.sci_note}]{\sphinxcrossref{\sphinxcode{\sphinxupquote{sci\_note}}}}}(x, pos)
&
\sphinxAtStartPar

\\
\sphinxbottomrule
\end{tabulary}
\sphinxtableafterendhook\par
\sphinxattableend\end{savenotes}


\subparagraph{Module Contents}
\label{\detokenize{autoapi/nexoclom2/utilities/sci_notation/index:module-contents}}\index{sci\_note() (in module nexoclom2.utilities.sci\_notation)@\spxentry{sci\_note()}\spxextra{in module nexoclom2.utilities.sci\_notation}}

\begin{fulllineitems}
\phantomsection\label{\detokenize{autoapi/nexoclom2/utilities/sci_notation/index:nexoclom2.utilities.sci_notation.sci_note}}
\pysigstartsignatures
\pysiglinewithargsret
{\sphinxcode{\sphinxupquote{nexoclom2.utilities.sci\_notation.}}\sphinxbfcode{\sphinxupquote{sci\_note}}}
{\sphinxparam{\DUrole{n}{x}}\sphinxparamcomma \sphinxparam{\DUrole{n}{pos}}}
{}
\pysigstopsignatures
\end{fulllineitems}



\subparagraph{Classes}
\label{\detokenize{autoapi/nexoclom2/utilities/index:classes}}

\begin{savenotes}\sphinxattablestart
\sphinxthistablewithglobalstyle
\sphinxthistablewithnovlinesstyle
\centering
\begin{tabulary}{\linewidth}[t]{\X{1}{2}\X{1}{2}}
\sphinxtoprule
\sphinxtableatstartofbodyhook
\sphinxAtStartPar
{\hyperref[\detokenize{autoapi/nexoclom2/utilities/index:nexoclom2.utilities.DatabaseOperations}]{\sphinxcrossref{\sphinxcode{\sphinxupquote{DatabaseOperations}}}}}
&
\sphinxAtStartPar
Manage the TinyDB database
\\
\sphinxbottomrule
\end{tabulary}
\sphinxtableafterendhook\par
\sphinxattableend\end{savenotes}


\subparagraph{Functions}
\label{\detokenize{autoapi/nexoclom2/utilities/index:functions}}

\begin{savenotes}\sphinxattablestart
\sphinxthistablewithglobalstyle
\sphinxthistablewithnovlinesstyle
\centering
\begin{tabulary}{\linewidth}[t]{\X{1}{2}\X{1}{2}}
\sphinxtoprule
\sphinxtableatstartofbodyhook
\sphinxAtStartPar
{\hyperref[\detokenize{autoapi/nexoclom2/utilities/index:nexoclom2.utilities.sci_note}]{\sphinxcrossref{\sphinxcode{\sphinxupquote{sci\_note}}}}}(x, pos)
&
\sphinxAtStartPar

\\
\sphinxbottomrule
\end{tabulary}
\sphinxtableafterendhook\par
\sphinxattableend\end{savenotes}


\subparagraph{Package Contents}
\label{\detokenize{autoapi/nexoclom2/utilities/index:package-contents}}\index{DatabaseOperations (class in nexoclom2.utilities)@\spxentry{DatabaseOperations}\spxextra{class in nexoclom2.utilities}}

\begin{fulllineitems}
\phantomsection\label{\detokenize{autoapi/nexoclom2/utilities/index:nexoclom2.utilities.DatabaseOperations}}
\pysigstartsignatures
\pysigline
{\sphinxbfcode{\sphinxupquote{class\DUrole{w}{ }}}\sphinxcode{\sphinxupquote{nexoclom2.utilities.}}\sphinxbfcode{\sphinxupquote{DatabaseOperations}}}
\pysigstopsignatures
\sphinxAtStartPar
Manage the TinyDB database
\begin{quote}\begin{description}
\sphinxlineitem{Parameters}\begin{description}
\sphinxlineitem{\sphinxstylestrong{None}}
\end{description}

\sphinxlineitem{Attributes}\begin{description}
\sphinxlineitem{\sphinxstylestrong{dp\_path}}{[}str{]}
\sphinxAtStartPar
Path to the database file as specified in the nexoclom configuration file.

\end{description}

\end{description}\end{quote}
\index{db\_path (nexoclom2.utilities.DatabaseOperations attribute)@\spxentry{db\_path}\spxextra{nexoclom2.utilities.DatabaseOperations attribute}}

\begin{fulllineitems}
\phantomsection\label{\detokenize{autoapi/nexoclom2/utilities/index:nexoclom2.utilities.DatabaseOperations.db_path}}
\pysigstartsignatures
\pysigline
{\sphinxbfcode{\sphinxupquote{db\_path}}}
\pysigstopsignatures
\end{fulllineitems}

\index{reset\_database() (nexoclom2.utilities.DatabaseOperations class method)@\spxentry{reset\_database()}\spxextra{nexoclom2.utilities.DatabaseOperations class method}}

\begin{fulllineitems}
\phantomsection\label{\detokenize{autoapi/nexoclom2/utilities/index:nexoclom2.utilities.DatabaseOperations.reset_database}}
\pysigstartsignatures
\pysiglinewithargsret
{\sphinxbfcode{\sphinxupquote{classmethod\DUrole{w}{ }}}\sphinxbfcode{\sphinxupquote{reset\_database}}}
{}
{}
\pysigstopsignatures
\end{fulllineitems}

\index{make\_acceptable() (nexoclom2.utilities.DatabaseOperations method)@\spxentry{make\_acceptable()}\spxextra{nexoclom2.utilities.DatabaseOperations method}}

\begin{fulllineitems}
\phantomsection\label{\detokenize{autoapi/nexoclom2/utilities/index:nexoclom2.utilities.DatabaseOperations.make_acceptable}}
\pysigstartsignatures
\pysiglinewithargsret
{\sphinxbfcode{\sphinxupquote{make\_acceptable}}}
{\sphinxparam{\DUrole{n}{inputs}}}
{{ $\rightarrow$ dict}}
\pysigstopsignatures
\sphinxAtStartPar
Takes a Python input\sphinxhyphen{}type class and converts it to a saveable format.

\sphinxAtStartPar
The input classes (Geometry, etc.) have objects that can not be stored in
the TinyDB format (cannot be converted to json). This function cleans
things up.
\begin{quote}\begin{description}
\sphinxlineitem{Returns}\begin{description}
\sphinxlineitem{Dict object that can be inserted ito the database}
\end{description}

\end{description}\end{quote}

\end{fulllineitems}

\index{insert\_inputs() (nexoclom2.utilities.DatabaseOperations method)@\spxentry{insert\_inputs()}\spxextra{nexoclom2.utilities.DatabaseOperations method}}

\begin{fulllineitems}
\phantomsection\label{\detokenize{autoapi/nexoclom2/utilities/index:nexoclom2.utilities.DatabaseOperations.insert_inputs}}
\pysigstartsignatures
\pysiglinewithargsret
{\sphinxbfcode{\sphinxupquote{insert\_inputs}}}
{\sphinxparam{\DUrole{n}{inputs}}}
{}
\pysigstopsignatures
\sphinxAtStartPar
Inserts inputs object into database
Adds the individual pieces of the inputs object into their respective
database tables and creates a unique record in the inputs table for the
complete set that will be associated with the saved outputs. There
should only be one savedfile for each unique set of inputs
\begin{quote}\begin{description}
\sphinxlineitem{Parameters}\begin{description}
\sphinxlineitem{\sphinxstylestrong{inputs}}{[}Input{]}
\sphinxlineitem{\sphinxstylestrong{kwargs}}
\sphinxAtStartPar
kwargs contains pieces from the output that need to be saved such as
number of packets run and whether the output is compressed. More
items TDB.

\end{description}

\sphinxlineitem{Returns}\begin{description}
\sphinxlineitem{Document id associated with the complete set of inputs}
\end{description}

\end{description}\end{quote}

\end{fulllineitems}

\index{search\_inputs() (nexoclom2.utilities.DatabaseOperations method)@\spxentry{search\_inputs()}\spxextra{nexoclom2.utilities.DatabaseOperations method}}

\begin{fulllineitems}
\phantomsection\label{\detokenize{autoapi/nexoclom2/utilities/index:nexoclom2.utilities.DatabaseOperations.search_inputs}}
\pysigstartsignatures
\pysiglinewithargsret
{\sphinxbfcode{\sphinxupquote{search\_inputs}}}
{\sphinxparam{\DUrole{n}{inputs}}}
{}
\pysigstopsignatures
\end{fulllineitems}

\index{delete\_inputs() (nexoclom2.utilities.DatabaseOperations method)@\spxentry{delete\_inputs()}\spxextra{nexoclom2.utilities.DatabaseOperations method}}

\begin{fulllineitems}
\phantomsection\label{\detokenize{autoapi/nexoclom2/utilities/index:nexoclom2.utilities.DatabaseOperations.delete_inputs}}
\pysigstartsignatures
\pysiglinewithargsret
{\sphinxbfcode{\sphinxupquote{delete\_inputs}}}
{\sphinxparam{\DUrole{n}{doc\_id}}}
{}
\pysigstopsignatures
\end{fulllineitems}

\index{return\_table() (nexoclom2.utilities.DatabaseOperations method)@\spxentry{return\_table()}\spxextra{nexoclom2.utilities.DatabaseOperations method}}

\begin{fulllineitems}
\phantomsection\label{\detokenize{autoapi/nexoclom2/utilities/index:nexoclom2.utilities.DatabaseOperations.return_table}}
\pysigstartsignatures
\pysiglinewithargsret
{\sphinxbfcode{\sphinxupquote{return\_table}}}
{\sphinxparam{\DUrole{n}{tablename}\DUrole{p}{:}\DUrole{w}{ }\DUrole{n}{str}}}
{}
\pysigstopsignatures
\sphinxAtStartPar
Retrieve all records for an input classes (inputs.geometry, etc.)
\begin{quote}\begin{description}
\sphinxlineitem{Parameters}\begin{description}
\sphinxlineitem{\sphinxstylestrong{tablename}}{[}str{]}
\end{description}

\sphinxlineitem{Returns}\begin{description}
\sphinxlineitem{List of TinyDB Documents if there are any; None if not.}
\end{description}

\end{description}\end{quote}

\end{fulllineitems}


\end{fulllineitems}

\index{sci\_note() (in module nexoclom2.utilities)@\spxentry{sci\_note()}\spxextra{in module nexoclom2.utilities}}

\begin{fulllineitems}
\phantomsection\label{\detokenize{autoapi/nexoclom2/utilities/index:nexoclom2.utilities.sci_note}}
\pysigstartsignatures
\pysiglinewithargsret
{\sphinxcode{\sphinxupquote{nexoclom2.utilities.}}\sphinxbfcode{\sphinxupquote{sci\_note}}}
{\sphinxparam{\DUrole{n}{x}}\sphinxparamcomma \sphinxparam{\DUrole{n}{pos}}}
{}
\pysigstopsignatures
\end{fulllineitems}



\subsubsection{Attributes}
\label{\detokenize{autoapi/nexoclom2/index:attributes}}

\begin{savenotes}\sphinxattablestart
\sphinxthistablewithglobalstyle
\sphinxthistablewithnovlinesstyle
\centering
\begin{tabulary}{\linewidth}[t]{\X{1}{2}\X{1}{2}}
\sphinxtoprule
\sphinxtableatstartofbodyhook
\sphinxAtStartPar
{\hyperref[\detokenize{autoapi/nexoclom2/index:nexoclom2.path}]{\sphinxcrossref{\sphinxcode{\sphinxupquote{path}}}}}
&
\sphinxAtStartPar

\\
\sphinxhline
\sphinxAtStartPar
{\hyperref[\detokenize{autoapi/nexoclom2/index:nexoclom2.config}]{\sphinxcrossref{\sphinxcode{\sphinxupquote{config}}}}}
&
\sphinxAtStartPar

\\
\sphinxhline
\sphinxAtStartPar
{\hyperref[\detokenize{autoapi/nexoclom2/index:nexoclom2.__name__}]{\sphinxcrossref{\sphinxcode{\sphinxupquote{\_\_name\_\_}}}}}
&
\sphinxAtStartPar

\\
\sphinxhline
\sphinxAtStartPar
{\hyperref[\detokenize{autoapi/nexoclom2/index:nexoclom2.__author__}]{\sphinxcrossref{\sphinxcode{\sphinxupquote{\_\_author\_\_}}}}}
&
\sphinxAtStartPar

\\
\sphinxhline
\sphinxAtStartPar
{\hyperref[\detokenize{autoapi/nexoclom2/index:nexoclom2.__email__}]{\sphinxcrossref{\sphinxcode{\sphinxupquote{\_\_email\_\_}}}}}
&
\sphinxAtStartPar

\\
\sphinxhline
\sphinxAtStartPar
{\hyperref[\detokenize{autoapi/nexoclom2/index:nexoclom2.__version__}]{\sphinxcrossref{\sphinxcode{\sphinxupquote{\_\_version\_\_}}}}}
&
\sphinxAtStartPar

\\
\sphinxbottomrule
\end{tabulary}
\sphinxtableafterendhook\par
\sphinxattableend\end{savenotes}


\subsubsection{Classes}
\label{\detokenize{autoapi/nexoclom2/index:classes}}

\begin{savenotes}\sphinxattablestart
\sphinxthistablewithglobalstyle
\sphinxthistablewithnovlinesstyle
\centering
\begin{tabulary}{\linewidth}[t]{\X{1}{2}\X{1}{2}}
\sphinxtoprule
\sphinxtableatstartofbodyhook
\sphinxAtStartPar
{\hyperref[\detokenize{autoapi/nexoclom2/index:nexoclom2.Input}]{\sphinxcrossref{\sphinxcode{\sphinxupquote{Input}}}}}
&
\sphinxAtStartPar
Class defining all input parameters for a NEXOCLOM2 model run.
\\
\sphinxhline
\sphinxAtStartPar
{\hyperref[\detokenize{autoapi/nexoclom2/index:nexoclom2.Output}]{\sphinxcrossref{\sphinxcode{\sphinxupquote{Output}}}}}
&
\sphinxAtStartPar
Class to store compute particle trajectories and store the results.
\\
\sphinxhline
\sphinxAtStartPar
{\hyperref[\detokenize{autoapi/nexoclom2/index:nexoclom2.ModelImage}]{\sphinxcrossref{\sphinxcode{\sphinxupquote{ModelImage}}}}}
&
\sphinxAtStartPar

\\
\sphinxhline
\sphinxAtStartPar
{\hyperref[\detokenize{autoapi/nexoclom2/index:nexoclom2.SSObject}]{\sphinxcrossref{\sphinxcode{\sphinxupquote{SSObject}}}}}
&
\sphinxAtStartPar
Physical data for solar system bodies.
\\
\sphinxhline
\sphinxAtStartPar
{\hyperref[\detokenize{autoapi/nexoclom2/index:nexoclom2.NexoclomConfig}]{\sphinxcrossref{\sphinxcode{\sphinxupquote{NexoclomConfig}}}}}
&
\sphinxAtStartPar
Configuration object based on the nexoclom2 configuration file.
\\
\sphinxbottomrule
\end{tabulary}
\sphinxtableafterendhook\par
\sphinxattableend\end{savenotes}


\subsubsection{Package Contents}
\label{\detokenize{autoapi/nexoclom2/index:package-contents}}\index{path (in module nexoclom2)@\spxentry{path}\spxextra{in module nexoclom2}}

\begin{fulllineitems}
\phantomsection\label{\detokenize{autoapi/nexoclom2/index:nexoclom2.path}}
\pysigstartsignatures
\pysigline
{\sphinxcode{\sphinxupquote{nexoclom2.}}\sphinxbfcode{\sphinxupquote{path}}}
\pysigstopsignatures
\end{fulllineitems}

\index{Input (class in nexoclom2)@\spxentry{Input}\spxextra{class in nexoclom2}}

\begin{fulllineitems}
\phantomsection\label{\detokenize{autoapi/nexoclom2/index:nexoclom2.Input}}
\pysigstartsignatures
\pysiglinewithargsret
{\sphinxbfcode{\sphinxupquote{class\DUrole{w}{ }}}\sphinxcode{\sphinxupquote{nexoclom2.}}\sphinxbfcode{\sphinxupquote{Input}}}
{\sphinxparam{\DUrole{n}{infile}\DUrole{p}{:}\DUrole{w}{ }\DUrole{n}{str}}}
{}
\pysigstopsignatures
\sphinxAtStartPar
Class defining all input parameters for a NEXOCLOM2 model run.
\begin{quote}\begin{description}
\sphinxlineitem{Parameters}\begin{description}
\sphinxlineitem{\sphinxstylestrong{infile}}{[}str{]}
\sphinxAtStartPar
plain text file containing model input parameters. See inputfiles for
a description of the input file format

\end{description}

\sphinxlineitem{Attributes}\begin{description}
\sphinxlineitem{\sphinxstylestrong{geometry}}{[}Geometry{]}
\sphinxlineitem{\sphinxstylestrong{forces}}{[}Forces{]}
\sphinxlineitem{\sphinxstylestrong{surfaceinteraction}}{[}ConstantSurfaceInteraction, etc{]}
\sphinxlineitem{\sphinxstylestrong{spatialdist}}{[}UniformSpatialDist, etc.{]}
\sphinxlineitem{\sphinxstylestrong{speeddist}}{[}GaussianSpeedDist, etc.{]}
\sphinxlineitem{\sphinxstylestrong{angulardist}}{[}RadialAngularDist, IsotropicAngularDist{]}
\sphinxlineitem{\sphinxstylestrong{lossinfo}}{[}LossInformation{]}
\sphinxlineitem{\sphinxstylestrong{options}}{[}Options{]}
\end{description}

\end{description}\end{quote}
\index{\_inputfile (nexoclom2.Input attribute)@\spxentry{\_inputfile}\spxextra{nexoclom2.Input attribute}}

\begin{fulllineitems}
\phantomsection\label{\detokenize{autoapi/nexoclom2/index:nexoclom2.Input._inputfile}}
\pysigstartsignatures
\pysigline
{\sphinxbfcode{\sphinxupquote{\_inputfile}}}
\pysigstopsignatures
\end{fulllineitems}

\index{\_classes (nexoclom2.Input attribute)@\spxentry{\_classes}\spxextra{nexoclom2.Input attribute}}

\begin{fulllineitems}
\phantomsection\label{\detokenize{autoapi/nexoclom2/index:nexoclom2.Input._classes}}
\pysigstartsignatures
\pysigline
{\sphinxbfcode{\sphinxupquote{\_classes}}\sphinxbfcode{\sphinxupquote{\DUrole{w}{ }\DUrole{p}{=}\DUrole{w}{ }{[}\textquotesingle{}geometry\textquotesingle{}, \textquotesingle{}surfaceinteraction\textquotesingle{}, \textquotesingle{}forces\textquotesingle{}, \textquotesingle{}spatialdist\textquotesingle{}, \textquotesingle{}speeddist\textquotesingle{}, \textquotesingle{}angulardist\textquotesingle{},...}}}
\pysigstopsignatures
\end{fulllineitems}

\index{config (nexoclom2.Input attribute)@\spxentry{config}\spxextra{nexoclom2.Input attribute}}

\begin{fulllineitems}
\phantomsection\label{\detokenize{autoapi/nexoclom2/index:nexoclom2.Input.config}}
\pysigstartsignatures
\pysigline
{\sphinxbfcode{\sphinxupquote{config}}}
\pysigstopsignatures
\end{fulllineitems}

\index{forces (nexoclom2.Input attribute)@\spxentry{forces}\spxextra{nexoclom2.Input attribute}}

\begin{fulllineitems}
\phantomsection\label{\detokenize{autoapi/nexoclom2/index:nexoclom2.Input.forces}}
\pysigstartsignatures
\pysigline
{\sphinxbfcode{\sphinxupquote{forces}}}
\pysigstopsignatures
\end{fulllineitems}

\index{lossinfo (nexoclom2.Input attribute)@\spxentry{lossinfo}\spxextra{nexoclom2.Input attribute}}

\begin{fulllineitems}
\phantomsection\label{\detokenize{autoapi/nexoclom2/index:nexoclom2.Input.lossinfo}}
\pysigstartsignatures
\pysigline
{\sphinxbfcode{\sphinxupquote{lossinfo}}}
\pysigstopsignatures
\end{fulllineitems}

\index{options (nexoclom2.Input attribute)@\spxentry{options}\spxextra{nexoclom2.Input attribute}}

\begin{fulllineitems}
\phantomsection\label{\detokenize{autoapi/nexoclom2/index:nexoclom2.Input.options}}
\pysigstartsignatures
\pysigline
{\sphinxbfcode{\sphinxupquote{options}}}
\pysigstopsignatures
\end{fulllineitems}

\index{\_\_str\_\_() (nexoclom2.Input method)@\spxentry{\_\_str\_\_()}\spxextra{nexoclom2.Input method}}

\begin{fulllineitems}
\phantomsection\label{\detokenize{autoapi/nexoclom2/index:nexoclom2.Input.__str__}}
\pysigstartsignatures
\pysiglinewithargsret
{\sphinxbfcode{\sphinxupquote{\_\_str\_\_}}}
{}
{}
\pysigstopsignatures
\end{fulllineitems}

\index{\_\_repr\_\_() (nexoclom2.Input method)@\spxentry{\_\_repr\_\_()}\spxextra{nexoclom2.Input method}}

\begin{fulllineitems}
\phantomsection\label{\detokenize{autoapi/nexoclom2/index:nexoclom2.Input.__repr__}}
\pysigstartsignatures
\pysiglinewithargsret
{\sphinxbfcode{\sphinxupquote{\_\_repr\_\_}}}
{}
{}
\pysigstopsignatures
\end{fulllineitems}

\index{\_\_eq\_\_() (nexoclom2.Input method)@\spxentry{\_\_eq\_\_()}\spxextra{nexoclom2.Input method}}

\begin{fulllineitems}
\phantomsection\label{\detokenize{autoapi/nexoclom2/index:nexoclom2.Input.__eq__}}
\pysigstartsignatures
\pysiglinewithargsret
{\sphinxbfcode{\sphinxupquote{\_\_eq\_\_}}}
{\sphinxparam{\DUrole{n}{other}}}
{}
\pysigstopsignatures
\end{fulllineitems}

\index{\_read\_params() (nexoclom2.Input method)@\spxentry{\_read\_params()}\spxextra{nexoclom2.Input method}}

\begin{fulllineitems}
\phantomsection\label{\detokenize{autoapi/nexoclom2/index:nexoclom2.Input._read_params}}
\pysigstartsignatures
\pysiglinewithargsret
{\sphinxbfcode{\sphinxupquote{\_read\_params}}}
{}
{}
\pysigstopsignatures
\end{fulllineitems}

\index{search() (nexoclom2.Input method)@\spxentry{search()}\spxextra{nexoclom2.Input method}}

\begin{fulllineitems}
\phantomsection\label{\detokenize{autoapi/nexoclom2/index:nexoclom2.Input.search}}
\pysigstartsignatures
\pysiglinewithargsret
{\sphinxbfcode{\sphinxupquote{search}}}
{}
{}
\pysigstopsignatures
\sphinxAtStartPar
This method allows users to search for inputs without having to
instantiate the database themselves.
\begin{quote}\begin{description}
\sphinxlineitem{Returns}\begin{description}
\sphinxlineitem{Database document ID associated with inputs}
\end{description}

\end{description}\end{quote}

\end{fulllineitems}

\index{make\_savefile() (nexoclom2.Input method)@\spxentry{make\_savefile()}\spxextra{nexoclom2.Input method}}

\begin{fulllineitems}
\phantomsection\label{\detokenize{autoapi/nexoclom2/index:nexoclom2.Input.make_savefile}}
\pysigstartsignatures
\pysiglinewithargsret
{\sphinxbfcode{\sphinxupquote{make\_savefile}}}
{\sphinxparam{\DUrole{n}{doc\_id}}}
{}
\pysigstopsignatures
\end{fulllineitems}


\end{fulllineitems}

\index{Output (class in nexoclom2)@\spxentry{Output}\spxextra{class in nexoclom2}}

\begin{fulllineitems}
\phantomsection\label{\detokenize{autoapi/nexoclom2/index:nexoclom2.Output}}
\pysigstartsignatures
\pysiglinewithargsret
{\sphinxbfcode{\sphinxupquote{class\DUrole{w}{ }}}\sphinxcode{\sphinxupquote{nexoclom2.}}\sphinxbfcode{\sphinxupquote{Output}}}
{\sphinxparam{\DUrole{n}{inputs}}\sphinxparamcomma \sphinxparam{\DUrole{n}{n\_packets}\DUrole{o}{=}\DUrole{default_value}{0}}\sphinxparamcomma \sphinxparam{\DUrole{n}{n\_iterations}\DUrole{o}{=}\DUrole{default_value}{1}}\sphinxparamcomma \sphinxparam{\DUrole{n}{compress}\DUrole{o}{=}\DUrole{default_value}{True}}\sphinxparamcomma \sphinxparam{\DUrole{n}{overwrite}\DUrole{o}{=}\DUrole{default_value}{False}}}
{}
\pysigstopsignatures
\sphinxAtStartPar
Class to store compute particle trajectories and store the results.
\begin{quote}\begin{description}
\sphinxlineitem{Parameters}\begin{description}
\sphinxlineitem{\sphinxstylestrong{inputs}}{[}Input{]}
\sphinxlineitem{\sphinxstylestrong{n\_packets}}{[}int{]}
\sphinxlineitem{\sphinxstylestrong{compress}}{[}bool, Default=True{]}
\end{description}

\sphinxlineitem{Attributes}\begin{description}
\sphinxlineitem{\sphinxstylestrong{inputs: Input}}
\sphinxAtStartPar
The inputs used in this model run.

\sphinxlineitem{\sphinxstylestrong{n\_packets: int, float}}
\sphinxAtStartPar
Total number of packets to run

\sphinxlineitem{\sphinxstylestrong{compress: Bool}}
\sphinxAtStartPar
If True removes packets with frac=0 from the saved output, Default = True

\sphinxlineitem{\sphinxstylestrong{starting\_point: ndarray}}
\sphinxAtStartPar
Initial state relative to startpoint with standard units. Columns are
time (s), x (km), y (km), z (km), r (km), vx (km/s), vy (km/s),
(km/s), v (km/s), frac, longitude (rad), latitude (rad),
local\_time (hr), altitude (rad), azimuth (rad)

\sphinxlineitem{\sphinxstylestrong{final\_state: ndarray}}
\end{description}

\end{description}\end{quote}
\subsubsection*{Notes}

\sphinxAtStartPar
The user will not generally call this directly but will instead use
\sphinxcode{\sphinxupquote{inputs.run()}}.

\sphinxAtStartPar
All calculations will be done in units of s, km, km/s, rad and converted
to more appropriate units at the end.
\index{inputs (nexoclom2.Output attribute)@\spxentry{inputs}\spxextra{nexoclom2.Output attribute}}

\begin{fulllineitems}
\phantomsection\label{\detokenize{autoapi/nexoclom2/index:nexoclom2.Output.inputs}}
\pysigstartsignatures
\pysigline
{\sphinxbfcode{\sphinxupquote{inputs}}}
\pysigstopsignatures
\end{fulllineitems}

\index{compress (nexoclom2.Output attribute)@\spxentry{compress}\spxextra{nexoclom2.Output attribute}}

\begin{fulllineitems}
\phantomsection\label{\detokenize{autoapi/nexoclom2/index:nexoclom2.Output.compress}}
\pysigstartsignatures
\pysigline
{\sphinxbfcode{\sphinxupquote{compress}}\sphinxbfcode{\sphinxupquote{\DUrole{w}{ }\DUrole{p}{=}\DUrole{w}{ }True}}}
\pysigstopsignatures
\end{fulllineitems}

\index{randgen (nexoclom2.Output attribute)@\spxentry{randgen}\spxextra{nexoclom2.Output attribute}}

\begin{fulllineitems}
\phantomsection\label{\detokenize{autoapi/nexoclom2/index:nexoclom2.Output.randgen}}
\pysigstartsignatures
\pysigline
{\sphinxbfcode{\sphinxupquote{randgen}}}
\pysigstopsignatures
\end{fulllineitems}

\index{center (nexoclom2.Output attribute)@\spxentry{center}\spxextra{nexoclom2.Output attribute}}

\begin{fulllineitems}
\phantomsection\label{\detokenize{autoapi/nexoclom2/index:nexoclom2.Output.center}}
\pysigstartsignatures
\pysigline
{\sphinxbfcode{\sphinxupquote{center}}}
\pysigstopsignatures
\end{fulllineitems}

\index{startpoint (nexoclom2.Output attribute)@\spxentry{startpoint}\spxextra{nexoclom2.Output attribute}}

\begin{fulllineitems}
\phantomsection\label{\detokenize{autoapi/nexoclom2/index:nexoclom2.Output.startpoint}}
\pysigstartsignatures
\pysigline
{\sphinxbfcode{\sphinxupquote{startpoint}}}
\pysigstopsignatures
\end{fulllineitems}

\index{objects (nexoclom2.Output attribute)@\spxentry{objects}\spxextra{nexoclom2.Output attribute}}

\begin{fulllineitems}
\phantomsection\label{\detokenize{autoapi/nexoclom2/index:nexoclom2.Output.objects}}
\pysigstartsignatures
\pysigline
{\sphinxbfcode{\sphinxupquote{objects}}}
\pysigstopsignatures
\end{fulllineitems}

\index{positions (nexoclom2.Output attribute)@\spxentry{positions}\spxextra{nexoclom2.Output attribute}}

\begin{fulllineitems}
\phantomsection\label{\detokenize{autoapi/nexoclom2/index:nexoclom2.Output.positions}}
\pysigstartsignatures
\pysigline
{\sphinxbfcode{\sphinxupquote{positions}}}
\pysigstopsignatures
\end{fulllineitems}

\index{unit (nexoclom2.Output attribute)@\spxentry{unit}\spxextra{nexoclom2.Output attribute}}

\begin{fulllineitems}
\phantomsection\label{\detokenize{autoapi/nexoclom2/index:nexoclom2.Output.unit}}
\pysigstartsignatures
\pysigline
{\sphinxbfcode{\sphinxupquote{unit}}\sphinxbfcode{\sphinxupquote{\DUrole{w}{ }\DUrole{p}{=}\DUrole{w}{ }None}}}
\pysigstopsignatures
\end{fulllineitems}

\index{species (nexoclom2.Output attribute)@\spxentry{species}\spxextra{nexoclom2.Output attribute}}

\begin{fulllineitems}
\phantomsection\label{\detokenize{autoapi/nexoclom2/index:nexoclom2.Output.species}}
\pysigstartsignatures
\pysigline
{\sphinxbfcode{\sphinxupquote{species}}}
\pysigstopsignatures
\end{fulllineitems}

\index{initialize\_objects() (nexoclom2.Output method)@\spxentry{initialize\_objects()}\spxextra{nexoclom2.Output method}}

\begin{fulllineitems}
\phantomsection\label{\detokenize{autoapi/nexoclom2/index:nexoclom2.Output.initialize_objects}}
\pysigstartsignatures
\pysiglinewithargsret
{\sphinxbfcode{\sphinxupquote{initialize\_objects}}}
{}
{}
\pysigstopsignatures
\end{fulllineitems}

\index{\_remove() (nexoclom2.Output method)@\spxentry{\_remove()}\spxextra{nexoclom2.Output method}}

\begin{fulllineitems}
\phantomsection\label{\detokenize{autoapi/nexoclom2/index:nexoclom2.Output._remove}}
\pysigstartsignatures
\pysiglinewithargsret
{\sphinxbfcode{\sphinxupquote{\_remove}}}
{}
{}
\pysigstopsignatures
\end{fulllineitems}

\index{\_start\_outputfile() (nexoclom2.Output method)@\spxentry{\_start\_outputfile()}\spxextra{nexoclom2.Output method}}

\begin{fulllineitems}
\phantomsection\label{\detokenize{autoapi/nexoclom2/index:nexoclom2.Output._start_outputfile}}
\pysigstartsignatures
\pysiglinewithargsret
{\sphinxbfcode{\sphinxupquote{\_start\_outputfile}}}
{}
{}
\pysigstopsignatures
\end{fulllineitems}

\index{\_save\_starting\_point() (nexoclom2.Output method)@\spxentry{\_save\_starting\_point()}\spxextra{nexoclom2.Output method}}

\begin{fulllineitems}
\phantomsection\label{\detokenize{autoapi/nexoclom2/index:nexoclom2.Output._save_starting_point}}
\pysigstartsignatures
\pysiglinewithargsret
{\sphinxbfcode{\sphinxupquote{\_save\_starting\_point}}}
{\sphinxparam{\DUrole{n}{starting\_point}}}
{}
\pysigstopsignatures
\end{fulllineitems}

\index{save\_final\_state() (nexoclom2.Output method)@\spxentry{save\_final\_state()}\spxextra{nexoclom2.Output method}}

\begin{fulllineitems}
\phantomsection\label{\detokenize{autoapi/nexoclom2/index:nexoclom2.Output.save_final_state}}
\pysigstartsignatures
\pysiglinewithargsret
{\sphinxbfcode{\sphinxupquote{save\_final\_state}}}
{\sphinxparam{\DUrole{n}{final\_state}}}
{}
\pysigstopsignatures
\end{fulllineitems}

\index{starting\_point() (nexoclom2.Output method)@\spxentry{starting\_point()}\spxextra{nexoclom2.Output method}}

\begin{fulllineitems}
\phantomsection\label{\detokenize{autoapi/nexoclom2/index:nexoclom2.Output.starting_point}}
\pysigstartsignatures
\pysiglinewithargsret
{\sphinxbfcode{\sphinxupquote{starting\_point}}}
{\sphinxparam{\DUrole{n}{iteration}\DUrole{o}{=}\DUrole{default_value}{None}}\sphinxparamcomma \sphinxparam{\DUrole{n}{n\_packets}\DUrole{o}{=}\DUrole{default_value}{None}}}
{}
\pysigstopsignatures\begin{quote}\begin{description}
\sphinxlineitem{Parameters}\begin{description}
\sphinxlineitem{\sphinxstylestrong{iteration}}
\sphinxlineitem{\sphinxstylestrong{n\_packets}}
\end{description}

\sphinxlineitem{Returns}\begin{description}
\sphinxlineitem{StartingPoint}
\end{description}

\end{description}\end{quote}
\subsubsection*{Notes}

\sphinxAtStartPar
If iteration and n\_packets are both given, iteration number takes
precedence.

\end{fulllineitems}

\index{initial\_state() (nexoclom2.Output method)@\spxentry{initial\_state()}\spxextra{nexoclom2.Output method}}

\begin{fulllineitems}
\phantomsection\label{\detokenize{autoapi/nexoclom2/index:nexoclom2.Output.initial_state}}
\pysigstartsignatures
\pysiglinewithargsret
{\sphinxbfcode{\sphinxupquote{initial\_state}}}
{\sphinxparam{\DUrole{n}{iteration}\DUrole{o}{=}\DUrole{default_value}{None}}\sphinxparamcomma \sphinxparam{\DUrole{n}{n\_packets}\DUrole{o}{=}\DUrole{default_value}{None}}}
{}
\pysigstopsignatures
\end{fulllineitems}

\index{to\_planet\_coords() (nexoclom2.Output method)@\spxentry{to\_planet\_coords()}\spxextra{nexoclom2.Output method}}

\begin{fulllineitems}
\phantomsection\label{\detokenize{autoapi/nexoclom2/index:nexoclom2.Output.to_planet_coords}}
\pysigstartsignatures
\pysiglinewithargsret
{\sphinxbfcode{\sphinxupquote{to\_planet\_coords}}}
{\sphinxparam{\DUrole{n}{time}}\sphinxparamcomma \sphinxparam{\DUrole{n}{X}}\sphinxparamcomma \sphinxparam{\DUrole{n}{V}}}
{}
\pysigstopsignatures
\end{fulllineitems}

\index{final\_state() (nexoclom2.Output method)@\spxentry{final\_state()}\spxextra{nexoclom2.Output method}}

\begin{fulllineitems}
\phantomsection\label{\detokenize{autoapi/nexoclom2/index:nexoclom2.Output.final_state}}
\pysigstartsignatures
\pysiglinewithargsret
{\sphinxbfcode{\sphinxupquote{final\_state}}}
{\sphinxparam{\DUrole{n}{which}\DUrole{o}{=}\DUrole{default_value}{None}}\sphinxparamcomma \sphinxparam{\DUrole{n}{to\_planet\_coords}\DUrole{o}{=}\DUrole{default_value}{False}}}
{}
\pysigstopsignatures
\end{fulllineitems}


\end{fulllineitems}

\index{ModelImage (class in nexoclom2)@\spxentry{ModelImage}\spxextra{class in nexoclom2}}

\begin{fulllineitems}
\phantomsection\label{\detokenize{autoapi/nexoclom2/index:nexoclom2.ModelImage}}
\pysigstartsignatures
\pysiglinewithargsret
{\sphinxbfcode{\sphinxupquote{class\DUrole{w}{ }}}\sphinxcode{\sphinxupquote{nexoclom2.}}\sphinxbfcode{\sphinxupquote{ModelImage}}}
{\sphinxparam{\DUrole{n}{output}}\sphinxparamcomma \sphinxparam{\DUrole{n}{params}}\sphinxparamcomma \sphinxparam{\DUrole{n}{overwrite}\DUrole{o}{=}\DUrole{default_value}{False}}\sphinxparamcomma \sphinxparam{\DUrole{n}{chunksize}\DUrole{o}{=}\DUrole{default_value}{1000000}}}
{}
\pysigstopsignatures
\sphinxAtStartPar
Bases: {\hyperref[\detokenize{autoapi/nexoclom2/data_simulation/ModelResult/index:nexoclom2.data_simulation.ModelResult.ModelResult}]{\sphinxcrossref{\sphinxcode{\sphinxupquote{nexoclom2.data\_simulation.ModelResult.ModelResult}}}}}
\index{dimensions (nexoclom2.ModelImage attribute)@\spxentry{dimensions}\spxextra{nexoclom2.ModelImage attribute}}

\begin{fulllineitems}
\phantomsection\label{\detokenize{autoapi/nexoclom2/index:nexoclom2.ModelImage.dimensions}}
\pysigstartsignatures
\pysigline
{\sphinxbfcode{\sphinxupquote{dimensions}}}
\pysigstopsignatures
\end{fulllineitems}

\index{yrange (nexoclom2.ModelImage attribute)@\spxentry{yrange}\spxextra{nexoclom2.ModelImage attribute}}

\begin{fulllineitems}
\phantomsection\label{\detokenize{autoapi/nexoclom2/index:nexoclom2.ModelImage.yrange}}
\pysigstartsignatures
\pysigline
{\sphinxbfcode{\sphinxupquote{yrange}}}
\pysigstopsignatures
\end{fulllineitems}

\index{zrange (nexoclom2.ModelImage attribute)@\spxentry{zrange}\spxextra{nexoclom2.ModelImage attribute}}

\begin{fulllineitems}
\phantomsection\label{\detokenize{autoapi/nexoclom2/index:nexoclom2.ModelImage.zrange}}
\pysigstartsignatures
\pysigline
{\sphinxbfcode{\sphinxupquote{zrange}}}
\pysigstopsignatures
\end{fulllineitems}

\index{subobs\_longitude (nexoclom2.ModelImage attribute)@\spxentry{subobs\_longitude}\spxextra{nexoclom2.ModelImage attribute}}

\begin{fulllineitems}
\phantomsection\label{\detokenize{autoapi/nexoclom2/index:nexoclom2.ModelImage.subobs_longitude}}
\pysigstartsignatures
\pysigline
{\sphinxbfcode{\sphinxupquote{subobs\_longitude}}}
\pysigstopsignatures
\end{fulllineitems}

\index{subobs\_latitude (nexoclom2.ModelImage attribute)@\spxentry{subobs\_latitude}\spxextra{nexoclom2.ModelImage attribute}}

\begin{fulllineitems}
\phantomsection\label{\detokenize{autoapi/nexoclom2/index:nexoclom2.ModelImage.subobs_latitude}}
\pysigstartsignatures
\pysigline
{\sphinxbfcode{\sphinxupquote{subobs\_latitude}}}
\pysigstopsignatures
\end{fulllineitems}

\index{packet\_image (nexoclom2.ModelImage attribute)@\spxentry{packet\_image}\spxextra{nexoclom2.ModelImage attribute}}

\begin{fulllineitems}
\phantomsection\label{\detokenize{autoapi/nexoclom2/index:nexoclom2.ModelImage.packet_image}}
\pysigstartsignatures
\pysigline
{\sphinxbfcode{\sphinxupquote{packet\_image}}}
\pysigstopsignatures
\end{fulllineitems}

\index{Apix (nexoclom2.ModelImage attribute)@\spxentry{Apix}\spxextra{nexoclom2.ModelImage attribute}}

\begin{fulllineitems}
\phantomsection\label{\detokenize{autoapi/nexoclom2/index:nexoclom2.ModelImage.Apix}}
\pysigstartsignatures
\pysigline
{\sphinxbfcode{\sphinxupquote{Apix}}}
\pysigstopsignatures
\end{fulllineitems}

\index{rotation (nexoclom2.ModelImage attribute)@\spxentry{rotation}\spxextra{nexoclom2.ModelImage attribute}}

\begin{fulllineitems}
\phantomsection\label{\detokenize{autoapi/nexoclom2/index:nexoclom2.ModelImage.rotation}}
\pysigstartsignatures
\pysigline
{\sphinxbfcode{\sphinxupquote{rotation}}}
\pysigstopsignatures
\end{fulllineitems}

\index{centers (nexoclom2.ModelImage attribute)@\spxentry{centers}\spxextra{nexoclom2.ModelImage attribute}}

\begin{fulllineitems}
\phantomsection\label{\detokenize{autoapi/nexoclom2/index:nexoclom2.ModelImage.centers}}
\pysigstartsignatures
\pysigline
{\sphinxbfcode{\sphinxupquote{centers}}}
\pysigstopsignatures
\end{fulllineitems}


\end{fulllineitems}

\index{SSObject (class in nexoclom2)@\spxentry{SSObject}\spxextra{class in nexoclom2}}

\begin{fulllineitems}
\phantomsection\label{\detokenize{autoapi/nexoclom2/index:nexoclom2.SSObject}}
\pysigstartsignatures
\pysiglinewithargsret
{\sphinxbfcode{\sphinxupquote{class\DUrole{w}{ }}}\sphinxcode{\sphinxupquote{nexoclom2.}}\sphinxbfcode{\sphinxupquote{SSObject}}}
{\sphinxparam{\DUrole{n}{obj}\DUrole{p}{:}\DUrole{w}{ }\DUrole{n}{str}}}
{}
\pysigstopsignatures\begin{description}
\sphinxlineitem{Physical data for solar system bodies.}
\sphinxAtStartPar
Object containing all the necessary physical data for solar system objects.
Data is stored in a table included with the package. A separate table
contains the NAIF IDs. If the object is not found in the data table, returns
an object with just the object name, type = Unknown, and if possible the
NAIF ID.

\end{description}
\begin{quote}\begin{description}
\sphinxlineitem{Parameters}\begin{description}
\sphinxlineitem{\sphinxstylestrong{obj}}{[}str{]}
\sphinxAtStartPar
Name of the solar system object to gather data for.

\end{description}

\sphinxlineitem{Attributes}\begin{description}
\sphinxlineitem{\sphinxstylestrong{object: str}}\begin{quote}

\sphinxAtStartPar
Name of solar system body. Source: input parameter
\end{quote}
\begin{description}
\sphinxlineitem{orbits: str}
\sphinxAtStartPar
Object the body orbits. Source: PlanetaryConstants.csv

\sphinxlineitem{radius}{[}distance quantity{]}
\sphinxAtStartPar
Object radius. Source: SPICE

\sphinxlineitem{unit: astropy unit}
\sphinxAtStartPar
Named: R\_\textless{}object\textgreater{}

\sphinxlineitem{GM: Quantity}
\sphinxAtStartPar
Mass times gravitational constant. Source: SPICE

\sphinxlineitem{mass: mass quantity}
\sphinxAtStartPar
Object mass in kg. Source: GM from SPICE

\sphinxlineitem{a: distance quantity}
\sphinxAtStartPar
Object semi\sphinxhyphen{}major axis. Source: SPICE

\sphinxlineitem{e: float}
\sphinxAtStartPar
Orbital eccentricity. For planets: Source SPICE. For moons: Set to 0.
This only affects calculations when a modeltime is not specified and
is a small affect.

\sphinxlineitem{tilt: angle quantity}
\sphinxAtStartPar
Tilt of rotation axis relative to ecliptic in degrees.
Source: PlanetaryConstants.csv

\sphinxlineitem{rotperiod: time quantity}
\sphinxAtStartPar
Siderial rotational period in hours. Source: PlanetaryConstants.csv

\sphinxlineitem{orbperiod: time quantity}
\sphinxAtStartPar
Sideral orbital period. Source: SPICE

\sphinxlineitem{orbvel: velocity quantity}
\sphinxAtStartPar
:math:{\color{red}\bfseries{}\textasciigrave{}}v\_\{orb\} =

\end{description}

\sphinxlineitem{\sphinxstylestrong{rac\{2 pi a\}\{orbperiod\}\textasciigrave{}}}\begin{description}
\sphinxlineitem{satellites: list of str or None}
\sphinxAtStartPar
List of satellites of the body. Source: PlanetaryConstants.csv

\sphinxlineitem{type}{[}\{‘Star’, ‘Planet’, or ‘Moon’\}{]}
\sphinxAtStartPar
Source: PlanetaryConstants.csv

\sphinxlineitem{naifid}{[}int{]}
\sphinxAtStartPar
Source: naifids.csv

\end{description}

\end{description}

\end{description}\end{quote}
\subsubsection*{Examples}

\begin{sphinxVerbatim}[commandchars=\\\{\}]
\PYG{g+gp}{\PYGZgt{}\PYGZgt{}\PYGZgt{} }\PYG{k+kn}{from}\PYG{+w}{ }\PYG{n+nn}{nexoclom2}\PYG{n+nn}{.}\PYG{n+nn}{solarsystem}\PYG{+w}{ }\PYG{k+kn}{import} \PYG{n}{SSObject}
\PYG{g+gp}{\PYGZgt{}\PYGZgt{}\PYGZgt{} }\PYG{n}{jupiter} \PYG{o}{=} \PYG{n}{SSObject}\PYG{p}{(}\PYG{l+s+s1}{\PYGZsq{}}\PYG{l+s+s1}{Jupiter}\PYG{l+s+s1}{\PYGZsq{}}\PYG{p}{)}
\PYG{g+gp}{\PYGZgt{}\PYGZgt{}\PYGZgt{} }\PYG{n+nb}{print}\PYG{p}{(}\PYG{n}{jupiter}\PYG{p}{)}
\PYG{g+go}{Object: Jupiter}
\PYG{g+go}{Type = Planet}
\PYG{g+go}{Orbits Sun}
\PYG{g+go}{Satellites: Io, Europa, Ganymede, Callisto}
\PYG{g+go}{Radius = 71492.00 km}
\PYG{g+go}{Mass = 1.90e+27 kg}
\PYG{g+go}{a = 5.20 AU}
\PYG{g+go}{Eccentricity = 0.05}
\PYG{g+go}{Tilt = 3.08 deg}
\PYG{g+go}{Rotation Period = 9.93 h}
\PYG{g+go}{Orbital Period = 4333.00 d}
\PYG{g+go}{GM = \PYGZhy{}1.27e+17 m3 / s2}
\PYG{g+go}{NAIFID = 599}
\PYG{g+gp}{\PYGZgt{}\PYGZgt{}\PYGZgt{} }\PYG{n+nb}{print}\PYG{p}{(}\PYG{n+nb}{len}\PYG{p}{(}\PYG{n}{jupiter}\PYG{p}{)}\PYG{p}{)}
\PYG{g+go}{5}
\PYG{g+gp}{\PYGZgt{}\PYGZgt{}\PYGZgt{} }\PYG{n}{hst} \PYG{o}{=} \PYG{n}{SSObject}\PYG{p}{(}\PYG{l+s+s1}{\PYGZsq{}}\PYG{l+s+s1}{HST}\PYG{l+s+s1}{\PYGZsq{}}\PYG{p}{)}
\PYG{g+gp}{\PYGZgt{}\PYGZgt{}\PYGZgt{} }\PYG{n+nb}{print}\PYG{p}{(}\PYG{n}{hst}\PYG{p}{)}
\PYG{g+go}{Object: Hst}
\PYG{g+go}{Type = Unknown}
\PYG{g+go}{NAIFID = \PYGZhy{}48}
\PYG{g+gp}{\PYGZgt{}\PYGZgt{}\PYGZgt{} }\PYG{n+nb}{print}\PYG{p}{(}\PYG{n}{jupiter} \PYG{o}{==} \PYG{n}{hst}\PYG{p}{)}
\PYG{g+go}{False}
\end{sphinxVerbatim}
\begin{quote}\begin{description}
\sphinxlineitem{Authors}
\sphinxAtStartPar
Matthew Burger

\end{description}\end{quote}
\index{object (nexoclom2.SSObject attribute)@\spxentry{object}\spxextra{nexoclom2.SSObject attribute}}

\begin{fulllineitems}
\phantomsection\label{\detokenize{autoapi/nexoclom2/index:nexoclom2.SSObject.object}}
\pysigstartsignatures
\pysigline
{\sphinxbfcode{\sphinxupquote{object}}}
\pysigstopsignatures
\end{fulllineitems}

\index{\_\_eq\_\_() (nexoclom2.SSObject method)@\spxentry{\_\_eq\_\_()}\spxextra{nexoclom2.SSObject method}}

\begin{fulllineitems}
\phantomsection\label{\detokenize{autoapi/nexoclom2/index:nexoclom2.SSObject.__eq__}}
\pysigstartsignatures
\pysiglinewithargsret
{\sphinxbfcode{\sphinxupquote{\_\_eq\_\_}}}
{\sphinxparam{\DUrole{n}{other}}}
{}
\pysigstopsignatures
\end{fulllineitems}

\index{\_\_len\_\_() (nexoclom2.SSObject method)@\spxentry{\_\_len\_\_()}\spxextra{nexoclom2.SSObject method}}

\begin{fulllineitems}
\phantomsection\label{\detokenize{autoapi/nexoclom2/index:nexoclom2.SSObject.__len__}}
\pysigstartsignatures
\pysiglinewithargsret
{\sphinxbfcode{\sphinxupquote{\_\_len\_\_}}}
{}
{}
\pysigstopsignatures
\sphinxAtStartPar
Returns number of satellites + 1

\end{fulllineitems}

\index{\_\_repr\_\_() (nexoclom2.SSObject method)@\spxentry{\_\_repr\_\_()}\spxextra{nexoclom2.SSObject method}}

\begin{fulllineitems}
\phantomsection\label{\detokenize{autoapi/nexoclom2/index:nexoclom2.SSObject.__repr__}}
\pysigstartsignatures
\pysiglinewithargsret
{\sphinxbfcode{\sphinxupquote{\_\_repr\_\_}}}
{}
{}
\pysigstopsignatures
\end{fulllineitems}

\index{\_\_str\_\_() (nexoclom2.SSObject method)@\spxentry{\_\_str\_\_()}\spxextra{nexoclom2.SSObject method}}

\begin{fulllineitems}
\phantomsection\label{\detokenize{autoapi/nexoclom2/index:nexoclom2.SSObject.__str__}}
\pysigstartsignatures
\pysiglinewithargsret
{\sphinxbfcode{\sphinxupquote{\_\_str\_\_}}}
{}
{}
\pysigstopsignatures
\end{fulllineitems}


\end{fulllineitems}

\index{NexoclomConfig (class in nexoclom2)@\spxentry{NexoclomConfig}\spxextra{class in nexoclom2}}

\begin{fulllineitems}
\phantomsection\label{\detokenize{autoapi/nexoclom2/index:nexoclom2.NexoclomConfig}}
\pysigstartsignatures
\pysigline
{\sphinxbfcode{\sphinxupquote{class\DUrole{w}{ }}}\sphinxcode{\sphinxupquote{nexoclom2.}}\sphinxbfcode{\sphinxupquote{NexoclomConfig}}}
\pysigstopsignatures
\sphinxAtStartPar
Configuration object based on the nexoclom2 configuration file.

\sphinxAtStartPar
The \sphinxtitleref{NEXCOCLOMCONFIG} environment variable must be set. This is automatically
set to \sphinxcode{\sphinxupquote{\$HOME/.nexoclom}} when the nexoclom2 Python environment is activated,
although the user is free to change it.

\sphinxAtStartPar
Each line in the nexoclom configuration file should be in the form
\sphinxcode{\sphinxupquote{key = value}} where the keys are highlighed below. Configuration
settings for nexoclom2 extensions (such as for working with speficic
instrument data) can also be placed here. All lines in the file with the
proper format are included in the returned object.

\sphinxAtStartPar
\sphinxcode{\sphinxupquote{savepath}}: Top\sphinxhyphen{}level path on disk where all model output is saved (Required)

\sphinxAtStartPar
\sphinxcode{\sphinxupquote{database}}: Name of the TinyDB database file (Optional). Defaults to
\sphinxcode{\sphinxupquote{thesolarsystemmb.db}}.

\sphinxAtStartPar
\sphinxcode{\sphinxupquote{user}}: username (Required if not set as an environment variable).
\begin{quote}\begin{description}
\sphinxlineitem{Parameters}\begin{description}
\sphinxlineitem{\sphinxstylestrong{None}}
\end{description}

\sphinxlineitem{Attributes}\begin{description}
\sphinxlineitem{\sphinxstylestrong{configfile: str}}
\sphinxAtStartPar
Name of configuration file used.

\sphinxlineitem{\sphinxstylestrong{savepath: str}}
\sphinxAtStartPar
Top\sphinxhyphen{}level path on disk where all model output is saved.

\sphinxlineitem{\sphinxstylestrong{database: str}}
\sphinxAtStartPar
Name of the TinyDB database file modelresults are cataloged in.

\sphinxlineitem{\sphinxstylestrong{user: str}}
\sphinxAtStartPar
User’s username on the system.

\sphinxlineitem{\sphinxstylestrong{other: str}}
\sphinxAtStartPar
Other configuration settings can be included.

\end{description}

\sphinxlineitem{Raises}\begin{description}
\sphinxlineitem{ConfigfileError}
\sphinxAtStartPar
If \sphinxcode{\sphinxupquote{NEXOCLOMCONFIG}} environment variable not set or a required
parameter in the configuration file is not given

\sphinxlineitem{FileNotFoundError}
\sphinxAtStartPar
If the configuration file is not found

\end{description}

\end{description}\end{quote}


\begin{sphinxseealso}{See also:}
\begin{description}
\sphinxlineitem{{\hyperref[\detokenize{autoapi/nexoclom2/utilities/exceptions/index:nexoclom2.utilities.exceptions.ConfigfileError}]{\sphinxcrossref{\sphinxcode{\sphinxupquote{nexoclom2.utilities.exceptions.ConfigfileError}}}}}}
\end{description}


\end{sphinxseealso}

\subsubsection*{Examples}

\sphinxAtStartPar
For a configuration file in \sphinxcode{\sphinxupquote{\$HOME/.nexoclom}} containing the following:

\begin{sphinxVerbatim}[commandchars=\\\{\}]
\PYG{n}{savepath} \PYG{o}{=} \PYG{o}{/}\PYG{n}{user}\PYG{o}{/}\PYG{n}{mburger}\PYG{o}{/}\PYG{n}{Data}\PYG{o}{/}\PYG{n}{ModelData}
\PYG{n}{database} \PYG{o}{=} \PYG{n}{thesolarsystemmb}\PYG{o}{.}\PYG{n}{db}
\PYG{n}{mesdatapath} \PYG{o}{=} \PYG{o}{/}\PYG{n}{Users}\PYG{o}{/}\PYG{n}{mburger}\PYG{o}{/}\PYG{n}{Work}\PYG{o}{/}\PYG{n}{Data}\PYG{o}{/}\PYG{n}{MESSENGER}\PYG{o}{/}\PYG{n}{UVVS}
\PYG{n}{mesdatabase} \PYG{o}{=} \PYG{n}{messengeruvvsdb}
\end{sphinxVerbatim}

\sphinxAtStartPar
In Python:

\begin{sphinxVerbatim}[commandchars=\\\{\}]
\PYG{g+gp}{\PYGZgt{}\PYGZgt{}\PYGZgt{} }\PYG{n}{config} \PYG{o}{=} \PYG{n}{NexoclomConfig}\PYG{p}{(}\PYG{p}{)}
\PYG{g+gp}{\PYGZgt{}\PYGZgt{}\PYGZgt{} }\PYG{n+nb}{print}\PYG{p}{(}\PYG{n}{config}\PYG{p}{)}
\PYG{g+go}{configfile = /Users/mburger/.nexoclom2\PYGZus{}dev}
\PYG{g+go}{savepath = /Volumes/nexoclom\PYGZus{}output/modeloutputs2\PYGZus{}dev}
\PYG{g+go}{database = thesolarsystemmb\PYGZus{}dev.db}
\PYG{g+go}{user = mburger}
\end{sphinxVerbatim}
\begin{quote}\begin{description}
\sphinxlineitem{Authors}
\sphinxAtStartPar
Matthew Burger

\end{description}\end{quote}
\index{configfile (nexoclom2.NexoclomConfig attribute)@\spxentry{configfile}\spxextra{nexoclom2.NexoclomConfig attribute}}

\begin{fulllineitems}
\phantomsection\label{\detokenize{autoapi/nexoclom2/index:nexoclom2.NexoclomConfig.configfile}}
\pysigstartsignatures
\pysigline
{\sphinxbfcode{\sphinxupquote{configfile}}}
\pysigstopsignatures
\end{fulllineitems}

\index{database (nexoclom2.NexoclomConfig attribute)@\spxentry{database}\spxextra{nexoclom2.NexoclomConfig attribute}}

\begin{fulllineitems}
\phantomsection\label{\detokenize{autoapi/nexoclom2/index:nexoclom2.NexoclomConfig.database}}
\pysigstartsignatures
\pysigline
{\sphinxbfcode{\sphinxupquote{database}}}
\pysigstopsignatures
\end{fulllineitems}

\index{user (nexoclom2.NexoclomConfig attribute)@\spxentry{user}\spxextra{nexoclom2.NexoclomConfig attribute}}

\begin{fulllineitems}
\phantomsection\label{\detokenize{autoapi/nexoclom2/index:nexoclom2.NexoclomConfig.user}}
\pysigstartsignatures
\pysigline
{\sphinxbfcode{\sphinxupquote{user}}}
\pysigstopsignatures
\end{fulllineitems}

\index{\_\_str\_\_() (nexoclom2.NexoclomConfig method)@\spxentry{\_\_str\_\_()}\spxextra{nexoclom2.NexoclomConfig method}}

\begin{fulllineitems}
\phantomsection\label{\detokenize{autoapi/nexoclom2/index:nexoclom2.NexoclomConfig.__str__}}
\pysigstartsignatures
\pysiglinewithargsret
{\sphinxbfcode{\sphinxupquote{\_\_str\_\_}}}
{}
{}
\pysigstopsignatures
\end{fulllineitems}

\index{\_\_repr\_\_() (nexoclom2.NexoclomConfig method)@\spxentry{\_\_repr\_\_()}\spxextra{nexoclom2.NexoclomConfig method}}

\begin{fulllineitems}
\phantomsection\label{\detokenize{autoapi/nexoclom2/index:nexoclom2.NexoclomConfig.__repr__}}
\pysigstartsignatures
\pysiglinewithargsret
{\sphinxbfcode{\sphinxupquote{\_\_repr\_\_}}}
{}
{}
\pysigstopsignatures
\end{fulllineitems}


\end{fulllineitems}

\index{config (in module nexoclom2)@\spxentry{config}\spxextra{in module nexoclom2}}

\begin{fulllineitems}
\phantomsection\label{\detokenize{autoapi/nexoclom2/index:nexoclom2.config}}
\pysigstartsignatures
\pysigline
{\sphinxcode{\sphinxupquote{nexoclom2.}}\sphinxbfcode{\sphinxupquote{config}}}
\pysigstopsignatures
\end{fulllineitems}

\index{\_\_name\_\_ (in module nexoclom2)@\spxentry{\_\_name\_\_}\spxextra{in module nexoclom2}}

\begin{fulllineitems}
\phantomsection\label{\detokenize{autoapi/nexoclom2/index:nexoclom2.__name__}}
\pysigstartsignatures
\pysigline
{\sphinxcode{\sphinxupquote{nexoclom2.}}\sphinxbfcode{\sphinxupquote{\_\_name\_\_}}\sphinxbfcode{\sphinxupquote{\DUrole{w}{ }\DUrole{p}{=}\DUrole{w}{ }\textquotesingle{}nexoclom2\textquotesingle{}}}}
\pysigstopsignatures
\end{fulllineitems}

\index{\_\_author\_\_ (in module nexoclom2)@\spxentry{\_\_author\_\_}\spxextra{in module nexoclom2}}

\begin{fulllineitems}
\phantomsection\label{\detokenize{autoapi/nexoclom2/index:nexoclom2.__author__}}
\pysigstartsignatures
\pysigline
{\sphinxcode{\sphinxupquote{nexoclom2.}}\sphinxbfcode{\sphinxupquote{\_\_author\_\_}}\sphinxbfcode{\sphinxupquote{\DUrole{w}{ }\DUrole{p}{=}\DUrole{w}{ }\textquotesingle{}Matthew Burger\textquotesingle{}}}}
\pysigstopsignatures
\end{fulllineitems}

\index{\_\_email\_\_ (in module nexoclom2)@\spxentry{\_\_email\_\_}\spxextra{in module nexoclom2}}

\begin{fulllineitems}
\phantomsection\label{\detokenize{autoapi/nexoclom2/index:nexoclom2.__email__}}
\pysigstartsignatures
\pysigline
{\sphinxcode{\sphinxupquote{nexoclom2.}}\sphinxbfcode{\sphinxupquote{\_\_email\_\_}}\sphinxbfcode{\sphinxupquote{\DUrole{w}{ }\DUrole{p}{=}\DUrole{w}{ }\textquotesingle{}mburger@stsci.edu\textquotesingle{}}}}
\pysigstopsignatures
\end{fulllineitems}

\index{\_\_version\_\_ (in module nexoclom2)@\spxentry{\_\_version\_\_}\spxextra{in module nexoclom2}}

\begin{fulllineitems}
\phantomsection\label{\detokenize{autoapi/nexoclom2/index:nexoclom2.__version__}}
\pysigstartsignatures
\pysigline
{\sphinxcode{\sphinxupquote{nexoclom2.}}\sphinxbfcode{\sphinxupquote{\_\_version\_\_}}}
\pysigstopsignatures
\end{fulllineitems}


\sphinxstepscope


\subsection{GeometryTime}
\label{\detokenize{autoapi/GeometryTime/index:module-GeometryTime}}\label{\detokenize{autoapi/GeometryTime/index:geometrytime}}\label{\detokenize{autoapi/GeometryTime/index::doc}}\index{module@\spxentry{module}!GeometryTime@\spxentry{GeometryTime}}\index{GeometryTime@\spxentry{GeometryTime}!module@\spxentry{module}}

\subsubsection{Classes}
\label{\detokenize{autoapi/GeometryTime/index:classes}}

\begin{savenotes}\sphinxattablestart
\sphinxthistablewithglobalstyle
\sphinxthistablewithnovlinesstyle
\centering
\begin{tabulary}{\linewidth}[t]{\X{1}{2}\X{1}{2}}
\sphinxtoprule
\sphinxtableatstartofbodyhook
\sphinxAtStartPar
{\hyperref[\detokenize{autoapi/GeometryTime/index:GeometryTime.GeometryTime}]{\sphinxcrossref{\sphinxcode{\sphinxupquote{GeometryTime}}}}}
&
\sphinxAtStartPar
Geometry with Time Class
\\
\sphinxbottomrule
\end{tabulary}
\sphinxtableafterendhook\par
\sphinxattableend\end{savenotes}


\subsubsection{Module Contents}
\label{\detokenize{autoapi/GeometryTime/index:module-contents}}\index{GeometryTime (class in GeometryTime)@\spxentry{GeometryTime}\spxextra{class in GeometryTime}}

\begin{fulllineitems}
\phantomsection\label{\detokenize{autoapi/GeometryTime/index:GeometryTime.GeometryTime}}
\pysigstartsignatures
\pysiglinewithargsret
{\sphinxbfcode{\sphinxupquote{class\DUrole{w}{ }}}\sphinxcode{\sphinxupquote{GeometryTime.}}\sphinxbfcode{\sphinxupquote{GeometryTime}}}
{\sphinxparam{\DUrole{n}{gparam}}}
{}
\pysigstopsignatures
\sphinxAtStartPar
Bases: \sphinxcode{\sphinxupquote{nexoclom2.initial\_state.Geometry}}

\sphinxAtStartPar
Geometry with Time Class

\sphinxAtStartPar
Set the system geometry using a timestamp.

\sphinxAtStartPar
Parameters set here
\begin{itemize}
\item {} 
\sphinxAtStartPar
modeltime : must be in a format recognized by astrpy.time.Time

\end{itemize}

\sphinxAtStartPar
Parameters set by Geometry base class
\begin{itemize}
\item {} 
\sphinxAtStartPar
center object

\item {} 
\sphinxAtStartPar
startpoint, Default = center

\item {} 
\sphinxAtStartPar
included, Default = (center, startpoint)

\end{itemize}

\sphinxAtStartPar
See {\hyperref[\detokenize{nexoclom2/inputfiles:geometry}]{\sphinxcrossref{\DUrole{std}{\DUrole{std-ref}{Geometry}}}}} for more information.
\begin{quote}\begin{description}
\sphinxlineitem{Parameters}\begin{description}
\sphinxlineitem{\sphinxstylestrong{gparam}}{[}dict{]}
\sphinxAtStartPar
keys, values for defining system geometry.

\end{description}

\sphinxlineitem{Attributes}\begin{description}
\sphinxlineitem{\sphinxstylestrong{\_\_name\_\_}}{[}‘GeometryTime’{]}
\sphinxlineitem{\sphinxstylestrong{center}}{[}str{]}
\sphinxAtStartPar
Central body for the model

\sphinxlineitem{\sphinxstylestrong{startpoint}}{[}str{]}
\sphinxAtStartPar
Object from which packets are ejected.

\sphinxlineitem{\sphinxstylestrong{included}}{[}list of str{]}
\sphinxAtStartPar
Objects included in calculations.

\sphinxlineitem{\sphinxstylestrong{modeltime}}{[}astropy.Time{]}
\sphinxAtStartPar
Model simulation time.

\end{description}

\end{description}\end{quote}
\index{\_\_name\_\_ (GeometryTime.GeometryTime attribute)@\spxentry{\_\_name\_\_}\spxextra{GeometryTime.GeometryTime attribute}}

\begin{fulllineitems}
\phantomsection\label{\detokenize{autoapi/GeometryTime/index:GeometryTime.GeometryTime.__name__}}
\pysigstartsignatures
\pysigline
{\sphinxbfcode{\sphinxupquote{\_\_name\_\_}}\sphinxbfcode{\sphinxupquote{\DUrole{w}{ }\DUrole{p}{=}\DUrole{w}{ }\textquotesingle{}GeometryTime\textquotesingle{}}}}
\pysigstopsignatures
\end{fulllineitems}

\index{\_\_str\_\_() (GeometryTime.GeometryTime method)@\spxentry{\_\_str\_\_()}\spxextra{GeometryTime.GeometryTime method}}

\begin{fulllineitems}
\phantomsection\label{\detokenize{autoapi/GeometryTime/index:GeometryTime.GeometryTime.__str__}}
\pysigstartsignatures
\pysiglinewithargsret
{\sphinxbfcode{\sphinxupquote{\_\_str\_\_}}}
{}
{}
\pysigstopsignatures
\sphinxAtStartPar
Override of superclass \_\_str\_\_

\end{fulllineitems}

\index{\_\_eq\_\_() (GeometryTime.GeometryTime method)@\spxentry{\_\_eq\_\_()}\spxextra{GeometryTime.GeometryTime method}}

\begin{fulllineitems}
\phantomsection\label{\detokenize{autoapi/GeometryTime/index:GeometryTime.GeometryTime.__eq__}}
\pysigstartsignatures
\pysiglinewithargsret
{\sphinxbfcode{\sphinxupquote{\_\_eq\_\_}}}
{\sphinxparam{\DUrole{n}{other}}}
{}
\pysigstopsignatures
\end{fulllineitems}


\end{fulllineitems}


\sphinxstepscope


\subsection{FlatSpeedDist}
\label{\detokenize{autoapi/FlatSpeedDist/index:module-FlatSpeedDist}}\label{\detokenize{autoapi/FlatSpeedDist/index:flatspeeddist}}\label{\detokenize{autoapi/FlatSpeedDist/index::doc}}\index{module@\spxentry{module}!FlatSpeedDist@\spxentry{FlatSpeedDist}}\index{FlatSpeedDist@\spxentry{FlatSpeedDist}!module@\spxentry{module}}

\subsubsection{Classes}
\label{\detokenize{autoapi/FlatSpeedDist/index:classes}}

\begin{savenotes}\sphinxattablestart
\sphinxthistablewithglobalstyle
\sphinxthistablewithnovlinesstyle
\centering
\begin{tabulary}{\linewidth}[t]{\X{1}{2}\X{1}{2}}
\sphinxtoprule
\sphinxtableatstartofbodyhook
\sphinxAtStartPar
{\hyperref[\detokenize{autoapi/FlatSpeedDist/index:FlatSpeedDist.FlatSpeedDist}]{\sphinxcrossref{\sphinxcode{\sphinxupquote{FlatSpeedDist}}}}}
&
\sphinxAtStartPar
Defines a distribution with constant speed probability between two values.
\\
\sphinxbottomrule
\end{tabulary}
\sphinxtableafterendhook\par
\sphinxattableend\end{savenotes}


\subsubsection{Module Contents}
\label{\detokenize{autoapi/FlatSpeedDist/index:module-contents}}\index{FlatSpeedDist (class in FlatSpeedDist)@\spxentry{FlatSpeedDist}\spxextra{class in FlatSpeedDist}}

\begin{fulllineitems}
\phantomsection\label{\detokenize{autoapi/FlatSpeedDist/index:FlatSpeedDist.FlatSpeedDist}}
\pysigstartsignatures
\pysiglinewithargsret
{\sphinxbfcode{\sphinxupquote{class\DUrole{w}{ }}}\sphinxcode{\sphinxupquote{FlatSpeedDist.}}\sphinxbfcode{\sphinxupquote{FlatSpeedDist}}}
{\sphinxparam{\DUrole{n}{sparam}\DUrole{p}{:}\DUrole{w}{ }\DUrole{n}{dict}}}
{}
\pysigstopsignatures
\sphinxAtStartPar
Bases: {\hyperref[\detokenize{autoapi/nexoclom2/initial_state/InputClass/index:nexoclom2.initial_state.InputClass.InputClass}]{\sphinxcrossref{\sphinxcode{\sphinxupquote{nexoclom2.initial\_state.InputClass.InputClass}}}}}

\sphinxAtStartPar
Defines a distribution with constant speed probability between two values.
\begin{quote}\begin{description}
\sphinxlineitem{Parameters}\begin{description}
\sphinxlineitem{\sphinxstylestrong{sparam}}{[}dict{]}
\sphinxAtStartPar
Key, value for defining the distribution

\end{description}

\sphinxlineitem{Attributes}\begin{description}
\sphinxlineitem{\sphinxstylestrong{vmin, vmin: astropy quantity}}
\sphinxAtStartPar
Minimum and maximum speed values. Defaults to 0 km/s and 10 km/s

\end{description}

\end{description}\end{quote}
\index{\_\_name\_\_ (FlatSpeedDist.FlatSpeedDist attribute)@\spxentry{\_\_name\_\_}\spxextra{FlatSpeedDist.FlatSpeedDist attribute}}

\begin{fulllineitems}
\phantomsection\label{\detokenize{autoapi/FlatSpeedDist/index:FlatSpeedDist.FlatSpeedDist.__name__}}
\pysigstartsignatures
\pysigline
{\sphinxbfcode{\sphinxupquote{\_\_name\_\_}}\sphinxbfcode{\sphinxupquote{\DUrole{w}{ }\DUrole{p}{=}\DUrole{w}{ }\textquotesingle{}FlatSpeedDist\textquotesingle{}}}}
\pysigstopsignatures
\end{fulllineitems}

\index{pdf() (FlatSpeedDist.FlatSpeedDist method)@\spxentry{pdf()}\spxextra{FlatSpeedDist.FlatSpeedDist method}}

\begin{fulllineitems}
\phantomsection\label{\detokenize{autoapi/FlatSpeedDist/index:FlatSpeedDist.FlatSpeedDist.pdf}}
\pysigstartsignatures
\pysiglinewithargsret
{\sphinxbfcode{\sphinxupquote{pdf}}}
{\sphinxparam{\DUrole{n}{v}}}
{}
\pysigstopsignatures
\sphinxAtStartPar
Probability Distribution Function

\end{fulllineitems}

\index{support() (FlatSpeedDist.FlatSpeedDist method)@\spxentry{support()}\spxextra{FlatSpeedDist.FlatSpeedDist method}}

\begin{fulllineitems}
\phantomsection\label{\detokenize{autoapi/FlatSpeedDist/index:FlatSpeedDist.FlatSpeedDist.support}}
\pysigstartsignatures
\pysiglinewithargsret
{\sphinxbfcode{\sphinxupquote{support}}}
{}
{}
\pysigstopsignatures
\end{fulllineitems}

\index{choose\_points() (FlatSpeedDist.FlatSpeedDist method)@\spxentry{choose\_points()}\spxextra{FlatSpeedDist.FlatSpeedDist method}}

\begin{fulllineitems}
\phantomsection\label{\detokenize{autoapi/FlatSpeedDist/index:FlatSpeedDist.FlatSpeedDist.choose_points}}
\pysigstartsignatures
\pysiglinewithargsret
{\sphinxbfcode{\sphinxupquote{choose\_points}}}
{\sphinxparam{\DUrole{n}{n\_packets}}\sphinxparamcomma \sphinxparam{\DUrole{n}{randgen}\DUrole{o}{=}\DUrole{default_value}{None}}}
{}
\pysigstopsignatures
\sphinxAtStartPar
Compute random deviates from arbitrary 1D distribution.
f\_x does not need to integrate to 1. The function normalizes the
distribution. Uses Transformation method (Numerical Recipes, 7.3.2)
\begin{quote}\begin{description}
\sphinxlineitem{Parameters}\begin{description}
\sphinxlineitem{\sphinxstylestrong{n\_packets}}{[}int{]}
\sphinxAtStartPar
The number of random deviates to compute

\sphinxlineitem{\sphinxstylestrong{randgen}}{[}numpy.random.\_generator.Generator{]}
\end{description}

\sphinxlineitem{Returns}\begin{description}
\sphinxlineitem{numpy array of length num chosen from the distribution f\_x.}
\end{description}

\end{description}\end{quote}

\end{fulllineitems}


\end{fulllineitems}


\sphinxstepscope


\subsection{RadialAngDist}
\label{\detokenize{autoapi/RadialAngDist/index:module-RadialAngDist}}\label{\detokenize{autoapi/RadialAngDist/index:radialangdist}}\label{\detokenize{autoapi/RadialAngDist/index::doc}}\index{module@\spxentry{module}!RadialAngDist@\spxentry{RadialAngDist}}\index{RadialAngDist@\spxentry{RadialAngDist}!module@\spxentry{module}}

\subsubsection{Classes}
\label{\detokenize{autoapi/RadialAngDist/index:classes}}

\begin{savenotes}\sphinxattablestart
\sphinxthistablewithglobalstyle
\sphinxthistablewithnovlinesstyle
\centering
\begin{tabulary}{\linewidth}[t]{\X{1}{2}\X{1}{2}}
\sphinxtoprule
\sphinxtableatstartofbodyhook
\sphinxAtStartPar
{\hyperref[\detokenize{autoapi/RadialAngDist/index:RadialAngDist.RadialAngDist}]{\sphinxcrossref{\sphinxcode{\sphinxupquote{RadialAngDist}}}}}
&
\sphinxAtStartPar
Angular distribution with all particles ejected radially from the surface
\\
\sphinxbottomrule
\end{tabulary}
\sphinxtableafterendhook\par
\sphinxattableend\end{savenotes}


\subsubsection{Module Contents}
\label{\detokenize{autoapi/RadialAngDist/index:module-contents}}\index{RadialAngDist (class in RadialAngDist)@\spxentry{RadialAngDist}\spxextra{class in RadialAngDist}}

\begin{fulllineitems}
\phantomsection\label{\detokenize{autoapi/RadialAngDist/index:RadialAngDist.RadialAngDist}}
\pysigstartsignatures
\pysiglinewithargsret
{\sphinxbfcode{\sphinxupquote{class\DUrole{w}{ }}}\sphinxcode{\sphinxupquote{RadialAngDist.}}\sphinxbfcode{\sphinxupquote{RadialAngDist}}}
{\sphinxparam{\DUrole{n}{sparam}\DUrole{o}{=}\DUrole{default_value}{None}}}
{}
\pysigstopsignatures
\sphinxAtStartPar
Bases: {\hyperref[\detokenize{autoapi/nexoclom2/initial_state/InputClass/index:nexoclom2.initial_state.InputClass.InputClass}]{\sphinxcrossref{\sphinxcode{\sphinxupquote{nexoclom2.initial\_state.InputClass.InputClass}}}}}

\sphinxAtStartPar
Angular distribution with all particles ejected radially from the surface
\begin{itemize}
\item {} 
\sphinxAtStartPar
altitude set to 90º

\item {} 
\sphinxAtStartPar
azimuth set to 0º

\end{itemize}
\begin{quote}\begin{description}
\sphinxlineitem{Parameters}\begin{description}
\sphinxlineitem{\sphinxstylestrong{sparam}}{[}None, TinyDB Document{]}
\end{description}

\sphinxlineitem{Attributes}\begin{description}
\sphinxlineitem{\sphinxstylestrong{None}}
\end{description}

\end{description}\end{quote}
\index{\_\_name\_\_ (RadialAngDist.RadialAngDist attribute)@\spxentry{\_\_name\_\_}\spxextra{RadialAngDist.RadialAngDist attribute}}

\begin{fulllineitems}
\phantomsection\label{\detokenize{autoapi/RadialAngDist/index:RadialAngDist.RadialAngDist.__name__}}
\pysigstartsignatures
\pysigline
{\sphinxbfcode{\sphinxupquote{\_\_name\_\_}}\sphinxbfcode{\sphinxupquote{\DUrole{w}{ }\DUrole{p}{=}\DUrole{w}{ }\textquotesingle{}RadialAngDist\textquotesingle{}}}}
\pysigstopsignatures
\end{fulllineitems}

\index{choose\_points() (RadialAngDist.RadialAngDist method)@\spxentry{choose\_points()}\spxextra{RadialAngDist.RadialAngDist method}}

\begin{fulllineitems}
\phantomsection\label{\detokenize{autoapi/RadialAngDist/index:RadialAngDist.RadialAngDist.choose_points}}
\pysigstartsignatures
\pysiglinewithargsret
{\sphinxbfcode{\sphinxupquote{choose\_points}}}
{\sphinxparam{\DUrole{n}{n\_packets}}\sphinxparamcomma \sphinxparam{\DUrole{n}{randgen}\DUrole{o}{=}\DUrole{default_value}{None}}}
{}
\pysigstopsignatures
\end{fulllineitems}


\end{fulllineitems}


\sphinxstepscope


\subsection{GeometryNoTime}
\label{\detokenize{autoapi/GeometryNoTime/index:module-GeometryNoTime}}\label{\detokenize{autoapi/GeometryNoTime/index:geometrynotime}}\label{\detokenize{autoapi/GeometryNoTime/index::doc}}\index{module@\spxentry{module}!GeometryNoTime@\spxentry{GeometryNoTime}}\index{GeometryNoTime@\spxentry{GeometryNoTime}!module@\spxentry{module}}

\subsubsection{Classes}
\label{\detokenize{autoapi/GeometryNoTime/index:classes}}

\begin{savenotes}\sphinxattablestart
\sphinxthistablewithglobalstyle
\sphinxthistablewithnovlinesstyle
\centering
\begin{tabulary}{\linewidth}[t]{\X{1}{2}\X{1}{2}}
\sphinxtoprule
\sphinxtableatstartofbodyhook
\sphinxAtStartPar
{\hyperref[\detokenize{autoapi/GeometryNoTime/index:GeometryNoTime.GeometryNoTime}]{\sphinxcrossref{\sphinxcode{\sphinxupquote{GeometryNoTime}}}}}
&
\sphinxAtStartPar
Geometry Without Time Class
\\
\sphinxbottomrule
\end{tabulary}
\sphinxtableafterendhook\par
\sphinxattableend\end{savenotes}


\subsubsection{Module Contents}
\label{\detokenize{autoapi/GeometryNoTime/index:module-contents}}\index{GeometryNoTime (class in GeometryNoTime)@\spxentry{GeometryNoTime}\spxextra{class in GeometryNoTime}}

\begin{fulllineitems}
\phantomsection\label{\detokenize{autoapi/GeometryNoTime/index:GeometryNoTime.GeometryNoTime}}
\pysigstartsignatures
\pysiglinewithargsret
{\sphinxbfcode{\sphinxupquote{class\DUrole{w}{ }}}\sphinxcode{\sphinxupquote{GeometryNoTime.}}\sphinxbfcode{\sphinxupquote{GeometryNoTime}}}
{\sphinxparam{\DUrole{n}{gparam}}}
{}
\pysigstopsignatures
\sphinxAtStartPar
Bases: \sphinxcode{\sphinxupquote{nexoclom2.initial\_state.Geometry}}

\sphinxAtStartPar
Geometry Without Time Class

\sphinxAtStartPar
Set the system geometry manually.

\sphinxAtStartPar
Parameters set here
\begin{itemize}
\item {} 
\sphinxAtStartPar
phi: orbital phase/subsolar longitude for each satellite in degrees

\item {} 
\sphinxAtStartPar
subsolarpoint

\item {} 
\sphinxAtStartPar
taa: true anomaly in degrees, Default = 0

\item {} 
\sphinxAtStartPar
dtaa: true anomaly tolerance in degrees, Default = 2º

\end{itemize}

\sphinxAtStartPar
Parameters set by Geometry base class
\begin{itemize}
\item {} 
\sphinxAtStartPar
center

\item {} 
\sphinxAtStartPar
startpoint: Default = center

\item {} 
\sphinxAtStartPar
included: Default = (center, startpoint)

\end{itemize}

\sphinxAtStartPar
See {\hyperref[\detokenize{nexoclom2/inputfiles:geometry}]{\sphinxcrossref{\DUrole{std}{\DUrole{std-ref}{Geometry}}}}} for more information.
\begin{quote}\begin{description}
\sphinxlineitem{Parameters}\begin{description}
\sphinxlineitem{\sphinxstylestrong{gparam}}{[}dict{]}
\sphinxAtStartPar
keys, values for defining system geometry.

\end{description}

\sphinxlineitem{Attributes}\begin{description}
\sphinxlineitem{\sphinxstylestrong{center}}{[}str{]}
\sphinxAtStartPar
Central body for the model

\sphinxlineitem{\sphinxstylestrong{startpoint}}{[}str{]}
\sphinxAtStartPar
Object from which packets are ejected.

\sphinxlineitem{\sphinxstylestrong{included}}{[}list of str{]}
\sphinxAtStartPar
Objects included in calculations.

\sphinxlineitem{\sphinxstylestrong{phi}}{[}dict{]}
\sphinxAtStartPar
Orbital phase angle for each included satellite.

\sphinxlineitem{\sphinxstylestrong{subsolarpoint}}{[}tuple of astropy quantities{]}
\sphinxAtStartPar
Longitude and latitude of the sub\sphinxhyphen{}solar point on the center object.

\sphinxlineitem{\sphinxstylestrong{taa}}{[}astropy quantity{]}
\sphinxAtStartPar
The center object’s true anomaly angle. If central object is a moon,
use planet’s TAA

\sphinxlineitem{\sphinxstylestrong{dtaa}}{[}astropy quantity{]}
\sphinxAtStartPar
Tolerance for true anomaly differences in model searches.

\end{description}

\end{description}\end{quote}
\index{\_\_name\_\_ (GeometryNoTime.GeometryNoTime attribute)@\spxentry{\_\_name\_\_}\spxextra{GeometryNoTime.GeometryNoTime attribute}}

\begin{fulllineitems}
\phantomsection\label{\detokenize{autoapi/GeometryNoTime/index:GeometryNoTime.GeometryNoTime.__name__}}
\pysigstartsignatures
\pysigline
{\sphinxbfcode{\sphinxupquote{\_\_name\_\_}}\sphinxbfcode{\sphinxupquote{\DUrole{w}{ }\DUrole{p}{=}\DUrole{w}{ }\textquotesingle{}GeometryNoTime\textquotesingle{}}}}
\pysigstopsignatures
\end{fulllineitems}

\index{\_\_str\_\_() (GeometryNoTime.GeometryNoTime method)@\spxentry{\_\_str\_\_()}\spxextra{GeometryNoTime.GeometryNoTime method}}

\begin{fulllineitems}
\phantomsection\label{\detokenize{autoapi/GeometryNoTime/index:GeometryNoTime.GeometryNoTime.__str__}}
\pysigstartsignatures
\pysiglinewithargsret
{\sphinxbfcode{\sphinxupquote{\_\_str\_\_}}}
{}
{}
\pysigstopsignatures
\sphinxAtStartPar
Override of superclass \_\_str\_\_

\end{fulllineitems}

\index{\_\_eq\_\_() (GeometryNoTime.GeometryNoTime method)@\spxentry{\_\_eq\_\_()}\spxextra{GeometryNoTime.GeometryNoTime method}}

\begin{fulllineitems}
\phantomsection\label{\detokenize{autoapi/GeometryNoTime/index:GeometryNoTime.GeometryNoTime.__eq__}}
\pysigstartsignatures
\pysiglinewithargsret
{\sphinxbfcode{\sphinxupquote{\_\_eq\_\_}}}
{\sphinxparam{\DUrole{n}{other}}}
{}
\pysigstopsignatures
\end{fulllineitems}


\end{fulllineitems}


\sphinxstepscope


\subsection{SurfMapSurfInt}
\label{\detokenize{autoapi/SurfMapSurfInt/index:module-SurfMapSurfInt}}\label{\detokenize{autoapi/SurfMapSurfInt/index:surfmapsurfint}}\label{\detokenize{autoapi/SurfMapSurfInt/index::doc}}\index{module@\spxentry{module}!SurfMapSurfInt@\spxentry{SurfMapSurfInt}}\index{SurfMapSurfInt@\spxentry{SurfMapSurfInt}!module@\spxentry{module}}
\sphinxstepscope


\subsection{UniformSpatDist}
\label{\detokenize{autoapi/UniformSpatDist/index:module-UniformSpatDist}}\label{\detokenize{autoapi/UniformSpatDist/index:uniformspatdist}}\label{\detokenize{autoapi/UniformSpatDist/index::doc}}\index{module@\spxentry{module}!UniformSpatDist@\spxentry{UniformSpatDist}}\index{UniformSpatDist@\spxentry{UniformSpatDist}!module@\spxentry{module}}

\subsubsection{Classes}
\label{\detokenize{autoapi/UniformSpatDist/index:classes}}

\begin{savenotes}\sphinxattablestart
\sphinxthistablewithglobalstyle
\sphinxthistablewithnovlinesstyle
\centering
\begin{tabulary}{\linewidth}[t]{\X{1}{2}\X{1}{2}}
\sphinxtoprule
\sphinxtableatstartofbodyhook
\sphinxAtStartPar
{\hyperref[\detokenize{autoapi/UniformSpatDist/index:UniformSpatDist.UniformSpatDist}]{\sphinxcrossref{\sphinxcode{\sphinxupquote{UniformSpatDist}}}}}
&
\sphinxAtStartPar
Defines a spatial distribution with uniform flux from the exobase.
\\
\sphinxbottomrule
\end{tabulary}
\sphinxtableafterendhook\par
\sphinxattableend\end{savenotes}


\subsubsection{Module Contents}
\label{\detokenize{autoapi/UniformSpatDist/index:module-contents}}\index{UniformSpatDist (class in UniformSpatDist)@\spxentry{UniformSpatDist}\spxextra{class in UniformSpatDist}}

\begin{fulllineitems}
\phantomsection\label{\detokenize{autoapi/UniformSpatDist/index:UniformSpatDist.UniformSpatDist}}
\pysigstartsignatures
\pysiglinewithargsret
{\sphinxbfcode{\sphinxupquote{class\DUrole{w}{ }}}\sphinxcode{\sphinxupquote{UniformSpatDist.}}\sphinxbfcode{\sphinxupquote{UniformSpatDist}}}
{\sphinxparam{\DUrole{n}{sparam}\DUrole{p}{:}\DUrole{w}{ }\DUrole{n}{dict\DUrole{p}{,}\DUrole{w}{ }tinydb.table.Document}}}
{}
\pysigstopsignatures
\sphinxAtStartPar
Bases: {\hyperref[\detokenize{autoapi/nexoclom2/initial_state/InputClass/index:nexoclom2.initial_state.InputClass.InputClass}]{\sphinxcrossref{\sphinxcode{\sphinxupquote{nexoclom2.initial\_state.InputClass.InputClass}}}}}

\sphinxAtStartPar
Defines a spatial distribution with uniform flux from the exobase.

\sphinxAtStartPar
Sets up an initial spatial distribution that ejects particles with constant
flux (in particles/cm\textasciicircum{}2/s) from the exobase. Limits can be placed on the
latitude/longitude range.

\sphinxAtStartPar
Parameters that can be set
\begin{itemize}
\item {} 
\sphinxAtStartPar
longitude

\item {} 
\sphinxAtStartPar
latitude

\item {} 
\sphinxAtStartPar
exobase

\item {} 
\sphinxAtStartPar
frane

\end{itemize}

\sphinxAtStartPar
See {\hyperref[\detokenize{nexoclom2/inputfiles:spatialdist}]{\sphinxcrossref{\DUrole{std}{\DUrole{std-ref}{SpatialDist}}}}} for more information.
\begin{quote}\begin{description}
\sphinxlineitem{Parameters}\begin{description}
\sphinxlineitem{\sphinxstylestrong{sparam}}{[}dict{]}
\sphinxAtStartPar
Key, value for defining the distribution

\end{description}

\sphinxlineitem{Attributes}\begin{description}
\sphinxlineitem{\sphinxstylestrong{longitude: tuple of astropy quantities}}
\sphinxAtStartPar
longitude range packets are ejected from. Default: (0º, 360º)

\sphinxlineitem{\sphinxstylestrong{laitutde: tuple of astropy quantities}}
\sphinxAtStartPar
latitude range packets are ejected from. Default: (\sphinxhyphen{}90º, 90º)

\sphinxlineitem{\sphinxstylestrong{exobase: float}}
\sphinxAtStartPar
Distance from starting object’s center from which to eject particles.
Measured relative to starting object’s radius. Default: 1.0

\sphinxlineitem{\sphinxstylestrong{frame: str}}
\sphinxAtStartPar
SPICE frame to use for latitude and longitude. Options are “IAU”,
“SOLAR”, and “SOLARFIXED”. See {\hyperref[\detokenize{nexoclom2/coordinate_systems:coordinate-systems}]{\sphinxcrossref{\DUrole{std}{\DUrole{std-ref}{Planetary Coordinate Systems and System Geometry}}}}} for more
information.

\end{description}

\end{description}\end{quote}
\index{\_\_name\_\_ (UniformSpatDist.UniformSpatDist attribute)@\spxentry{\_\_name\_\_}\spxextra{UniformSpatDist.UniformSpatDist attribute}}

\begin{fulllineitems}
\phantomsection\label{\detokenize{autoapi/UniformSpatDist/index:UniformSpatDist.UniformSpatDist.__name__}}
\pysigstartsignatures
\pysigline
{\sphinxbfcode{\sphinxupquote{\_\_name\_\_}}\sphinxbfcode{\sphinxupquote{\DUrole{w}{ }\DUrole{p}{=}\DUrole{w}{ }\textquotesingle{}UniformSpatDist\textquotesingle{}}}}
\pysigstopsignatures
\end{fulllineitems}

\index{pdf\_longitude() (UniformSpatDist.UniformSpatDist method)@\spxentry{pdf\_longitude()}\spxextra{UniformSpatDist.UniformSpatDist method}}

\begin{fulllineitems}
\phantomsection\label{\detokenize{autoapi/UniformSpatDist/index:UniformSpatDist.UniformSpatDist.pdf_longitude}}
\pysigstartsignatures
\pysiglinewithargsret
{\sphinxbfcode{\sphinxupquote{pdf\_longitude}}}
{\sphinxparam{\DUrole{n}{lon}}}
{}
\pysigstopsignatures
\end{fulllineitems}

\index{cdf\_longitude() (UniformSpatDist.UniformSpatDist method)@\spxentry{cdf\_longitude()}\spxextra{UniformSpatDist.UniformSpatDist method}}

\begin{fulllineitems}
\phantomsection\label{\detokenize{autoapi/UniformSpatDist/index:UniformSpatDist.UniformSpatDist.cdf_longitude}}
\pysigstartsignatures
\pysiglinewithargsret
{\sphinxbfcode{\sphinxupquote{cdf\_longitude}}}
{\sphinxparam{\DUrole{n}{lon}}}
{}
\pysigstopsignatures
\end{fulllineitems}

\index{support\_longitude() (UniformSpatDist.UniformSpatDist method)@\spxentry{support\_longitude()}\spxextra{UniformSpatDist.UniformSpatDist method}}

\begin{fulllineitems}
\phantomsection\label{\detokenize{autoapi/UniformSpatDist/index:UniformSpatDist.UniformSpatDist.support_longitude}}
\pysigstartsignatures
\pysiglinewithargsret
{\sphinxbfcode{\sphinxupquote{support\_longitude}}}
{}
{}
\pysigstopsignatures\begin{quote}\begin{description}
\sphinxlineitem{Returns}\begin{description}
\sphinxlineitem{tuple with valid range for the PDF}
\end{description}

\end{description}\end{quote}

\end{fulllineitems}

\index{pdf\_latitude() (UniformSpatDist.UniformSpatDist method)@\spxentry{pdf\_latitude()}\spxextra{UniformSpatDist.UniformSpatDist method}}

\begin{fulllineitems}
\phantomsection\label{\detokenize{autoapi/UniformSpatDist/index:UniformSpatDist.UniformSpatDist.pdf_latitude}}
\pysigstartsignatures
\pysiglinewithargsret
{\sphinxbfcode{\sphinxupquote{pdf\_latitude}}}
{\sphinxparam{\DUrole{n}{lat}}}
{}
\pysigstopsignatures
\end{fulllineitems}

\index{cdf\_latitude() (UniformSpatDist.UniformSpatDist method)@\spxentry{cdf\_latitude()}\spxextra{UniformSpatDist.UniformSpatDist method}}

\begin{fulllineitems}
\phantomsection\label{\detokenize{autoapi/UniformSpatDist/index:UniformSpatDist.UniformSpatDist.cdf_latitude}}
\pysigstartsignatures
\pysiglinewithargsret
{\sphinxbfcode{\sphinxupquote{cdf\_latitude}}}
{\sphinxparam{\DUrole{n}{lat}}}
{}
\pysigstopsignatures
\end{fulllineitems}

\index{support\_latitude() (UniformSpatDist.UniformSpatDist method)@\spxentry{support\_latitude()}\spxextra{UniformSpatDist.UniformSpatDist method}}

\begin{fulllineitems}
\phantomsection\label{\detokenize{autoapi/UniformSpatDist/index:UniformSpatDist.UniformSpatDist.support_latitude}}
\pysigstartsignatures
\pysiglinewithargsret
{\sphinxbfcode{\sphinxupquote{support\_latitude}}}
{}
{}
\pysigstopsignatures\begin{quote}\begin{description}
\sphinxlineitem{Returns}\begin{description}
\sphinxlineitem{tuple with valid range for the PDF}
\end{description}

\end{description}\end{quote}

\end{fulllineitems}

\index{choose\_points() (UniformSpatDist.UniformSpatDist method)@\spxentry{choose\_points()}\spxextra{UniformSpatDist.UniformSpatDist method}}

\begin{fulllineitems}
\phantomsection\label{\detokenize{autoapi/UniformSpatDist/index:UniformSpatDist.UniformSpatDist.choose_points}}
\pysigstartsignatures
\pysiglinewithargsret
{\sphinxbfcode{\sphinxupquote{choose\_points}}}
{\sphinxparam{\DUrole{n}{n\_packets}}\sphinxparamcomma \sphinxparam{\DUrole{n}{randgen}\DUrole{o}{=}\DUrole{default_value}{None}}}
{}
\pysigstopsignatures
\sphinxAtStartPar
Returns initial x, y, and z. For a moon, a rotation will be done later
to move packets into the proper orbital position
\begin{quote}\begin{description}
\sphinxlineitem{Parameters}\begin{description}
\sphinxlineitem{\sphinxstylestrong{npack}}{[}int{]}
\sphinxAtStartPar
Number of packets to generate

\sphinxlineitem{\sphinxstylestrong{randgen}}{[}numpy.random.\_generator.Generator{]}
\sphinxAtStartPar
Optional. Default=None

\end{description}

\sphinxlineitem{Returns}\begin{description}
\sphinxlineitem{dictionary with longitude and latitude.}
\end{description}

\end{description}\end{quote}

\end{fulllineitems}


\end{fulllineitems}


\sphinxstepscope


\subsection{SurfMapSpatDist}
\label{\detokenize{autoapi/SurfMapSpatDist/index:module-SurfMapSpatDist}}\label{\detokenize{autoapi/SurfMapSpatDist/index:surfmapspatdist}}\label{\detokenize{autoapi/SurfMapSpatDist/index::doc}}\index{module@\spxentry{module}!SurfMapSpatDist@\spxentry{SurfMapSpatDist}}\index{SurfMapSpatDist@\spxentry{SurfMapSpatDist}!module@\spxentry{module}}
\sphinxstepscope


\subsection{ConstantSurfInt}
\label{\detokenize{autoapi/ConstantSurfInt/index:module-ConstantSurfInt}}\label{\detokenize{autoapi/ConstantSurfInt/index:constantsurfint}}\label{\detokenize{autoapi/ConstantSurfInt/index::doc}}\index{module@\spxentry{module}!ConstantSurfInt@\spxentry{ConstantSurfInt}}\index{ConstantSurfInt@\spxentry{ConstantSurfInt}!module@\spxentry{module}}

\subsubsection{Classes}
\label{\detokenize{autoapi/ConstantSurfInt/index:classes}}

\begin{savenotes}\sphinxattablestart
\sphinxthistablewithglobalstyle
\sphinxthistablewithnovlinesstyle
\centering
\begin{tabulary}{\linewidth}[t]{\X{1}{2}\X{1}{2}}
\sphinxtoprule
\sphinxtableatstartofbodyhook
\sphinxAtStartPar
{\hyperref[\detokenize{autoapi/ConstantSurfInt/index:ConstantSurfInt.ConstantSurfInt}]{\sphinxcrossref{\sphinxcode{\sphinxupquote{ConstantSurfInt}}}}}
&
\sphinxAtStartPar
Surface interaction with constant sticking coefficient and accommodation factor
\\
\sphinxbottomrule
\end{tabulary}
\sphinxtableafterendhook\par
\sphinxattableend\end{savenotes}


\subsubsection{Module Contents}
\label{\detokenize{autoapi/ConstantSurfInt/index:module-contents}}\index{ConstantSurfInt (class in ConstantSurfInt)@\spxentry{ConstantSurfInt}\spxextra{class in ConstantSurfInt}}

\begin{fulllineitems}
\phantomsection\label{\detokenize{autoapi/ConstantSurfInt/index:ConstantSurfInt.ConstantSurfInt}}
\pysigstartsignatures
\pysiglinewithargsret
{\sphinxbfcode{\sphinxupquote{class\DUrole{w}{ }}}\sphinxcode{\sphinxupquote{ConstantSurfInt.}}\sphinxbfcode{\sphinxupquote{ConstantSurfInt}}}
{\sphinxparam{\DUrole{n}{sparam}\DUrole{p}{:}\DUrole{w}{ }\DUrole{n}{dict\DUrole{p}{,}\DUrole{w}{ }tinydb.table.Document}}}
{}
\pysigstopsignatures
\sphinxAtStartPar
Bases: {\hyperref[\detokenize{autoapi/nexoclom2/initial_state/InputClass/index:nexoclom2.initial_state.InputClass.InputClass}]{\sphinxcrossref{\sphinxcode{\sphinxupquote{nexoclom2.initial\_state.InputClass.InputClass}}}}}

\sphinxAtStartPar
Surface interaction with constant sticking coefficient and accommodation factor

\sphinxAtStartPar
Parameters that can to be set
\begin{itemize}
\item {} 
\sphinxAtStartPar
stickcoef

\item {} 
\sphinxAtStartPar
accomfactor

\end{itemize}

\sphinxAtStartPar
See {\hyperref[\detokenize{nexoclom2/inputfiles:surfaceinteractions}]{\sphinxcrossref{\DUrole{std}{\DUrole{std-ref}{SurfaceInteraction}}}}} for more information.
\begin{quote}\begin{description}
\sphinxlineitem{Parameters}
\sphinxAtStartPar
\sphinxstyleliteralstrong{\sphinxupquote{sparams}} (\sphinxstyleliteralemphasis{\sphinxupquote{dict}}) \textendash{} keys, values for defining surface interactions.

\sphinxlineitem{Attributes}\begin{description}
\sphinxlineitem{\sphinxstylestrong{stickcoef}}{[}float{]}
\sphinxAtStartPar
Sticking coefficient between 0 (no sticking) and 1 (complete sticking).
Default = 1.

\sphinxlineitem{\sphinxstylestrong{accomfactor}}{[}float{]}
\sphinxAtStartPar
Accommodation factor between 0 (elastic scattering) and 1 (complete
accommodation to surface temperature). Required if stickcoef \textless{} 1.

\end{description}

\end{description}\end{quote}
\index{\_\_name\_\_ (ConstantSurfInt.ConstantSurfInt attribute)@\spxentry{\_\_name\_\_}\spxextra{ConstantSurfInt.ConstantSurfInt attribute}}

\begin{fulllineitems}
\phantomsection\label{\detokenize{autoapi/ConstantSurfInt/index:ConstantSurfInt.ConstantSurfInt.__name__}}
\pysigstartsignatures
\pysigline
{\sphinxbfcode{\sphinxupquote{\_\_name\_\_}}\sphinxbfcode{\sphinxupquote{\DUrole{w}{ }\DUrole{p}{=}\DUrole{w}{ }\textquotesingle{}ConstantSurfInt\textquotesingle{}}}}
\pysigstopsignatures
\end{fulllineitems}


\end{fulllineitems}


\sphinxstepscope


\subsection{IsotropicAngDist}
\label{\detokenize{autoapi/IsotropicAngDist/index:module-IsotropicAngDist}}\label{\detokenize{autoapi/IsotropicAngDist/index:isotropicangdist}}\label{\detokenize{autoapi/IsotropicAngDist/index::doc}}\index{module@\spxentry{module}!IsotropicAngDist@\spxentry{IsotropicAngDist}}\index{IsotropicAngDist@\spxentry{IsotropicAngDist}!module@\spxentry{module}}

\subsubsection{Classes}
\label{\detokenize{autoapi/IsotropicAngDist/index:classes}}

\begin{savenotes}\sphinxattablestart
\sphinxthistablewithglobalstyle
\sphinxthistablewithnovlinesstyle
\centering
\begin{tabulary}{\linewidth}[t]{\X{1}{2}\X{1}{2}}
\sphinxtoprule
\sphinxtableatstartofbodyhook
\sphinxAtStartPar
{\hyperref[\detokenize{autoapi/IsotropicAngDist/index:IsotropicAngDist.IsotropicAngDist}]{\sphinxcrossref{\sphinxcode{\sphinxupquote{IsotropicAngDist}}}}}
&
\sphinxAtStartPar
Eject particles isotropically from surface into outward facing hemisphere
\\
\sphinxbottomrule
\end{tabulary}
\sphinxtableafterendhook\par
\sphinxattableend\end{savenotes}


\subsubsection{Module Contents}
\label{\detokenize{autoapi/IsotropicAngDist/index:module-contents}}\index{IsotropicAngDist (class in IsotropicAngDist)@\spxentry{IsotropicAngDist}\spxextra{class in IsotropicAngDist}}

\begin{fulllineitems}
\phantomsection\label{\detokenize{autoapi/IsotropicAngDist/index:IsotropicAngDist.IsotropicAngDist}}
\pysigstartsignatures
\pysiglinewithargsret
{\sphinxbfcode{\sphinxupquote{class\DUrole{w}{ }}}\sphinxcode{\sphinxupquote{IsotropicAngDist.}}\sphinxbfcode{\sphinxupquote{IsotropicAngDist}}}
{\sphinxparam{\DUrole{n}{sparam}\DUrole{p}{:}\DUrole{w}{ }\DUrole{n}{dict\DUrole{p}{,}\DUrole{w}{ }tinydb.table.Document}}}
{}
\pysigstopsignatures
\sphinxAtStartPar
Bases: {\hyperref[\detokenize{autoapi/nexoclom2/initial_state/InputClass/index:nexoclom2.initial_state.InputClass.InputClass}]{\sphinxcrossref{\sphinxcode{\sphinxupquote{nexoclom2.initial\_state.InputClass.InputClass}}}}}

\sphinxAtStartPar
Eject particles isotropically from surface into outward facing hemisphere

\sphinxAtStartPar
Parameters that can be set
\begin{itemize}
\item {} 
\sphinxAtStartPar
altitude in degrees from 0º \sphinxhyphen{} 90º

\item {} 
\sphinxAtStartPar
azimuth in degrees from 0º \sphinxhyphen{} 360º

\end{itemize}
\begin{quote}\begin{description}
\sphinxlineitem{Parameters}\begin{description}
\sphinxlineitem{\sphinxstylestrong{sparam: dict, TinyDB Document}}
\sphinxAtStartPar
Key, value for defining the distribution

\end{description}

\sphinxlineitem{Attributes}\begin{description}
\sphinxlineitem{\sphinxstylestrong{azimuth: tuple of astropy Angles}}
\sphinxAtStartPar
azimuth range packets are ejected from measured north from east.
Default: (0º, 360º)

\sphinxlineitem{\sphinxstylestrong{altitude: tuple of astropy Angles}}
\sphinxAtStartPar
altitude range packets are ejected from relative to surface tangent.
Default: (0º, 90º)

\end{description}

\end{description}\end{quote}
\index{\_\_name\_\_ (IsotropicAngDist.IsotropicAngDist attribute)@\spxentry{\_\_name\_\_}\spxextra{IsotropicAngDist.IsotropicAngDist attribute}}

\begin{fulllineitems}
\phantomsection\label{\detokenize{autoapi/IsotropicAngDist/index:IsotropicAngDist.IsotropicAngDist.__name__}}
\pysigstartsignatures
\pysigline
{\sphinxbfcode{\sphinxupquote{\_\_name\_\_}}\sphinxbfcode{\sphinxupquote{\DUrole{w}{ }\DUrole{p}{=}\DUrole{w}{ }\textquotesingle{}IsotropicAngDist\textquotesingle{}}}}
\pysigstopsignatures
\end{fulllineitems}

\index{pdf\_azimuth() (IsotropicAngDist.IsotropicAngDist method)@\spxentry{pdf\_azimuth()}\spxextra{IsotropicAngDist.IsotropicAngDist method}}

\begin{fulllineitems}
\phantomsection\label{\detokenize{autoapi/IsotropicAngDist/index:IsotropicAngDist.IsotropicAngDist.pdf_azimuth}}
\pysigstartsignatures
\pysiglinewithargsret
{\sphinxbfcode{\sphinxupquote{pdf\_azimuth}}}
{\sphinxparam{\DUrole{n}{az}}}
{}
\pysigstopsignatures
\end{fulllineitems}

\index{cdf\_azimuth() (IsotropicAngDist.IsotropicAngDist method)@\spxentry{cdf\_azimuth()}\spxextra{IsotropicAngDist.IsotropicAngDist method}}

\begin{fulllineitems}
\phantomsection\label{\detokenize{autoapi/IsotropicAngDist/index:IsotropicAngDist.IsotropicAngDist.cdf_azimuth}}
\pysigstartsignatures
\pysiglinewithargsret
{\sphinxbfcode{\sphinxupquote{cdf\_azimuth}}}
{\sphinxparam{\DUrole{n}{az}}}
{}
\pysigstopsignatures
\end{fulllineitems}

\index{support\_azimuth() (IsotropicAngDist.IsotropicAngDist method)@\spxentry{support\_azimuth()}\spxextra{IsotropicAngDist.IsotropicAngDist method}}

\begin{fulllineitems}
\phantomsection\label{\detokenize{autoapi/IsotropicAngDist/index:IsotropicAngDist.IsotropicAngDist.support_azimuth}}
\pysigstartsignatures
\pysiglinewithargsret
{\sphinxbfcode{\sphinxupquote{support\_azimuth}}}
{}
{}
\pysigstopsignatures\begin{quote}\begin{description}
\sphinxlineitem{Returns}\begin{description}
\sphinxlineitem{tuple with valid range for the PDF}
\end{description}

\end{description}\end{quote}

\end{fulllineitems}

\index{pdf\_altitude() (IsotropicAngDist.IsotropicAngDist method)@\spxentry{pdf\_altitude()}\spxextra{IsotropicAngDist.IsotropicAngDist method}}

\begin{fulllineitems}
\phantomsection\label{\detokenize{autoapi/IsotropicAngDist/index:IsotropicAngDist.IsotropicAngDist.pdf_altitude}}
\pysigstartsignatures
\pysiglinewithargsret
{\sphinxbfcode{\sphinxupquote{pdf\_altitude}}}
{\sphinxparam{\DUrole{n}{alt}}}
{}
\pysigstopsignatures
\end{fulllineitems}

\index{cdf\_altitude() (IsotropicAngDist.IsotropicAngDist method)@\spxentry{cdf\_altitude()}\spxextra{IsotropicAngDist.IsotropicAngDist method}}

\begin{fulllineitems}
\phantomsection\label{\detokenize{autoapi/IsotropicAngDist/index:IsotropicAngDist.IsotropicAngDist.cdf_altitude}}
\pysigstartsignatures
\pysiglinewithargsret
{\sphinxbfcode{\sphinxupquote{cdf\_altitude}}}
{\sphinxparam{\DUrole{n}{alt}}}
{}
\pysigstopsignatures
\end{fulllineitems}

\index{support\_altitude() (IsotropicAngDist.IsotropicAngDist method)@\spxentry{support\_altitude()}\spxextra{IsotropicAngDist.IsotropicAngDist method}}

\begin{fulllineitems}
\phantomsection\label{\detokenize{autoapi/IsotropicAngDist/index:IsotropicAngDist.IsotropicAngDist.support_altitude}}
\pysigstartsignatures
\pysiglinewithargsret
{\sphinxbfcode{\sphinxupquote{support\_altitude}}}
{}
{}
\pysigstopsignatures\begin{quote}\begin{description}
\sphinxlineitem{Returns}\begin{description}
\sphinxlineitem{tuple with valid range for the PDF}
\end{description}

\end{description}\end{quote}

\end{fulllineitems}

\index{choose\_points() (IsotropicAngDist.IsotropicAngDist method)@\spxentry{choose\_points()}\spxextra{IsotropicAngDist.IsotropicAngDist method}}

\begin{fulllineitems}
\phantomsection\label{\detokenize{autoapi/IsotropicAngDist/index:IsotropicAngDist.IsotropicAngDist.choose_points}}
\pysigstartsignatures
\pysiglinewithargsret
{\sphinxbfcode{\sphinxupquote{choose\_points}}}
{\sphinxparam{\DUrole{n}{n\_packets}}\sphinxparamcomma \sphinxparam{\DUrole{n}{randgen}\DUrole{o}{=}\DUrole{default_value}{None}}}
{}
\pysigstopsignatures
\sphinxAtStartPar
Returns initial x, y, and z. For a moon, a rotation will be done later
to move packets into the proper orbital position
\begin{quote}\begin{description}
\sphinxlineitem{Parameters}\begin{description}
\sphinxlineitem{\sphinxstylestrong{npack}}{[}int{]}
\sphinxAtStartPar
Number of packets to generate

\sphinxlineitem{\sphinxstylestrong{randgen}}{[}numpy.random.\_generator.Generator{]}
\sphinxAtStartPar
Optional. Default=None

\end{description}

\sphinxlineitem{Returns}\begin{description}
\sphinxlineitem{npack x 3 ndarray with x, y, and z positions}
\end{description}

\end{description}\end{quote}

\end{fulllineitems}


\end{fulllineitems}


\sphinxstepscope


\subsection{SurfSpotSpatDist}
\label{\detokenize{autoapi/SurfSpotSpatDist/index:module-SurfSpotSpatDist}}\label{\detokenize{autoapi/SurfSpotSpatDist/index:surfspotspatdist}}\label{\detokenize{autoapi/SurfSpotSpatDist/index::doc}}\index{module@\spxentry{module}!SurfSpotSpatDist@\spxentry{SurfSpotSpatDist}}\index{SurfSpotSpatDist@\spxentry{SurfSpotSpatDist}!module@\spxentry{module}}
\sphinxstepscope


\subsection{reformat\_gvalues}
\label{\detokenize{autoapi/reformat_gvalues/index:module-reformat_gvalues}}\label{\detokenize{autoapi/reformat_gvalues/index:reformat-gvalues}}\label{\detokenize{autoapi/reformat_gvalues/index::doc}}\index{module@\spxentry{module}!reformat\_gvalues@\spxentry{reformat\_gvalues}}\index{reformat\_gvalues@\spxentry{reformat\_gvalues}!module@\spxentry{module}}
\sphinxAtStartPar
Used once to convert gvalue text files to csv and ecsv


\subsubsection{Attributes}
\label{\detokenize{autoapi/reformat_gvalues/index:attributes}}

\begin{savenotes}\sphinxattablestart
\sphinxthistablewithglobalstyle
\sphinxthistablewithnovlinesstyle
\centering
\begin{tabulary}{\linewidth}[t]{\X{1}{2}\X{1}{2}}
\sphinxtoprule
\sphinxtableatstartofbodyhook
\sphinxAtStartPar
{\hyperref[\detokenize{autoapi/reformat_gvalues/index:reformat_gvalues.species}]{\sphinxcrossref{\sphinxcode{\sphinxupquote{species}}}}}
&
\sphinxAtStartPar

\\
\sphinxbottomrule
\end{tabulary}
\sphinxtableafterendhook\par
\sphinxattableend\end{savenotes}


\subsubsection{Functions}
\label{\detokenize{autoapi/reformat_gvalues/index:functions}}

\begin{savenotes}\sphinxattablestart
\sphinxthistablewithglobalstyle
\sphinxthistablewithnovlinesstyle
\centering
\begin{tabulary}{\linewidth}[t]{\X{1}{2}\X{1}{2}}
\sphinxtoprule
\sphinxtableatstartofbodyhook
\sphinxAtStartPar
{\hyperref[\detokenize{autoapi/reformat_gvalues/index:reformat_gvalues.reformat_gvalues}]{\sphinxcrossref{\sphinxcode{\sphinxupquote{reformat\_gvalues}}}}}(species)
&
\sphinxAtStartPar

\\
\sphinxhline
\sphinxAtStartPar
{\hyperref[\detokenize{autoapi/reformat_gvalues/index:reformat_gvalues.gvalue_csv_to_ecsv}]{\sphinxcrossref{\sphinxcode{\sphinxupquote{gvalue\_csv\_to\_ecsv}}}}}(species)
&
\sphinxAtStartPar

\\
\sphinxbottomrule
\end{tabulary}
\sphinxtableafterendhook\par
\sphinxattableend\end{savenotes}


\subsubsection{Module Contents}
\label{\detokenize{autoapi/reformat_gvalues/index:module-contents}}\index{reformat\_gvalues() (in module reformat\_gvalues)@\spxentry{reformat\_gvalues()}\spxextra{in module reformat\_gvalues}}

\begin{fulllineitems}
\phantomsection\label{\detokenize{autoapi/reformat_gvalues/index:reformat_gvalues.reformat_gvalues}}
\pysigstartsignatures
\pysiglinewithargsret
{\sphinxcode{\sphinxupquote{reformat\_gvalues.}}\sphinxbfcode{\sphinxupquote{reformat\_gvalues}}}
{\sphinxparam{\DUrole{n}{species}}}
{}
\pysigstopsignatures
\end{fulllineitems}

\index{gvalue\_csv\_to\_ecsv() (in module reformat\_gvalues)@\spxentry{gvalue\_csv\_to\_ecsv()}\spxextra{in module reformat\_gvalues}}

\begin{fulllineitems}
\phantomsection\label{\detokenize{autoapi/reformat_gvalues/index:reformat_gvalues.gvalue_csv_to_ecsv}}
\pysigstartsignatures
\pysiglinewithargsret
{\sphinxcode{\sphinxupquote{reformat\_gvalues.}}\sphinxbfcode{\sphinxupquote{gvalue\_csv\_to\_ecsv}}}
{\sphinxparam{\DUrole{n}{species}}}
{}
\pysigstopsignatures
\end{fulllineitems}

\index{species (in module reformat\_gvalues)@\spxentry{species}\spxextra{in module reformat\_gvalues}}

\begin{fulllineitems}
\phantomsection\label{\detokenize{autoapi/reformat_gvalues/index:reformat_gvalues.species}}
\pysigstartsignatures
\pysigline
{\sphinxcode{\sphinxupquote{reformat\_gvalues.}}\sphinxbfcode{\sphinxupquote{species}}\sphinxbfcode{\sphinxupquote{\DUrole{w}{ }\DUrole{p}{=}\DUrole{w}{ }{[}\textquotesingle{}Al\textquotesingle{}, \textquotesingle{}Ca\textquotesingle{}, \textquotesingle{}CaII\textquotesingle{}, \textquotesingle{}H\textquotesingle{}, \textquotesingle{}He\textquotesingle{}, \textquotesingle{}K\textquotesingle{}, \textquotesingle{}Mg\textquotesingle{}, \textquotesingle{}MgII\textquotesingle{}, \textquotesingle{}Mn\textquotesingle{}, \textquotesingle{}Na\textquotesingle{}, \textquotesingle{}O\textquotesingle{}, \textquotesingle{}S\textquotesingle{}{]}}}}
\pysigstopsignatures
\end{fulllineitems}


\sphinxstepscope


\subsection{SputteringFluxDist}
\label{\detokenize{autoapi/SputteringFluxDist/index:module-SputteringFluxDist}}\label{\detokenize{autoapi/SputteringFluxDist/index:sputteringfluxdist}}\label{\detokenize{autoapi/SputteringFluxDist/index::doc}}\index{module@\spxentry{module}!SputteringFluxDist@\spxentry{SputteringFluxDist}}\index{SputteringFluxDist@\spxentry{SputteringFluxDist}!module@\spxentry{module}}

\subsubsection{Classes}
\label{\detokenize{autoapi/SputteringFluxDist/index:classes}}

\begin{savenotes}\sphinxattablestart
\sphinxthistablewithglobalstyle
\sphinxthistablewithnovlinesstyle
\centering
\begin{tabulary}{\linewidth}[t]{\X{1}{2}\X{1}{2}}
\sphinxtoprule
\sphinxtableatstartofbodyhook
\sphinxAtStartPar
{\hyperref[\detokenize{autoapi/SputteringFluxDist/index:SputteringFluxDist.SputteringFluxDist}]{\sphinxcrossref{\sphinxcode{\sphinxupquote{SputteringFluxDist}}}}}
&
\sphinxAtStartPar
Defines a Sputtering flux distribution from the surface.
\\
\sphinxbottomrule
\end{tabulary}
\sphinxtableafterendhook\par
\sphinxattableend\end{savenotes}


\subsubsection{Module Contents}
\label{\detokenize{autoapi/SputteringFluxDist/index:module-contents}}\index{SputteringFluxDist (class in SputteringFluxDist)@\spxentry{SputteringFluxDist}\spxextra{class in SputteringFluxDist}}

\begin{fulllineitems}
\phantomsection\label{\detokenize{autoapi/SputteringFluxDist/index:SputteringFluxDist.SputteringFluxDist}}
\pysigstartsignatures
\pysiglinewithargsret
{\sphinxbfcode{\sphinxupquote{class\DUrole{w}{ }}}\sphinxcode{\sphinxupquote{SputteringFluxDist.}}\sphinxbfcode{\sphinxupquote{SputteringFluxDist}}}
{\sphinxparam{\DUrole{n}{sparam}\DUrole{p}{:}\DUrole{w}{ }\DUrole{n}{dict}}}
{}
\pysigstopsignatures
\sphinxAtStartPar
Bases: {\hyperref[\detokenize{autoapi/nexoclom2/initial_state/InputClass/index:nexoclom2.initial_state.InputClass.InputClass}]{\sphinxcrossref{\sphinxcode{\sphinxupquote{nexoclom2.initial\_state.InputClass.InputClass}}}}}

\sphinxAtStartPar
Defines a Sputtering flux distribution from the surface.

\sphinxAtStartPar
Sets up an initial flux distribution with a sputting speed distribution.
\begin{quote}\begin{description}
\sphinxlineitem{Parameters}\begin{description}
\sphinxlineitem{\sphinxstylestrong{sparam}}{[}dict{]}
\sphinxAtStartPar
Key, vaue for defining the distribution

\end{description}

\sphinxlineitem{Attributes}\begin{description}
\sphinxlineitem{\sphinxstylestrong{alpha}}{[}float{]}
\sphinxlineitem{\sphinxstylestrong{beta}}{[}float{]}
\sphinxlineitem{\sphinxstylestrong{U}}{[}astropy quantity{]}
\sphinxAtStartPar
Surface binding energy

\sphinxlineitem{\sphinxstylestrong{species}}{[}Atom{]}
\end{description}

\end{description}\end{quote}
\index{\_\_name\_\_ (SputteringFluxDist.SputteringFluxDist attribute)@\spxentry{\_\_name\_\_}\spxextra{SputteringFluxDist.SputteringFluxDist attribute}}

\begin{fulllineitems}
\phantomsection\label{\detokenize{autoapi/SputteringFluxDist/index:SputteringFluxDist.SputteringFluxDist.__name__}}
\pysigstartsignatures
\pysigline
{\sphinxbfcode{\sphinxupquote{\_\_name\_\_}}\sphinxbfcode{\sphinxupquote{\DUrole{w}{ }\DUrole{p}{=}\DUrole{w}{ }\textquotesingle{}SputteringFluxDist\textquotesingle{}}}}
\pysigstopsignatures
\end{fulllineitems}

\index{pdf() (SputteringFluxDist.SputteringFluxDist method)@\spxentry{pdf()}\spxextra{SputteringFluxDist.SputteringFluxDist method}}

\begin{fulllineitems}
\phantomsection\label{\detokenize{autoapi/SputteringFluxDist/index:SputteringFluxDist.SputteringFluxDist.pdf}}
\pysigstartsignatures
\pysiglinewithargsret
{\sphinxbfcode{\sphinxupquote{pdf}}}
{\sphinxparam{\DUrole{n}{v}}}
{}
\pysigstopsignatures
\end{fulllineitems}

\index{support() (SputteringFluxDist.SputteringFluxDist method)@\spxentry{support()}\spxextra{SputteringFluxDist.SputteringFluxDist method}}

\begin{fulllineitems}
\phantomsection\label{\detokenize{autoapi/SputteringFluxDist/index:SputteringFluxDist.SputteringFluxDist.support}}
\pysigstartsignatures
\pysiglinewithargsret
{\sphinxbfcode{\sphinxupquote{support}}}
{}
{}
\pysigstopsignatures
\end{fulllineitems}

\index{choose\_points() (SputteringFluxDist.SputteringFluxDist method)@\spxentry{choose\_points()}\spxextra{SputteringFluxDist.SputteringFluxDist method}}

\begin{fulllineitems}
\phantomsection\label{\detokenize{autoapi/SputteringFluxDist/index:SputteringFluxDist.SputteringFluxDist.choose_points}}
\pysigstartsignatures
\pysiglinewithargsret
{\sphinxbfcode{\sphinxupquote{choose\_points}}}
{\sphinxparam{\DUrole{n}{n\_packets}}\sphinxparamcomma \sphinxparam{\DUrole{n}{randgen}\DUrole{o}{=}\DUrole{default_value}{None}}}
{}
\pysigstopsignatures
\sphinxAtStartPar
Compute random deviates from arbitrary 1D distribution.
f\_x does not need to integrate to 1. The function normalizes the
distribution. Uses Transformation method (Numerical Recipes, 7.3.2)
\begin{quote}\begin{description}
\sphinxlineitem{Parameters}\begin{description}
\sphinxlineitem{\sphinxstylestrong{n\_packets}}{[}int{]}
\sphinxAtStartPar
The number of random deviates to compute

\sphinxlineitem{\sphinxstylestrong{randgen}}{[}numpy.random.\_generator.Generator{]}
\end{description}

\sphinxlineitem{Returns}\begin{description}
\sphinxlineitem{numpy array of length num chosen from the distribution f\_x.}
\end{description}

\end{description}\end{quote}

\end{fulllineitems}


\end{fulllineitems}


\sphinxstepscope


\subsection{MaxwellianFluxDist}
\label{\detokenize{autoapi/MaxwellianFluxDist/index:module-MaxwellianFluxDist}}\label{\detokenize{autoapi/MaxwellianFluxDist/index:maxwellianfluxdist}}\label{\detokenize{autoapi/MaxwellianFluxDist/index::doc}}\index{module@\spxentry{module}!MaxwellianFluxDist@\spxentry{MaxwellianFluxDist}}\index{MaxwellianFluxDist@\spxentry{MaxwellianFluxDist}!module@\spxentry{module}}

\subsubsection{Classes}
\label{\detokenize{autoapi/MaxwellianFluxDist/index:classes}}

\begin{savenotes}\sphinxattablestart
\sphinxthistablewithglobalstyle
\sphinxthistablewithnovlinesstyle
\centering
\begin{tabulary}{\linewidth}[t]{\X{1}{2}\X{1}{2}}
\sphinxtoprule
\sphinxtableatstartofbodyhook
\sphinxAtStartPar
{\hyperref[\detokenize{autoapi/MaxwellianFluxDist/index:MaxwellianFluxDist.MaxwellianFluxDist}]{\sphinxcrossref{\sphinxcode{\sphinxupquote{MaxwellianFluxDist}}}}}
&
\sphinxAtStartPar
Defines a Maxwellian flux distribution from the surface.
\\
\sphinxbottomrule
\end{tabulary}
\sphinxtableafterendhook\par
\sphinxattableend\end{savenotes}


\subsubsection{Module Contents}
\label{\detokenize{autoapi/MaxwellianFluxDist/index:module-contents}}\index{MaxwellianFluxDist (class in MaxwellianFluxDist)@\spxentry{MaxwellianFluxDist}\spxextra{class in MaxwellianFluxDist}}

\begin{fulllineitems}
\phantomsection\label{\detokenize{autoapi/MaxwellianFluxDist/index:MaxwellianFluxDist.MaxwellianFluxDist}}
\pysigstartsignatures
\pysiglinewithargsret
{\sphinxbfcode{\sphinxupquote{class\DUrole{w}{ }}}\sphinxcode{\sphinxupquote{MaxwellianFluxDist.}}\sphinxbfcode{\sphinxupquote{MaxwellianFluxDist}}}
{\sphinxparam{\DUrole{n}{sparam}\DUrole{p}{:}\DUrole{w}{ }\DUrole{n}{dict}}}
{}
\pysigstopsignatures
\sphinxAtStartPar
Bases: {\hyperref[\detokenize{autoapi/nexoclom2/initial_state/InputClass/index:nexoclom2.initial_state.InputClass.InputClass}]{\sphinxcrossref{\sphinxcode{\sphinxupquote{nexoclom2.initial\_state.InputClass.InputClass}}}}}

\sphinxAtStartPar
Defines a Maxwellian flux distribution from the surface.

\sphinxAtStartPar
Sets up an initial flux distribution with a Maxwellian speed distribution.
see {\hyperref[\detokenize{nexoclom2/inputfiles:maxwellianspeeddist}]{\sphinxcrossref{\DUrole{std}{\DUrole{std-ref}{Maxwellian Distribution}}}}} for more details.
\begin{quote}\begin{description}
\sphinxlineitem{Parameters}\begin{description}
\sphinxlineitem{\sphinxstylestrong{sparam}}{[}dict{]}
\sphinxAtStartPar
Key, value for defining the distribution

\end{description}

\sphinxlineitem{Attributes}\begin{description}
\sphinxlineitem{\sphinxstylestrong{temperature}}{[}astropy quantity{]}
\sphinxlineitem{\sphinxstylestrong{species}}{[}Atom{]}
\sphinxAtStartPar
Particle being ejected

\sphinxlineitem{\sphinxstylestrong{v\_th}}{[}astropy quantity{]}
\sphinxAtStartPar
Thermal speed

\end{description}

\end{description}\end{quote}
\index{\_\_name\_\_ (MaxwellianFluxDist.MaxwellianFluxDist attribute)@\spxentry{\_\_name\_\_}\spxextra{MaxwellianFluxDist.MaxwellianFluxDist attribute}}

\begin{fulllineitems}
\phantomsection\label{\detokenize{autoapi/MaxwellianFluxDist/index:MaxwellianFluxDist.MaxwellianFluxDist.__name__}}
\pysigstartsignatures
\pysigline
{\sphinxbfcode{\sphinxupquote{\_\_name\_\_}}\sphinxbfcode{\sphinxupquote{\DUrole{w}{ }\DUrole{p}{=}\DUrole{w}{ }\textquotesingle{}MaxwellianFluxDist\textquotesingle{}}}}
\pysigstopsignatures
\end{fulllineitems}

\index{pdf() (MaxwellianFluxDist.MaxwellianFluxDist method)@\spxentry{pdf()}\spxextra{MaxwellianFluxDist.MaxwellianFluxDist method}}

\begin{fulllineitems}
\phantomsection\label{\detokenize{autoapi/MaxwellianFluxDist/index:MaxwellianFluxDist.MaxwellianFluxDist.pdf}}
\pysigstartsignatures
\pysiglinewithargsret
{\sphinxbfcode{\sphinxupquote{pdf}}}
{\sphinxparam{\DUrole{n}{v}}}
{}
\pysigstopsignatures
\sphinxAtStartPar
Probability Distribution Function
Needs to be unitless

\end{fulllineitems}

\index{support() (MaxwellianFluxDist.MaxwellianFluxDist method)@\spxentry{support()}\spxextra{MaxwellianFluxDist.MaxwellianFluxDist method}}

\begin{fulllineitems}
\phantomsection\label{\detokenize{autoapi/MaxwellianFluxDist/index:MaxwellianFluxDist.MaxwellianFluxDist.support}}
\pysigstartsignatures
\pysiglinewithargsret
{\sphinxbfcode{\sphinxupquote{support}}}
{}
{}
\pysigstopsignatures\begin{quote}\begin{description}
\sphinxlineitem{Returns}\begin{description}
\sphinxlineitem{tuple with valid range for the PDF}
\end{description}

\end{description}\end{quote}

\end{fulllineitems}

\index{choose\_points() (MaxwellianFluxDist.MaxwellianFluxDist method)@\spxentry{choose\_points()}\spxextra{MaxwellianFluxDist.MaxwellianFluxDist method}}

\begin{fulllineitems}
\phantomsection\label{\detokenize{autoapi/MaxwellianFluxDist/index:MaxwellianFluxDist.MaxwellianFluxDist.choose_points}}
\pysigstartsignatures
\pysiglinewithargsret
{\sphinxbfcode{\sphinxupquote{choose\_points}}}
{\sphinxparam{\DUrole{n}{n\_packets}}\sphinxparamcomma \sphinxparam{\DUrole{n}{randgen}\DUrole{o}{=}\DUrole{default_value}{None}}}
{}
\pysigstopsignatures
\sphinxAtStartPar
Compute random deviates from arbitrary 1D distribution.
f\_x does not need to integrate to 1. The function normalizes the
distribution. Uses Transformation method (Numerical Recipes, 7.3.2)
\begin{quote}\begin{description}
\sphinxlineitem{Parameters}\begin{description}
\sphinxlineitem{\sphinxstylestrong{n\_packets}}{[}int{]}
\sphinxAtStartPar
The number of random deviates to compute

\sphinxlineitem{\sphinxstylestrong{randgen}}{[}numpy.random.\_generator.Generator{]}
\end{description}

\sphinxlineitem{Returns}\begin{description}
\sphinxlineitem{numpy array of length num chosen from the distribution f\_x.}
\end{description}

\end{description}\end{quote}

\end{fulllineitems}


\end{fulllineitems}


\sphinxstepscope


\subsection{GoldenSpiralSpatDist}
\label{\detokenize{autoapi/GoldenSpiralSpatDist/index:module-GoldenSpiralSpatDist}}\label{\detokenize{autoapi/GoldenSpiralSpatDist/index:goldenspiralspatdist}}\label{\detokenize{autoapi/GoldenSpiralSpatDist/index::doc}}\index{module@\spxentry{module}!GoldenSpiralSpatDist@\spxentry{GoldenSpiralSpatDist}}\index{GoldenSpiralSpatDist@\spxentry{GoldenSpiralSpatDist}!module@\spxentry{module}}

\subsubsection{Attributes}
\label{\detokenize{autoapi/GoldenSpiralSpatDist/index:attributes}}

\begin{savenotes}\sphinxattablestart
\sphinxthistablewithglobalstyle
\sphinxthistablewithnovlinesstyle
\centering
\begin{tabulary}{\linewidth}[t]{\X{1}{2}\X{1}{2}}
\sphinxtoprule
\sphinxtableatstartofbodyhook
\sphinxAtStartPar
{\hyperref[\detokenize{autoapi/GoldenSpiralSpatDist/index:GoldenSpiralSpatDist.golden_pts}]{\sphinxcrossref{\sphinxcode{\sphinxupquote{golden\_pts}}}}}
&
\sphinxAtStartPar

\\
\sphinxbottomrule
\end{tabulary}
\sphinxtableafterendhook\par
\sphinxattableend\end{savenotes}


\subsubsection{Classes}
\label{\detokenize{autoapi/GoldenSpiralSpatDist/index:classes}}

\begin{savenotes}\sphinxattablestart
\sphinxthistablewithglobalstyle
\sphinxthistablewithnovlinesstyle
\centering
\begin{tabulary}{\linewidth}[t]{\X{1}{2}\X{1}{2}}
\sphinxtoprule
\sphinxtableatstartofbodyhook
\sphinxAtStartPar
{\hyperref[\detokenize{autoapi/GoldenSpiralSpatDist/index:GoldenSpiralSpatDist.GoldenSpiralSpatDist}]{\sphinxcrossref{\sphinxcode{\sphinxupquote{GoldenSpiralSpatDist}}}}}
&
\sphinxAtStartPar
Choose points roughly equally spaced on a sphere.
\\
\sphinxbottomrule
\end{tabulary}
\sphinxtableafterendhook\par
\sphinxattableend\end{savenotes}


\subsubsection{Module Contents}
\label{\detokenize{autoapi/GoldenSpiralSpatDist/index:module-contents}}\index{golden\_pts (in module GoldenSpiralSpatDist)@\spxentry{golden\_pts}\spxextra{in module GoldenSpiralSpatDist}}

\begin{fulllineitems}
\phantomsection\label{\detokenize{autoapi/GoldenSpiralSpatDist/index:GoldenSpiralSpatDist.golden_pts}}
\pysigstartsignatures
\pysigline
{\sphinxcode{\sphinxupquote{GoldenSpiralSpatDist.}}\sphinxbfcode{\sphinxupquote{golden\_pts}}}
\pysigstopsignatures
\end{fulllineitems}

\index{GoldenSpiralSpatDist (class in GoldenSpiralSpatDist)@\spxentry{GoldenSpiralSpatDist}\spxextra{class in GoldenSpiralSpatDist}}

\begin{fulllineitems}
\phantomsection\label{\detokenize{autoapi/GoldenSpiralSpatDist/index:GoldenSpiralSpatDist.GoldenSpiralSpatDist}}
\pysigstartsignatures
\pysiglinewithargsret
{\sphinxbfcode{\sphinxupquote{class\DUrole{w}{ }}}\sphinxcode{\sphinxupquote{GoldenSpiralSpatDist.}}\sphinxbfcode{\sphinxupquote{GoldenSpiralSpatDist}}}
{\sphinxparam{\DUrole{n}{sparam}}}
{}
\pysigstopsignatures
\sphinxAtStartPar
Bases: {\hyperref[\detokenize{autoapi/nexoclom2/initial_state/InputClass/index:nexoclom2.initial_state.InputClass.InputClass}]{\sphinxcrossref{\sphinxcode{\sphinxupquote{nexoclom2.initial\_state.InputClass.InputClass}}}}}

\sphinxAtStartPar
Choose points roughly equally spaced on a sphere.

\sphinxAtStartPar
This uses the astropy.coordinates.golden\_spiral\_grid to choose the points

\sphinxAtStartPar
Parameters that can be set:
\begin{itemize}
\item {} 
\sphinxAtStartPar
exobase

\end{itemize}
\begin{quote}\begin{description}
\sphinxlineitem{Parameters}\begin{description}
\sphinxlineitem{\sphinxstylestrong{sparam: dict}}
\sphinxAtStartPar
Key, value for defining the distribution

\end{description}

\sphinxlineitem{Attributes}\begin{description}
\sphinxlineitem{\sphinxstylestrong{exobase: float}}
\sphinxAtStartPar
Distance from starting object’s center from which to eject particles.
Measured relative to starting object’s radius. Default: 1.0

\sphinxlineitem{\sphinxstylestrong{frame: str}}
\sphinxAtStartPar
Defaults to SOLAR. It is possible to change this, but there isn’t much
reason to.

\end{description}

\end{description}\end{quote}
\index{\_\_name\_\_ (GoldenSpiralSpatDist.GoldenSpiralSpatDist attribute)@\spxentry{\_\_name\_\_}\spxextra{GoldenSpiralSpatDist.GoldenSpiralSpatDist attribute}}

\begin{fulllineitems}
\phantomsection\label{\detokenize{autoapi/GoldenSpiralSpatDist/index:GoldenSpiralSpatDist.GoldenSpiralSpatDist.__name__}}
\pysigstartsignatures
\pysigline
{\sphinxbfcode{\sphinxupquote{\_\_name\_\_}}\sphinxbfcode{\sphinxupquote{\DUrole{w}{ }\DUrole{p}{=}\DUrole{w}{ }\textquotesingle{}TwoDRegularSpatDist\textquotesingle{}}}}
\pysigstopsignatures
\end{fulllineitems}

\index{frame (GoldenSpiralSpatDist.GoldenSpiralSpatDist attribute)@\spxentry{frame}\spxextra{GoldenSpiralSpatDist.GoldenSpiralSpatDist attribute}}

\begin{fulllineitems}
\phantomsection\label{\detokenize{autoapi/GoldenSpiralSpatDist/index:GoldenSpiralSpatDist.GoldenSpiralSpatDist.frame}}
\pysigstartsignatures
\pysigline
{\sphinxbfcode{\sphinxupquote{frame}}\sphinxbfcode{\sphinxupquote{\DUrole{w}{ }\DUrole{p}{=}\DUrole{w}{ }\textquotesingle{}SOLAR\textquotesingle{}}}}
\pysigstopsignatures
\end{fulllineitems}

\index{choose\_points() (GoldenSpiralSpatDist.GoldenSpiralSpatDist method)@\spxentry{choose\_points()}\spxextra{GoldenSpiralSpatDist.GoldenSpiralSpatDist method}}

\begin{fulllineitems}
\phantomsection\label{\detokenize{autoapi/GoldenSpiralSpatDist/index:GoldenSpiralSpatDist.GoldenSpiralSpatDist.choose_points}}
\pysigstartsignatures
\pysiglinewithargsret
{\sphinxbfcode{\sphinxupquote{choose\_points}}}
{\sphinxparam{\DUrole{n}{npackets}}}
{}
\pysigstopsignatures\begin{quote}\begin{description}
\sphinxlineitem{Parameters}\begin{description}
\sphinxlineitem{\sphinxstylestrong{npackets}}
\end{description}

\sphinxlineitem{Returns}\begin{description}
\sphinxlineitem{dictionary with longitude and latitude.}
\end{description}

\end{description}\end{quote}

\end{fulllineitems}


\end{fulllineitems}


\sphinxstepscope


\subsection{TemperatureDependentSurfInt}
\label{\detokenize{autoapi/TemperatureDependentSurfInt/index:module-TemperatureDependentSurfInt}}\label{\detokenize{autoapi/TemperatureDependentSurfInt/index:temperaturedependentsurfint}}\label{\detokenize{autoapi/TemperatureDependentSurfInt/index::doc}}\index{module@\spxentry{module}!TemperatureDependentSurfInt@\spxentry{TemperatureDependentSurfInt}}\index{TemperatureDependentSurfInt@\spxentry{TemperatureDependentSurfInt}!module@\spxentry{module}}

\section{Indices and tables}
\label{\detokenize{index:indices-and-tables}}
\sphinxAtStartPar
\DUrole{xref}{\DUrole{std}{\DUrole{std-ref}{genindex}}}
\DUrole{xref}{\DUrole{std}{\DUrole{std-ref}{modindex}}}
\DUrole{xref}{\DUrole{std}{\DUrole{std-ref}{search}}}

\begin{sphinxthebibliography}{Huebner2}
\bibitem[Huebner2015]{nexoclom2/atomicdata:huebner2015}
\sphinxAtStartPar
Huebner and Mukherjee, Photoionization and photodissociation
rates in solar and blackbody radiation fields, Planetary and Space Science, 106,
11\sphinxhyphen{}45, 2015, \sphinxhref{https://www.sciencedirect.com/science/article/pii/S003206331400381X?via\%3Dihub}{10.1016/j.pss.2014.11.022}
\bibitem[Huebner2015]{nexoclom2/references:huebner2015}
\sphinxAtStartPar
Huebner and Mukherjee, Photoionization and photodissociation
rates in solar and blackbody radiation fields, Planetary and Space Science, 106,
11\sphinxhyphen{}45, 2015, \sphinxhref{https://www.sciencedirect.com/science/article/pii/S003206331400381X?via\%3Dihub}{10.1016/j.pss.2014.11.022}
\end{sphinxthebibliography}


\renewcommand{\indexname}{Python Module Index}
\begin{sphinxtheindex}
\let\bigletter\sphinxstyleindexlettergroup
\bigletter{c}
\item\relax\sphinxstyleindexentry{ConstantSurfInt}\sphinxstyleindexpageref{autoapi/ConstantSurfInt/index:\detokenize{module-ConstantSurfInt}}
\indexspace
\bigletter{f}
\item\relax\sphinxstyleindexentry{FlatSpeedDist}\sphinxstyleindexpageref{autoapi/FlatSpeedDist/index:\detokenize{module-FlatSpeedDist}}
\indexspace
\bigletter{g}
\item\relax\sphinxstyleindexentry{Geometry}\sphinxstyleindexpageref{autoapi/Geometry/index:\detokenize{module-Geometry}}
\item\relax\sphinxstyleindexentry{GeometryNoTime}\sphinxstyleindexpageref{autoapi/GeometryNoTime/index:\detokenize{module-GeometryNoTime}}
\item\relax\sphinxstyleindexentry{GeometryTime}\sphinxstyleindexpageref{autoapi/GeometryTime/index:\detokenize{module-GeometryTime}}
\item\relax\sphinxstyleindexentry{GoldenSpiralSpatDist}\sphinxstyleindexpageref{autoapi/GoldenSpiralSpatDist/index:\detokenize{module-GoldenSpiralSpatDist}}
\indexspace
\bigletter{i}
\item\relax\sphinxstyleindexentry{IsotropicAngDist}\sphinxstyleindexpageref{autoapi/IsotropicAngDist/index:\detokenize{module-IsotropicAngDist}}
\indexspace
\bigletter{m}
\item\relax\sphinxstyleindexentry{MaxwellianFluxDist}\sphinxstyleindexpageref{autoapi/MaxwellianFluxDist/index:\detokenize{module-MaxwellianFluxDist}}
\indexspace
\bigletter{n}
\item\relax\sphinxstyleindexentry{nexoclom2}\sphinxstyleindexpageref{autoapi/nexoclom2/index:\detokenize{module-nexoclom2}}
\item\relax\sphinxstyleindexentry{nexoclom2.atomicdata}\sphinxstyleindexpageref{autoapi/nexoclom2/atomicdata/index:\detokenize{module-nexoclom2.atomicdata}}
\item\relax\sphinxstyleindexentry{nexoclom2.atomicdata.atom}\sphinxstyleindexpageref{autoapi/nexoclom2/atomicdata/atom/index:\detokenize{module-nexoclom2.atomicdata.atom}}
\item\relax\sphinxstyleindexentry{nexoclom2.atomicdata.charge\_exchange}\sphinxstyleindexpageref{autoapi/nexoclom2/atomicdata/charge_exchange/index:\detokenize{module-nexoclom2.atomicdata.charge_exchange}}
\item\relax\sphinxstyleindexentry{nexoclom2.atomicdata.eimp\_emission\_coef}\sphinxstyleindexpageref{autoapi/nexoclom2/atomicdata/eimp_emission_coef/index:\detokenize{module-nexoclom2.atomicdata.eimp_emission_coef}}
\item\relax\sphinxstyleindexentry{nexoclom2.atomicdata.eimp\_ionization\_coef}\sphinxstyleindexpageref{autoapi/nexoclom2/atomicdata/eimp_ionization_coef/index:\detokenize{module-nexoclom2.atomicdata.eimp_ionization_coef}}
\item\relax\sphinxstyleindexentry{nexoclom2.atomicdata.extract\_chX\_coefs}\sphinxstyleindexpageref{autoapi/nexoclom2/atomicdata/extract_chX_coefs/index:\detokenize{module-nexoclom2.atomicdata.extract_chX_coefs}}
\item\relax\sphinxstyleindexentry{nexoclom2.atomicdata.extract\_eimp\_coefs}\sphinxstyleindexpageref{autoapi/nexoclom2/atomicdata/extract_eimp_coefs/index:\detokenize{module-nexoclom2.atomicdata.extract_eimp_coefs}}
\item\relax\sphinxstyleindexentry{nexoclom2.atomicdata.gvalues}\sphinxstyleindexpageref{autoapi/nexoclom2/atomicdata/gvalues/index:\detokenize{module-nexoclom2.atomicdata.gvalues}}
\item\relax\sphinxstyleindexentry{nexoclom2.atomicdata.lossrate}\sphinxstyleindexpageref{autoapi/nexoclom2/atomicdata/lossrate/index:\detokenize{module-nexoclom2.atomicdata.lossrate}}
\item\relax\sphinxstyleindexentry{nexoclom2.atomicdata.reformat\_gvalues}\sphinxstyleindexpageref{autoapi/nexoclom2/atomicdata/reformat_gvalues/index:\detokenize{module-nexoclom2.atomicdata.reformat_gvalues}}
\item\relax\sphinxstyleindexentry{nexoclom2.data}\sphinxstyleindexpageref{autoapi/nexoclom2/data/index:\detokenize{module-nexoclom2.data}}
\item\relax\sphinxstyleindexentry{nexoclom2.data\_simulation}\sphinxstyleindexpageref{autoapi/nexoclom2/data_simulation/index:\detokenize{module-nexoclom2.data_simulation}}
\item\relax\sphinxstyleindexentry{nexoclom2.data\_simulation.ModelImage}\sphinxstyleindexpageref{autoapi/nexoclom2/data_simulation/ModelImage/index:\detokenize{module-nexoclom2.data_simulation.ModelImage}}
\item\relax\sphinxstyleindexentry{nexoclom2.data\_simulation.ModelResult}\sphinxstyleindexpageref{autoapi/nexoclom2/data_simulation/ModelResult/index:\detokenize{module-nexoclom2.data_simulation.ModelResult}}
\item\relax\sphinxstyleindexentry{nexoclom2.initial\_state}\sphinxstyleindexpageref{autoapi/nexoclom2/initial_state/index:\detokenize{module-nexoclom2.initial_state}}
\item\relax\sphinxstyleindexentry{nexoclom2.initial\_state.Forces}\sphinxstyleindexpageref{autoapi/nexoclom2/initial_state/Forces/index:\detokenize{module-nexoclom2.initial_state.Forces}}
\item\relax\sphinxstyleindexentry{nexoclom2.initial\_state.Input}\sphinxstyleindexpageref{autoapi/nexoclom2/initial_state/Input/index:\detokenize{module-nexoclom2.initial_state.Input}}
\item\relax\sphinxstyleindexentry{nexoclom2.initial\_state.InputClass}\sphinxstyleindexpageref{autoapi/nexoclom2/initial_state/InputClass/index:\detokenize{module-nexoclom2.initial_state.InputClass}}
\item\relax\sphinxstyleindexentry{nexoclom2.initial\_state.LossInformation}\sphinxstyleindexpageref{autoapi/nexoclom2/initial_state/LossInformation/index:\detokenize{module-nexoclom2.initial_state.LossInformation}}
\item\relax\sphinxstyleindexentry{nexoclom2.initial\_state.Options}\sphinxstyleindexpageref{autoapi/nexoclom2/initial_state/Options/index:\detokenize{module-nexoclom2.initial_state.Options}}
\item\relax\sphinxstyleindexentry{nexoclom2.math}\sphinxstyleindexpageref{autoapi/nexoclom2/math/index:\detokenize{module-nexoclom2.math}}
\item\relax\sphinxstyleindexentry{nexoclom2.math.histogram}\sphinxstyleindexpageref{autoapi/nexoclom2/math/histogram/index:\detokenize{module-nexoclom2.math.histogram}}
\item\relax\sphinxstyleindexentry{nexoclom2.math.ks\_test}\sphinxstyleindexpageref{autoapi/nexoclom2/math/ks_test/index:\detokenize{module-nexoclom2.math.ks_test}}
\item\relax\sphinxstyleindexentry{nexoclom2.math.mod\_close}\sphinxstyleindexpageref{autoapi/nexoclom2/math/mod_close/index:\detokenize{module-nexoclom2.math.mod_close}}
\item\relax\sphinxstyleindexentry{nexoclom2.math.rotation\_matrix}\sphinxstyleindexpageref{autoapi/nexoclom2/math/rotation_matrix/index:\detokenize{module-nexoclom2.math.rotation_matrix}}
\item\relax\sphinxstyleindexentry{nexoclom2.particle\_tracking}\sphinxstyleindexpageref{autoapi/nexoclom2/particle_tracking/index:\detokenize{module-nexoclom2.particle_tracking}}
\item\relax\sphinxstyleindexentry{nexoclom2.particle\_tracking.compute\_accel}\sphinxstyleindexpageref{autoapi/nexoclom2/particle_tracking/compute_accel/index:\detokenize{module-nexoclom2.particle_tracking.compute_accel}}
\item\relax\sphinxstyleindexentry{nexoclom2.particle\_tracking.ConstantIntegrator}\sphinxstyleindexpageref{autoapi/nexoclom2/particle_tracking/ConstantIntegrator/index:\detokenize{module-nexoclom2.particle_tracking.ConstantIntegrator}}
\item\relax\sphinxstyleindexentry{nexoclom2.particle\_tracking.final\_state}\sphinxstyleindexpageref{autoapi/nexoclom2/particle_tracking/final_state/index:\detokenize{module-nexoclom2.particle_tracking.final_state}}
\item\relax\sphinxstyleindexentry{nexoclom2.particle\_tracking.Output}\sphinxstyleindexpageref{autoapi/nexoclom2/particle_tracking/Output/index:\detokenize{module-nexoclom2.particle_tracking.Output}}
\item\relax\sphinxstyleindexentry{nexoclom2.particle\_tracking.packets}\sphinxstyleindexpageref{autoapi/nexoclom2/particle_tracking/packets/index:\detokenize{module-nexoclom2.particle_tracking.packets}}
\item\relax\sphinxstyleindexentry{nexoclom2.particle\_tracking.rk5\_integrator}\sphinxstyleindexpageref{autoapi/nexoclom2/particle_tracking/rk5_integrator/index:\detokenize{module-nexoclom2.particle_tracking.rk5_integrator}}
\item\relax\sphinxstyleindexentry{nexoclom2.particle\_tracking.starting\_point}\sphinxstyleindexpageref{autoapi/nexoclom2/particle_tracking/starting_point/index:\detokenize{module-nexoclom2.particle_tracking.starting_point}}
\item\relax\sphinxstyleindexentry{nexoclom2.particle\_tracking.state\_vectors}\sphinxstyleindexpageref{autoapi/nexoclom2/particle_tracking/state_vectors/index:\detokenize{module-nexoclom2.particle_tracking.state_vectors}}
\item\relax\sphinxstyleindexentry{nexoclom2.particle\_tracking.VariableIntegrator}\sphinxstyleindexpageref{autoapi/nexoclom2/particle_tracking/VariableIntegrator/index:\detokenize{module-nexoclom2.particle_tracking.VariableIntegrator}}
\item\relax\sphinxstyleindexentry{nexoclom2.solarsystem}\sphinxstyleindexpageref{autoapi/nexoclom2/solarsystem/index:\detokenize{module-nexoclom2.solarsystem}}
\item\relax\sphinxstyleindexentry{nexoclom2.solarsystem.IoTorus}\sphinxstyleindexpageref{autoapi/nexoclom2/solarsystem/IoTorus/index:\detokenize{module-nexoclom2.solarsystem.IoTorus}}
\item\relax\sphinxstyleindexentry{nexoclom2.solarsystem.IoTorus\_bak}\sphinxstyleindexpageref{autoapi/nexoclom2/solarsystem/IoTorus_bak/index:\detokenize{module-nexoclom2.solarsystem.IoTorus_bak}}
\item\relax\sphinxstyleindexentry{nexoclom2.solarsystem.load\_kernels}\sphinxstyleindexpageref{autoapi/nexoclom2/solarsystem/load_kernels/index:\detokenize{module-nexoclom2.solarsystem.load_kernels}}
\item\relax\sphinxstyleindexentry{nexoclom2.solarsystem.SSObject}\sphinxstyleindexpageref{autoapi/nexoclom2/solarsystem/SSObject/index:\detokenize{module-nexoclom2.solarsystem.SSObject}}
\item\relax\sphinxstyleindexentry{nexoclom2.solarsystem.SSObject\_bak}\sphinxstyleindexpageref{autoapi/nexoclom2/solarsystem/SSObject_bak/index:\detokenize{module-nexoclom2.solarsystem.SSObject_bak}}
\item\relax\sphinxstyleindexentry{nexoclom2.solarsystem.SSPosition}\sphinxstyleindexpageref{autoapi/nexoclom2/solarsystem/SSPosition/index:\detokenize{module-nexoclom2.solarsystem.SSPosition}}
\item\relax\sphinxstyleindexentry{nexoclom2.solarsystem.SSPositionTime}\sphinxstyleindexpageref{autoapi/nexoclom2/solarsystem/SSPositionTime/index:\detokenize{module-nexoclom2.solarsystem.SSPositionTime}}
\item\relax\sphinxstyleindexentry{nexoclom2.solarsystem.TorusImage}\sphinxstyleindexpageref{autoapi/nexoclom2/solarsystem/TorusImage/index:\detokenize{module-nexoclom2.solarsystem.TorusImage}}
\item\relax\sphinxstyleindexentry{nexoclom2.solarsystem.VoyagerTorus}\sphinxstyleindexpageref{autoapi/nexoclom2/solarsystem/VoyagerTorus/index:\detokenize{module-nexoclom2.solarsystem.VoyagerTorus}}
\item\relax\sphinxstyleindexentry{nexoclom2.utilities}\sphinxstyleindexpageref{autoapi/nexoclom2/utilities/index:\detokenize{module-nexoclom2.utilities}}
\item\relax\sphinxstyleindexentry{nexoclom2.utilities.database\_operations}\sphinxstyleindexpageref{autoapi/nexoclom2/utilities/database_operations/index:\detokenize{module-nexoclom2.utilities.database_operations}}
\item\relax\sphinxstyleindexentry{nexoclom2.utilities.exceptions}\sphinxstyleindexpageref{autoapi/nexoclom2/utilities/exceptions/index:\detokenize{module-nexoclom2.utilities.exceptions}}
\item\relax\sphinxstyleindexentry{nexoclom2.utilities.NexoclomConfig}\sphinxstyleindexpageref{autoapi/nexoclom2/utilities/NexoclomConfig/index:\detokenize{module-nexoclom2.utilities.NexoclomConfig}}
\item\relax\sphinxstyleindexentry{nexoclom2.utilities.sci\_notation}\sphinxstyleindexpageref{autoapi/nexoclom2/utilities/sci_notation/index:\detokenize{module-nexoclom2.utilities.sci_notation}}
\indexspace
\bigletter{r}
\item\relax\sphinxstyleindexentry{RadialAngDist}\sphinxstyleindexpageref{autoapi/RadialAngDist/index:\detokenize{module-RadialAngDist}}
\item\relax\sphinxstyleindexentry{reformat\_gvalues}\sphinxstyleindexpageref{autoapi/reformat_gvalues/index:\detokenize{module-reformat_gvalues}}
\indexspace
\bigletter{s}
\item\relax\sphinxstyleindexentry{SputteringFluxDist}\sphinxstyleindexpageref{autoapi/SputteringFluxDist/index:\detokenize{module-SputteringFluxDist}}
\item\relax\sphinxstyleindexentry{SurfMapSpatDist}\sphinxstyleindexpageref{autoapi/SurfMapSpatDist/index:\detokenize{module-SurfMapSpatDist}}
\item\relax\sphinxstyleindexentry{SurfMapSurfInt}\sphinxstyleindexpageref{autoapi/SurfMapSurfInt/index:\detokenize{module-SurfMapSurfInt}}
\item\relax\sphinxstyleindexentry{SurfSpotSpatDist}\sphinxstyleindexpageref{autoapi/SurfSpotSpatDist/index:\detokenize{module-SurfSpotSpatDist}}
\indexspace
\bigletter{t}
\item\relax\sphinxstyleindexentry{TemperatureDependentSurfInt}\sphinxstyleindexpageref{autoapi/TemperatureDependentSurfInt/index:\detokenize{module-TemperatureDependentSurfInt}}
\indexspace
\bigletter{u}
\item\relax\sphinxstyleindexentry{UniformSpatDist}\sphinxstyleindexpageref{autoapi/UniformSpatDist/index:\detokenize{module-UniformSpatDist}}
\end{sphinxtheindex}

\renewcommand{\indexname}{Index}
\printindex
\end{document}